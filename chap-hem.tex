\chapter{适应异质界面催化的高通量自动流程软件}
\section{适应异质界面催化的高通量自动流程软件}
高通量自动流程为变革材料研发模式,加速材料的研发进程提供了强有力的工具,为实现“材料基因组”基本思想提供了软件支持。目前高通量自动流程主要通过建立典型材料的数据库(如无机材料数据库、合金材料数据库、催化材料数据库、光电材料数据库等)来辅助提升和优化结构和组分,获得理想的材料物性。现有的高通量自动流程计算软件都是针对特定的应用场景开发的,MatCloud 对应用场景的依赖最低,却只适用一般简单的应用研究。
1. 基本设想
合金材料与催化活性材料是当前材料研究领域重点方向,这两类材料的物理、化学性能都与组成元素的组分、结构有着密切的关联,它们的性质最终都需要用组成元素的原子间相互作用势描述。对于合金材料而言,微观原子间相互作用决定了体系的宏观力学性质;对于催化材料,其与反应物的相互作用强弱,将直接影响着反应活化能的大小。遗憾的是,现有的高通量计算软件都无法满足这些材料的原子间相互作用的需求。我们充分调研了各类高通量计算软件后,提出 了针对描写复杂的原子间相互作用势应用场景的高通量计算流程实现方案:

(1)增强结构建模功能——除了可以产生原子、分子、表面、体相结构外,还可以根据要求对结构进行约束,如允许表面态原子在指定方向运动、分子在表面以特定方式堆积、体相结构的缺陷、嵌隙等;为提升模块的对称性分析功能,引入 AFLOW 的“标准化”对称性分析方案。
(2)扩大计算模拟尺度——多尺度模拟是材料计算中的难点,简单地集成多个尺度的计算软件,如果不考虑不同尺度间的耦合,很难得到理想的结果,利用 MongoDB 支持的FireWorks 计算流程管理,可以设计更合理完善的程序流程,通过微观尺度的第一原理计算获得介观或宏观尺度的计算物性或者使不同尺度的计算结果更好地实现耦合自洽。
(3)计算结果的分析重点是获得准确的多体相互作用的势函数——传统的多体相互作用势是通过双体相互作用原子势叠加得到的,一般误差较大,近年来,随着机器学习方法的兴起,利用高斯过程回归(Gaussian Process Regression, GPR)、贝叶斯优化(Bayesian Optimization, BO)和人工神经网络(Artificial Neural Network, ANN )优化多体相互作用势取得了很大的进步,有望成为探索原子间相互作用势的新方法。利用 Python 模块组合的灵活性,我们将已有高通量计算软件中的开源模块,重新组合、扩展成基本符合要求的软件,结构功能模块主要以 ASE的模块为基础,修改扩充后得到;计算模块则在 ASE 接口基础上,采用 MP 的 FireWorks管理模式;结果分析部分在 ASE 和 MP 的 Pymatgen 基础上,充分利用 Python 提供的机器学习功能模块,扩充现有软件的计算分析能力。软件的程序结构见 Fig. 3。

我们将分别选取无定型碳在催化活性材料 TiO 2 的表面上的相互作用为典型应用,检验我们的软件效果。
Fig. 3 适应多体相互作用的高通量计算流程结构示意.

\section{$\vec k\cdot\vec p$方法加速计算}
材料的电子结构计算的主要任务是得到体系的能带结构和基态总能,能带结构是电子态在倒空间中沿高对称方向的分布曲线,基态总能则是通过微观结构获得宏观物性的基础。传统的电子结构计算,高对称方向的选择完全依赖经验和人工,有很大的随意性,往往不能涵盖全部高对称性k点。AFLOW软件首先提出标准化对称性分析的思想并程序化,但由于AFLOW是非开源软件,我们开发了自己的材料模型对称性分析模块,根据具体计算任务的特点,在保证计算精度的同时,充分利用材料的结构对称性降低计算量,提高现有“结构弛 豫-静态计算”自动流程的计算效率及准确性。对称性分析模块能在对结构原胞对称性分析基础上生成“标准化”的能带路径k-path,消除能带计算由于Brillouin区路径选择不恰当造成的计算结果可比性较差的问题。
Fig. 4 (a)对称性分析模块给出的 FCC 结构的“标准化”能带路径 k-path;
根据标准能带路径 k-path 绘制的 GeF 4
能带图。

Fig. 4a 给出了 FCC 结构的 Brillouin 区的标准化能带路径 k-path 和能带图。与传统的能带图(Fig. 4a)对比,采用标准化路径能带图表现的电子态在空间的分布考虑的对称性方向更丰富,表现的电子态在能量空间的分布也更全面。标准化能带路径 k-path 选取的这一特点,对于 GeF 4 (Fig. 4b),传统能带计算一般把最小带隙定位为 Γ 点上的价带和导带间能量差,是直接带隙。但是标准化能带路径计算表明,实际上 GeF 4 是间接带隙,价带顶位于 H 点,导带底位于 Γ 点,因为传统能带计算中,经验选取的能带路径中通常不包含 H 点,因此出现误判,而通过标准化能带路径可以有效地避免这类问题。类似地,对称性分析模块标准化,使得基态时组成受力分析更准确,对后续宏观物理力学计算精度的提高也有重要的贡献。

适用于简单结构的对称性分析很难处理复杂体系,对于这样的问题,我们将重点放在复杂体系的原子相互作用势。对于催化材料 TiO2 表面上无定型碳的 C-C 原子间相互作用,我们分别分析了二体(2b)、三体(3b)和 Smooth Overlap of Atomic Positions (SOAP) [47] descriptor优化计算的多原子间相互作用,最终叠加得到原子间相互作用曲线(Fig. 5)。结果表明,单独考虑 C-C 原子间二体相互作用,仅在平衡位置附近与第一原理计算结果符合,当原子间距增大时,误差明显增大,考虑二体-三体相互作用后,结果有所改善;单纯的 SOAP 优化,当原子间距较小时误差非常大(此时原子间相互作用势以双体-三体效应为主)。只有综合考虑多体相互作用叠加后的结果,才可以得到与 DFT 计算吻合的结果。
Fig.5 ANN 优化的 TiO 2 表面的 C-C 相互作用势能面曲线,以 DFT 计算(蓝色)为参考。不同结果考虑的原子 间相互作用对势能面贡献不同

3. 后续的工作
Ni-基单晶高温合金是重要的先进航空发动机的关键材料,在材料基因工程课题“高通量并发式材料计算算法和软件”中,重点研究的内容之一就是通过高通量计算得到该类合金材料的原子间相互势函数的正确描述。高温强度与高熔点元素的含量有直接关系。第三、四代单晶高温合金中加入的高熔点元素浓度高达 20wt.\%以上,高浓度的高熔点元素使得合金极易析出不稳定相,明显影响合金的使用寿命。课题以调控合金化学元素的适宜匹配为关键科学问题之一。同时平行的结合热力学及动力学计算研究合金元素对不稳定相析出行为的影响,并进行对相应的理论分析及预测。获得 Ni-基单晶高温合金的组织稳定性控制方法。将在高通量并发式集成计算结果的基础上,研究 Ni-基单晶高温合金多元体系的稳态及亚稳状态,优化合金组织结构。同时基于热力学及动力学模拟结合显微组织分析的方法,研究合金元素在固溶和时效过程中的扩散规律以及相的演变规律。这部分工作的核心微观机制,就是确定组分元素原子间的相互作用。确定重元素间的多体相互作用势一直是计算物理中的重点和难点,近年来随着机器学习方法和软件的普及,利用多种优化算法获得多体相互作用势的研究也取得了不小的进步,我们设想在机器学习研究 TiO 2 表面多体相互作用势研究基础上,引入 GPR、BO 优化方法研究 Ni-基单晶高温合金中的多体相互作用。由于 Ni-基单晶高温合金的元素组分复杂,为获得足够多的机器学习的训练数据,必须有强大的高通量多尺度计算支持,因此除了引入和发展 Python 的机器学习模块,针对 Ni-基单晶高温合金计算过程中,要求完成优化≥500 作业量级的高通量并发式计算任务。在FireWorks 支持的框架中,发展高通量并发式计算与多尺度建模方法相结合,构建高通量多尺度序列算法,实现量子力学、多尺度与高通量并发式计算相集成的全链条计算方法和软件系统;多尺度建模方法学主要由材料科学与物理学发展所驱动。实现多尺度建模与算法,主要取决于对问题的物理洞察,预期与高通量并发式计算相集成,可解决具有多自由度的复杂体系问题,对揭示材料内禀性质及合金设计具有重要意义。多尺度建模方法学中的一个关键性问题为:在跨越空间尺度与学科交叉中创新发展“桥接”模式。发展多尺度建模及跨层次桥接算法须将量子力学、热力学、动力学及力性自洽于或分别内涵于解析表述之中,特别的物理及材料科学具本质意义,离散的微观结构是唯象行为的基础或源泉。多尺度现象的物理基础,源于原子间相互作用规律及对相关电子或电荷分布的深刻认识和表述。多尺度科学被外部技术需求和信息科学发展所牵引,同时也被材料科学快速低耗发展所推动。

