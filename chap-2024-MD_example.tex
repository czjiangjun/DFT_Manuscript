\newpage
\noindent{\heiti 课后习题}

%例题 1:双原子分子的 Lennard-Jones 势能与力计算
{\heiti 1. 已知\ch{Ar}原子对的 \textrm{Lennard-Jones}势中参数分别为 $\varepsilon=0.238~\mathrm{kcal/mol}$,$\sigma=3.4~\mathrm{\AA}$,计算当两个\ch{Ar}原子的间距为$r=4.0~\mathrm{\AA}$ 时的相互作用势能及原子所受的作用力。}

%解题过程:
\noindent{\heiti 答案:}\\
由式\eqref{eq:Potential_L-J}\textrm{Lennard-Jones}~势函数公式:
\begin{displaymath}
	U_{\mathrm{LJ}}(r)=4\varepsilon\bigg[\bigg(\dfrac{\sigma}{r}\bigg)^{12}−\bigg(\dfrac{\sigma}{r}\bigg)^6\bigg]
\end{displaymath}
    代入参数:
    \begin{displaymath}
	    \begin{aligned}
		    \bigg(\dfrac{\sigma}{r}\bigg)=&\dfrac{3.4}{4.0}=0.85\\
		    \bigg(\dfrac{\sigma}{r}\bigg)^{12}=&0.85^{12}\approx0.142\\
		    \bigg(\dfrac{\sigma}{r}\bigg)^{6}=&0.85^{6}\approx0.377\\
	    \end{aligned}
    \end{displaymath}
    \begin{displaymath}
	 \therefore~U_{\mathrm{LJ}}(r)=4\times0.238\times(0.142−0.377)\approx−0.226~\mathrm{kcal/mol}
    \end{displaymath}
    类似地,原子间相互作用力公式\eqref{eq:Focrce_L-J}%(沿径向,斥力为正)
:
\begin{displaymath}
	F_{\mathrm{LJ}}(r)=-\dfrac{\mathrm{d}U}{\mathrm{d}r}=\dfrac{24\varepsilon}{r}\bigg[2\bigg(\dfrac{\sigma}r\bigg)^{13}-\bigg(\dfrac{\sigma}r\bigg)^7\bigg]
\end{displaymath}
    代入参数:
    \begin{displaymath}
	    \begin{aligned}
		   2\bigg(\dfrac{\sigma}{r}\bigg)^{13}=&2\times0.85^{13}\approx2\times0.121=0.242\\
		    \bigg(\dfrac{\sigma}{r}\bigg)^{7}=&0.85^{7}\approx0.319\\
	    \end{aligned}
    \end{displaymath}
    \begin{displaymath}
	\therefore~F_{\mathrm{LJ}}(r)=24\times0.238\times(0.242−0.319)\approx−0.109~\mathrm{kcal/(mol\cdot\AA)}
    \end{displaymath}
    {\heiti 答}:~两个\ch{Ar}原子间的势能为 $-0.226~\mathrm{kcal/mol}$,作用力为 $-0.109~\mathrm{kcal/(mol\cdot\AA)}$,负值表示吸引力。

%例题 6:多原子分子的键角弯曲能计算
{\heiti 2. 水分子(\ch{H2O})中\ch{O}-\ch{H}的键长$r=0.958~\mathrm{\AA}$,\ch{H}-\ch{O}-\ch{H}形成的键角$\theta=104.5^{\circ}$。若已知力场函数形式为:
\begin{displaymath}
	\mbox{键角弯曲势}~V(\theta)=k(\theta−\theta_0)^2
\end{displaymath}
其中键角弯曲参数$k=500~\mathrm{kcal}/(\mathrm{mol}\cdot\mathrm{rad}^2)$,并已经$\theta_0=109.5^{\circ}$。计算该分子的键角弯曲能。}

%解题过程:
\noindent{\heiti 答案:}\\
    首先将各角度转换为弧度:
    \begin{displaymath}
	    \begin{aligned}
	    \theta=104.5^{\circ}\times\dfrac{\pi}{180^{\circ}}\approx1.824~\mathrm{rad}\\
	    \theta_0=109.5^{\circ}\times\dfrac{\pi}{180^{\circ}}\approx1.911~\mathrm{rad} 
	    \end{aligned}
    \end{displaymath}
    因此有角度差:
    \begin{displaymath}
	    \Delta\theta=\theta−\theta_0=1.824−1.911\approx−0.087~\mathrm{rad}
    \end{displaymath}
    计算弯曲能:
    \begin{displaymath}
	    \begin{aligned}
		    V(\theta)=&k\times(\theta-\theta_0)^2=k\times(\Delta\theta)^2\\
		    =&500\times(−0.087)^2\\
		    \approx&500\times0.00757\\
		    \approx&3.785~\mathrm{kcal/mol}
	    \end{aligned}
    \end{displaymath}
{\heiti 答}:水分子的键角弯曲能为 $3.785~\mathrm{kcal/mol}$。

%例题 2:Verlet 算法求解原子运动轨迹
    {\heiti 3. \ch{Ar}原子质量$m=39.95~\mathrm{a.m.u}$,初始位置$r_0=0~\mathrm{\AA}$,速度$v_0=100~\mathrm{m/s}$,在$t=0$ 时受力$F_0=1.0~\mathrm{pN}$。使用\textrm{Verlet}算法计算$t=1~\mathrm{fs}$时,原子位于$r_1$和速度$v_{0.5}$~(时间步长取为$\Delta t=1~\mathrm{fs}$)。}

%解题过程:
    \noindent{\heiti 答案:}\\
    单位转换:
    \begin{displaymath}
	    \begin{aligned}
%		    1~\mathrm{a.m.u}=&1.6605\times10^{-27}\mathrm{kg}\\
		    \mbox{原子质量}:~39.95~\textrm{a.m.u}=&39.95\times1.6605\times10^{-27}=6.635\times10^{-26}~\mathrm{kg}\\
		    \mbox{原子受力}:~1~\mathrm{pN} =&1.0\times10^{-12}~\mathrm{N}\\
		    \mbox{初始速度}:~1~\mathrm{fs} =&1.0\times10^{-15}~\mathrm{s}
	    \end{aligned} 
    \end{displaymath}
    因此有\ch{Ar}原子加速度
	\begin{displaymath}
		a_0=\dfrac{F_0}{m}=\dfrac{10^{-12}}{6.635\times10^{-26}}\approx1.507\times10^{13}~\mathrm{m/s}
	\end{displaymath}
	根据\textrm{Verlet}算法,原子位置的更新:
	\begin{displaymath}
		r_1=r_0+v_0\Delta t+\frac12a_0\Delta t^2
	\end{displaymath}
	将速度$v_0=100~\mathrm{m/s}$和时间步长$\Delta t=10^{−15}~\mathrm{s}$代入,有
	\begin{displaymath}
		\begin{aligned}
			\mbox{原式}=&0+100\times10^{−15}+\frac12\times1.507\times10^{13}\times(10^{−15})2\\
			=&10^{−13}+7.535\times10^{−18}\approx1.000075\times10^{−13}~\mathrm{m}\\
			=&1.000~\mathrm{\AA}
		\end{aligned}
	\end{displaymath}
    类似地,由半时间步长速度公式:
    \begin{displaymath}
\begin{aligned}
	v_{0.5}=&v_0+\frac12a_0\Delta t\\
	=&100+\frac12\times1.507\times10^{13}\times10^{−15}\\
	=&100+7.535×10^{−3}
	\approx100.0075~\mathrm{m/s}
\end{aligned}
    \end{displaymath}
    {\heiti 答}:$t=1~\mathrm{fs}$时,\ch{Ar}原子的位置为$1.000~\mathrm{\AA}$处,半时间步场速度为$100.0075~\mathrm{m/s}$。

%例题 3:三维原子体系的动能与温度计算
    {\heiti 4. 模拟盒中有3个\ch{Ar}原子,速度分别是:
    \begin{itemize}
	    \item 原子1:~$v_x=100~\mathrm{m/s}$, $v_y=50~\mathrm{m/s}$, $v_z=20~\mathrm{m/s}$
	    \item 原子2:~$v_x=-50~\mathrm{m/s}$, $v_y=80~\mathrm{m/s}$, $v_z=30~\mathrm{m/s}$
	    \item 原子3:~$v_x=30~\mathrm{m/s}$, $v_y=-20~\mathrm{m/s}$, $v_z=-40~\mathrm{m/s}$
    \end{itemize}
    计算体系动能$E_k$和温度$T$~(\textrm{Boltzmann}常数取为$k_{\mathrm B}=0.0019872~\mathrm{kcal/(mol\cdot K)}$)。}

%    解题过程:
    \noindent{\heiti 答案:}\\
    单个原子动能计算:
    \ch{Ar}原子的质量
    \begin{displaymath}
	    m=39.95~\mathrm{a.m.u}=39.95\times1.6605\times10^{-27}=6.635\times10^{-26}~\mathrm{kg}    
    \end{displaymath}
    各原子的动能:
    \begin{displaymath}
	    \begin{aligned}
		    \varepsilon_k^1=&\frac12m(v_{x1}^2+v_{y1}^2+v_{z1}^2)\\
		    =&\frac12\times6.635\times10^{-26}\times\big[100^2+50^2+20^2\big]\\
		    =&\frac12\times6.635\times10^{-26}\times12900\\
		    \varepsilon_k^2=&\frac12m(v_{x2}^2+v_{y2}^2+v_{z2}^2)\\
		    =&\frac12\times6.635\times10^{-26}\times\big[(-50)^2+80^2+30^2\big]\\
		    =&\frac12\times6.635\times10^{-26}\times9800\\
		    \varepsilon_k^3=&\frac12m(v_{x3}^2+v_{y3}^2+v_{z3}^2)\\
		    =&\frac12\times6.635\times10^{-26}\times\big[30^2+(-20)^2+(-40)^2\big]\\
		    =&\frac12\times6.635\times10^{-26}\times2900
	    \end{aligned}
    \end{displaymath}
   三个原子的总动能为
   \begin{displaymath}
	   \begin{aligned}
		   \varepsilon_k^{\mathrm{tot}}=&E_k^1+E_k^2+E_k^3\\
		   =&\frac12\times6.635\times10^{-26}\times(12900+9800+2900)\\
		   =&3.3175\times10^{-26}\times25600\\
		   \approx&8.593\times10^{-22}~\mathrm{J}
	   \end{aligned}
   \end{displaymath}
   根据转换关系:
   \begin{displaymath}
	   \begin{aligned}
		   1~\mathrm{mol}=&6.022\times10^{23}~\mbox{原子}\\
		   1~\mathrm{J}=&0.239\times10^{-3}~\mathrm{kcal}
	   \end{aligned}
   \end{displaymath}
   对于3个\ch{Ar}原子的模拟盒子代表的宏观体系,有总的动能为
   \begin{displaymath}
	   \begin{aligned}
	   E_k^{tot}=&\sum\varepsilon_k^{tot}\\
	   =&8.593\times10^{-22}\times0.239\times10^{-3}\times6.022\times10^{22}\\
	   \approx&12.34~\mathrm{kcal/mol}
	   \end{aligned}
   \end{displaymath}
   温度计算,%对于三维体系($n=3$)自由度: $N=3n=9$
   由温度估算公式
   \begin{displaymath}
	   E_k^{tot}=\frac{n}2Nk_{\mathrm{B}}T=\frac32Nk_{\mathrm{B}}T\Rightarrow T=\dfrac{2E_k^{tot}}{nNk_{\mathrm{B}}}=\dfrac23\dfrac{E_k^{tot}}{Nk_{\mathrm{B}}}
   \end{displaymath}
   因此,取$N=1~\mathrm{mol}$,有
   \begin{displaymath}
	   T=\dfrac23\times\dfrac{12.34}{1\times0.0019872}=4147~\mathrm{K}
   \end{displaymath}
   {\heiti 答}:模拟宏观体系的总动能为$12.34~\mathrm{kcal/mol}$,温度为$4147~\mathrm{K}$

%例题 5:分子体系的总能量与压力计算
{\heiti 5. 在边长$L=10~\mathrm{\AA}$的立方模拟盒中,通过100个\ch{Ar}原子的获得的体系宏观总动能为$E_k^{tot}=200~\mathrm{kcal/mol}$,总势能为$E_p^{tot}=-500~\mathrm{kcal/mol}$,计算该\ch{Ar}原子体系宏观总能量$E_{\mathrm{total}}$和压力$P$~(提示:~使用\textrm{virial}定理,考虑理想气体,取$1~\mathrm{atm}=1.013\times10^5~\mathrm{Pa}$)。}

% 解题过程:
\noindent{\heiti 答案:}\\
体系宏观总能量:
\begin{displaymath}
	\begin{aligned}
		E_{\mathrm{total}}=&E_k^{tot}+E_p^{tot}\\
		=&200+(-500)\\
		=&-300~\mathrm{kcal/mol}
	\end{aligned}
\end{displaymath}
\begin{displaymath}
	\begin{aligned}
		\mbox{根据\textrm{virial}定理,宏观压力可以表示为:~}P=&\dfrac1{V}\bigg(Nk_{\mathrm{B}}T+\frac13\bigg\langle\sum_{i=1}^N\vec r_i\cdot\nabla_iV_N(R)\bigg\rangle\bigg)\\
		\mbox{对于气体,结合动能与\textrm{virial}项,则原式}=&\dfrac{2E_k}{3V}+\dfrac1{3V}\bigg\langle\sum_{i=1}^N\vec r_i\cdot\nabla_iV_N(R)\bigg\rangle
	\end{aligned}
\end{displaymath}
如果进一步简化,利用利用理想气体近似,忽略势能\textrm{virial}项,假设动能主导,则:
\begin{displaymath}
	E_k^{tot}=\frac32Nk_{\mathrm{B}}T\Rightarrow k_{\mathrm{B}}T=\dfrac{2E_k^{tot}}{3N}
\end{displaymath}
将模拟对象$N=100~\mathrm{atoms}$(与模拟对象对应的宏观体系 $1~\mathrm{mol}$时,对于单原子\ch{Ar}体系,有$N=6.022\times10^{23}$),有系数:
\begin{displaymath}
	k_{\mathrm{B}}T=\dfrac{2E_k^{tot}}{3N}=\dfrac{2\times200}{3\times100}=1.333~\mathrm{kcal/mol}
\end{displaymath}
    模拟对象的微观体积:
    \begin{displaymath}
	    V=L^3=10^3=1000~\mathrm{\AA}^3=10000\times10^{-30}=1.0\times10^{-27}~\mathrm{m^{-3}}
    \end{displaymath}
则,宏观体系对应的理想气体压力:
\begin{displaymath}
	\begin{aligned}
		P=&\dfrac{Nk_{\mathrm{B}}T}{V}=\dfrac{100\times1.333\times4184}{1.0\times10^{-27}}~\mbox{此处利用}1~\mathrm{kcal}=4184~\mathrm{J}\\
		=&5.57\times10^{32}~\mathrm{Pa}
	\end{aligned}
\end{displaymath}
压力单位换算成\textrm{atm}则有
\begin{displaymath}
	P=\dfrac{5.57\times10^{32}}{1.1013\times10^5}\approx5.5\times5.5\times10^{27}~\mathrm{atm}
\end{displaymath}
注意:实际计算需考虑\textrm{virial}项~(由势能导出的力贡献),此处的计算过程简化为理想气体模型,仅作示例。\\
{\heiti 答}:宏观体系的总能量为$-300~\mathrm{kcal/mol}$,采用理想气体近似,压力约为$5.5\times10^{27}~\mathrm{atm}$。

%例题 4:均方位移(MSD)与扩散系数计算
   {\heiti 6. 某原子在 $100~\mathrm{ps}$内的均方位移\textrm{(MSD)}数据如下表所示(单位:$\mathrm{\AA}^2$):
\begin{table}[!h]
	\centering
  \begin{tabular}{lcccccc}
    \toprule
    时间$t~(\mathrm{ps})$	&50	&100	&150	&200	&250	&300\\
    \midrule
    \textrm{MSD}($\mathrm{t}$)	&2.5	&5.2	&7.8	&10.5	&13.2	&15.9\\
    \bottomrule
  \end{tabular}
\end{table}
\\若已知\textrm{Einstein}三维扩散关系表达式为
\begin{displaymath}
	\mathrm{MSD(t)}=6Dt%(三维扩散)
\end{displaymath}
计算扩散系数D~(单位:$\mathrm{cm}^2/\mathrm{s}$)。}

%解题过程:						
\noindent{\heiti 答案:}\\
对数据进行线性拟合,取时间范围$t=100\sim300~\mathrm{ps}$区间~(满足扩散线性区),计算近似线性斜率$k$:
\begin{displaymath}
	k=\dfrac{15.9-5.2}{300-100}=\dfrac{10.7}{200}=0.0535~\mathrm{\AA}^2/\mathrm{ps}
\end{displaymath}
因此有
\begin{displaymath}
	\begin{aligned}
		\mbox{线性斜率} k=&6D\\
		D=&\dfrac{k}6=\dfrac{0.0535}6~\mathrm{\AA}^2/\mathrm{ps}\approx0.00892~\mathrm{\AA}^2/\mathrm{ps}
	\end{aligned}
\end{displaymath}
注意单位转换关系:
\begin{displaymath}
	\begin{aligned}
		1~\mathrm{\AA} =&1.0\times10^{-8}~\mathrm{cm}\\
		1~\mathrm{ps}=&1.0\times10^{-12}~\mathrm{s}\\
		1~\mathrm{\AA}^2/\mathrm{ps}=&(10^{-8})^2/10^{-12}=10^{-4}~\mathrm{cm^2/s}
	\end{aligned}
\end{displaymath}
因此有
\begin{displaymath}
	D=0.00892~\mathrm{\AA}^2/\mathrm{ps}=0.00892\times10^{-4}=8.92\times10^{-7}~\mathrm{cm^2/s}
\end{displaymath}
{\heiti 答}:根据\textrm{Einstein}关系计算的扩散系数为 $8.92\times10^{-7}~\mathrm{cm^2/s}$。

%总结
%以上例题覆盖了分子动力学模拟的核心计算模块:
%
%    力场相互作用(LJ 势、键角势);
%    运动方程求解(Verlet 算法);
%    热力学性质(能量、温度、压力);
%    扩散行为(MSD 与扩散系数)。
%    实际模拟中需结合编程实现(如 LAMMPS、GROMACS),但手算例题可帮助理解底层物理原理和数学逻辑。
%
