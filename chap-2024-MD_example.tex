%例题 1:双原子分子的 Lennard-Jones 势能与力计算
1. 已知\textrm{Ar}<++>原子对的 Lennard-Jones 参数为ϵ=0.238 kcal/mol,σ=3.4 Å,计算两原子间距r=4.0 Å 时的相互势能及作用力。
解题过程:

    Lennard-Jones 势能公式:U(r)=4ϵ[(rσ​)12−(rσ​)6]
    代入参数:(rσ​)=4.03.4​=0.85(rσ​)12=0.8512≈0.142,(rσ​)6=0.856≈0.377U(r)=4×0.238×(0.142−0.377)≈−0.226 kcal/mol
    相互作用力公式(沿径向,斥力为正):F(r)=−drdU​=24ϵ[2(rσ​)13−(rσ​)7]r1​
    代入参数:2(rσ​)13=2×0.8513≈2×0.121=0.242(rσ​)7=0.857≈0.319

Å


    结论:势能为 - 0.226 kcal/mol,作用力为 - 0.109 kcal/(mol・Å)(负值表示吸引力)。

例题 2:Verlet 算法求解原子运动轨迹
问题:某原子质量m=39.95 amu,初始位置r0​=0 Å,速度v0​=100 m/s,在t=0 时受力F0​=1.0 pN。使用 Verlet 算法计算t=1 fs 时的位置r1​和速度v0.5​(时间步长Δt=1 fs)。
解题过程:

    单位转换:
        1 amu = 1.6605×10⁻²⁷ kg,39.95 amu = 6.635×10⁻²⁶ kg
        1 pN = 10⁻¹² N,1 fs = 10⁻¹⁵ s
        加速度

        ²
    Verlet 算法位置更新公式:r1​=r0​+v0​Δt+21​a0​Δt2
    代入数值:v0​=100 m/s,Δt=10−15 sr1​=0+100×10−15+21​×1.507×1013×(10−15)2=10−13+7.535×10−18≈1.000075×10−13 m=1.000 A˚
    半时间步速度公式:v0.5​=v0​+21​a0​Δt=100+21​×1.507×1013×10−15=100+7.535×10−3≈100.0075 m/s
    结论:t=1 fs 时位置为 1.000 Å,半时间步速度为 100.0075 m/s。

例题 3:三维原子体系的动能与温度计算
问题:模拟盒中有 3 个氩原子,速度分别为:

    原子 1:vx​=100 m/s, vy​=50 m/s, vz​=20 m/s
    原子 2:vx​=-50 m/s, vy​=80 m/s, vz​=30 m/s
    原子 3:vx​=30 m/s, vy​=-20 m/s, vz​=-40 m/s
    计算体系动能Ek​和温度T(玻尔兹曼常数kB​=0.0019872 kcal/(mol·K))。
    解题过程:


    单个原子动能计算:
    氩原子质量m=39.95 amu=6.635×10⁻²⁶ kg,单个原子速度平方:
        原子 1:

²²²
²
原子 2:
²²²
²
原子 3:
²²²

        ²
    总动能:Ek​=21​m(v12​+v22​+v32​)=21​×6.635×10−26×(12900+9800+2900)=3.3175×10−26×25600≈8.593×10−22 J
    转换为 kcal/mol:1 J=0.239×10⁻³ kcal,1 mol=6.022×10²³ 原子Ek​=8.593×10−22×0.239×10−3×6.022×1023≈12.34 kcal/mol
    温度计算(三维体系自由度n=3N=9):Ek​=2n​NkB​T⇒T=nNkB​2Ek​​
    (注:N=1 mol 时,Ek​=23​NkB​T)T=3×0.00198722×12.34​≈4147 K
    结论:总动能为 12.34 kcal/mol,温度为 4147 K。

例题 4:均方位移(MSD)与扩散系数计算
问题:某原子在 100 ps 内的 MSD 数据如下(单位:Ų):

时间t (ps)	50	100	150	200	250	300
MSD(t)	2.5	5.2	7.8	10.5	13.2	15.9
使用爱因斯坦关系式计算扩散系数D(单位:cm²/s)。						
解题过程:						

    爱因斯坦关系式:MSD(t)=6Dt(三维扩散)
    对数据进行线性拟合,取t=100~300 ps 区间(满足扩散线性区),计算斜率:
        斜率

Ų
单位转换:

    1 Å=10⁻⁸ cm,1 ps=10⁻¹² s

Ų²Å²Å²
转换为 cm²/s:²


        结论:扩散系数为 8.92×10⁻⁷ cm²/s。

例题 5:分子体系的总能量与压力计算
问题:在边长L=10 Å 的立方模拟盒中,100 个氩原子的总动能Ek​=200 kcal/mol,总势能Ep​=-500 kcal/mol,计算体系总能量Etotal​和压力P(使用维里定理,1 atm=1.013×10⁵ Pa)。
解题过程:

    总能量:Etotal​=Ek​+Ep​=200+(−500)=−300 kcal/mol
    维里定理计算压力:P=V1​(1NkB​T​+31​⟨∑i<j​Fij​⋅rij​⟩)
    另一种形式(结合动能与维里项):P=3V2Ek​​+3V1​⟨∑i<j​Fij​⋅rij​⟩
    但更简单的方法是利用理想气体近似(忽略势能维里项,假设动能主导):Ek​=23​NkB​T⇒kB​T=3N2Ek​​
    代入N=100 atoms(视为 1 mol 时N=6.022×10²³,此处以单原子计):kB​T=3×1002×200​=1.333 kcal/mol
    体积

ų³
压力公式:³(

    )=10−27100×1.333×4184​=5.57×1032 Pa
    转换为 atm:P=1.013×1055.57×1032​≈5.5×1027 atm
    注:实际计算需考虑维里项(由势能导出的力贡献),此处简化为理想气体模型,仅作示例。
    结论:总能量为 - 300 kcal/mol,压力约为 5.5×10²⁷ atm(理想气体近似)。

例题 6:多原子分子的键角弯曲能计算
问题:水分子(H₂O)中 O-H 键长r=0.958 Å,H-O-H 键角θ=104.5°,力场参数为:键角弯曲势V(θ)=k(θ−θ0​)2,其中k=500 kcal/(mol·rad²),θ0​=109.5°。计算该分子的键角弯曲能。
解题过程:

    角度转换为弧度:θ=104.5°×180°π​≈1.824 radθ0​=109.5°×180°π​≈1.911 rad
    角度偏差:Δθ=θ−θ0​=1.824−1.911≈−0.087 rad
    弯曲能计算:V(θ)=k(Δθ)2=500×(−0.087)2≈500×0.00757≈3.785 kcal/mol
    结论:水分子的键角弯曲能为 3.785 kcal/mol。

总结
以上例题覆盖了分子动力学模拟的核心计算模块:

    力场相互作用(LJ 势、键角势);
    运动方程求解(Verlet 算法);
    热力学性质(能量、温度、压力);
    扩散行为(MSD 与扩散系数)。
    实际模拟中需结合编程实现(如 LAMMPS、GROMACS),但手算例题可帮助理解底层物理原理和数学逻辑。

