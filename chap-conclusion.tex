\chapter{结论}
高通量自动流程软件是近年来材料计算软件研发的一个重要领域,通过对主流的高通量
自动流程软件调研和分析发现,这类软件的开发主要由特定的应用需求和应用场景(如合金
材料、电池材料、催化材料、高分子材料等)所牵引,当前主要集中在围绕电子结构的第一
原理计算层次,有少量发展到分子动力学层次。对软件结构及其数据库维护和管理功能实现
的分析表明,目前软件主体结构和实现方案重点解决电子结构计算的流程自动化,但对于关
心原子间多体相互作用的复杂合金材料和催化活性材料研究,这些高通量自动流程软件都无
法完全满足需求。我们围绕催化合金材料模拟计算的应用场景,综合评估各类高通量计算软
件后,充分利用 Python 强大的功能扩展和可视化工具支持,整合 MP 的 FireWorks 和 ASE
的灵活建模功能,扩展对称性分析模块,实现标准化能带路径 k-path 生成,获得更完备的
材料电子结构;分析 TiO 2 催化表面无定形碳模拟原子间相互作用时;我们基于 FireWorks
和 MongoDB 支撑,创造性地引入机器学习分析工具,利用少量 DFT 计算结果,得到 TiO 2
表面上无定形碳的原子间相互作用势的精确表示,与精确的第一原理计算相比,误差<10%。
随着研究对象的日益复杂,对高通量自动流程软件的需求也将更高。该思路已经应用到
国家重点研发计划“材料基因工程关键技术与支撑平台”中的“高通量并发式材料计算算法和
软件”项目(项目编号: 2017YFB0701500)中,我们将以 Ni-基单晶高温合金材料为牵引,发展
可支持≥1000 作业量级的高通量并发式计算、可将量子力学、热力学、动力学及力性自洽的多尺度建模及跨层次桥接流程的软件平台,该平台集成的机器学习模块,将瞄准的 Ni-基单
晶高温合金材料的原子间相互作用势精确表示,服务 Ni-基单晶高温合金材料的组分-性能优
化。通过该应用需求的牵引,提升高通量自动流程软件平台对复杂体系的原子间多体相互作
用的模拟能力。
