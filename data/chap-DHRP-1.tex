\chapter{\rm{DHR}的合成}
%\section{\rm{DHR}的合成}
\textrm{DHR}的合成路线间图\ref{Fig:Synthesis-DHR},根据\textrm{DHR}的合成条件,以前驱体化合物1为起始原料,前驱体化合物1通过四氯化钛~(\ch{TiCl4})促进二苯并环庚酮二聚的改进方法制备。前驱体化合物1经两步反应转化为DHR:~在$0^{\circ}\mathrm{C}$下,前驱体化合物1与12当量的三氯化铁~(\ch{FeCl3})在\textrm{~(DCM)}/\ch{CH3NO2}混合溶剂中发生\textrm{Scholl}反应,同时生成两个五元环,得到前驱体化合物3,产率为87\%。通过优化后处理步骤~(采用萃取/沉淀法替代柱层析),前驱体化合物3的合成可到到克级的规模,且产率提升至98\%。前驱体化合物3与2,3-二氯-5,6-二氰基-1,4-苯醌\textrm{~(DDQ)}在二氧六环中发生脱氢反应,生成\textrm{DHR}中的两个七元环。优化反应条件后,使用6.0当量的\textrm{DDQ}时,\textrm{DHR}的产率为87\%。这里顺便提一下该步骤的无柱后处理方法:~通过物理气相传输\textrm{~(PVT)}纯化,可获得克级规模的纯\textrm{DHR},产率为60\%。
\begin{figure}[h!]
\centering
\vspace*{-0.1in}
\includegraphics[height=4.0in]{Figures/Synthesis-DHR.png}
%\caption{\fontsize{7.2pt}{4.2pt}\selectfont{含有\textrm{5/7/5}环系的共轭分子模型示意图,具有特定的多层堆积特性.}}%
\caption{\textrm{Dicyclohepta[ijkl,uvwx]rubicene~~(DHR)}的合成.}%
\label{Fig:Synthesis-DHR}
\end{figure}

作为对比,还尝试了“先脱氢形成两个七元环、后\textrm{Scholl}反应,形成两个五元环”的\textrm{DHR}合成路线~(见图\ref{Fig:Synthesis-DHR}):~在2.6当量\textrm{DDQ}存在下,前驱体化合物1脱氢生成前驱体化合物2的反应效率极高,15分钟内即可完成,产率为80\%;但在\ch{FeCl3}存在下,前驱体化合物2经\textrm{Scholl}反应生成\textrm{DHR}的尝试,但是尝试了很多方案,都没有能成功。

\section{前驱体化合物1的合成}
化学物1合成的反应式如图\ref{Fig:DHR_Synthesis-equation-1}所示。
\begin{figure}[h!]
\centering
\vspace*{-0.1in}
\includegraphics[height=1.9in]{Figures/DHR_Synthesis-1.png}
\caption{反应物1的合成方程式.}%
\label{Fig:DHR_Synthesis-equation-1}
\end{figure}
前驱体化合物1通过改进的文献\cite{JOC55-4943_1990,JCSCC5-381_1987}方法合成。该反应在1000\textrm{mL}三颈圆底烧瓶中进行,烧瓶配备回流冷凝管、滴液漏斗和磁力搅拌器。向烧瓶中加入锌粉~(37.97~\textrm{g},576~\textrm{mmol},3.0当量)和新鲜蒸馏的无水四氢呋喃~(\textrm{THF},400~\textrm{mL}),通氮气鼓泡20分钟。将混合物冷却至$\text{-}78^{\circ}\mathrm{C}$~(丙酮-液氮浴),通过注射器在约15分钟内逐滴注入\ch{TiCl4}~(24~\textrm{mL},211~\textrm{mmol},1.1当量)。逐渐升温至$0^{\circ}\mathrm{C}$,保持$0^{\circ}\mathrm{C}$,反应30分钟,然后升温至室温反应1小时,最后升温至$80^{\circ}\mathrm{C}$回流1小时。随后,在约1小时内逐滴加入溶解于无水\textrm{THF}~(100~\textrm{mL})中的二苯并环庚酮~(40~\textrm{g},192~\textrm{mmol},1.0当量)溶液。

		回流反应1天,通过薄层色谱\textrm{~(TLC)}监测原料消失。冷却至室温后,进一步冷却至$0^{\circ}\mathrm{C}$,加入二氯甲烷~(300~\textrm{mL})稀释反应液,并用1~\textrm{mol/L}盐酸水溶液淬灭。过滤有机相,用无水硫酸钠~(\ch{Na2SO4})干燥。真空除去溶剂,得到粗产物,将其在二氯甲烷-石油醚混合溶剂中多次重结晶纯化,得到前驱体化合物1~(5.8~\textrm{g}),收率15.8\%,为黄色粉末。

\subsection{前驱体化合物1的基本性质}
\begin{figure}[h!]
\centering
\vspace*{-0.05in}
\includegraphics[height=2.8in]{Figures/DHR_experiment-UV_Absorption.png}
\caption{THF溶液中的紫外-可见吸收光谱.前驱体化合物1~(绿),前驱体化合物2~(橙),前驱体化合物3~(红)}%
\label{Fig:DHR_UV}
\end{figure}

根据¹H NMR、¹³C NMR、高分辨质谱~(HRMS)及元素分析,确定前驱体化合物1的化学结构(下同)。

\begin{minipage}{0.40\textwidth}
%\begin{figure}[h!]
%\centering
\vspace*{-0.05in}
\hspace*{-0.4in}
\includegraphics[height=2.0in]{Figures/DHR_synthesis-1-struct.png}
%\caption{反应物1的合成方程式.}%
%\label{Fig:DHR_Compound-1}
%\end{figure}
\end{minipage}
\hspace{1pt}
\begin{minipage}{0.58\textwidth}
		熔点:~$251.3\text{-}253.1^{\circ}\mathrm{C}$

		\textrm{THF}溶液中紫外-可见吸收光谱:\\$\lambda_{\max}/\textrm{nm}~(\epsilon)$:~276~(80300)、389~(5900)、418~(12600)、443~(12600)。如图\ref{Fig:DHR_UV}的绿色谱线所示。

		$^1$H NMR~(400 MHz,\ch{CDCl3})~$\delta$~(ppm):~7.70~(d,J=8.8 Hz,2H,H-8)、7.35-7.34~(m,4H,H-3、2)、7.23-7.19~(m,4H,H-6、1)、7.12-7.06~(m,4H,H-7、9)、3.51~(m,4H,H-5、4)、2.94~(m,2H,H-4)。如图\ref{Fig:DHR_NMR-1_H}所示。
\end{minipage}
\begin{figure}[h!]
\centering
\vspace*{-0.05in}
\includegraphics[height=3.5in]{Figures/DHR_experiment-NMR-1-H.png}
\caption{反应物1的NMR氢谱.}%
\label{Fig:DHR_NMR-1_H}
\end{figure}
\noindent $^{13}$C NMR~(100 MHz,\ch{CDCl3})$~\delta$~(ppm):~143.04~(C-11)、139.59~(C-10)、136.99~(C-12)、135.37~(C-15)、135.06~(C-9)、133.12~(C-14)、129.00~(C-13)、127.80~(C-2)、127.44~(C-6)、126.48~(C-3)、126.24~(C-8)、125.18~(C-1)、123.34~(C-7)、40.71~(C-5)、33.06~(C-4)。如图\ref{Fig:DHR_NMR-1_C}所示。
\begin{figure}[h!]
\centering
\vspace*{-0.05in}
\includegraphics[height=3.5in]{Figures/DHR_experiment-NMR-1-C.png}
\caption{反应物1的NMR碳谱.}%
\label{Fig:DHR_NMR-1_C}
\end{figure}

高分辨质谱~(EI):~分子式\ch{C30H22},理论值382.1722,实测值382.1725。

高分辨质谱~(MALDI-TOF):~$[\mathrm{M}]^+$,分子式\ch{C30H22},理论值382.1716,实测值382.1717。如图\ref{Fig:DHR_MS-1}所示。
\begin{figure}[h!]
\centering
\vspace*{-0.05in}
\includegraphics[height=2.8in]{Figures/DHR_experiment-MS-1.png}
\caption{反应物1的质谱.}%
\label{Fig:DHR_MS-1}
\end{figure}

元素分析:~分子式\ch{C30H22},理论值C 94.20\%、H 5.80\%;实测值C 94.16\%、H 5.79\%。

\section{前驱体化合物2的合成}
化学物2的合成反应式如图\ref{Fig:DHR_Synthesis-equation-2}所示。
\begin{figure}[h!]
\centering
\vspace*{-0.1in}
\includegraphics[height=1.9in]{Figures/DHR_Synthesis-2.png}
\caption{反应物2的合成方程式.}%
\label{Fig:DHR_Synthesis-equation-2}
\end{figure}

在配备回流冷凝管的25 mL双颈圆底烧瓶中,于空气氛围下加入前驱体化合物1~(100 mg,0.26 mmol,1.0当量)。通过抽真空-充氮气循环三次后,用注射器加入无水二噁烷~(4 mL)。在$105^{\circ}\mathrm{C}$搅拌下,通过注射器加入溶解于2 mL二噁烷中的DDQ~(153 mg,0.68 mmol,2.6当量)溶液。15分钟后,通过薄层色谱~(TLC)监测原料消失。将反应混合物冷却至室温,过滤,真空浓缩滤液。通过硅胶柱层析~(洗脱剂:~石油醚/二氯甲烷=4:1)分离产物,得到前驱体化合物2~(79.2 mg),收率80\%,为橙红色粉末。

\subsection{前驱体化合物2的基本性质}
\begin{minipage}{0.40\textwidth}
%\begin{figure}[h!]
%\centering
\vspace*{-0.05in}
\hspace*{-0.4in}
\includegraphics[height=2.0in]{Figures/DHR_synthesis-2-struct.png}
%\caption{反应物2的合成方程式.}%
%\label{Fig:DHR_Compound-1}
%\end{figure}
\end{minipage}
\hspace{1pt}
\begin{minipage}{0.58\textwidth}
	熔点:~$253.1\text{-}253.4^{\circ}\mathrm{C}$。

四氢呋喃溶液中紫外-可见吸收光谱:\\$\lambda_{\max}/\mathrm{nm}$~($\varepsilon$)272~(64300)、299~(28400)、470~(11300)、497~(12800)。如图\ref{Fig:DHR_UV}的橙色谱线所示。

$^1$H NMR~(400 MHz,\ch{CDCl3})δ~(ppm):~7.69~(d,J=8.4 Hz,2H,H-8)、7.28-7.19~(m,6H,H-3、2、1)、7.10~(t,2H,H-7)、7.05~(d,J=6.4 Hz,2H,H-6)、7.00~(d,J=7.6 Hz,2H,H-9)、6.68~(d,J=11.6 Hz,2H,H-4)、6.62~(d,J=12.0 Hz,2H,H-5)。
\end{minipage}
\begin{figure}[h!]
\centering
\vspace*{-0.05in}
\includegraphics[height=3.5in]{Figures/DHR_experiment-NMR-2-H.png}
\caption{反应物1的NMR氢谱.}%
\label{Fig:DHR_NMR-2_H}
\end{figure}

$^{13}$C NMR~(100 MHz,\ch{CDCl3})δ~(ppm):~139.39~(C-10)、138.06~(C-11)、136.52~(C-5)、135.80~(C-13)、135.58~(C-9)、134.67~(C-14、15)、133.01~(C-12)、131.53~(C-4)、129.26~(C-3)、128.28~(C-1)、128.10~(C-6)、127.75~(C-2)、126.55~(C-8)、124.55~(C-7)。
\begin{figure}[h!]
\centering
\vspace*{-0.05in}
\includegraphics[height=3.5in]{Figures/DHR_experiment-NMR-2-C.png}
\caption{反应物2的NMR碳谱.}%
\label{Fig:DHR_NMR-2_C}
\end{figure}

高分辨质谱~(EI):~分子式\ch{C30H18},理论值378.1409,实测值378.1415。

高分辨质谱~(MALDI-TOF):~[M]$^+$,分子式\ch{C30H18},理论值378.1403,实测值378.1404。如图\ref{Fig:DHR_MS-1}所示。
\begin{figure}[h!]
\centering
\vspace*{-0.05in}
\includegraphics[height=2.8in]{Figures/DHR_experiment-MS-2.png}
\caption{反应物2的质谱.}%
\label{Fig:DHR_MS-2}
\end{figure}

元素分析:~分子式\ch{C30H18},理论值C 95.21\%、H 4.79\%;实测值C 94.92\%、H 4.94\%。

\section{前驱体化合物3的合成}
化学物3的合成反应式如图\ref{Fig:DHR_Synthesis-equation-3}所示。
\begin{figure}[h!]
\centering
\vspace*{-0.1in}
\includegraphics[height=1.9in]{Figures/DHR_Synthesis-3.png}
\caption{反应物3的合成方程式.}%
\label{Fig:DHR_Synthesis-equation-3}
\end{figure}
前驱体化合物3的合成路线有两条:
\begin{itemize}
	\item 方法A:~在50 mL双颈圆底烧瓶中,于空气氛围下加入前驱体化合物1~(200 mg,0.52 mmol,1.0当量)。通过抽真空-充氮气循环三次后,用注射器加入无水二氯甲烷~(50 mL)。将混合物冷却至$0^{\circ}\mathrm{C}$~(冰浴),搅拌10分钟。随后,通过注射器加入溶解于10 mL硝基甲烷中的三氯化铁~(1.0 g,6.24 mmol,12.0当量)溶液,反应液颜色由黄绿色变为深绿色。10分钟后,通过薄层色谱~(TLC)监测原料消失。用肼淬灭反应,溶液颜色由绿色变为红色。将反应混合物通过硅藻土短柱过滤,真空浓缩滤液。通过硅胶柱层析~(洗脱剂:~石油醚/二氯甲烷=4:1)纯化残留物,得到前驱体化合物3~(172 mg),收率87.8\%,为红棕色粉末。
	\item 方法B~(放大合成工艺):~在500 mL双颈圆底烧瓶中,于空气氛围下加入前驱体化合物1~(1.0 g,2.62 mmol,1.0当量)。通过抽真空-充氮气循环三次后,用注射器加入无水二氯甲烷~(250 mL)。将混合物冷却至$0^{\circ}\mathrm{C}$~(冰浴),搅拌20分钟。随后,通过注射器加入溶解于50 mL硝基甲烷中的三氯化铁~(5.0 g,31.44 mmol,12.0当量)溶液,反应液颜色由黄绿色变为深绿色。10分钟后,通过薄层色谱~(TLC)监测原料消失。用肼淬灭反应,溶液颜色由绿色变为红色。将反应混合物通过硅藻土短柱过滤,用二氯甲烷洗涤滤饼。滤液用饱和氯化钠溶液萃取三次,无水硫酸钠干燥,真空浓缩。将粗产物溶解于二氯甲烷中,在旋转蒸发仪上加入少量石油醚和甲醇作为不良溶剂进行重结晶,得到纯前驱体化合物3~(971.8 mg),收率98.1\%。
\end{itemize}

\subsection{前驱体化合物3的基本性质}
\begin{minipage}{0.40\textwidth}
%\begin{figure}[h!]
%\centering
\vspace*{-0.05in}
\hspace*{-0.4in}
\includegraphics[height=2.0in]{Figures/DHR_synthesis-3-struct.png}
%\caption{反应物3的合成方程式.}%
%\label{Fig:DHR_Compound-1}
%\end{figure}
\end{minipage}
\hspace{1pt}
\begin{minipage}{0.58\textwidth}
	熔点:~$273.4\text{-}275.2^{\circ}\mathrm{C}$。

四氢呋喃溶液中紫外-可见吸收光谱:~$\lambda_{\max}/\mathrm{nm}$~($\varepsilon$)256~(32800)、305~(20900)、478~(5800)、508~(7900)、541~(6200)。如图\ref{Fig:DHR_UV}的红色谱线所示。

$^1$H NMR~(400 MHz,CDCl₃)δ~(ppm):~7.82~(d,J=6.4 Hz,2H,H-7)、7.71~(d,J=7.2 Hz,2H,H-1)、7.37~(d,J=6.8 Hz,2H,H-6)、7.20~(t,2H,H-2)、7.12~(d,J=7.2 Hz,2H,H-3)、3.51~(t,J=5.2 Hz,4H,H-5)、3.37~(t,J=5.2 Hz,4H,H-4)。
\end{minipage}
\begin{figure}[h!]
\centering
\vspace*{-0.05in}
\includegraphics[height=3.5in]{Figures/DHR_experiment-NMR-3-H.png}
\caption{反应物3的NMR氢谱.}%
\label{Fig:DHR_NMR-3_H}
\end{figure}
$^{13}$C NMR~(100 MHz,CDCl₃)δ~(ppm):~140.11~(C-12)、139.29~(C-10)、138.74~(C-9)、137.81~(C-11)、136.42~(C-8)、133.55~(C-15)、132.47~(C-14)、127.78~(C-3)、127.59~(C-6)、126.97~(C-2)、125.09~(C-13)、121.64~(C-7)、119.83~(C-1)、34.38~(C-5)、33.78~(C-4)。
\begin{figure}[h!]
\centering
\vspace*{-0.05in}
\includegraphics[height=3.5in]{Figures/DHR_experiment-NMR-3-C.png}
\caption{反应物3的NMR碳谱.}%
\label{Fig:DHR_NMR-3_C}
\end{figure}

高分辨质谱~(EI):~分子式\ch{C30H18},理论值378.1409,实测值378.1404。

高分辨质谱~(MALDI-TOF):~[M]$^+$,分子式C₃₀H₁₈,理论值378.1403,实测值378.1404。
\begin{figure}[h!]
\centering
\vspace*{-0.05in}
\includegraphics[height=2.8in]{Figures/DHR_experiment-MS-3.png}
\caption{反应物3的质谱.}%
\label{Fig:DHR_MS-3}
\end{figure}

元素分析:~分子式\ch{C30H18},理论值C 95.21\%、H 4.79\%;实测值C 94.93\%、H 4.72\%。

\section{DHR的合成}
DHR的合成反应式如图\ref{Fig:DHR_Synthesis-equation}所示。
\begin{figure}[h!]
\centering
\vspace*{-0.1in}
\includegraphics[height=1.9in]{Figures/DHR_Synthesis.png}
\caption{\textrm{DRH}的合成方程式.}%
\label{Fig:DHR_Synthesis-equation}
\end{figure}

在配备回流冷凝管的50 mL双颈圆底烧瓶中,加入前驱体化合物3~(100 mg,0.26 mmol,1.0当量)。通过抽真空-充氮气循环三次后,用注射器加入无水二噁烷~(10 mL)。在$105^{\circ}\mathrm{C}$搅拌下,通过注射器加入溶解于2 mL二噁烷中的DDQ~(360 mg,1.56 mmol,6.0当量)溶液。1小时后,通过薄层色谱~(TLC)监测原料消失。将反应混合物冷却至室温,过滤,用甲醇、石油醚、丙酮和乙腈洗涤滤渣以除去杂质。随后,用二氯甲烷和四氢呋喃作为溶剂多次洗涤滤渣以萃取产物,通过薄层色谱~(TLC)监测产物消失。合并滤液,浓缩得到粗产物,将其在旋转蒸发仪上用四氢呋喃-甲醇混合溶剂重结晶,得到纯DHR~(84.5 mg),收率86.9\%,为黑色粉末。

DHR的放大合成工艺:~在配备回流冷凝管的250 mL双颈圆底烧瓶中,加入前驱体化合物3~(1.0 g,2.67 mmol,1.0当量)。通过抽真空-充氮气循环三次后,用注射器加入无水二噁烷~(100 mL)。在$105^{\circ}\mathrm{C}$搅拌下,通过注射器加入溶解于35 mL二噁烷中的DDQ~(3.60 g,15.9 mmol,6.0当量)溶液,反应体系变为蓝紫色。回流反应2小时后,通过薄层色谱~(TLC)监测原料消失。将反应混合物冷却至室温,过滤,用甲醇、石油醚、丙酮和乙腈洗涤滤渣以除去杂质。收集滤渣作为粗产物,在约280℃、20 Pa条件下通过物理气相传输~(PVT)法进一步纯化,得到纯DHR~(599 mg),收率60\%,为黑色粉末。

\subsection{DHR的基本性质}
\begin{minipage}{0.40\textwidth}
%\begin{figure}[h!]
%\centering
\vspace*{-0.05in}
\hspace*{-0.4in}
\includegraphics[height=2.0in]{Figures/DHR_synthesis-struct.png}
%\caption{反应物3的合成方程式.}%
%\label{Fig:DHR_Compound-1}
%\end{figure}
\end{minipage}
\hspace{1pt}
\begin{minipage}{0.58\textwidth}
熔点:~$363.3\text{-}365.1$。

四氢呋喃溶液中紫外-可见吸收光谱:~$\lambda_{\max}/\mathrm{nm}$~(ε)300~(50900)、345~(26800)、565~(7700)、609~(14500)、666~(15800)。如图\ref{Fig:DHR_UV}的红色谱线所示。

$^1$H NMR~(400 MHz,$d8$-THF)$\delta$~(ppm):~8.32~(d,J=6.8 Hz,2H,H-7)、8.24~(d,J=6.4 Hz,2H,H-1)、7.70~(d,J=7.2 Hz,2H,H-6)、7.60-7.58~(m,4H,H-2、3)、7.27~(d,J=12.4 Hz,2H,H-5)、7.05~(d,J=12.4 Hz,2H,H-4)。
\end{minipage}
\begin{figure}[h!]
\centering
\vspace*{-0.05in}
\includegraphics[height=3.5in]{Figures/DHR_experiment-NMR-H_1.png}
\caption{产物DHR在溶剂$d8$-THF中的NMR氢谱.}%
\label{Fig:DHR_NMR_H-1}
\end{figure}

$^{13}$C NMR~(273 MHz,$d8$-THF)$\delta$~(ppm):~138.50~(C-10)、137.36~(C-12)、136.48~(C-9)、134.20~(C-8)、133.75~(C-11)、133.02~(C-5)、131.78~(C-15)、131.55~(C-4)、129.19~(C-14)、128.90~(C-3)、127.82~(C-6)、127.16~(C-2)、126.78~(C-13)、123.03~(C-7)、122.55~(C-1)。
\begin{figure}[h!]
\centering
\vspace*{-0.05in}
\includegraphics[height=3.5in]{Figures/DHR_experiment-NMR-C_1.png}
\caption{产物DHR在溶剂$d8$-THF中的NMR碳谱.}%
\label{Fig:DHR_NMR_C-1}
\end{figure}

由于产物DHR在室温下溶解度较低,在$100^{\circ}\mathrm{C}$下使用500 MHz核磁共振波谱仪,以\ch{CDCl2CDCl2}为溶剂进行高温NMR表征:~

$^1$H NMR~(500 MHz,\ch{CDCl2CDCl2})$\delta$~(ppm):~8.13~(d,J=7.0 Hz,2H)、8.10~(d,J=7.0 Hz,2H)、7.55-7.50~(m,4H)、7.46~(d,J=7.5 Hz,2H)、7.14~(d,J=12.5 Hz,2H)、6.93~(d,J=12.5 Hz,2H)。
\begin{figure}[h!]
\centering
\vspace*{-0.05in}
\includegraphics[height=3.5in]{Figures/DHR_experiment-NMR-H_2.png}
\caption{产物DHR在溶剂\ch{CDCl2CDCl2}中的NMR氢谱.}%
\label{Fig:DHR_NMR_H-2}
\end{figure}

$^{13}$C NMR~(126 MHz,\ch{CDCl2CDCl2})$\delta$~(ppm):~138.70、137.35、136.51、134.21、133.73、133.16、131.95、131.65、129.39、128.95、127.70、127.16、126.92、122.78、122.49。
\begin{figure}[h!]
\centering
\vspace*{-0.05in}
\includegraphics[height=3.5in]{Figures/DHR_experiment-NMR-C_2.png}
\caption{产物DHR在溶剂\ch{CDCl2CDCl2}中的NMR氢谱.}%
\label{Fig:DHR_NMR_C-2}
\end{figure}

高分辨质谱~(EI):~分子式\ch{C30H14},理论值374.1096,实测值374.1099。

高分辨质谱~(MALDI-TOF):~[M]$^+$,分子式\ch{C30H14},理论值374.1090,实测值374.1089。
\begin{figure}[h!]
\centering
\vspace*{-0.05in}
\includegraphics[height=2.8in]{Figures/DHR_experiment-MS.png}
\caption{DHR的质谱.}%
\label{Fig:DHR_MS}
\end{figure}

元素分析:~分子式\ch{C30H14},理论值C 96.23\%、H 3.77\%;实测值C 96.05\%、H 3.73\%。

\chapter{\rm{DHR}的表征}
\section{一般表征技术与试剂}
\subsection{表征技术}
表征DHR及其前驱体的实验设备包括:~
\begin{itemize}
	\item $^1$H NMR和$^{13}$C NMR光谱采用布鲁克~(Bruker)AVANCE III 400 MHz、500 MHz、600 MHz或950 MHz核磁共振波谱仪测定。质谱通过布鲁克Solarix-XR高分辨质谱仪测定。元素分析在Carlo-Erba-1106型元素分析仪上进行。
	\item 快速柱层析采用200-300目硅胶,按标准技术使用指定洗脱剂进行分离。分析型薄层色谱~(TLC)采用预制玻璃背板硅胶板。除非另有说明,展开后的色谱图通过紫外吸收~(254 nm)进行可视化检测。
	\item 吸收光谱使用日立~(HITACHI)UH4150紫外-可见分光光度计记录。循环伏安法测试在三电极体系中进行:~工作电极为玻碳电极,辅助电极为铂丝电极,参比电极为Ag/AgCl~(饱和KCl)电极,测试仪器为计算机控制的CHI660C电化学工作站,测试温度为室温,扫描速率为100 mV·s$^{-1}$,支持电解质为0.1 mol/L四丁基六氟磷酸铵~(n-Bu$_4$NPF$_6$)的干燥1,2-二氯苯/二氯甲烷~(体积比1:1)溶液。校准过程中,二茂铁/二茂铁鎓离子对~(Fc/Fc$^+$)的氧化还原电势在相同条件下测定。
	\item 热重分析~(TGA)在TGA8000型热重分析仪上进行,测试条件为氮气氛围,升温速率$10^{\circ}\mathrm{C}/\mathrm{min}$,温度范围$50^{\circ}\mathrm{C}$至$550^{\circ}\mathrm{C}$。熔点通过布奇~(BUCHI)B540型熔点仪测定。薄膜的X射线衍射~(XRD)图谱在室温下采用2 kW理学~(Rigaku)X射线衍射系统以反射模式测定。单晶衍射数据通过配备电荷耦合器件~(CCD)面探测器的理学Saturn衍射仪收集。%本文报道的晶体结构数据~(不含结构因子)已存入剑桥晶体学数据中心~(CCDC):~前驱体化合物1的CCDC编号为1953759,前驱体化合物2为1953602,前驱体化合物3为1953606,化合物DHR为1971948。
\end{itemize}

\subsection{实验试剂}
二苯并环庚酮购自东京化成工业~(TCI)株式会社。四氯化钛~(TiCl$_4$)和2,3-二氯-5,6-二氰基-1,4-苯醌~(DDQ)购自Alfa Aesar公司,直接使用。三氯化铁~(FeCl$_3$)和硝基甲烷~(CH$_3$NO$_2$)购自国药集团化学试剂有限公司,直接使用。其他试剂均为市售品,未经进一步纯化直接使用。四氢呋喃~(THF)使用前经金属钠新鲜蒸馏提纯。二噁烷和二氯甲烷购自J\&K公司的超干溶剂,直接使用。其他溶剂除%非另有说明,
均直接使用。%本文中,目标化合物双环庚烯[ijkl,uvwx]红荧烯简称为DHR。

\section{DHR的性能表征}
\subsection{单晶结构分析}
\begin{figure}[h!]
\centering
\vspace*{-0.05in}
\includegraphics[height=1.3in]{Figures/DHR_ORTEP-1.png}
\includegraphics[height=1.3in]{Figures/DHR_ORTEP-2.png}
\includegraphics[height=1.3in]{Figures/DHR_ORTEP-3.png}
\caption{ORTEP绘制的DHR在晶体中的分子结构~(概率水平30\%)~a)两个构象体i和ii;b)带原子编号的构象体i;c)构象体i的堆积方式~(显示π-π距离及短C∙∙∙H)}%
\label{Fig:DHR_ORTEP}
\end{figure}
通过物理气相传输法~($280^{\circ}\mathrm{C}$,20 Pa)缓慢升华,将获得适用于单晶结构分析的DHR单晶。DHR分子在晶格中存在无序性:~两个构象体~(i和ii)占据每个晶格位点,占有率分别为46\%和54\%~(见图\ref{Fig:DHR_ORTEP})。构象体i和ii的键长与键角数据见表\ref{Tab:DHR-Bond}与\ref{Tab:DHR-Angle}。在构象体i的C-C键中,C2-C3和C9-C10键长最长,达1.487 Å;而C8-C9、C12-C13和C14-C2键长最短,低至1.33 Å;构象体ii表现出类似的键长趋势。与母体薁的C-C键长~(1.387-1.427 Å[35])相比,DHR中两个薁单元的C-C键长~(构象体i:~1.333-1.487 Å;构象体ii:~1.354-1.458 Å)离散度更大。

如图\ref{Fig:DHR_ORTEP}b和\ref{Fig:DHR_ORTEP}c所示,DHR分子呈近完全平面构型,具有$\mathrm{C}_{2h}$对称性。作为对比,我们还给出了前驱体1、2、3的晶体结构,发现它们的中心苯环B与末端苯环E分别形成-54.97°、-47.23°和9.75°的二面角~(见图\ref{Fig:DHR_dihedral-angle}和表\ref{Tab:DHR_dihedral-angle})。由此推测,“形式薁单元”中五边形与七边形的同时存在,是DHR实现平面结构的关键因素。

如图\ref{Fig:DHR_ORTEP}2c%及图S9~(支持信息)
所示,DHR分子以“人字形”~(herringbone)方式堆积。由于晶格中分子的无序性,分子间π-π堆积距离~(3.438-3.465 Å)和C∙∙∙H原子间接触距离~(2.684-3.194 Å)存在轻微差异。这种人字形分子间排列理论上有利于电荷传输,但如后文所述,分子无序性会导致电荷传输性能下降。

\begin{table}[h!]
    \centering
    \begin{tabular}{|l|c|c|c|c|}
        \hline
        化合物 & DHR~(构象体ii) & 1 & 2~(D环非平面) & 3 \\
        \hline
        二面角~(环B与环E)/° & 11.29 & -54.97 & -47.23 & 9.75 \\
        \hline
    \end{tabular}
    \caption{化合物的二面角数据}
\label{Tab:DHR_dihedral-angle}
\end{table}

\begin{table}[h!]
    \centering
    \begin{tabular}{|l|l|l|l|l|l|}
        \hline
        \multicolumn{3}{|c|}{构象体i的键长} & \multicolumn{3}{c|}{构象体ii的键长} \\
        \hline
        原子 & 原子 & 长度/Å & 原子 & 原子 & 长度/Å \\
        \hline
        C1 & C2 & 1.394~(7) & C1 & C2 & 1.408~(8) \\
        \hline
        C1 & C5 & 1.400~(7) & C1 & C5 & 1.392~(6) \\
        \hline
        C1 & C6’ & 1.400~(7) & C1 & C6’ & 1.395~(5) \\
        \hline
        C2 & C3 & 1.479~(9) & C2 & C3 & 1.457~(8) \\
        \hline
        C2 & C14 & 1.381~(7) & C2 & C14 & 1.362~(7) \\
        \hline
        C3 & C4 & 1.401~(8) & C3 & C4 & 1.395~(10) \\
        \hline
        C3 & C13 & 1.381~(8) & C3 & C13 & 1.379~(7) \\
        \hline
        C4 & C5 & 1.462~(15) & C4 & C5 & 1.449~(13) \\
        \hline
        C4 & C10 & 1.403~(10) & C4 & C10 & 1.409~(9) \\
        \hline
        C5 & C6 & 1.410~(8) & C5 & C6 & 1.412~(6) \\
        \hline
        C6 & C7 & 1.436~(8) & C6 & C7 & 1.444~(7) \\
        \hline
        C7 & C8 & 1.459~(9) & C7 & C8 & 1.443~(7) \\
        \hline
        C7 & C15’ & 1.39~(2) & C7 & C15 & 1.419~(16) \\
        \hline
        C8 & C9 & 1.333~(8) & C8 & C9 & 1.354~(7) \\
        \hline
        C9 & C10 & 1.487~(8) & C9 & C10 & 1.458~(8) \\
        \hline
        C10 & C11 & 1.381~(10) & C10 & C11 & 1.422~(9) \\
        \hline
        C11 & C12 & 1.407~(8) & C11 & C12 & 1.387~(8) \\
        \hline
        C12 & C13 & 1.371~(8) & C12 & C13 & 1.388~(10) \\
        \hline
        C14 & C15 & 1.42~(2) & C14 & C15 & 1.41~(2) \\
        \hline
    \end{tabular}
    \caption{DHR的键长数据}
    \label{Tab:DHR-Bond}
\end{table}

\begin{table}[h!]
    \centering
    \begin{tabular}{|l|l|l|l|l|l|l|l|}
        \hline
        \multicolumn{4}{|c|}{构象体i的键角} & \multicolumn{4}{c|}{构象体ii的键角} \\
        \hline
        原子 & 原子 & 原子 & 角度/° & 原子 & 原子 & 原子 & 角度/° \\
        \hline
        C2 & C1 & C6’ & 123.9~(12) & C5 & C1 & C2 & 111.3~(10) \\
        \hline
        C6’ & C1 & C5 & 123.5~(6) & C6’ & C1 & C2 & 124.0~(10) \\
        \hline
        C1 & C2 & C3 & 105.8~(7) & C1 & C2 & C3 & 106.2~(6) \\
        \hline
        C14 & C2 & C1 & 118.0~(7) & C14 & C2 & C1 & 116.5~(6) \\
        \hline
        C14 & C2 & C3 & 136.2~(4) & C14 & C2 & C3 & 137.3~(4) \\
        \hline
        C4 & C3 & C2 & 106.8~(6) & C4 & C3 & C2 & 106.9~(5) \\
        \hline
        C13 & C3 & C2 & 133.2~(5) & C13 & C3 & C2 & 133.4~(5) \\
        \hline
        C13 & C3 & C4 & 120.0~(7) & C13 & C3 & C4 & 119.6~(7) \\
        \hline
        C3 & C4 & C5 & 109.6~(9) & C3 & C4 & C5 & 109.7~(8) \\
        \hline
        C3 & C4 & C10 & 122.3~(8) & C3 & C4 & C10 & 122.7~(9) \\
        \hline
        C10 & C4 & C5 & 127.1~(9) & C10 & C4 & C5 & 126.9~(9) \\
        \hline
        C1 & C5 & C4 & 104.6~(12) & C1 & C5 & C4 & 105.2~(10) \\
        \hline
        C1 & C5 & C6 & 119.8~(8) & C1 & C5 & C6 & 119.7~(7) \\
        \hline
        C6 & C5 & C4 & 134.1~(13) & C6 & C5 & C4 & 134.3~(11) \\
        \hline
        C1’ & C6 & C5 & 116.5~(7) & C1’ & C6 & C5 & 115.5~(6) \\
        \hline
        C1’ & C6 & C7 & 119.2~(12) & C1’ & C6 & C7 & 119.3~(10) \\
        \hline
        C5 & C6 & C7 & 124.1~(12) & C5 & C6 & C7 & 125.1~(10) \\
        \hline
        C6 & C7 & C8 & 125.8~(9) & C8 & C7 & C6 & 124.8~(7) \\
        \hline
        C15’ & C7 & C6 & 115.1~(10) & C15 & C7 & C6 & 115.6~(10) \\
        \hline
        C15’ & C7 & C8 & 119.0~(9) & C15 & C7 & C8 & 119.4~(9) \\
        \hline
        C9 & C8 & C7 & 132.3~(5) & C9A & C8A & C7A & 132.6~(5) \\
        \hline
        C8 & C9 & C10 & 131.7~(6) & C8 & C9 & C10 & 131.7~(5) \\
        \hline
        C4 & C10 & C9 & 123.2~(7) & C4 & C10 & C9 & 123.8~(7) \\
        \hline
        C11 & C10 & C4 & 116.1~(6) & C4 & C10 & C11 & 116.2~(6) \\
        \hline
        C11 & C10 & C9 & 120.6~(6) & C11 & C10 & C9 & 119.9~(5) \\
        \hline
        C10 & C11 & C12 & 121.5~(8) & C12 & C11 & C10 & 120.1~(7) \\
        \hline
        C13 & C12 & C11 & 121.3~(8) & C11 & C12 & C13 & 122.2~(7) \\
        \hline
        C12 & C13 & C3 & 118.5~(6) & C3 & C13 & C12 & 118.9~(6) \\
        \hline
        C2 & C14 & C15 & 118.4~(8) & C14’ & C15 & C7 & 121.9~(13) \\
        \hline
        C7’ & C15 & C14 & 125.1~(11) & C2 & C1 & C5 & 112.6~(11) \\
        \hline
    \end{tabular}
    \caption{DHR的键角数据}
    \label{Tab:DHR-Angle}
\end{table}

\begin{figure}[h!]
\centering
\vspace*{-0.05in}
\includegraphics[height=2.0in]{Figures/DHR_DFT-di-angle-1.png}
\includegraphics[height=2.0in]{Figures/DHR_DFT-di-angle-2.png}
\includegraphics[height=2.0in]{Figures/DHR_DFT-di-angle-3.png}
\includegraphics[height=2.0in]{Figures/DHR_DFT-di-angle-4.png}
\caption{化合物DHR~(a)、前驱体化合物1~(b)、2~(c)和3~(d)的选定二面角}%
\label{Fig:DHR_dihedral-angle}
\end{figure}

%\begin{figure}[h!]
%\centering
%\vspace*{-0.05in}
%\includegraphics[height=1.3in]{Figures/DHR_ORTEP-1.png}
%\includegraphics[height=1.3in]{Figures/DHR_ORTEP-2.png}
%\includegraphics[height=1.3in]{Figures/DHR_ORTEP-3.png}
%\caption{DHR 不同构象体~(构象体 i 和构象体 ii)之间的选定堆积模式。图中标记①和②的分子分别代表构象体 i 或 ii,展示了单胞中两种分子间的相互作用。关键分子间距离~(单位:~Å)标注为 d:~d=3.002~(a)、3.153~(b)、3.194~(c)、2.705~(d)、2.685~(e)、2.824~(f)、2.684~(g)、2.880~(h)。}
%\label{Fig:DHR_Configure}
%\end{figure}

\subsection{稳定性分析}
\begin{figure}[h!]
\centering
\vspace*{-0.05in}
\includegraphics[height=4.0in]{Figures/DHR_TGA-1.png}
\caption{DHR的热重分析~(TGA),温度从$30^{\circ}\mathrm{C}$起,以$10^{\circ}/\mathrm{min}$的速度升至$550^{\circ}\mathrm{C}$}
\label{Fig:DHR_TGA-1}
\end{figure}
热重分析~(TGA)数据显示,DHR具有优异的热稳定性:~温度升至$400^{\circ}\mathrm{C}$时,重量损失仍低于5\%~(见图\ref{Fig:DHR_TGA-1})。DHR在室温空气中存放超过6个月,结构无变化;即使在空气中$200^{\circ}\mathrm{C}$加热1小时,其¹H NMR和质谱谱图仍保持不变。因此,DHR的稳定性与Mastalerz团队报道的含两个嵌入式薁单元的多环芳烃\upcite{ACIE58-17577_2019,AC131-17741_2019}相近,而不同于Müllen和Feng近期报道的含两个五边形和两个七边形的纳米石墨烯~(空气稳定性差\upcite{JACS141-12011_2019})。

\subsection{DHR的光物理与电化学性质}
\begin{figure}[h!]
\centering
\vspace*{-0.05in}
\includegraphics[height=3.0in]{Figures/DHR_experiment-CVM-1.png}
\includegraphics[height=3.0in]{Figures/DHR_experiment-CVM-2.png}
\caption{DHR在四氢呋喃~(THF)中的紫外-可见吸收光谱。}
\label{Fig:DHR_CVM}
\end{figure}
\subsubsection{紫外-可见吸收与荧光发射}
图\ref{Fig:DHR_CVM}a显示了DHR在四氢呋喃~(THF)中的紫外-可见吸收光谱。DHR在紫外区~(305-345 nm)和可见光区~(610-666 nm)均有吸收。%密度泛函理论~(DFT)计算表明,610-666 nm的吸收源于S₀→S₁跃迁,而紫外区的吸收源于S₀→S₂跃迁~(图4a)。DHR溶液在670 nm和400 nm处有发射峰~(见图S15,支持信息),结合DFT计算可将其分别归为S₁→S₀跃迁和反常反卡莎规则S₂→S₀跃迁。有趣的是,轨道分析显示,DHR中的“形式薁单元”主要对S₂→S₀跃迁有贡献[36]。

\subsubsection{循环伏安法与差分脉冲伏安法}
在二氯甲烷~(DCM)与邻二氯苯~(o-DCB)的1:1混合溶剂中,对DHR进行了循环伏安法~(CV)和差分脉冲伏安法~(DPV)测试。如图\ref{Fig:DHR_CVM}b所示,DHR表现出四个氧化峰,对应电位分别为$E_p^{ox1}=0.22~\mathrm{V}$、$E_p^{ox2}=0.52~\mathrm{V}$、$E_p^{ox3}=0.96~\mathrm{V}$、$E_p^{ox4}=1.06~\mathrm{V}$~(\textrm{vs.}$\mathrm{Fc}/\mathrm{Fc}^+$),以及两个还原峰,对应电位为$E_p^{\mathrm{red1}}=-1.74~\mathrm{V}$、$E_p^{\mathrm{red2}}=-2.19~\mathrm{V}$~(相对于$\mathrm{Fc}/\mathrm{Fc}^+$);其中第三和第四个氧化峰间距较近,推测可能是由于溶液中DHR形成分子聚集体所致。这一假设得到了以下实验的支持:~当DHR浓度降至$8.9\times10^{-4}~\mathrm{mol/L}$,并在$60^{\circ}\mathrm{C}$下测试时,第三和第四个氧化峰合并为一个峰。%~(见图S16,支持信息)。

%基于起始氧化电位~(0.18 V vs Fc/Fc⁺)和起始还原电位~(-1.64 V vs Fc/Fc⁺),估算出DHR的最高占据分子轨道~(HOMO)能量为-4.98 eV,最低未占据分子轨道~(LUMO)能量为-3.16 eV,进而计算出其带隙为1.82 eV。

\subsection{化学氧化行为}
\begin{figure}[h!]
\centering
\vspace*{-0.05in}
\includegraphics[height=4.0in]{Figures/DHR_experiment-Absorption.png}
\caption{TFA作为溶剂,DHR经1当量和6当量的\ch{NOBF_4}吸收光谱。}
\label{Fig:DHR_Absorption}
\end{figure}
DHR经\ch{NOBF4}化学氧化后的吸收光谱变化。氧化产物在THF等常规有机溶剂中的溶解度极低,因此选用三氟乙酸~(TFA)作为溶剂进行测试。如图\ref{Fig:DHR_Absorption}所示,加入1当量\ch{NOBF4}后,体系在900-1400 nm处出现新的吸收峰;电子顺磁共振~(ESR)测试显示,经\ch{NOBF4}处理后的DHR溶液出现强ESR信号~(见图\ref{Fig:DHR_ESR}),表明900-1400 nm的新吸收峰可归为DHR自由基阳离子的形成。当\ch{NOBF4}用量增加至6当量时,溶液的ESR信号消失,同时在950 nm处出现新的吸收峰,这可能是由于形成了DHR二价阳离子。
\begin{figure}[h!]
\centering
\vspace*{-0.05in}
\includegraphics[height=4.0in]{Figures/DHR_experiment-ESR.png}
\caption{DHR经过不同当量的\ch{NOBF_4}的ESR谱信号。}
\label{Fig:DHR_ESR}
\end{figure}

\subsection{DHR的半导体性能}
\begin{figure}[h!]
\centering
\vspace*{-0.05in}
\includegraphics[height=3.0in]{Figures/DHR_experiment-OFET-1.png}
\includegraphics[height=3.0in]{Figures/DHR_experiment-OFET-2.png}
\caption{DHR 晶体 OFET 器件的电学性能:~~(a) 转移曲线~(横坐标:~栅极电压 V$_G$ ~(V),纵坐标:~漏极电流 I$_{DS}$ ~(A));~(b) 输出曲线~(横坐标:~漏源电压 V$_{DS}$ ~(V),纵坐标:~漏极电流 I$_{DS}$ ~(A))。}
\label{Fig:DHR_OFET}
\end{figure}
通过物理气相传输~(PVT)法生长DHR晶体,研究其固态半导体性能。将晶体置于十八烷基三氯硅烷~(OTS)修饰的Si/\ch{SiO2}衬底上,制备底栅顶接触~(BGTC)结构的场效应晶体管~(FET)器件:~以晶体为传输沟道,采用掩模板遮挡,随后沉积15 nm厚的三氧化钼~(\ch{MoO3})作为修饰层,并沉积30 nm厚的金~(Au)作为顶源极和漏极。在空气中测试器件的迁移曲线和输出曲线~(见图\ref{Fig:DHR_OFET}),

饱和区的电流可以体通过迁移率公式
\begin{equation}
I_{DS} = (W/2L)C_i\mu~(V_G - V_T)^2
	\label{eq:Transfer}
\end{equation}
计算,其中 $W$ 为沟道宽度,$L$ 为沟道长度,$C_i$ 为栅介质电容,$\mu$ 为迁移率,$V_T$ 为阈值电压。基于20个器件的转移曲线,提取出DHR的空穴迁移率:~最高空穴迁移率可达$0.082\mathrm{cm}^2\cdot\mathrm{V}^{⁻1}\cdot\mathrm{S}^{-1}$,开关比为$5.45\times10^4$;平均空穴迁移率为$0.049\mathrm{cm}^2\cdot\mathrm{V}^{-1}\cdot\mathrm{s}^{-1}$。结果表明DHR晶体表现出典型的$p$型半导体行为。


%\section{4. 热重分析~(TGA)曲线}
%图S1. DHR的热重分析曲线,升温范围30℃至550℃,升温速率10℃/min。

%图S2. 前驱体化合物1、2和3的热重分析曲线,升温范围30℃至500℃,升温速率10℃/min。

%\section{5. DHR的稳定性}
%图S3. DHR的高分辨基质辅助激光解吸电离飞行时间~(HR-MALDI-TOF)质谱对比:~~(a) 新鲜制备的样品;~(b) 在空气中200℃加热1小时后的样品。

%图S4. DHR在d₈-THF中25℃下的$^1$H NMR光谱对比~(400 MHz)。标注:~溶剂残留峰~(d₈-THF)、水峰~(H₂O)、旋转边带~(*)。

%\section*{7. Absorption spectra}
%\begin{figure}[h]
%    \centering
%    \includegraphics[width=0.9\textwidth]{figure_S11.pdf} % 需替换为实际图片路径
%    \caption{DHR 在 THF 溶液~(1.0×10$^{-5}$ mol/L,蓝色曲线)和薄膜状态~(红色曲线)下的吸收光谱对比。横坐标为波长~(nm),纵坐标为吸光度~(Absorption)。}
%    \label{fig:dhr_absorption_solution_film}
%\end{figure}
%
%\begin{figure}[h]
%    \centering
%    \includegraphics[width=0.9\textwidth]{figure_S12.pdf} % 需替换为实际图片路径
%    \caption{化合物 1~(绿色曲线)、2~(橙色曲线)和 3~(红色曲线)在 THF 溶液~(1.0×10$^{-5}$ mol/L)中的紫外-可见吸收光谱。横坐标为波长~(nm),纵坐标为吸光度~(Absorption)。}
%    \label{fig:compounds_123_absorption}
%\end{figure}
%\section*{9. Cyclic voltammetry and differential pulse voltammetry measurements for DHR}
%\begin{figure}[h]
%    \centering
%    \includegraphics[width=0.8\textwidth]{figure_S16.pdf} % 需替换为实际图片路径
%    \caption{DHR~(8.9×10$^{-4}$ M)在 DCM/o-DCB~(1:1)混合溶剂中的循环伏安图~(CV,黑色曲线)和差分脉冲伏安图~(DPV,红色曲线)。测试条件:~60 °C,支持电解质为 0.1 M Bu$_4$NPF$_6$,工作电极为玻碳电极,对电极为铂丝,参比电极为 Ag/AgCl,扫描速率为 100 mV/s。二茂铁~(Fc)用作外部参比,电位以 E$_{1/2}$~(Fc/Fc$^+$) 为基准表示。}
%    \label{fig:dhr_cv_dpv}
%\end{figure}

%\section*{10. Absorption and ESR spectra of DHR after chemical oxidation}
%\begin{figure}[h]
%    \centering
%    \includegraphics[width=0.8\textwidth]{figure_S17.pdf} % 需替换为实际图片路径
%    \caption{DHR~(1.0×10$^{-3}$ M)在 CF$_3$COOH 中分别加入 1 当量和 6 当量 NOBF$_4$ 氧化后的吸收光谱。横坐标为波长~(nm),纵坐标为吸光度~(Absorption),不同曲线对应不同氧化当量。}
%    \label{fig:dhr_oxidation_absorption}
%\end{figure}
%
%\begin{figure}[h]
%    \centering
%    \includegraphics[width=0.8\textwidth]{figure_S18.pdf} % 需替换为实际图片路径
%    \caption{DHR 加入不同量 NOBF$_4$ 氧化后的 ESR 光谱。横坐标为磁场强度~(G),纵坐标为信号强度~(Intensity),不同曲线对应不同氧化剂用量。}
%    \label{fig:dhr_oxidation_esr}
%\end{figure}

%\section*{11. OFET devices}
%\subsection{11.1 Substrate treatment}
%\begin{verbatim}
%基底依次用纯水、食人鱼溶液~(H₂SO₄:H₂O₂ = 3:7)、纯水、异丙醇清洗,然后进行氧等离子体处理~(5 min,100 W)。采用气相沉积法用十八烷基三氯硅烷~(OTS)对 Si/SiO₂ 晶片进行修饰:~将清洁后的 Si/SiO₂ 晶片在 90 °C 真空下干燥 90 min 以去除水分,冷却至室温后滴加少量 OTS,在真空条件下加热至 120 °C 并保持 2 h。
%\end{verbatim}

%\subsection{11.2 Crystal growth}
%\begin{verbatim}
%采用双区水平炉通过物理气相传输~(PVT)法生长 DHR 微晶:~将装有 DHR 粉末的石英舟置于高温区~(200-210 °C),OTS/Si/SiO₂ 晶片置于结晶区,以高纯度 Ar 为载气~(流速 30-50 sccm),系统压力维持在 21 Pa,生长 5 h 后,微晶条带沉积在 OTS/Si/SiO₂ 基底上。
%\end{verbatim}

%\subsection{11.3 OFETs fabrication and measurement}
%\begin{verbatim}
%基于 OTS 修饰的 Si/SiO₂~(300 nm)基底,采用“条带掩模技术”制备底栅顶接触~(BGTC)结构的 OFET 器件。重掺杂 n 型 Si 晶片用作栅电极,为增强载流子注入,通过热蒸发~(速率 0.1 Å/s)在 DHR 微晶与漏/源电极之间插入 1.5 nm 厚的 MoO₃ 缓冲层,随后以 0.1 Å/s 的低升华速率热蒸发金膜,通过“条带掩模技术”在 DHR 微单晶上沉积 30 nm 厚的漏/源电极。
%\end{verbatim}

%\subsection{11.4 Characterization}
%\begin{figure}[h]
%    \centering
%    \subfigure[DHR 微晶的光学显微镜图]{
%        \includegraphics[width=0.45\textwidth]{figure_S19a.pdf} % 需替换为实际图片路径
%    }
%    \subfigure[FET 器件的光学显微镜图]{
%        \includegraphics[width=0.45\textwidth]{figure_S19b.pdf} % 需替换为实际图片路径
%    }
%    \subfigure[TEM 图像及对应的 SAED 图谱]{
%        \includegraphics[width=0.45\textwidth]{figure_S19c.pdf} % 需替换为实际图片路径
%    }
%    \subfigure[晶体的 XRD 图谱]{
%        \includegraphics[width=0.45\textwidth]{figure_S19d.pdf} % 需替换为实际图片路径
%    }
%    \caption{DHR 微晶及 OFET 器件的表征:~~(a) 微晶光学显微镜图~(标尺:~40 μm);~(b) 器件光学显微镜图~(标尺:~20 μm);~(c) TEM 图像~(标尺:~50 nm)及 SAED 图谱;~(d) XRD 图谱~(横坐标:~2θ ~(°),纵坐标:~强度 ~(a.u.))。}
%    \label{fig:dhr_crystal_device_char}
%\end{figure}
%
%\begin{figure}[h]
%    \centering
%    \subfigure[转移特性曲线]{
%        \includegraphics[width=0.45\textwidth]{figure_S20a.pdf} % 需替换为实际图片路径
%    }
%    \subfigure[输出特性曲线]{
%        \includegraphics[width=0.45\textwidth]{figure_S20b.pdf} % 需替换为实际图片路径
%    }
%    \caption{DHR 晶体 OFET 器件的电学性能:~~(a) 转移曲线~(横坐标:~栅极电压 V$_G$ ~(V),纵坐标:~漏极电流 I$_{DS}$ ~(A));~(b) 输出曲线~(横坐标:~漏源电压 V$_{DS}$ ~(V),纵坐标:~漏极电流 I$_{DS}$ ~(A))。迁移率通过饱和区公式 I$_{DS} = ~(W/~(2L))C_i\mu~(V_G - V_T)^2$ 提取,其中 W 为沟道宽度,L 为沟道长度,C_i 为栅介质电容,μ 为迁移率,V_T 为阈值电压。}
%    \label{fig:dhr_ofet_electrical}
%\end{figure}

