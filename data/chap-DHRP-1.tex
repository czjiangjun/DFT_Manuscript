\chapter{\rm{DHR}的合成与表征}
\section{\rm{DHR}的合成}
根据\textrm{DHR}的合成以化合物1为起始原料,化合物1通过四氯化钛(TiCl₄)促进二苯并环庚酮二聚的改进方法制备(见支持信息中的反应式S1)。化合物1经两步反应转化为DHR(方案1):在0℃下,化合物1与12当量的三氯化铁(FeCl₃)在二氯甲烷(DCM)/硝基甲烷(CH₃NO₂)混合溶剂中发生肖尔反应,同时生成两个五边形,得到化合物3,产率为87\%。通过优化后处理步骤(采用萃取/沉淀法替代柱层析),化合物3的合成可实现克级规模,且产率提升至98\%。
\begin{figure}[h!]
\centering
\vspace*{-0.1in}
\includegraphics[height=1.8in]{Figures/Synthesis-DHR.png}
%\caption{\fontsize{7.2pt}{4.2pt}\selectfont{含有\textrm{5/7/5}环系的共轭分子模型示意图,具有特定的多层堆积特性.}}%
\caption{\textrm{Dicyclohepta[ijkl,uvwx]rubicene~(DHR)}的合成.}%
\label{Fig:Synthesis-DHR}
\end{figure}

化合物3与2,3-二氯-5,6-二氰基-1,4-苯醌(DDQ)在二氧六环中发生脱氢反应,生成DHR中的两个七边形。优化反应条件后,使用6.0当量的DDQ时,DHR的产率为87\%。此外,本研究还开发了该步骤的无柱后处理方法:通过物理气相传输(PVT)纯化法,可获得克级规模的纯DHR,产率为60\%。

![方案1. 二环庚[ijkl,uvwx]红荧烯(DHR)的合成路线](注:原文含合成方案图,此处为方案标题翻译)
作为对比,我们还尝试了“先脱氢形成两个七边形、后肖尔反应形成两个五边形”的DHR合成路线(方案1):在2.6当量DDQ存在下,化合物1脱氢生成化合物2的反应效率极高,15分钟内即可完成,产率为80\%;但在FeCl₃存在下,化合物2经肖尔反应生成DHR的尝试在多种条件下均未成功。


 DHR的结构表征
 单晶结构分析
通过物理气相传输法(280℃,20 Pa)缓慢升华,获得了适用于单晶结构分析的DHR单晶。DHR分子在晶格中存在无序性:两个构象体(i和ii)占据每个晶格位点,占有率分别为46\%和54\%(见图2a)。构象体i和ii的键长与键角数据见表S1-S4(支持信息)。在构象体i的C-C键中,C2-C3和C9-C10键长最长,达1.487 Å;而C8-C9、C12-C13和C14-C2键长最短,低至1.33 Å;构象体ii表现出类似的键长趋势。与母体薁的C-C键长(1.387-1.427 Å[35])相比,DHR中两个薁单元的C-C键长(构象体i:1.333-1.487 Å;构象体ii:1.354-1.458 Å)离散度更大。

图2. DHR在晶体中的分子结构ORTEP图(概率水平30\%)(注:原文含图,此处为图注翻译)
a)两个构象体i和ii;b)带原子编号的构象体i;c)构象体i的堆积方式(显示π-π距离及短C∙∙∙H接触)

如图2b和2c所示,DHR分子呈近完全平面构型,具有C₂h对称性。作为对比,我们还获得了前驱体1、2、3的晶体结构,发现它们的中心苯环B与末端苯环E分别形成-54.97°、-47.23°和9.75°的二面角(见图S5,支持信息)。由此推测,“形式薁单元”中五边形与七边形的同时存在,是DHR实现平面结构的关键因素。

如图2c及图S9(支持信息)所示,DHR分子以“人字形”(herringbone)方式堆积。由于晶格中分子的无序性,分子间π-π堆积距离(3.438-3.465 Å)和C∙∙∙H原子间接触距离(2.684-3.194 Å)存在轻微差异。这种人字形分子间排列理论上有利于电荷传输,但如后文所述,分子无序性会导致电荷传输性能下降。


 稳定性分析
通过¹H NMR、¹³C NMR、高分辨质谱(HRMS)及元素分析,确认了DHR的化学结构。热重分析(TGA)数据显示,DHR具有优异的热稳定性:温度升至400℃时,重量损失仍低于5\%(见图S1,支持信息)。DHR在室温空气中存放超过6个月,结构无变化;即使在空气中200℃加热1小时,其¹H NMR和质谱谱图仍保持不变(见图S3-S4,支持信息)。因此,DHR的稳定性与Mastalerz团队报道的含两个嵌入式薁单元的多环芳烃[33]相近,而不同于Müllen和Feng近期报道的含两个五边形和两个七边形的纳米石墨烯(空气稳定性差[32])。


 芳香性与电子离域性计算
计算了DHR中不同环的Z轴1Å处各向同性化学屏蔽表面(ICSS(1)zz)及核独立化学位移(NICS(1))值。需注意,ICSS(1)zz值越正,表明化学屏蔽作用越强,芳香性越高。如图3a所示,化学屏蔽强度顺序为:环B > 环E/H > 环A/C > 环F/I > 环D/G。此外,NICS(1)计算结果为:环B(-25.7)、环E/H(-22.4)、环A/C(-18.8)、环F/I(-4.3)、环D/G(12.6)(见图S10,支持信息),表明六元环具有芳香性,五元环呈弱芳香性,七元环呈弱反芳香性。这与¹H NMR谱中“六元环上质子的化学位移比七元环上质子更显著低场位移”的现象一致(见图S27,支持信息)。

π电子定域化轨道指示函数(LOL-π)计算显示,E/F/B/I/H环内的电子离域性更强(图3b);相比之下,D环中C8-C9键和A环中C2-C14键的电子主要呈定域状态,这与这些环中键长的交替性特征一致(见表S1-S4,支持信息)。

[图3. DHR的ICSS(1)zz图(a)与LOL-π图(b)](注:原文含图,此处为图注翻译)
a)Z轴1Å处各向同性化学屏蔽表面(ICSS(1)zz):深橙色区域表示芳香性强;b)π电子定域化轨道指示函数(LOL-π):橙色表示键内π电子呈定域状态


 DHR的光物理与电化学性质
 紫外-可见吸收与荧光发射
图4a显示了DHR在四氢呋喃(THF)中的紫外-可见吸收光谱。DHR在紫外区(305-345 nm)和可见光区(610-666 nm)均有吸收。密度泛函理论(DFT)计算表明,610-666 nm的吸收源于S₀→S₁跃迁,而紫外区的吸收源于S₀→S₂跃迁(图4a)。DHR溶液在670 nm和400 nm处有发射峰(见图S15,支持信息),结合DFT计算可将其分别归为S₁→S₀跃迁和反常反卡莎规则S₂→S₀跃迁。有趣的是,轨道分析显示,DHR中的“形式薁单元”主要对S₂→S₀跃迁有贡献[36]。


 循环伏安法与差分脉冲伏安法
 在二氯甲烷(DCM)与邻二氯苯(o-DCB)的1:1混合溶剂中,对DHR进行了循环伏安法(CV)和差分脉冲伏安法(DPV)测试。如图4b所示,DHR表现出四个氧化峰,对应电位分别为$E_p^{ox1}=0.22~\mathrm{V}$、$E_p^{ox2}=0.52~\mathrm{V}$、$E_p^{ox3}=0.96~\mathrm{V}$、$E_p^{ox4}=1.06~\mathrm{V}$(\textrm{vs.}$\mathrm{Fc}/\mathrm{Fc}^+$),以及两个还原峰,对应电位为$E_p^{\mathrm{red1}}=-1.74~\mathrm{V}$、$E_p^{\mathrm{red2}}=-2.19~\mathrm{V}$(相对于$\mathrm{Fc}/\mathrm{Fc}^+$);其中第三和第四个氧化峰间距较近,推测可能是由于溶液中DHR形成分子聚集体所致。这一假设得到了以下实验的支持:当DHR浓度降至$8.9\times10^{-4}~\mathrm{mol/L}$,并在60℃下测试时,第三和第四个氧化峰合并为一个峰(见图S16,支持信息)。

基于起始氧化电位(0.18 V vs Fc/Fc⁺)和起始还原电位(-1.64 V vs Fc/Fc⁺),估算出DHR的最高占据分子轨道(HOMO)能量为-4.98 eV,最低未占据分子轨道(LUMO)能量为-3.16 eV,进而计算出其带隙为1.82 eV。


 化学氧化行为
研究了DHR经四氟硼酸硝鎓(NOBF₄)化学氧化后的吸收光谱变化。氧化产物在THF等常规有机溶剂中的溶解度极低,因此选用三氟乙酸(TFA)作为溶剂进行测试。如图S17(支持信息)所示,加入1当量NOBF₄后,体系在900-1400 nm处出现新的吸收峰;电子顺磁共振(ESR)测试显示,经NOBF₄处理后的DHR溶液出现强ESR信号(见图S18,支持信息),表明900-1400 nm的新吸收峰可归为DHR自由基阳离子的形成。当NOBF₄用量增加至6当量时,溶液的ESR信号消失,同时在950 nm处出现新的吸收峰,这可能是由于形成了DHR二价阳离子。


 DHR的半导体性能
通过物理气相传输(PVT)法生长DHR晶体,研究其固态半导体性能。将晶体置于十八烷基三氯硅烷(OTS)修饰的Si/SiO₂衬底上,制备底栅顶接触(BGTC)结构的场效应晶体管(FET)器件:以晶体为传输沟道,采用掩模板遮挡,随后沉积15 nm厚的三氧化钼(MoO₃)作为修饰层,并沉积30 nm厚的金(Au)作为顶源极和漏极。在空气中测试器件的转移曲线和输出曲线(见图S20,支持信息),结果表明DHR晶体表现出典型的p型半导体行为。

基于20个器件的转移曲线,提取出DHR的空穴迁移率:最高空穴迁移率可达0.082 cm²·V⁻¹·s⁻¹,开关比为5.45×10⁴;平均空穴迁移率为0.049 cm²·V⁻¹·s⁻¹。

通过优化DHR的中性基态和离子态几何结构,计算出其空穴传输的重组能为251 meV(见支持信息)。较小的重组能与DHR的刚性结构(所有共轭环均稠合)相符。进一步计算了电荷转移积分,结果见表S6(支持信息):分子间π-π相互作用和C∙∙∙H相互作用相关的转移积分受分子无序性影响——构象体i或ii的同型分子对之间的π-π相互作用转移积分,明显大于构象体i与ii的异型分子对。这一计算结果表明,分子无序性不利于高效电荷传输,与DHR晶体表现出较低电荷迁移率的实验事实一致。

\documentclass{article}
\usepackage{ctex}
\usepackage{graphicx}
\usepackage{amsmath}
\usepackage{amssymb}
\usepackage{array}
\usepackage{geometry}
\geometry{a4paper, margin=1in}

\title{含两个五元环和两个七元环的双环庚烯[ijkl,uvwx]红荧烯:一种稳定且平面的非苯类纳米石墨烯}
\author{张熙沙+、黄艳英+、张静、孟伟、彭倩、孔蕊蕊、肖振伟、刘杰、黄妙飞、易远平、陈亮亮、范庆瑞、林高波、刘子彤、张冠新、江浪、张德庆*\\
(+共同第一作者;*通讯作者)}
\date{}

\begin{document}
\maketitle

\section*{目录}
\begin{enumerate}
    \item 一般表征技术与试剂 \dotfill S2
    \begin{enumerate}
        \item[1.1] 表征技术 \dotfill S2
        \item[1.2] 试剂 \dotfill S2
    \end{enumerate}
    \item 一般计算方法 \dotfill S3
    \begin{enumerate}
        \item[2.1] 芳香性计算细节 \dotfill S3
        \item[2.2] 吸收光谱与发射光谱模拟计算细节 \dotfill S3
        \item[2.3] 重组能与转移积分计算细节 \dotfill S4
    \end{enumerate}
    \item 合成与表征 \dotfill S6
    \item 热重分析(TGA)曲线 \dotfill S10
    \item 双环庚烯[ijkl,uvwx]红荧烯(DHR)的稳定性 \dotfill S12
    \item DHR及前驱体化合物1、2、3的晶体结构 \dotfill S14
    \begin{enumerate}
        \item[6.1] DHR的X射线晶体学数据 \dotfill S14
        \item[6.2] 化合物1的X射线晶体学数据 \dotfill S17
        \item[6.3] 化合物2的X射线晶体学数据 \dotfill S18
        \item[6.4] 化合物3的X射线晶体学数据 \dotfill S19
        \item[6.5] 5/7元环对分子的平面化作用 \dotfill S20
        \item[6.6] DHR的结构分析 \dotfill S21
    \end{enumerate}
    \item 吸收光谱 \dotfill S24
    \item DHR的模拟吸收与发射光谱 \dotfill S25
    \begin{enumerate}
        \item[8.1] DHR前线轨道的优化几何结构与电子结构 \dotfill S25
        \item[8.2] 模拟发射光谱 \dotfill S27
    \end{enumerate}
    \item DHR的循环伏安法(CV)与差分脉冲伏安法(DPV)测试 \dotfill S28
    \item 化学氧化后DHR的吸收光谱与电子自旋共振(ESR)光谱 \dotfill S29
    \item 有机场效应晶体管(OFET)器件 \dotfill S30
    \item 重组能与转移积分计算 \dotfill S32
    \item 核磁共振(NMR)光谱 \dotfill S33
    \begin{enumerate}
        \item[13.1] DHR的NMR光谱 \dotfill S33
        \item[13.2] 化合物1、2、3的NMR光谱 \dotfill S39
    \end{enumerate}
    \item 质谱(MS) \dotfill S51
\end{enumerate}

\section{1. 一般表征技术与试剂}
\subsection{1.1 表征技术}
除非另有说明,$^1$H NMR和$^{13}$C NMR光谱采用布鲁克(Bruker)AVANCE III 400 MHz、500 MHz、600 MHz或950 MHz核磁共振波谱仪测定。质谱通过布鲁克Solarix-XR高分辨质谱仪测定。元素分析在Carlo-Erba-1106型元素分析仪上进行。

快速柱层析采用200-300目硅胶,按标准技术使用指定洗脱剂进行分离。分析型薄层色谱(TLC)采用预制玻璃背板硅胶板。除非另有说明,展开后的色谱图通过紫外吸收(254 nm)进行可视化检测。

吸收光谱使用日立(HITACHI)UH4150紫外-可见分光光度计记录。循环伏安法测试在三电极体系中进行:工作电极为玻碳电极,辅助电极为铂丝电极,参比电极为Ag/AgCl(饱和KCl)电极,测试仪器为计算机控制的CHI660C电化学工作站,测试温度为室温,扫描速率为100 mV·s$^{-1}$,支持电解质为0.1 mol/L四丁基六氟磷酸铵(n-Bu$_4$NPF$_6$)的干燥1,2-二氯苯/二氯甲烷(体积比1:1)溶液。校准过程中,二茂铁/二茂铁鎓离子对(Fc/Fc$^+$)的氧化还原电势在相同条件下测定。分子的最高占据分子轨道(HOMO)和最低未占据分子轨道(LUMO)能量通过以下公式估算:
\[
\text{HOMO} = -(E^{\text{ox}}_{\text{onset}} + 4.8)\text{eV}
\]
\[
\text{LUMO} = -(E^{\text{red}}_{\text{onset}} + 4.8)\text{eV}
\]

热重分析(TGA)在TGA8000型热重分析仪上进行,测试条件为氮气氛围,升温速率10℃/min,温度范围50℃至550℃。熔点通过布奇(BUCHI)B540型熔点仪测定。薄膜的X射线衍射(XRD)图谱在室温下采用2 kW理学(Rigaku)X射线衍射系统以反射模式测定。单晶衍射数据通过配备电荷耦合器件(CCD)面探测器的理学Saturn衍射仪收集。本文报道的晶体结构数据(不含结构因子)已存入剑桥晶体学数据中心(CCDC):化合物1的CCDC编号为1953759,化合物2为1953602,化合物3为1953606,化合物DHR为1971948。

\subsection{1.2 试剂}
二苯并环庚酮购自东京化成工业(TCI)株式会社。四氯化钛(TiCl$_4$)和2,3-二氯-5,6-二氰基-1,4-苯醌(DDQ)购自Alfa Aesar公司,直接使用。三氯化铁(FeCl$_3$)和硝基甲烷(CH$_3$NO$_2$)购自国药集团化学试剂有限公司,直接使用。其他试剂均为市售品,除非另有说明,未经进一步纯化直接使用。四氢呋喃(THF)使用前经金属钠新鲜蒸馏提纯。二噁烷和二氯甲烷购自J&K公司的超干溶剂,直接使用。其他溶剂除非另有说明,均直接使用。本文中,目标化合物双环庚烯[ijkl,uvwx]红荧烯简称为DHR。

\section{2. 一般计算方法}
\subsection{2.1 芳香性计算细节}
所有计算均使用Gaussian 09程序$^1$进行。所有稳定点的几何结构优化采用M06-2X泛函$^2$,该泛函已被证实适用于描述色散效应。分子中所有元素均采用6-31G(d)$^3$基组。在相同水平下进行频率计算,以确认每个稳定点为极小值点或过渡态结构。各向同性化学屏蔽表面(ICSS)$^4$及相关参数通过Multiwfn 3.7程序$^5$生成。核独立化学位移(NICS)值在B3LYP/def2svp理论水平$^6$下计算。

\paragraph{参考文献:}
(1) Gaussian 09,修订版D.01,Frisch, M. J.;Trucks, G. W.;Schlegel, H. B.;Scuseria, G. E.;Robb, M. A.;Cheeseman, J. R.;Scalmani, G.;Barone, V.;Mennucci, B.;Petersson, G. A.;Nakatsuji, H.;Caricato, M.;Li, X.;Hratchian, H. P.;Izmaylov, A. F.;Bloino, J.;Zheng, G.;Sonnenberg, J. L.;Hada, M.;Ehara, M.;Toyota, K.;Fukuda, R.;Hasegawa, J.;Ishida, M.;Nakajima, T.;Honda, Y.;Kitao, O.;Nakai, H.;Vreven, T.;Montgomery, Jr., J. A.;Peralta, J. E.;Ogliaro, F.;Bearpark, M.;Heyd, J. J.;Brothers, E.;Kudin, K. N.;Staroverov, V. N.;Keith, T.;Kobayashi, R.;Normand, J.;Raghavachari, K.;Rendell, A.;Burant, J. C.;Iyengar, S. S.;Tomasi, J.;Cossi, M.;Rega, N.;Millam, J. M.;Klene, M.;Knox, J. E.;Cross, J. B.;Bakken, V.;Adamo, C.;Jaramillo, J.;Gomperts, R.;Stratmann, R. E.;Yazyev, O.;Austin, A. J.;Cammi, R.;Pomelli, C.;Ochterski, J. W.;Martin, R. L.;Morokuma, K.;Zakrzewski, V. G.;Voth, G. A.;Salvador, P.;Dannenberg, J. J.;Dapprich, S.;Daniels, A. D.;Farkas, O.;Foresman, J. B.;Ortiz, J. V.;Cioslowski, J.;Fox, D. J.;Gaussian, Inc.,美国康涅狄格州沃林福德,2013。

(2) (a) Zhao, Y.;Truhlar, D. G.《化学研究评述》(Acc. Chem. Res.)2008, 41, 157. (b) Zhao, Y.;Truhlar, D. G.《理论化学学报》(Theor. Chem. Acc.)2008, 120, 215.

(3) 关于6-31G(d)基组,参见:(a) Ditchfield, R.;Hehre, W. J.;Pople, J. A.《化学物理学报》(J. Chem. Phys.)1971, 54, 724. (b) Hehre, W. J.;Ditchfield, R.;Pople, J. A.《化学物理学报》(J. Chem. Phys.)1972, 56, 2257. (c) Hariharan, P. C.;Pople, J. A.《理论化学学报》(Theor. Chim. Acta)1973, 28, 213. (d) Dill, J. D.;Pople, J. A.《化学物理学报》(J. Chem. Phys.)1975, 62, 2921. (e) Francl, M. M.;Pietro, W. J.;Hehre, W. J.;Binkley, J. S.;Gordon, M. S.;DeFrees, D. J.;Pople, J. A.《化学物理学报》(J. Chem. Phys.)1982, 77, 3654. (f) Hehre, W. J.;Radom, L.;Schleyer, P. v. R.;Pople, J. A.《从头算分子轨道理论》(Ab Initio Molecular Orbital Theory);Wiley出版社:美国纽约,1986。

(4) Klod, S.;Kleinpeter, E.《英国皇家化学会会刊,珀金会刊2》(J. Chem. Soc. Perkin Trans. 2)2001, 1893.

(5) (a) Lu, T.;Chen, F.《计算化学杂志》(J. Comput. Chem.)2012, 33, 580. (b) Lu, T.;Chen, F.《分子图形与建模杂志》(J. Mol. Graph. Model.)2012, 38, 314.

(6) (a) Lee, C.;Yang, W.;Parr, R. G.《物理评论B:凝聚态与材料物理》(Phys. Rev. B: Condens. Matter Mater. Phys.)1988, 37, 785-789;(b) Miehlich, B.;Savin, A.;Stoll, H.;Preuss, H.《化学物理快报》(Chem. Phys. Lett.)1989, 157, 200-206;(c) Becke, A. D.《化学物理学报》(J. Chem. Phys.)1993, 98, 5648-5652;(d) Schäfer, A.;Horn, H.;Ahlrichs, R.《化学物理学报》(J. Chem. Phys.)1992, 97, 2571-2577。

\subsection{2.2 吸收光谱与发射光谱模拟计算细节}
S₀、S₁和S₂态的所有分子几何结构在CASSCF/6-31G*/(12,12)水平下优化,对称性限制为D₂h点群。基于CASSCF轨道,在CASPT2/6-31G*/(12,12)水平下计算激发能和振子强度。

上述计算使用MOLCAS 7.0程序$^7$进行。由于CASSCF方法计算成本高昂,无法进行频率计算,因此采用密度泛函理论(DFT)和含时密度泛函理论(TDDFT),在ωB97xd/6-311+g*水平下通过Gaussian 16程序$^8$分别计算DHR在S₀、S₁和S₂态的频率并分析简正模式。此外,通过MOMAP程序$^9$中的热振动相关函数法模拟孤立DHR的振动吸收光谱和发射光谱,并详细指认光谱的精细结构。

\paragraph{参考文献:}
(7) Aquilante, F.;De Vico, L.;Ferré, N.;Ghigo, G.;Malmqvist, P.Å.;Neogrády, P.;Pedersen, T.;Pitoňák, M.;Reiher, M.;Roos, B. O.;等《计算化学杂志》(J. Comput. Chem.)2010, 31, 224-247.

(8) Gaussian 16,修订版B.01,Frisch, M. J.;Trucks, G. W.;Schlegel, H. B.;Scuseria, G. E.;Robb, M. A.;Cheeseman, J. R.;Scalmani, G.;Barone, V.;Petersson, G. A.;Nakatsuji, H.;Li, X.;Caricato, M.;Marenich, A. V.;Bloino, J.;Janesko, B. G.;Gomperts, R.;Mennucci, B.;Hratchian, H. P.;Ortiz, J. V.;Izmaylov, A. F.;Sonnenberg, J. L.;Williams-Young, D.;Ding, F.;Lipparini, F.;Egidi, F.;Goings, J.;Peng, B.;Petrone, A.;Henderson, T.;Ranasinghe, D.;Zakrzewski, V. G.;Gao, J.;Rega, N.;Zheng, G.;Liang, W.;Hada, M.;Ehara, M.;Toyota, K.;Fukuda, R.;Hasegawa, J.;Ishida, M.;Nakajima, T.;Honda, Y.;Kitao, O.;Nakai, H.;Vreven, T.;Throssell, K.;Montgomery, J. A., Jr.;Peralta, J. E.;Ogliaro, F.;Bearpark, M. J.;Heyd, J. J.;Brothers, E. N.;Kudin, K. N.;Staroverov, V. N.;Keith, T. A.;Kobayashi, R.;Normand, J.;Raghavachari, K.;Rendell, A. P.;Burant, J. C.;Iyengar, S. S.;Tomasi, J.;Cossi, M.;Millam, J. M.;Klene, M.;Adamo, C.;Cammi, R.;Ochterski, J. W.;Martin, R. L.;Morokuma, K.;Farkas, O.;Foresman, J. B.;Fox, D. J.;Gaussian, Inc.,美国康涅狄格州沃林福德,2016。

(9) Niu, Y.;Li, W.;Peng, Q.;Geng, H.;Yi, Y.;Wang, L.;Nan, G.;Wang, D.;Shuai, Z.《分子材料性质预测软件包(MOMAP)1.0:用于预测有机功能材料发光性质和迁移率的软件包》(MOlecular MAterials Property Prediction Package (MOMAP) 1.0: a software package for predicting the luminescent properties and mobility of organic functional materials)《分子物理学报》(Mol. Phys.)2018, 116, 1078-1090.

\subsection{2.3 重组能与转移积分计算细节}
DHR基态中性态和离子态的几何结构通过密度泛函理论(DFT)在ωB97XD/6-311+G*水平下优化。基于优化后的几何结构,采用四点势能面法计算空穴转移的重组能。结合极化连续介质模型(PCM)进一步计算第一、第二和第三电离势,介电常数ε设为20以考虑溶剂化效应。基于晶体结构中的分子对,通过片段轨道法$^{10}$结合基组正交化程序$^{11}$,在DFT-ωB97XD/6-31G**水平下评估转移积分。所有DFT计算均使用Gaussian 16软件包$^{12}$进行。

\paragraph{参考文献:}
(10) Senthilkumar, K.;Grozema, F. C.;Bickelhaupt, F. M.;Siebbeles, L. D. A.《化学物理学报》(J. Chem. Phys.)2003, 119, 9809.

(11) Valeev, E. F.;Coropceanu, V.;da Silva Filho, D. A.;Salman, S.;Brédas, J.-L.《美国化学会志》(J. Am. Chem. Soc.)2006, 128, 9882.

(12) Frisch, M. J.;Trucks, G. W.;Schlegel, H. B.;Scuseria, G. E.;Robb, M. A.;Cheeseman, J. R.;Scalmani, G.;Barone, V.;Petersson, G. A.;Nakatsuji, H.;Li, X.;Caricato, M.;Marenich, A. V.;Bloino, J.;Janesko, B. G.;Gomperts, R.;Mennucci, B.;Hratchian, H. P.;Ortiz, J. V.;Izmaylov, A. F.;Sonnenberg, J. L.;Williams-Young, D.;Ding, F.;Lipparini, F.;Egidi, F.;Goings, J.;Peng, B.;Petrone, A.;Henderson, T.;Ranasinghe, D.;Zakrzewski, V. G.;Gao, J.;Rega, N.;Zheng, G.;Liang, W.;Hada, M.;Ehara, M.;Toyota, K.;Fukuda, R.;Hasegawa, J.;Ishida, M.;Nakajima, T.;Honda, Y.;Kitao, O.;Nakai, H.;Vreven, T.;Throssell, K.;Montgomery, J. A.;Peralta, J. E.;Ogliaro, F.;Bearpark, M. J.;Heyd, J. J.;Brothers, E. N.;Kudin, K. N.;Staroverov, V. N.;Keith, T. A.;Kobayashi, R.;Normand, J.;Raghavachari, K.;Rendell, A. P.;Burant, J. C.;Iyengar, S. S.;Tomasi, J.;Cossi, M.;Millam, J. M.;Klene, M.;Adamo, C.;Cammi, R.;Ochterski, J. W.;Martin, R. L.;Morokuma, K.;Farkas, O.;Foresman, J. B.;Fox, D. J.;Gaussian 16,修订版A.03,Gaussian, Inc.,美国康涅狄格州沃林福德,2016。

\section{3. 合成与表征}
\subsection{化合物1的合成}
(反应式S1)

化合物1通过改进的文献方法$^{13}$合成。该反应在1000 mL三颈圆底烧瓶中进行,烧瓶配备回流冷凝管、滴液漏斗和磁力搅拌器。向烧瓶中加入锌粉(37.97 g,576 mmol,3.0当量)和新鲜蒸馏的无水四氢呋喃(400 mL),通氮气鼓泡20分钟。将混合物冷却至-78℃(丙酮-液氮浴),通过注射器在约15分钟内逐滴注入四氯化钛(24 mL,211 mmol,1.1当量)。逐渐升温至0℃,保持0℃反应30分钟,然后升温至室温反应1小时,最后升温至80℃回流1小时。随后,在约1小时内逐滴加入溶解于无水四氢呋喃(100 mL)中的二苯并环庚酮(40 g,192 mmol,1.0当量)溶液。

回流反应1天,通过薄层色谱(TLC)监测原料消失。冷却至室温后,进一步冷却至0℃,加入二氯甲烷(300 mL)稀释反应液,并用1 mol/L盐酸水溶液淬灭。过滤有机相,用无水硫酸钠(Na₂SO₄)干燥。真空除去溶剂,得到粗产物,将其在二氯甲烷-石油醚混合溶剂中多次重结晶纯化,得到化合物1(5.8 g),收率15.8%,为黄色粉末。

熔点:251.3-253.1℃。

四氢呋喃溶液中紫外-可见吸收光谱:λₘₐₓ/nm(ε)276(80300)、389(5900)、418(12600)、443(12600)。

$^1$H NMR(400 MHz,CDCl₃)δ(ppm):7.70(d,J=8.8 Hz,2H,H-8)、7.35-7.34(m,4H,H-3、2)、7.23-7.19(m,4H,H-6、1)、7.12-7.06(m,4H,H-7、9)、3.51(m,4H,H-5、4)、2.94(m,2H,H-4)。

$^{13}$C NMR(100 MHz,CDCl₃)δ(ppm):143.04(C-11)、139.59(C-10)、136.99(C-12)、135.37(C-15)、135.06(C-9)、133.12(C-14)、129.00(C-13)、127.80(C-2)、127.44(C-6)、126.48(C-3)、126.24(C-8)、125.18(C-1)、123.34(C-7)、40.71(C-5)、33.06(C-4)。

高分辨质谱(EI):分子式C₃₀H₂₂,理论值382.1722,实测值382.1725。

高分辨质谱(MALDI-TOF):[M]$^+$,分子式C₃₀H₂₂,理论值382.1716,实测值382.1717。

元素分析:分子式C₃₀H₂₂,理论值C 94.20%、H 5.80%;实测值C 94.16%、H 5.79%。

\paragraph{参考文献:}
(13) (a) Agranat, I.;Cohen, S.;Isaksson, R.;Sandstroem, J.;Suissa, M. R.《有机化学杂志》(J. Org. Chem.)1990, 55 (16), 4943-50. (b) Agranat, I.;Suissa, M. R.;Cohen, S.;Isaksson, R.;Sandstroem, J.;Dale, J.;Grace, D.《英国皇家化学会会刊,化学通讯》(J. Chem. Soc., Chem. Commun.)1987, (5), 381-3.

\subsection{化合物2的合成}
(反应式S2)

在配备回流冷凝管的25 mL双颈圆底烧瓶中,于空气氛围下加入化合物1(100 mg,0.26 mmol,1.0当量)。通过抽真空-充氮气循环三次后,用注射器加入无水二噁烷(4 mL)。在105℃搅拌下,通过注射器加入溶解于2 mL二噁烷中的DDQ(153 mg,0.68 mmol,2.6当量)溶液。15分钟后,通过薄层色谱(TLC)监测原料消失。将反应混合物冷却至室温,过滤,真空浓缩滤液。通过硅胶柱层析(洗脱剂:石油醚/二氯甲烷=4:1)分离产物,得到化合物2(79.2 mg),收率80%,为橙红色粉末。

熔点:253.1-253.4℃。

四氢呋喃溶液中紫外-可见吸收光谱:λₘₐₓ/nm(ε)272(64300)、299(28400)、470(11300)、497(12800)。

$^1$H NMR(400 MHz,CDCl₃)δ(ppm):7.69(d,J=8.4 Hz,2H,H-8)、7.28-7.19(m,6H,H-3、2、1)、7.10(t,2H,H-7)、7.05(d,J=6.4 Hz,2H,H-6)、7.00(d,J=7.6 Hz,2H,H-9)、6.68(d,J=11.6 Hz,2H,H-4)、6.62(d,J=12.0 Hz,2H,H-5)。

$^{13}$C NMR(100 MHz,CDCl₃)δ(ppm):139.39(C-10)、138.06(C-11)、136.52(C-5)、135.80(C-13)、135.58(C-9)、134.67(C-14、15)、133.01(C-12)、131.53(C-4)、129.26(C-3)、128.28(C-1)、128.10(C-6)、127.75(C-2)、126.55(C-8)、124.55(C-7)。

高分辨质谱(EI):分子式C₃₀H₁₈,理论值378.1409,实测值378.1415。

高分辨质谱(MALDI-TOF):[M]$^+$,分子式C₃₀H₁₈,理论值378.1403,实测值378.1404。

元素分析:分子式C₃₀H₁₈,理论值C 95.21%、H 4.79%;实测值C 94.92%、H 4.94%。

\subsection{化合物3的合成}
(反应式S3)

方法A:在50 mL双颈圆底烧瓶中,于空气氛围下加入化合物1(200 mg,0.52 mmol,1.0当量)。通过抽真空-充氮气循环三次后,用注射器加入无水二氯甲烷(50 mL)。将混合物冷却至0℃(冰浴),搅拌10分钟。随后,通过注射器加入溶解于10 mL硝基甲烷中的三氯化铁(1.0 g,6.24 mmol,12.0当量)溶液,反应液颜色由黄绿色变为深绿色。10分钟后,通过薄层色谱(TLC)监测原料消失。用肼淬灭反应,溶液颜色由绿色变为红色。将反应混合物通过硅藻土短柱过滤,真空浓缩滤液。通过硅胶柱层析(洗脱剂:石油醚/二氯甲烷=4:1)纯化残留物,得到化合物3(172 mg),收率87.8%,为红棕色粉末。

方法B(放大合成工艺):在500 mL双颈圆底烧瓶中,于空气氛围下加入化合物1(1.0 g,2.62 mmol,1.0当量)。通过抽真空-充氮气循环三次后,用注射器加入无水二氯甲烷(250 mL)。将混合物冷却至0℃(冰浴),搅拌20分钟。随后,通过注射器加入溶解于50 mL硝基甲烷中的三氯化铁(5.0 g,31.44 mmol,12.0当量)溶液,反应液颜色由黄绿色变为深绿色。10分钟后,通过薄层色谱(TLC)监测原料消失。用肼淬灭反应,溶液颜色由绿色变为红色。将反应混合物通过硅藻土短柱过滤,用二氯甲烷洗涤滤饼。滤液用饱和氯化钠溶液萃取三次,无水硫酸钠干燥,真空浓缩。将粗产物溶解于二氯甲烷中,在旋转蒸发仪上加入少量石油醚和甲醇作为不良溶剂进行重结晶,得到纯化合物3(971.8 mg),收率98.1%。

熔点:273.4-275.2℃。

四氢呋喃溶液中紫外-可见吸收光谱:λₘₐₓ/nm(ε)256(32800)、305(20900)、478(5800)、508(7900)、541(6200)。

$^1$H NMR(400 MHz,CDCl₃)δ(ppm):7.82(d,J=6.4 Hz,2H,H-7)、7.71(d,J=7.2 Hz,2H,H-1)、7.37(d,J=6.8 Hz,2H,H-6)、7.20(t,2H,H-2)、7.12(d,J=7.2 Hz,2H,H-3)、3.51(t,J=5.2 Hz,4H,H-5)、3.37(t,J=5.2 Hz,4H,H-4)。

$^{13}$C NMR(100 MHz,CDCl₃)δ(ppm):140.11(C-12)、139.29(C-10)、138.74(C-9)、137.81(C-11)、136.42(C-8)、133.55(C-15)、132.47(C-14)、127.78(C-3)、127.59(C-6)、126.97(C-2)、125.09(C-13)、121.64(C-7)、119.83(C-1)、34.38(C-5)、33.78(C-4)。

高分辨质谱(EI):分子式C₃₀H₁₈,理论值378.1409,实测值378.1404。

高分辨质谱(MALDI-TOF):[M]$^+$,分子式C₃₀H₁₈,理论值378.1403,实测值378.1404。

元素分析:分子式C₃₀H₁₈,理论值C 95.21%、H 4.79%;实测值C 94.93%、H 4.72%。

\subsection{DHR的合成}
(反应式S4)

在配备回流冷凝管的50 mL双颈圆底烧瓶中,加入化合物3(100 mg,0.26 mmol,1.0当量)。通过抽真空-充氮气循环三次后,用注射器加入无水二噁烷(10 mL)。在105℃搅拌下,通过注射器加入溶解于2 mL二噁烷中的DDQ(360 mg,1.56 mmol,6.0当量)溶液。1小时后,通过薄层色谱(TLC)监测原料消失。将反应混合物冷却至室温,过滤,用甲醇、石油醚、丙酮和乙腈洗涤滤渣以除去杂质。随后,用二氯甲烷和四氢呋喃作为溶剂多次洗涤滤渣以萃取产物,通过薄层色谱(TLC)监测产物消失。合并滤液,浓缩得到粗产物,将其在旋转蒸发仪上用四氢呋喃-甲醇混合溶剂重结晶,得到纯DHR(84.5 mg),收率86.9%,为黑色粉末。

DHR的放大合成工艺:在配备回流冷凝管的250 mL双颈圆底烧瓶中,加入化合物3(1.0 g,2.67 mmol,1.0当量)。通过抽真空-充氮气循环三次后,用注射器加入无水二噁烷(100 mL)。在105℃搅拌下,通过注射器加入溶解于35 mL二噁烷中的DDQ(3.60 g,15.9 mmol,6.0当量)溶液,反应体系变为蓝紫色。回流反应2小时后,通过薄层色谱(TLC)监测原料消失。将反应混合物冷却至室温,过滤,用甲醇、石油醚、丙酮和乙腈洗涤滤渣以除去杂质。收集滤渣作为粗产物,在约280℃、20 Pa条件下通过物理气相传输(PVT)法进一步纯化,得到纯DHR(599 mg),收率60%,为黑色粉末。

熔点:363.3-365.1℃。

四氢呋喃溶液中紫外-可见吸收光谱:λₘₐₓ/nm(ε)300(50900)、345(26800)、565(7700)、609(14500)、666(15800)。

$^1$H NMR(400 MHz,d₈-THF)δ(ppm):8.32(d,J=6.8 Hz,2H,H-7)、8.24(d,J=6.4 Hz,2H,H-1)、7.70(d,J=7.2 Hz,2H,H-6)、7.60-7.58(m,4H,H-2、3)、7.27(d,J=12.4 Hz,2H,H-5)、7.05(d,J=12.4 Hz,2H,H-4)。

$^{13}$C NMR(273 MHz,d₈-THF)δ(ppm):138.50(C-10)、137.36(C-12)、136.48(C-9)、134.20(C-8)、133.75(C-11)、133.02(C-5)、131.78(C-15)、131.55(C-4)、129.19(C-14)、128.90(C-3)、127.82(C-6)、127.16(C-2)、126.78(C-13)、123.03(C-7)、122.55(C-1)。

由于产物DHR在室温下溶解度较低,在100℃下使用500 MHz核磁共振波谱仪,以CDCl₂CDCl₂为溶剂进行高温NMR表征:

$^1$H NMR(500 MHz,CDCl₂CDCl₂)δ(ppm):8.13(d,J=7.0 Hz,2H)、8.10(d,J=7.0 Hz,2H)、7.55-7.50(m,4H)、7.46(d,J=7.5 Hz,2H)、7.14(d,J=12.5 Hz,2H)、6.93(d,J=12.5 Hz,2H)。

$^{13}$C NMR(126 MHz,CDCl₂CDCl₂)δ(ppm):138.70、137.35、136.51、134.21、133.73、133.16、131.95、131.65、129.39、128.95、127.70、127.16、126.92、122.78、122.49。

高分辨质谱(EI):分子式C₃₀H₁₄,理论值374.1096,实测值374.1099。

高分辨质谱(MALDI-TOF):[M]$^+$,分子式C₃₀H₁₄,理论值374.1090,实测值374.1089。

元素分析:分子式C₃₀H₁₄,理论值C 96.23%、H 3.77%;实测值C 96.05%、H 3.73%。

\section{4. 热重分析(TGA)曲线}
图S1. DHR的热重分析曲线,升温范围30℃至550℃,升温速率10℃/min。

图S2. 化合物1、2和3的热重分析曲线,升温范围30℃至500℃,升温速率10℃/min。

\section{5. DHR的稳定性}
图S3. DHR的高分辨基质辅助激光解吸电离飞行时间(HR-MALDI-TOF)质谱对比:(a) 新鲜制备的样品;(b) 在空气中200℃加热1小时后的样品。

图S4. DHR在d₈-THF中25℃下的$^1$H NMR光谱对比(400 MHz)。标注:溶剂残留峰(d₈-THF)、水峰(H₂O)、旋转边带(*)。

\section{6. DHR及前驱体化合物1、2、3的晶体结构}
\subsection{6.1 DHR的X射线晶体学数据}
DHR的CCDC编号:1971948

\begin{table}[h!]
    \centering
    \begin{tabular}{|l|l|}
        \hline
        经验式 & C₃₀H₁₄ \\
        \hline
        分子量 & 374.41 \\
        \hline
        温度/K & 169.99(11) \\
        \hline
        晶系 & 单斜晶系 \\
        \hline
        空间群 & P2₁/c \\
        \hline
        a/Å & 10.5545(5) \\
        \hline
        b/Å & 3.9469(2) \\
        \hline
        c/Å & 20.5542(10) \\
        \hline
        α/° & 90 \\
        \hline
        β/° & 97.509(5) \\
        \hline
        γ/° & 90 \\
        \hline
        体积/ų & 848.89(7) \\
        \hline
        Z值 & 2 \\
        \hline
        计算密度/g·cm⁻³ & 1.465 \\
        \hline
        吸收系数/mm⁻¹ & 0.638 \\
        \hline
        F(000) & 388.0 \\
        \hline
        晶体尺寸/mm³ & 0.1×0.05×0.01 \\
        \hline
        辐射源 & CuKα(λ=1.54184) \\
        \hline
        数据收集的2θ范围/° & 8.678至131.95 \\
        \hline
        指标范围 & -12≤h≤12,-4≤k≤4,-24≤l≤16 \\
        \hline
        收集的反射点数 & 7988 \\
        \hline
        独立反射点数 & 1486 [R_int=0.0485,R_sigma=0.0429] \\
        \hline
        数据/限制条件/参数 & 1486/67/272 \\
        \hline
        拟合优度(F²) & 1.043 \\
        \hline
        最终R指数[I≥2σ(I)] & R₁=0.0405,wR₂=0.0959 \\
        \hline
        最终R指数(所有数据) & R₁=0.0680,wR₂=0.1123 \\
        \hline
        最大差值峰/孔/e·Å⁻³ & 0.10/-0.14 \\
        \hline
    \end{tabular}
    \caption{DHR的X射线晶体学数据}
\end{table}

表S1. DHR的键长数据

\begin{table}[h!]
    \centering
    \begin{tabular}{|l|l|l|l|l|l|}
        \hline
        \multicolumn{3}{|c|}{构象体i的键长} & \multicolumn{3}{c|}{构象体ii的键长} \\
        \hline
        原子 & 原子 & 长度/Å & 原子 & 原子 & 长度/Å \\
        \hline
        C1 & C2 & 1.394(7) & C1 & C2 & 1.408(8) \\
        \hline
        C1 & C5 & 1.400(7) & C1 & C5 & 1.392(6) \\
        \hline
        C1 & C6’ & 1.400(7) & C1 & C6’ & 1.395(5) \\
        \hline
        C2 & C3 & 1.479(9) & C2 & C3 & 1.457(8) \\
        \hline
        C2 & C14 & 1.381(7) & C2 & C14 & 1.362(7) \\
        \hline
        C3 & C4 & 1.401(8) & C3 & C4 & 1.395(10) \\
        \hline
        C3 & C13 & 1.381(8) & C3 & C13 & 1.379(7) \\
        \hline
        C4 & C5 & 1.462(15) & C4 & C5 & 1.449(13) \\
        \hline
        C4 & C10 & 1.403(10) & C4 & C10 & 1.409(9) \\
        \hline
        C5 & C6 & 1.410(8) & C5 & C6 & 1.412(6) \\
        \hline
        C6 & C7 & 1.436(8) & C6 & C7 & 1.444(7) \\
        \hline
        C7 & C8 & 1.459(9) & C7 & C8 & 1.443(7) \\
        \hline
        C7 & C15’ & 1.39(2) & C7 & C15 & 1.419(16) \\
        \hline
        C8 & C9 & 1.333(8) & C8 & C9 & 1.354(7) \\
        \hline
        C9 & C10 & 1.487(8) & C9 & C10 & 1.458(8) \\
        \hline
        C10 & C11 & 1.381(10) & C10 & C11 & 1.422(9) \\
        \hline
        C11 & C12 & 1.407(8) & C11 & C12 & 1.387(8) \\
        \hline
        C12 & C13 & 1.371(8) & C12 & C13 & 1.388(10) \\
        \hline
        C14 & C15 & 1.42(2) & C14 & C15 & 1.41(2) \\
        \hline
    \end{tabular}
    \caption{DHR的键长数据}
\end{table}

表S2. DHR的键角数据

\begin{table}[h!]
    \centering
    \begin{tabular}{|l|l|l|l|l|l|l|l|}
        \hline
        \multicolumn{4}{|c|}{构象体i的键角} & \multicolumn{4}{c|}{构象体ii的键角} \\
        \hline
        原子 & 原子 & 原子 & 角度/° & 原子 & 原子 & 原子 & 角度/° \\
        \hline
        C2 & C1 & C6’ & 123.9(12) & C5 & C1 & C2 & 111.3(10) \\
        \hline
        C6’ & C1 & C5 & 123.5(6) & C6’ & C1 & C2 & 124.0(10) \\
        \hline
        C1 & C2 & C3 & 105.8(7) & C1 & C2 & C3 & 106.2(6) \\
        \hline
        C14 & C2 & C1 & 118.0(7) & C14 & C2 & C1 & 116.5(6) \\
        \hline
        C14 & C2 & C3 & 136.2(4) & C14 & C2 & C3 & 137.3(4) \\
        \hline
        C4 & C3 & C2 & 106.8(6) & C4 & C3 & C2 & 106.9(5) \\
        \hline
        C13 & C3 & C2 & 133.2(5) & C13 & C3 & C2 & 133.4(5) \\
        \hline
        C13 & C3 & C4 & 120.0(7) & C13 & C3 & C4 & 119.6(7) \\
        \hline
        C3 & C4 & C5 & 109.6(9) & C3 & C4 & C5 & 109.7(8) \\
        \hline
        C3 & C4 & C10 & 122.3(8) & C3 & C4 & C10 & 122.7(9) \\
        \hline
        C10 & C4 & C5 & 127.1(9) & C10 & C4 & C5 & 126.9(9) \\
        \hline
        C1 & C5 & C4 & 104.6(12) & C1 & C5 & C4 & 105.2(10) \\
        \hline
        C1 & C5 & C6 & 119.8(8) & C1 & C5 & C6 & 119.7(7) \\
        \hline
        C6 & C5 & C4 & 134.1(13) & C6 & C5 & C4 & 134.3(11) \\
        \hline
        C1’ & C6 & C5 & 116.5(7) & C1’ & C6 & C5 & 115.5(6) \\
        \hline
        C1’ & C6 & C7 & 119.2(12) & C1’ & C6 & C7 & 119.3(10) \\
        \hline
        C5 & C6 & C7 & 124.1(12) & C5 & C6 & C7 & 125.1(10) \\
        \hline
        C6 & C7 & C8 & 125.8(9) & C8 & C7 & C6 & 124.8(7) \\
        \hline
        C15’ & C7 & C6 & 115.1(10) & C15 & C7 & C6 & 115.6(10) \\
        \hline
        C15’ & C7 & C8 & 119.0(9) & C15 & C7 & C8 & 119.4(9) \\
        \hline
        C9 & C8 & C7 & 132.3(5) & C9A & C8A & C7A & 132.6(5) \\
        \hline
        C8 & C9 & C10 & 131.7(6) & C8 & C9 & C10 & 131.7(5) \\
        \hline
        C4 & C10 & C9 & 123.2(7) & C4 & C10 & C9 & 123.8(7) \\
        \hline
        C11 & C10 & C4 & 116.1(6) & C4 & C10 & C11 & 116.2(6) \\
        \hline
        C11 & C10 & C9 & 120.6(6) & C11 & C10 & C9 & 119.9(5) \\
        \hline
        C10 & C11 & C12 & 121.5(8) & C12 & C11 & C10 & 120.1(7) \\
        \hline
        C13 & C12 & C11 & 121.3(8) & C11 & C12 & C13 & 122.2(7) \\
        \hline
        C12 & C13 & C3 & 118.5(6) & C3 & C13 & C12 & 118.9(6) \\
        \hline
        C2 & C14 & C15 & 118.4(8) & C14’ & C15 & C7 & 121.9(13) \\
        \hline
        C7’ & C15 & C14 & 125.1(11) & C2 & C1 & C5 & 112.6(11) \\
        \hline
    \end{tabular}
    \caption{DHR的键角数据}
\end{table}

\subsection{6.2 化合物1的X射线晶体学数据}
化合物1的CCDC编号:1953759

\begin{table}[h!]
    \centering
    \begin{tabular}{|l|l|}
        \hline
        经验式 & C₃₀H₂₂ \\
        \hline
        分子量 & 382.47 \\
        \hline
        温度/K & 173.15 \\
        \hline
        波长/Å & 0.71073 \\
        \hline
        晶系 & 单斜晶系 \\
        \hline
        空间群 & P2₁/c \\
        \hline
        晶胞参数 & a=7.6867(15) Å,α=90° \\
        \hline
        & b=10.954(2) Å,β=94.39(3)° \\
        \hline
        体积/ų & 1937.5(7) \\
        \hline
        Z值 & 4 \\
        \hline
        计算密度/Mg·m⁻³ & 1.311 \\
        \hline
        吸收系数/mm⁻¹ & 0.074 \\
        \hline
        F(000) & 808 \\
        \hline
        晶体尺寸/mm³ & 0.337×0.291×0.035 \\
        \hline
        数据收集的θ范围/° & 2.059至27.489 \\
        \hline
        指标范围 & -9≤h≤9,-14≤k≤14,-26≤l≤29 \\
        \hline
        收集的反射点数 & 22413 \\
        \hline
        独立反射点数 & 4406 [R(int)=0.0514] \\
        \hline
        θ=25.242°时的完整性 & 99.5% \\
        \hline
        吸收校正 & 基于等效反射的半经验校正 \\
        \hline
        最大/最小透射率 & 1.00000/0.86939 \\
        \hline
        精修方法 & 基于F²的全矩阵最小二乘法 \\
        \hline
        数据/限制条件/参数 & 4406/0/271 \\
        \hline
        拟合优度(F²) & 1.288 \\
        \hline
        最终R指数[I≥2σ(I)] & R₁=0.0663,wR₂=0.1281 \\
        \hline
        最终R指数(所有数据) & R₁=0.0695,wR₂=0.1296 \\
        \hline
        消光系数 & 无 \\
        \hline
        最大差值峰/孔/e·Å⁻³ & 0.235/-0.158 \\
        \hline
    \end{tabular}
    \caption{化合物1的X射线晶体学数据}
\end{table}

\subsection{6.3 化合物2的X射线晶体学数据数据}
化合物2的CCDC编号:1953602

\begin{table}[h!]
    \centering
    \begin{tabular}{|l|l|}
        \hline
        经验式 & C₃₀H₁₈ \\
        \hline
        分子量 & 378.44 \\
        \hline
        温度/K & 173.15 \\
        \hline
        波长/Å & 0.71073 \\
        \hline
        晶系 & 单斜晶系 \\
        \hline
        空间群 & P2₁/c \\
        \hline
        晶胞参数 & a=10.141(3) Å,α=90° \\
        \hline
        & b=13.470(4) Å,β=96.330(3)° \\
        \hline
        体积/ų & 1874.6(9) \\
        \hline
        Z值 & 4 \\
        \hline
        计算密度/Mg·m⁻³ & 1.341 \\
        \hline
        吸收系数/mm⁻¹ & 0.076 \\
        \hline
        F(000) & 792 \\
        \hline
        晶体尺寸/mm³ & 0.352×0.261×0.109 \\
        \hline
        数据收集的θ范围/° & 3.039至27.479 \\
        \hline
        指标范围 & -13≤h≤12,-17≤k≤16,-17≤l≤17 \\
        \hline
        收集的反射点数 & 15054 \\
        \hline
        独立反射点数 & 4273 [R(int)=0.0527] \\
        \hline
        θ=25.242°时的完整性 & 99.5% \\
        \hline
        吸收校正 & 基于等效反射的半经验校正 \\
        \hline
        最大/最小透射率 & 1.00000/0.75431 \\
        \hline
        精修方法 & 基于F²的全矩阵最小二乘法 \\
        \hline
        数据/限制条件/参数 & 4273/0/271 \\
        \hline
        拟合优度(F²) & 1.164 \\
        \hline
        最终R指数[I≥2σ(I)] & R₁=0.0581,wR₂=0.1197 \\
        \hline
        最终R指数(所有数据) & R₁=0.0662,wR₂=0.1238 \\
        \hline
        消光系数 & 无 \\
        \hline
        最大差值峰/孔/e·Å⁻³ & 0.271/-0.180 \\
        \hline
    \end{tabular}
    \caption{化合物2的X射线晶体学数据}
\end{table}

\subsection{6.4 化合物3的X射线晶体学数据}
化合物3的CCDC编号:1953606

\begin{table}[h!]
    \centering
    \begin{tabular}{|l|l|}
        \hline
        经验式 & C₁₅H₉ \\
        \hline
        分子量 & 189.22 \\
        \hline
        温度/K & 173.15 \\
        \hline
        波长/Å & 0.71073 \\
        \hline
        晶系 & 单斜晶系 \\
        \hline
        空间群 & P2₁/n \\
        \hline
        晶胞参数 & a=10.012(3) Å,α=90° \\
        \hline
        & b=8.313(3) Å,β=110.939(6)° \\
        \hline
        体积/ų & 902.6(5) \\
        \hline
        Z值 & 4 \\
        \hline
        计算密度/Mg·m⁻³ & 1.393 \\
        \hline
        吸收系数/mm⁻¹ & 0.079 \\
        \hline
        F(000) & 396 \\
        \hline
        晶体尺寸/mm³ & 0.522×0.315×0.159 \\
        \hline
        数据收集的θ范围/° & 3.347至27.495 \\
        \hline
        指标范围 & -12≤h≤12,-10≤k≤9,-15≤l≤15 \\
        \hline
        收集的反射点数 & 9824 \\
        \hline
        独立反射点数 & 2055 [R(int)=0.0409] \\
        \hline
        θ=25.242°时的完整性 & 99.3% \\
        \hline
        吸收校正 & 基于等效反射的半经验校正 \\
        \hline
        最大/最小透射率 & 1.00000/0.79415 \\
        \hline
        精修方法 & 基于F²的全矩阵最小二乘法 \\
        \hline
        数据/限制条件/参数 & 2055/0/136 \\
        \hline
        拟合优度(F²) & 1.142 \\
        \hline
        最终R指数[I≥2σ(I)] & R₁=0.0495,wR₂=0.1197 \\
        \hline
        最终R指数(所有数据) & R₁=0.0514,wR₂=0.1223 \\
        \hline
        消光系数 & 无 \\
        \hline
        最大差值峰/孔/e·Å⁻³ & 0.232/-0.160 \\
        \hline
    \end{tabular}
    \caption{化合物3的X射线晶体学数据}
\end{table}

\subsection{6.5 5/7元环对分子的平面化作用}
图S5. 化合物DHR(a)、1(b)、2(c)和3(d)的选定二面角。

\begin{table}[h!]
    \centering
    \begin{tabular}{|l|l|l|l|l|}
        \hline
        化合物 & DHR(构象体ii) & 1 & 2(D环非平面) & 3 \\
        \hline
        二面角(环B与环E)/° & 11.29 & -54.97 & -47.23 & 9.75 \\
        \hline
    \end{tabular}
    \caption{化合物的二面角数据}
\end{table}

\subsection{6.6 DHR的结构分析}
表S3. DHR构象体i的选定键长和键角汇总与分析

\begin{table}[h!]
    \centering
    \begin{tabular}{|l|l|l|l|l|l|}
        \hline
        环 & 平均键长/Å & 最长键长/Å & 最短键长/Å & 最大键角/° & 最小键角/° \\
        \hline
        环A(C)(六元环) & 1.403 & 1.436(C7’-C6’) & 1.381(C14-C2) & 125.02(∠C14C15C7’) & 115.15(∠C15C7‘C6’) \\
        \hline
        环B(六元环) & 1.403 & 1.410(C6’-C5’) & 1.399(C1-C6’) & 123.45(∠C5C1C6’) & 116.53(∠C5C6C1’) \\
        \hline
        环E(H)(六元环) & 1.391 & 1.407(C11-C12) & 1.371(C12-C13) & 122.33(∠C3C4C10) & 116.12(∠C4C10C11) \\
        \hline
        环D(G)(七元环) & 1.427 & 1.487(C9-C10) & 1.333(C8-C9) & 134.11(∠C4C5C6) & 123.29(∠C4C10C9) \\
        \hline
        环F(I)(五元环) & 1.427 & 1.479(C2-C3) & 1.394(C2-C1) & 112.55(∠C5C1C2) & 104.65(∠C1C5C4) \\
        \hline
    \end{tabular}
    \caption{DHR构象体i的键长和键角数据}
\end{table}

图S6. DHR构象体i的选定键角(°)。显示线框顶视图,氢原子已省略以清晰展示。

表S4. DHR构象体ii的选定键长和键角汇总与分析

\begin{table}[h!]
    \centering
    \begin{tabular}{|l|l|l|l|l|l|}
        \hline
        环 & 平均键长/Å & 最长键长/Å & 最短键长/Å & 最大键角/° & 最小键角/° \\
        \hline
        环A(C)(六元环) & 1.407 & 1.444(C7’-C6’) & 1.362(C14-C2) & 123.98(∠C2C1C
