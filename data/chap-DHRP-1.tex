\chapter{\rm{DHR}的合成}
%\section{\rm{DHR}的合成}
\textrm{DHR}的合成路线间图\ref{Fig:Synthesis-DHR},根据\textrm{DHR}的合成条件,以化合物1为起始原料,化合物1通过四氯化钛~(\ch{TiCl4})促进二苯并环庚酮二聚的改进方法制备。化合物1经两步反应转化为DHR:~在$0^{\circ}\mathrm{C}$下,化合物1与12当量的三氯化铁~(\ch{FeCl3})在\textrm{~(DCM)}/\ch{CH3NO2}混合溶剂中发生\textrm{Scholl}反应,同时生成两个五元环,得到化合物3,产率为87\%。通过优化后处理步骤~(采用萃取/沉淀法替代柱层析),化合物3的合成可到到克级的规模,且产率提升至98\%。化合物3与2,3-二氯-5,6-二氰基-1,4-苯醌\textrm{~(DDQ)}在二氧六环中发生脱氢反应,生成\textrm{DHR}中的两个七元环。优化反应条件后,使用6.0当量的\textrm{DDQ}时,\textrm{DHR}的产率为87\%。这里顺便提一下该步骤的无柱后处理方法:~通过物理气相传输\textrm{~(PVT)}纯化,可获得克级规模的纯\textrm{DHR},产率为60\%。
\begin{figure}[h!]
\centering
\vspace*{-0.1in}
\includegraphics[height=1.8in]{Figures/Synthesis-DHR.png}
%\caption{\fontsize{7.2pt}{4.2pt}\selectfont{含有\textrm{5/7/5}环系的共轭分子模型示意图,具有特定的多层堆积特性.}}%
\caption{\textrm{Dicyclohepta[ijkl,uvwx]rubicene~~(DHR)}的合成.}%
\label{Fig:Synthesis-DHR}
\end{figure}

作为对比,还尝试了“先脱氢形成两个七元环、后\textrm{Scholl}反应,形成两个五元环”的\textrm{DHR}合成路线~(见图\ref{Fig:Synthesis-DHR}):~在2.6当量\textrm{DDQ}存在下,化合物1脱氢生成化合物2的反应效率极高,15分钟内即可完成,产率为80\%;但在\ch{FeCl3}存在下,化合物2经\textrm{Scholl}反应生成\textrm{DHR}的尝试,但是尝试了很多方案,都没有能成功。

\section{化合物1的合成}
化学反应式如图\ref{Fig:DHR_Synthesis-equation-1}所示。
\begin{figure}[h!]
\centering
\vspace*{-0.1in}
\includegraphics[height=1.9in]{Figures/DHR_Synthesis-1.png}
\caption{反应物1的合成方程式.}%
\label{Fig:DHR_Synthesis-equation-1}
\end{figure}
化合物1通过改进的文献\cite{JOC55-4943_1990,JCSCC5-381_1987}方法合成。该反应在1000\textrm{mL}三颈圆底烧瓶中进行,烧瓶配备回流冷凝管、滴液漏斗和磁力搅拌器。向烧瓶中加入锌粉~(37.97~\textrm{g},576~\textrm{mmol},3.0当量)和新鲜蒸馏的无水四氢呋喃~(\textrm{THF},400~\textrm{mL}),通氮气鼓泡20分钟。将混合物冷却至$\text{-}78^{\circ}\mathrm{C}$~(丙酮-液氮浴),通过注射器在约15分钟内逐滴注入\ch{TiCl4}~(24~\textrm{mL},211~\textrm{mmol},1.1当量)。逐渐升温至$0^{\circ}\mathrm{C}$,保持$0^{\circ}\mathrm{C}$,反应30分钟,然后升温至室温反应1小时,最后升温至$80^{\circ}\mathrm{C}$回流1小时。随后,在约1小时内逐滴加入溶解于无水\textrm{THF}~(100~\textrm{mL})中的二苯并环庚酮~(40~\textrm{g},192~\textrm{mmol},1.0当量)溶液。

		回流反应1天,通过薄层色谱\textrm{~(TLC)}监测原料消失。冷却至室温后,进一步冷却至$0^{\circ}\mathrm{C}$,加入二氯甲烷~(300~\textrm{mL})稀释反应液,并用1~\textrm{mol/L}盐酸水溶液淬灭。过滤有机相,用无水硫酸钠~(\ch{Na2SO4})干燥。真空除去溶剂,得到粗产物,将其在二氯甲烷-石油醚混合溶剂中多次重结晶纯化,得到化合物1~(5.8~\textrm{g}),收率15.8\%,为黄色粉末。

\subsection{化合物1的基本性质}
\begin{figure}[h!]
\centering
\vspace*{-0.05in}
\includegraphics[height=2.8in]{Figures/DHR_experiment-UV_Absorption.png}
\caption{THF溶液中的紫外-可见吸收光谱.化合物1~(绿),化合物2~(橙),化合物3~(红)}%
\label{Fig:DHR_UV}
\end{figure}
\begin{minipage}{0.40\textwidth}
%\begin{figure}[h!]
%\centering
\vspace*{-0.05in}
\hspace*{-0.4in}
\includegraphics[height=2.0in]{Figures/DHR_synthesis-1-struct.png}
%\caption{反应物1的合成方程式.}%
%\label{Fig:DHR_Compound-1}
%\end{figure}
\end{minipage}
\hspace{1pt}
\begin{minipage}{0.58\textwidth}
		熔点:~$251.3\text{-}253.1^{\circ}\mathrm{C}$

		\textrm{THF}溶液中紫外-可见吸收光谱:\\$\lambda_{\max}/\textrm{nm}~(\epsilon)$:~276~(80300)、389~(5900)、418~(12600)、443~(12600)。如图\ref{Fig:DHR_UV}的绿色谱线所示。

		$^1$H NMR~(400 MHz,\ch{CDCl3})~$\delta$~(ppm):~7.70~(d,J=8.8 Hz,2H,H-8)、7.35-7.34~(m,4H,H-3、2)、7.23-7.19~(m,4H,H-6、1)、7.12-7.06~(m,4H,H-7、9)、3.51~(m,4H,H-5、4)、2.94~(m,2H,H-4)。如图\ref{Fig:DHR_NMR-1_H}所示。
\end{minipage}
\begin{figure}[h!]
\centering
\vspace*{-0.05in}
\includegraphics[height=3.5in]{Figures/DHR_experiment-NMR-1-H.png}
\caption{反应物1的NMR氢谱.}%
\label{Fig:DHR_NMR-1_H}
\end{figure}
\noindent $^{13}$C NMR~(100 MHz,\ch{CDCl3})$~\delta$~(ppm):~143.04~(C-11)、139.59~(C-10)、136.99~(C-12)、135.37~(C-15)、135.06~(C-9)、133.12~(C-14)、129.00~(C-13)、127.80~(C-2)、127.44~(C-6)、126.48~(C-3)、126.24~(C-8)、125.18~(C-1)、123.34~(C-7)、40.71~(C-5)、33.06~(C-4)。如图\ref{Fig:DHR_NMR-1_C}所示。
\begin{figure}[h!]
\centering
\vspace*{-0.05in}
\includegraphics[height=3.5in]{Figures/DHR_experiment-NMR-1-C.png}
\caption{反应物1的NMR碳谱.}%
\label{Fig:DHR_NMR-1_C}
\end{figure}

高分辨质谱~(EI):~分子式\ch{C30H22},理论值382.1722,实测值382.1725。

高分辨质谱~(MALDI-TOF):~$[\mathrm{M}]^+$,分子式\ch{C30H22},理论值382.1716,实测值382.1717。如图\ref{Fig:DHR_MS-1}所示。
\begin{figure}[h!]
\centering
\vspace*{-0.05in}
\includegraphics[height=2.8in]{Figures/DHR_experiment-MS-1.png}
\caption{反应物1的质谱.}%
\label{Fig:DHR_MS-1}
\end{figure}

元素分析:~分子式\ch{C30H22},理论值C 94.20\%、H 5.80\%;实测值C 94.16\%、H 5.79\%。

\section{化合物2的合成}
化学反应式如图\ref{Fig:DHR_Synthesis-equation-2}所示。
\begin{figure}[h!]
\centering
\vspace*{-0.1in}
\includegraphics[height=1.9in]{Figures/DHR_Synthesis-2.png}
\caption{反应物2的合成方程式.}%
\label{Fig:DHR_Synthesis-equation-2}
\end{figure}

在配备回流冷凝管的25 mL双颈圆底烧瓶中,于空气氛围下加入化合物1~(100 mg,0.26 mmol,1.0当量)。通过抽真空-充氮气循环三次后,用注射器加入无水二噁烷~(4 mL)。在$105^{\circ}\mathrm{C}$搅拌下,通过注射器加入溶解于2 mL二噁烷中的DDQ~(153 mg,0.68 mmol,2.6当量)溶液。15分钟后,通过薄层色谱~(TLC)监测原料消失。将反应混合物冷却至室温,过滤,真空浓缩滤液。通过硅胶柱层析~(洗脱剂:~石油醚/二氯甲烷=4:1)分离产物,得到化合物2~(79.2 mg),收率80\%,为橙红色粉末。

\begin{minipage}{0.40\textwidth}
%\begin{figure}[h!]
%\centering
\vspace*{-0.05in}
\hspace*{-0.4in}
\includegraphics[height=2.0in]{Figures/DHR_synthesis-2-struct.png}
%\caption{反应物2的合成方程式.}%
%\label{Fig:DHR_Compound-1}
%\end{figure}
\end{minipage}
\hspace{1pt}
\begin{minipage}{0.58\textwidth}
熔点:~$253.1-253.4^{\circ}\mathrm{C}$。

四氢呋喃溶液中紫外-可见吸收光谱:\\$\lambda_{\max}/\mathrm{nm}$~($\varepsilon$)272~(64300)、299~(28400)、470~(11300)、497~(12800)。如图\ref{Fig:DHR_UV}的橙色谱线所示。

$^1$H NMR~(400 MHz,\ch{CDCl3})δ~(ppm):~7.69~(d,J=8.4 Hz,2H,H-8)、7.28-7.19~(m,6H,H-3、2、1)、7.10~(t,2H,H-7)、7.05~(d,J=6.4 Hz,2H,H-6)、7.00~(d,J=7.6 Hz,2H,H-9)、6.68~(d,J=11.6 Hz,2H,H-4)、6.62~(d,J=12.0 Hz,2H,H-5)。
\end{minipage}

$^{13}$C NMR~(100 MHz,\ch{CDCl3})δ~(ppm):~139.39~(C-10)、138.06~(C-11)、136.52~(C-5)、135.80~(C-13)、135.58~(C-9)、134.67~(C-14、15)、133.01~(C-12)、131.53~(C-4)、129.26~(C-3)、128.28~(C-1)、128.10~(C-6)、127.75~(C-2)、126.55~(C-8)、124.55~(C-7)。

高分辨质谱~(EI):~分子式\ch{C30H18},理论值378.1409,实测值378.1415。

高分辨质谱~(MALDI-TOF):~[M]$^+$,分子式\ch{C30H18},理论值378.1403,实测值378.1404。如图\ref{Fig:DHR_MS-1}所示。
\begin{figure}[h!]
\centering
\vspace*{-0.05in}
\includegraphics[height=2.8in]{Figures/DHR_experiment-MS-2.png}
\caption{反应物2的质谱.}%
\label{Fig:DHR_MS-2}
\end{figure}

元素分析:~分子式\ch{C30H18},理论值C 95.21\%、H 4.79\%;实测值C 94.92\%、H 4.94\%。

\subsection{化合物3的合成}
~(反应式S3)

方法A:~在50 mL双颈圆底烧瓶中,于空气氛围下加入化合物1~(200 mg,0.52 mmol,1.0当量)。通过抽真空-充氮气循环三次后,用注射器加入无水二氯甲烷~(50 mL)。将混合物冷却至0℃~(冰浴),搅拌10分钟。随后,通过注射器加入溶解于10 mL硝基甲烷中的三氯化铁~(1.0 g,6.24 mmol,12.0当量)溶液,反应液颜色由黄绿色变为深绿色。10分钟后,通过薄层色谱~(TLC)监测原料消失。用肼淬灭反应,溶液颜色由绿色变为红色。将反应混合物通过硅藻土短柱过滤,真空浓缩滤液。通过硅胶柱层析~(洗脱剂:~石油醚/二氯甲烷=4:1)纯化残留物,得到化合物3~(172 mg),收率87.8%,为红棕色粉末。

方法B~(放大合成工艺):~在500 mL双颈圆底烧瓶中,于空气氛围下加入化合物1~(1.0 g,2.62 mmol,1.0当量)。通过抽真空-充氮气循环三次后,用注射器加入无水二氯甲烷~(250 mL)。将混合物冷却至0℃~(冰浴),搅拌20分钟。随后,通过注射器加入溶解于50 mL硝基甲烷中的三氯化铁~(5.0 g,31.44 mmol,12.0当量)溶液,反应液颜色由黄绿色变为深绿色。10分钟后,通过薄层色谱~(TLC)监测原料消失。用肼淬灭反应,溶液颜色由绿色变为红色。将反应混合物通过硅藻土短柱过滤,用二氯甲烷洗涤滤饼。滤液用饱和氯化钠溶液萃取三次,无水硫酸钠干燥,真空浓缩。将粗产物溶解于二氯甲烷中,在旋转蒸发仪上加入少量石油醚和甲醇作为不良溶剂进行重结晶,得到纯化合物3~(971.8 mg),收率98.1%。

熔点:~273.4-275.2℃。

四氢呋喃溶液中紫外-可见吸收光谱:~λₘₐₓ/nm~(ε)256~(32800)、305~(20900)、478~(5800)、508~(7900)、541~(6200)。

$^1$H NMR~(400 MHz,CDCl₃)δ~(ppm):~7.82~(d,J=6.4 Hz,2H,H-7)、7.71~(d,J=7.2 Hz,2H,H-1)、7.37~(d,J=6.8 Hz,2H,H-6)、7.20~(t,2H,H-2)、7.12~(d,J=7.2 Hz,2H,H-3)、3.51~(t,J=5.2 Hz,4H,H-5)、3.37~(t,J=5.2 Hz,4H,H-4)。

$^{13}$C NMR~(100 MHz,CDCl₃)δ~(ppm):~140.11~(C-12)、139.29~(C-10)、138.74~(C-9)、137.81~(C-11)、136.42~(C-8)、133.55~(C-15)、132.47~(C-14)、127.78~(C-3)、127.59~(C-6)、126.97~(C-2)、125.09~(C-13)、121.64~(C-7)、119.83~(C-1)、34.38~(C-5)、33.78~(C-4)。

高分辨质谱~(EI):~分子式C₃₀H₁₈,理论值378.1409,实测值378.1404。

高分辨质谱~(MALDI-TOF):~[M]$^+$,分子式C₃₀H₁₈,理论值378.1403,实测值378.1404。

元素分析:~分子式C₃₀H₁₈,理论值C 95.21%、H 4.79%;实测值C 94.93%、H 4.72%。

\subsection{DHR的合成}
~(反应式S4)

在配备回流冷凝管的50 mL双颈圆底烧瓶中,加入化合物3~(100 mg,0.26 mmol,1.0当量)。通过抽真空-充氮气循环三次后,用注射器加入无水二噁烷~(10 mL)。在105℃搅拌下,通过注射器加入溶解于2 mL二噁烷中的DDQ~(360 mg,1.56 mmol,6.0当量)溶液。1小时后,通过薄层色谱~(TLC)监测原料消失。将反应混合物冷却至室温,过滤,用甲醇、石油醚、丙酮和乙腈洗涤滤渣以除去杂质。随后,用二氯甲烷和四氢呋喃作为溶剂多次洗涤滤渣以萃取产物,通过薄层色谱~(TLC)监测产物消失。合并滤液,浓缩得到粗产物,将其在旋转蒸发仪上用四氢呋喃-甲醇混合溶剂重结晶,得到纯DHR~(84.5 mg),收率86.9%,为黑色粉末。

DHR的放大合成工艺:~在配备回流冷凝管的250 mL双颈圆底烧瓶中,加入化合物3~(1.0 g,2.67 mmol,1.0当量)。通过抽真空-充氮气循环三次后,用注射器加入无水二噁烷~(100 mL)。在105℃搅拌下,通过注射器加入溶解于35 mL二噁烷中的DDQ~(3.60 g,15.9 mmol,6.0当量)溶液,反应体系变为蓝紫色。回流反应2小时后,通过薄层色谱~(TLC)监测原料消失。将反应混合物冷却至室温,过滤,用甲醇、石油醚、丙酮和乙腈洗涤滤渣以除去杂质。收集滤渣作为粗产物,在约280℃、20 Pa条件下通过物理气相传输~(PVT)法进一步纯化,得到纯DHR~(599 mg),收率60%,为黑色粉末。

熔点:~363.3-365.1℃。

四氢呋喃溶液中紫外-可见吸收光谱:~λₘₐₓ/nm~(ε)300~(50900)、345~(26800)、565~(7700)、609~(14500)、666~(15800)。

$^1$H NMR~(400 MHz,d₈-THF)δ~(ppm):~8.32~(d,J=6.8 Hz,2H,H-7)、8.24~(d,J=6.4 Hz,2H,H-1)、7.70~(d,J=7.2 Hz,2H,H-6)、7.60-7.58~(m,4H,H-2、3)、7.27~(d,J=12.4 Hz,2H,H-5)、7.05~(d,J=12.4 Hz,2H,H-4)。

$^{13}$C NMR~(273 MHz,d₈-THF)δ~(ppm):~138.50~(C-10)、137.36~(C-12)、136.48~(C-9)、134.20~(C-8)、133.75~(C-11)、133.02~(C-5)、131.78~(C-15)、131.55~(C-4)、129.19~(C-14)、128.90~(C-3)、127.82~(C-6)、127.16~(C-2)、126.78~(C-13)、123.03~(C-7)、122.55~(C-1)。

由于产物DHR在室温下溶解度较低,在100℃下使用500 MHz核磁共振波谱仪,以CDCl₂CDCl₂为溶剂进行高温NMR表征:~

$^1$H NMR~(500 MHz,CDCl₂CDCl₂)δ~(ppm):~8.13~(d,J=7.0 Hz,2H)、8.10~(d,J=7.0 Hz,2H)、7.55-7.50~(m,4H)、7.46~(d,J=7.5 Hz,2H)、7.14~(d,J=12.5 Hz,2H)、6.93~(d,J=12.5 Hz,2H)。

$^{13}$C NMR~(126 MHz,CDCl₂CDCl₂)δ~(ppm):~138.70、137.35、136.51、134.21、133.73、133.16、131.95、131.65、129.39、128.95、127.70、127.16、126.92、122.78、122.49。

高分辨质谱~(EI):~分子式C₃₀H₁₄,理论值374.1096,实测值374.1099。

高分辨质谱~(MALDI-TOF):~[M]$^+$,分子式C₃₀H₁₄,理论值374.1090,实测值374.1089。

元素分析:~分子式C₃₀H₁₄,理论值C 96.23%、H 3.77%;实测值C 96.05%、H 3.73%。


 DHR的结构表征
 单晶结构分析
通过物理气相传输法~(280℃,20 Pa)缓慢升华,获得了适用于单晶结构分析的DHR单晶。DHR分子在晶格中存在无序性:~两个构象体~(i和ii)占据每个晶格位点,占有率分别为46\%和54\%~(见图2a)。构象体i和ii的键长与键角数据见表S1-S4~(支持信息)。在构象体i的C-C键中,C2-C3和C9-C10键长最长,达1.487 Å;而C8-C9、C12-C13和C14-C2键长最短,低至1.33 Å;构象体ii表现出类似的键长趋势。与母体薁的C-C键长~(1.387-1.427 Å[35])相比,DHR中两个薁单元的C-C键长~(构象体i:~1.333-1.487 Å;构象体ii:~1.354-1.458 Å)离散度更大。

图2. DHR在晶体中的分子结构ORTEP图~(概率水平30\%)~(注:~原文含图,此处为图注翻译)
a)两个构象体i和ii;b)带原子编号的构象体i;c)构象体i的堆积方式~(显示π-π距离及短C∙∙∙H接触)

如图2b和2c所示,DHR分子呈近完全平面构型,具有C₂h对称性。作为对比,我们还获得了前驱体1、2、3的晶体结构,发现它们的中心苯环B与末端苯环E分别形成-54.97°、-47.23°和9.75°的二面角~(见图S5,支持信息)。由此推测,“形式薁单元”中五边形与七边形的同时存在,是DHR实现平面结构的关键因素。

如图2c及图S9~(支持信息)所示,DHR分子以“人字形”~(herringbone)方式堆积。由于晶格中分子的无序性,分子间π-π堆积距离~(3.438-3.465 Å)和C∙∙∙H原子间接触距离~(2.684-3.194 Å)存在轻微差异。这种人字形分子间排列理论上有利于电荷传输,但如后文所述,分子无序性会导致电荷传输性能下降。


 稳定性分析
通过¹H NMR、¹³C NMR、高分辨质谱~(HRMS)及元素分析,确认了DHR的化学结构。热重分析~(TGA)数据显示,DHR具有优异的热稳定性:~温度升至400℃时,重量损失仍低于5\%~(见图S1,支持信息)。DHR在室温空气中存放超过6个月,结构无变化;即使在空气中200℃加热1小时,其¹H NMR和质谱谱图仍保持不变~(见图S3-S4,支持信息)。因此,DHR的稳定性与Mastalerz团队报道的含两个嵌入式薁单元的多环芳烃[33]相近,而不同于Müllen和Feng近期报道的含两个五边形和两个七边形的纳米石墨烯~(空气稳定性差[32])。


 芳香性与电子离域性计算
计算了DHR中不同环的Z轴1Å处各向同性化学屏蔽表面~(ICSS~(1)zz)及核独立化学位移~(NICS~(1))值。需注意,ICSS~(1)zz值越正,表明化学屏蔽作用越强,芳香性越高。如图3a所示,化学屏蔽强度顺序为:~环B > 环E/H > 环A/C > 环F/I > 环D/G。此外,NICS~(1)计算结果为:~环B~(-25.7)、环E/H~(-22.4)、环A/C~(-18.8)、环F/I~(-4.3)、环D/G~(12.6)~(见图S10,支持信息),表明六元环具有芳香性,五元环呈弱芳香性,七元环呈弱反芳香性。这与¹H NMR谱中“六元环上质子的化学位移比七元环上质子更显著低场位移”的现象一致~(见图S27,支持信息)。

π电子定域化轨道指示函数~(LOL-π)计算显示,E/F/B/I/H环内的电子离域性更强~(图3b);相比之下,D环中C8-C9键和A环中C2-C14键的电子主要呈定域状态,这与这些环中键长的交替性特征一致~(见表S1-S4,支持信息)。

[图3. DHR的ICSS~(1)zz图~(a)与LOL-π图~(b)]~(注:~原文含图,此处为图注翻译)
a)Z轴1Å处各向同性化学屏蔽表面~(ICSS~(1)zz):~深橙色区域表示芳香性强;b)π电子定域化轨道指示函数~(LOL-π):~橙色表示键内π电子呈定域状态


 DHR的光物理与电化学性质
 紫外-可见吸收与荧光发射
图4a显示了DHR在四氢呋喃~(THF)中的紫外-可见吸收光谱。DHR在紫外区~(305-345 nm)和可见光区~(610-666 nm)均有吸收。密度泛函理论~(DFT)计算表明,610-666 nm的吸收源于S₀→S₁跃迁,而紫外区的吸收源于S₀→S₂跃迁~(图4a)。DHR溶液在670 nm和400 nm处有发射峰~(见图S15,支持信息),结合DFT计算可将其分别归为S₁→S₀跃迁和反常反卡莎规则S₂→S₀跃迁。有趣的是,轨道分析显示,DHR中的“形式薁单元”主要对S₂→S₀跃迁有贡献[36]。


 循环伏安法与差分脉冲伏安法
 在二氯甲烷~(DCM)与邻二氯苯~(o-DCB)的1:1混合溶剂中,对DHR进行了循环伏安法~(CV)和差分脉冲伏安法~(DPV)测试。如图4b所示,DHR表现出四个氧化峰,对应电位分别为$E_p^{ox1}=0.22~\mathrm{V}$、$E_p^{ox2}=0.52~\mathrm{V}$、$E_p^{ox3}=0.96~\mathrm{V}$、$E_p^{ox4}=1.06~\mathrm{V}$~(\textrm{vs.}$\mathrm{Fc}/\mathrm{Fc}^+$),以及两个还原峰,对应电位为$E_p^{\mathrm{red1}}=-1.74~\mathrm{V}$、$E_p^{\mathrm{red2}}=-2.19~\mathrm{V}$~(相对于$\mathrm{Fc}/\mathrm{Fc}^+$);其中第三和第四个氧化峰间距较近,推测可能是由于溶液中DHR形成分子聚集体所致。这一假设得到了以下实验的支持:~当DHR浓度降至$8.9\times10^{-4}~\mathrm{mol/L}$,并在60℃下测试时,第三和第四个氧化峰合并为一个峰~(见图S16,支持信息)。

基于起始氧化电位~(0.18 V vs Fc/Fc⁺)和起始还原电位~(-1.64 V vs Fc/Fc⁺),估算出DHR的最高占据分子轨道~(HOMO)能量为-4.98 eV,最低未占据分子轨道~(LUMO)能量为-3.16 eV,进而计算出其带隙为1.82 eV。


 化学氧化行为
研究了DHR经四氟硼酸硝鎓~(NOBF₄)化学氧化后的吸收光谱变化。氧化产物在THF等常规有机溶剂中的溶解度极低,因此选用三氟乙酸~(TFA)作为溶剂进行测试。如图S17~(支持信息)所示,加入1当量NOBF₄后,体系在900-1400 nm处出现新的吸收峰;电子顺磁共振~(ESR)测试显示,经NOBF₄处理后的DHR溶液出现强ESR信号~(见图S18,支持信息),表明900-1400 nm的新吸收峰可归为DHR自由基阳离子的形成。当NOBF₄用量增加至6当量时,溶液的ESR信号消失,同时在950 nm处出现新的吸收峰,这可能是由于形成了DHR二价阳离子。


 DHR的半导体性能
通过物理气相传输~(PVT)法生长DHR晶体,研究其固态半导体性能。将晶体置于十八烷基三氯硅烷~(OTS)修饰的Si/SiO₂衬底上,制备底栅顶接触~(BGTC)结构的场效应晶体管~(FET)器件:~以晶体为传输沟道,采用掩模板遮挡,随后沉积15 nm厚的三氧化钼~(MoO₃)作为修饰层,并沉积30 nm厚的金~(Au)作为顶源极和漏极。在空气中测试器件的转移曲线和输出曲线~(见图S20,支持信息),结果表明DHR晶体表现出典型的p型半导体行为。

基于20个器件的转移曲线,提取出DHR的空穴迁移率:~最高空穴迁移率可达0.082 cm²·V⁻¹·s⁻¹,开关比为5.45×10⁴;平均空穴迁移率为0.049 cm²·V⁻¹·s⁻¹。

通过优化DHR的中性基态和离子态几何结构,计算出其空穴传输的重组能为251 meV~(见支持信息)。较小的重组能与DHR的刚性结构~(所有共轭环均稠合)相符。进一步计算了电荷转移积分,结果见表S6~(支持信息):~分子间π-π相互作用和C∙∙∙H相互作用相关的转移积分受分子无序性影响——构象体i或ii的同型分子对之间的π-π相互作用转移积分,明显大于构象体i与ii的异型分子对。这一计算结果表明,分子无序性不利于高效电荷传输,与DHR晶体表现出较低电荷迁移率的实验事实一致。

\chapter{\rm{DHR}的表征}
\section{1. 一般表征技术与试剂}
\subsection{1.1 表征技术}
除非另有说明,$^1$H NMR和$^{13}$C NMR光谱采用布鲁克~(Bruker)AVANCE III 400 MHz、500 MHz、600 MHz或950 MHz核磁共振波谱仪测定。质谱通过布鲁克Solarix-XR高分辨质谱仪测定。元素分析在Carlo-Erba-1106型元素分析仪上进行。

快速柱层析采用200-300目硅胶,按标准技术使用指定洗脱剂进行分离。分析型薄层色谱~(TLC)采用预制玻璃背板硅胶板。除非另有说明,展开后的色谱图通过紫外吸收~(254 nm)进行可视化检测。

吸收光谱使用日立~(HITACHI)UH4150紫外-可见分光光度计记录。循环伏安法测试在三电极体系中进行:~工作电极为玻碳电极,辅助电极为铂丝电极,参比电极为Ag/AgCl~(饱和KCl)电极,测试仪器为计算机控制的CHI660C电化学工作站,测试温度为室温,扫描速率为100 mV·s$^{-1}$,支持电解质为0.1 mol/L四丁基六氟磷酸铵~(n-Bu$_4$NPF$_6$)的干燥1,2-二氯苯/二氯甲烷~(体积比1:1)溶液。校准过程中,二茂铁/二茂铁鎓离子对~(Fc/Fc$^+$)的氧化还原电势在相同条件下测定。分子的最高占据分子轨道~(HOMO)和最低未占据分子轨道~(LUMO)能量通过以下公式估算:~
\[
\text{HOMO} = -~(E^{\text{ox}}_{\text{onset}} + 4.8)\text{eV}
\]
\[
\text{LUMO} = -~(E^{\text{red}}_{\text{onset}} + 4.8)\text{eV}
\]

热重分析~(TGA)在TGA8000型热重分析仪上进行,测试条件为氮气氛围,升温速率10℃/min,温度范围50℃至550℃。熔点通过布奇~(BUCHI)B540型熔点仪测定。薄膜的X射线衍射~(XRD)图谱在室温下采用2 kW理学~(Rigaku)X射线衍射系统以反射模式测定。单晶衍射数据通过配备电荷耦合器件~(CCD)面探测器的理学Saturn衍射仪收集。本文报道的晶体结构数据~(不含结构因子)已存入剑桥晶体学数据中心~(CCDC):~化合物1的CCDC编号为1953759,化合物2为1953602,化合物3为1953606,化合物DHR为1971948。

\subsection{1.2 试剂}
二苯并环庚酮购自东京化成工业~(TCI)株式会社。四氯化钛~(TiCl$_4$)和2,3-二氯-5,6-二氰基-1,4-苯醌~(DDQ)购自Alfa Aesar公司,直接使用。三氯化铁~(FeCl$_3$)和硝基甲烷~(CH$_3$NO$_2$)购自国药集团化学试剂有限公司,直接使用。其他试剂均为市售品,除非另有说明,未经进一步纯化直接使用。四氢呋喃~(THF)使用前经金属钠新鲜蒸馏提纯。二噁烷和二氯甲烷购自J\&K公司的超干溶剂,直接使用。其他溶剂除非另有说明,均直接使用。本文中,目标化合物双环庚烯[ijkl,uvwx]红荧烯简称为DHR。


\section{3. 合成与表征}
\section{4. 热重分析~(TGA)曲线}
图S1. DHR的热重分析曲线,升温范围30℃至550℃,升温速率10℃/min。

图S2. 化合物1、2和3的热重分析曲线,升温范围30℃至500℃,升温速率10℃/min。

\section{5. DHR的稳定性}
图S3. DHR的高分辨基质辅助激光解吸电离飞行时间~(HR-MALDI-TOF)质谱对比:~~(a) 新鲜制备的样品;~(b) 在空气中200℃加热1小时后的样品。

图S4. DHR在d₈-THF中25℃下的$^1$H NMR光谱对比~(400 MHz)。标注:~溶剂残留峰~(d₈-THF)、水峰~(H₂O)、旋转边带~(*)。

\section{6. DHR及前驱体化合物1、2、3的晶体结构}
\subsection{6.1 DHR的X射线晶体学数据}
DHR的CCDC编号:~1971948

\begin{table}[h!]
    \centering
    \begin{tabular}{|l|l|}
        \hline
        经验式 & C₃₀H₁₄ \\
        \hline
        分子量 & 374.41 \\
        \hline
        温度/K & 169.99~(11) \\
        \hline
        晶系 & 单斜晶系 \\
        \hline
        空间群 & P2₁/c \\
        \hline
        a/Å & 10.5545~(5) \\
        \hline
        b/Å & 3.9469~(2) \\
        \hline
        c/Å & 20.5542~(10) \\
        \hline
        α/° & 90 \\
        \hline
        β/° & 97.509~(5) \\
        \hline
        γ/° & 90 \\
        \hline
        体积/ų & 848.89~(7) \\
        \hline
        Z值 & 2 \\
        \hline
        计算密度/g·cm⁻³ & 1.465 \\
        \hline
        吸收系数/mm⁻¹ & 0.638 \\
        \hline
        F~(000) & 388.0 \\
        \hline
        晶体尺寸/mm³ & 0.1×0.05×0.01 \\
        \hline
        辐射源 & CuKα~(λ=1.54184) \\
        \hline
        数据收集的2θ范围/° & 8.678至131.95 \\
        \hline
        指标范围 & -12≤h≤12,-4≤k≤4,-24≤l≤16 \\
        \hline
        收集的反射点数 & 7988 \\
        \hline
        独立反射点数 & 1486 $[R_int=0.0485,R_sigma=0.0429]$ \\
        \hline
        数据/限制条件/参数 & 1486/67/272 \\
        \hline
        拟合优度~(F²) & 1.043 \\
        \hline
        最终R指数[I≥2σ~(I)] & R₁=0.0405,wR₂=0.0959 \\
        \hline
        最终R指数~(所有数据) & R₁=0.0680,wR₂=0.1123 \\
        \hline
        最大差值峰/孔/e·Å⁻³ & 0.10/-0.14 \\
        \hline
    \end{tabular}
    \caption{DHR的X射线晶体学数据}
\end{table}

表S1. DHR的键长数据

\begin{table}[h!]
    \centering
    \begin{tabular}{|l|l|l|l|l|l|}
        \hline
        \multicolumn{3}{|c|}{构象体i的键长} & \multicolumn{3}{c|}{构象体ii的键长} \\
        \hline
        原子 & 原子 & 长度/Å & 原子 & 原子 & 长度/Å \\
        \hline
        C1 & C2 & 1.394~(7) & C1 & C2 & 1.408~(8) \\
        \hline
        C1 & C5 & 1.400~(7) & C1 & C5 & 1.392~(6) \\
        \hline
        C1 & C6’ & 1.400~(7) & C1 & C6’ & 1.395~(5) \\
        \hline
        C2 & C3 & 1.479~(9) & C2 & C3 & 1.457~(8) \\
        \hline
        C2 & C14 & 1.381~(7) & C2 & C14 & 1.362~(7) \\
        \hline
        C3 & C4 & 1.401~(8) & C3 & C4 & 1.395~(10) \\
        \hline
        C3 & C13 & 1.381~(8) & C3 & C13 & 1.379~(7) \\
        \hline
        C4 & C5 & 1.462~(15) & C4 & C5 & 1.449~(13) \\
        \hline
        C4 & C10 & 1.403~(10) & C4 & C10 & 1.409~(9) \\
        \hline
        C5 & C6 & 1.410~(8) & C5 & C6 & 1.412~(6) \\
        \hline
        C6 & C7 & 1.436~(8) & C6 & C7 & 1.444~(7) \\
        \hline
        C7 & C8 & 1.459~(9) & C7 & C8 & 1.443~(7) \\
        \hline
        C7 & C15’ & 1.39~(2) & C7 & C15 & 1.419~(16) \\
        \hline
        C8 & C9 & 1.333~(8) & C8 & C9 & 1.354~(7) \\
        \hline
        C9 & C10 & 1.487~(8) & C9 & C10 & 1.458~(8) \\
        \hline
        C10 & C11 & 1.381~(10) & C10 & C11 & 1.422~(9) \\
        \hline
        C11 & C12 & 1.407~(8) & C11 & C12 & 1.387~(8) \\
        \hline
        C12 & C13 & 1.371~(8) & C12 & C13 & 1.388~(10) \\
        \hline
        C14 & C15 & 1.42~(2) & C14 & C15 & 1.41~(2) \\
        \hline
    \end{tabular}
    \caption{DHR的键长数据}
\end{table}

表S2. DHR的键角数据

\begin{table}[h!]
    \centering
    \begin{tabular}{|l|l|l|l|l|l|l|l|}
        \hline
        \multicolumn{4}{|c|}{构象体i的键角} & \multicolumn{4}{c|}{构象体ii的键角} \\
        \hline
        原子 & 原子 & 原子 & 角度/° & 原子 & 原子 & 原子 & 角度/° \\
        \hline
        C2 & C1 & C6’ & 123.9~(12) & C5 & C1 & C2 & 111.3~(10) \\
        \hline
        C6’ & C1 & C5 & 123.5~(6) & C6’ & C1 & C2 & 124.0~(10) \\
        \hline
        C1 & C2 & C3 & 105.8~(7) & C1 & C2 & C3 & 106.2~(6) \\
        \hline
        C14 & C2 & C1 & 118.0~(7) & C14 & C2 & C1 & 116.5~(6) \\
        \hline
        C14 & C2 & C3 & 136.2~(4) & C14 & C2 & C3 & 137.3~(4) \\
        \hline
        C4 & C3 & C2 & 106.8~(6) & C4 & C3 & C2 & 106.9~(5) \\
        \hline
        C13 & C3 & C2 & 133.2~(5) & C13 & C3 & C2 & 133.4~(5) \\
        \hline
        C13 & C3 & C4 & 120.0~(7) & C13 & C3 & C4 & 119.6~(7) \\
        \hline
        C3 & C4 & C5 & 109.6~(9) & C3 & C4 & C5 & 109.7~(8) \\
        \hline
        C3 & C4 & C10 & 122.3~(8) & C3 & C4 & C10 & 122.7~(9) \\
        \hline
        C10 & C4 & C5 & 127.1~(9) & C10 & C4 & C5 & 126.9~(9) \\
        \hline
        C1 & C5 & C4 & 104.6~(12) & C1 & C5 & C4 & 105.2~(10) \\
        \hline
        C1 & C5 & C6 & 119.8~(8) & C1 & C5 & C6 & 119.7~(7) \\
        \hline
        C6 & C5 & C4 & 134.1~(13) & C6 & C5 & C4 & 134.3~(11) \\
        \hline
        C1’ & C6 & C5 & 116.5~(7) & C1’ & C6 & C5 & 115.5~(6) \\
        \hline
        C1’ & C6 & C7 & 119.2~(12) & C1’ & C6 & C7 & 119.3~(10) \\
        \hline
        C5 & C6 & C7 & 124.1~(12) & C5 & C6 & C7 & 125.1~(10) \\
        \hline
        C6 & C7 & C8 & 125.8~(9) & C8 & C7 & C6 & 124.8~(7) \\
        \hline
        C15’ & C7 & C6 & 115.1~(10) & C15 & C7 & C6 & 115.6~(10) \\
        \hline
        C15’ & C7 & C8 & 119.0~(9) & C15 & C7 & C8 & 119.4~(9) \\
        \hline
        C9 & C8 & C7 & 132.3~(5) & C9A & C8A & C7A & 132.6~(5) \\
        \hline
        C8 & C9 & C10 & 131.7~(6) & C8 & C9 & C10 & 131.7~(5) \\
        \hline
        C4 & C10 & C9 & 123.2~(7) & C4 & C10 & C9 & 123.8~(7) \\
        \hline
        C11 & C10 & C4 & 116.1~(6) & C4 & C10 & C11 & 116.2~(6) \\
        \hline
        C11 & C10 & C9 & 120.6~(6) & C11 & C10 & C9 & 119.9~(5) \\
        \hline
        C10 & C11 & C12 & 121.5~(8) & C12 & C11 & C10 & 120.1~(7) \\
        \hline
        C13 & C12 & C11 & 121.3~(8) & C11 & C12 & C13 & 122.2~(7) \\
        \hline
        C12 & C13 & C3 & 118.5~(6) & C3 & C13 & C12 & 118.9~(6) \\
        \hline
        C2 & C14 & C15 & 118.4~(8) & C14’ & C15 & C7 & 121.9~(13) \\
        \hline
        C7’ & C15 & C14 & 125.1~(11) & C2 & C1 & C5 & 112.6~(11) \\
        \hline
    \end{tabular}
    \caption{DHR的键角数据}
\end{table}

\subsection{6.2 化合物1的X射线晶体学数据}
化合物1的CCDC编号:~1953759

\begin{table}[h!]
    \centering
    \begin{tabular}{|l|l|}
        \hline
        经验式 & C₃₀H₂₂ \\
        \hline
        分子量 & 382.47 \\
        \hline
        温度/K & 173.15 \\
        \hline
        波长/Å & 0.71073 \\
        \hline
        晶系 & 单斜晶系 \\
        \hline
        空间群 & P2₁/c \\
        \hline
        晶胞参数 & a=7.6867~(15) Å,α=90° \\
        \hline
        & b=10.954~(2) Å,β=94.39~(3)° \\
        \hline
        体积/ų & 1937.5~(7) \\
        \hline
        Z值 & 4 \\
        \hline
        计算密度/Mg·m⁻³ & 1.311 \\
        \hline
        吸收系数/mm⁻¹ & 0.074 \\
        \hline
        F~(000) & 808 \\
        \hline
        晶体尺寸/mm³ & 0.337×0.291×0.035 \\
        \hline
        数据收集的θ范围/° & 2.059至27.489 \\
        \hline
        指标范围 & -9≤h≤9,-14≤k≤14,-26≤l≤29 \\
        \hline
        收集的反射点数 & 22413 \\
        \hline
        独立反射点数 & 4406 [R~(int)=0.0514] \\
        \hline
        θ=25.242°时的完整性 & 99.5\% \\
        \hline
        吸收校正 & 基于等效反射的半经验校正 \\
        \hline
        最大/最小透射率 & 1.00000/0.86939 \\
        \hline
        精修方法 & 基于F²的全矩阵最小二乘法 \\
        \hline
        数据/限制条件/参数 & 4406/0/271 \\
        \hline
        拟合优度~(F²) & 1.288 \\
        \hline
        最终R指数[I≥2σ~(I)] & R₁=0.0663,wR₂=0.1281 \\
        \hline
        最终R指数~(所有数据) & R₁=0.0695,wR₂=0.1296 \\
        \hline
        消光系数 & 无 \\
        \hline
        最大差值峰/孔/e·Å⁻³ & 0.235/-0.158 \\
        \hline
    \end{tabular}
    \caption{化合物1的X射线晶体学数据}
\end{table}

\subsection{6.3 化合物2的X射线晶体学数据数据}
化合物2的CCDC编号:~1953602

\begin{table}[h!]
    \centering
    \begin{tabular}{|l|l|}
        \hline
        经验式 & C₃₀H₁₈ \\
        \hline
        分子量 & 378.44 \\
        \hline
        温度/K & 173.15 \\
        \hline
        波长/Å & 0.71073 \\
        \hline
        晶系 & 单斜晶系 \\
        \hline
        空间群 & P2₁/c \\
        \hline
        晶胞参数 & a=10.141~(3) Å,α=90° \\
        \hline
        & b=13.470~(4) Å,β=96.330~(3)° \\
        \hline
        体积/ų & 1874.6~(9) \\
        \hline
        Z值 & 4 \\
        \hline
        计算密度/Mg·m⁻³ & 1.341 \\
        \hline
        吸收系数/mm⁻¹ & 0.076 \\
        \hline
        F~(000) & 792 \\
        \hline
        晶体尺寸/mm³ & 0.352×0.261×0.109 \\
        \hline
        数据收集的θ范围/° & 3.039至27.479 \\
        \hline
        指标范围 & -13≤h≤12,-17≤k≤16,-17≤l≤17 \\
        \hline
        收集的反射点数 & 15054 \\
        \hline
        独立反射点数 & 4273 [R~(int)=0.0527] \\
        \hline
        θ=25.242°时的完整性 & 99.5\% \\
        \hline
        吸收校正 & 基于等效反射的半经验校正 \\
        \hline
        最大/最小透射率 & 1.00000/0.75431 \\
        \hline
        精修方法 & 基于F²的全矩阵最小二乘法 \\
        \hline
        数据/限制条件/参数 & 4273/0/271 \\
        \hline
        拟合优度~(F²) & 1.164 \\
        \hline
        最终R指数[I≥2σ~(I)] & R₁=0.0581,wR₂=0.1197 \\
        \hline
        最终R指数~(所有数据) & R₁=0.0662,wR₂=0.1238 \\
        \hline
        消光系数 & 无 \\
        \hline
        最大差值峰/孔/e·Å⁻³ & 0.271/-0.180 \\
        \hline
    \end{tabular}
    \caption{化合物2的X射线晶体学数据}
\end{table}

\subsection{6.4 化合物3的X射线晶体学数据}
化合物3的CCDC编号:~1953606
\begin{table}[h]
    \centering
    \begin{tabular}{ll}
        \hline
        经验式 & $\mathrm{C}_{15}\mathrm{H}_{9}$ \\
        式量 & 189.22 \\
        温度 & 173.15 K \\
        波长 & 0.71073 Å \\
        晶体系统 & 单斜晶系 \\
        空间群 & $P2_1/n$ \\
        晶胞参数 & $a = 10.012~(3)$ Å, $\alpha = 90^\circ$ \\
        & $b = 8.313~(3)$ Å, $\beta = 110.939~(6)^\circ$ \\
        体积 & 902.6~(5) Å$^3$ \\
        $Z$ 值 & 4 \\
        计算密度 & 1.393 Mg/m$^3$ \\
        吸收系数 & 0.079 mm$^{-1}$ \\
        $F~(000)$ & 396 \\
        晶体尺寸 & 0.522 × 0.315 × 0.159 mm \\
        数据收集的 $\theta$ 范围 & 3.347 至 27.495° \\
        指标范围 & $-12 \leq h \leq 12$, $-10 \leq k \leq 9$, $-15 \leq l \leq 15$ \\
        收集到的反射点 & 9824 \\
        独立反射点 & 2055 [$\mathrm{R_{int}} = 0.0409$] \\
        对 $\theta = 25.242^\circ$ 的完整性 & 99.3\% \\
        吸收校正 & 基于等效反射的半经验法 \\
        最大和最小透射率 & 1.00000 和 0.79415 \\
        精修方法 & 基于 $F^2$ 的全矩阵最小二乘法 \\
        数据/约束/参数 & 2055 / 0 / 136 \\
        拟合优度 $\mathrm{GOF}$ on $F^2$ & 1.142 \\
        最终 $\mathrm{R}$ 指数 [$\mathrm{I>2\sigma~(I)}$] & $\mathrm{R_1} = 0.0495$, $\mathrm{wR_2} = 0.1197$ \\
        最终 $\mathrm{R}$ 指数 [所有数据] & $\mathrm{R_1} = 0.0514$, $\mathrm{wR_2} = 0.1223$ \\
        消光系数 & 无 \\
        最大差值峰/孔 & 0.232 和 -0.160 e·Å$^{-3}$ \\
        \hline
    \end{tabular}
    \caption{化合物 3 的 X 射线晶体学数据}
    \label{tab:crystal_data_3}
\end{table}
\subsection{6.5 5/7元环对分子的平面化作用}
图S5. 化合物DHR~(a)、1~(b)、2~(c)和3~(d)的选定二面角。

\begin{table}[h!]
    \centering
    \begin{tabular}{|l|l|l|l|l|}
        \hline
        化合物 & DHR~(构象体ii) & 1 & 2~(D环非平面) & 3 \\
        \hline
        二面角~(环B与环E)/° & 11.29 & -54.97 & -47.23 & 9.75 \\
        \hline
    \end{tabular}
    \caption{化合物的二面角数据}
\end{table}

\subsection{6.6 Structure analysis of DHR}
\subsubsection{Table S3  DHR 构象体 i 的选定键长和键角总结与分析}
\begin{table}[h]
    \centering
    \begin{tabular}{cccccc}
        \hline
        环 & 平均键长 ~(Å) & 最长键长 ~(Å) & 最短键长 ~(Å) & 最大键角 ~(°) & 最小键角 ~(°) \\
        \hline
        环 A ~(C)~(六元环) & 1.403 & 1.436 ~(C7’-C6’) & 1.381 ~(C14-C2) & 125.02 ~($\angle$C14C15C7’) & 115.15 ~($\angle$C15C7‘C6’) \\
        环 B~(六元环) & 1.403 & 1.410 ~(C6’-C5’) & 1.399 ~(C1-C6’) & 123.45 ~($\angle$C5C1C6’) & 116.53 ~($\angle$C5C6C1’) \\
        环 E ~(H)~(六元环) & 1.391 & 1.407 ~(C11-C12) & 1.371 ~(C12-C13) & 122.33 ~($\angle$C3C4C10) & 116.12 ~($\angle$C4C10C11) \\
        环 D ~(G)~(七元环) & 1.427 & 1.487 ~(C9-C10) & 1.333 ~(C8-C9) & 134.11 ~($\angle$C4C5C6) & 123.29 ~($\angle$C4C10C9) \\
        环 F ~(I)~(五元环) & 1.427 & 1.479 ~(C2-C3) & 1.394 ~(C2-C1) & 112.55 ~($\angle$C5C1C2) & 104.65 ~($\angle$C1C5C4) \\
        \hline
    \end{tabular}
    \caption{DHR 构象体 i 的环结构键长与键角分析}
    \label{tab:dhr_conformer_i_analysis}
\end{table}

%\begin{figure}[h]
%    \centering
%    \includegraphics[width=0.9\textwidth]{figure_S6.pdf} % 需替换为实际图片路径
%    \caption{DHR 构象体 i 中选定的键角~(单位:~°)。采用线框顶视图展示,氢原子已省略以清晰呈现环结构。关键键角数值标注于对应位置。}
%    \label{fig:dhr_conformer_i_bond_angles}
%\end{figure}

\subsubsection{Table S4  DHR 构象体 ii 的选定键长和键角总结与分析}
\begin{table}[h]
    \centering
    \begin{tabular}{cccccc}
        \hline
        环 & 平均键长 ~(Å) & 最长键长 ~(Å) & 最短键长 ~(Å) & 最大键角 ~(°) & 最小键角 ~(°) \\
        \hline
        环 A ~(C)~(六元环) & 1.407 & 1.444 ~(C7’-C6’) & 1.362 ~(C14-C2) & 123.98 ~($\angle$C2C1C6’) & 115.69 ~($\angle$C15C7‘C6’) \\
        环 B~(六元环) & 1.400 & 1.412 ~(C6’-C5’) & 1.392 ~(C1-C5) & 124.61 ~($\angle$C5C1C6’) & 115.48 ~($\angle$C1C6‘C5’) \\
        环 E ~(H)~(六元环) & 1.397 & 1.422 ~(C10-C11) & 1.379 ~(C3-C13) & 122.70 ~($\angle$C3C4C10) & 116.22 ~($\angle$C4C10C11) \\
        环 D ~(G)~(七元环) & 1.431 & 1.458 ~(C9-C10) & 1.354 ~(C8-C9) & 134.33 ~($\angle$C4C5C6) & 123.77 ~($\angle$C4C10C9) \\
        环 F ~(I)~(五元环) & 1.426 & 1.457 ~(C2-C3) & 1.392 ~(C1-C5) & 111.36 ~($\angle$C2C1C5) & 105.12 ~($\angle$C1C5C4) \\
        \hline
    \end{tabular}
    \caption{DHR 构象体 ii 的环结构键长与键角分析}
    \label{tab:dhr_conformer_ii_analysis}
\end{table}

%\begin{figure}[h]
%    \centering
%    \includegraphics[width=0.9\textwidth]{figure_S7.pdf} % 需替换为实际图片路径
%    \caption{DHR 构象体 ii 中选定的键角~(单位:~°)。采用线框顶视图展示,氢原子已省略以清晰呈现环结构。关键键角数值标注于对应位置。}
%    \label{fig:dhr_conformer_ii_bond_angles}
%\end{figure}
%
%\begin{figure}[h]
%    \centering
%    \includegraphics[width=0.6\textwidth]{figure_S8.pdf} % 需替换为实际图片路径
%    \caption{薁~(azulene)的选定键长和键角总结。数据来源于已报道文献\cite{ref14},关键键长~(单位:~Å)和键角~(单位:~°)标注于对应结构位置。}
%    \label{fig:azulene_bond_data}
%\end{figure}
%
%\begin{figure}[h]
%    \centering
%    \includegraphics[width=0.8\textwidth]{figure_S9.pdf} % 需替换为实际图片路径
%    \caption{DHR 不同构象体~(构象体 i 和构象体 ii)之间的选定堆积模式。图中标记①和②的分子分别代表构象体 i 或 ii,展示了单胞中两种分子间的相互作用。关键分子间距离~(单位:~Å)标注为 d:~d=3.002~(a)、3.153~(b)、3.194~(c)、2.705~(d)、2.685~(e)、2.824~(f)、2.684~(g)、2.880~(h)。}
%    \label{fig:dhr_packing_modes}
%\end{figure}
%
%\begin{figure}[h]
%    \centering
%    \includegraphics[width=0.7\textwidth]{figure_S10.pdf} % 需替换为实际图片路径
%    \caption{DHR 中不同环的计算 NICS~(1)~(核独立化学位移)值。各环~(A、B、D、E、F、G、H)对应的 NICS~(1) 数值标注于环中心位置,单位:~ppm。}
%    \label{fig:dhr_nics_values}
%\end{figure}

\section*{7. Absorption spectra}
%\begin{figure}[h]
%    \centering
%    \includegraphics[width=0.9\textwidth]{figure_S11.pdf} % 需替换为实际图片路径
%    \caption{DHR 在 THF 溶液~(1.0×10$^{-5}$ mol/L,蓝色曲线)和薄膜状态~(红色曲线)下的吸收光谱对比。横坐标为波长~(nm),纵坐标为吸光度~(Absorption)。}
%    \label{fig:dhr_absorption_solution_film}
%\end{figure}
%
%\begin{figure}[h]
%    \centering
%    \includegraphics[width=0.9\textwidth]{figure_S12.pdf} % 需替换为实际图片路径
%    \caption{化合物 1~(绿色曲线)、2~(橙色曲线)和 3~(红色曲线)在 THF 溶液~(1.0×10$^{-5}$ mol/L)中的紫外-可见吸收光谱。横坐标为波长~(nm),纵坐标为吸光度~(Absorption)。}
%    \label{fig:compounds_123_absorption}
%\end{figure}
%
\section*{8. Simulated absorptions and emissions of DHR}
\subsection{8.1 The optimized geometries and electronic structures of frontier orbitals of DHR}
%\begin{figure}[h]
%    \centering
%    \includegraphics[width=0.8\textwidth]{figure_S13.pdf} % 需替换为实际图片路径
%    \caption{DHR 在 S$_0$、S$_1$ 和 S$_2$ 态下的优化几何结构。所有结构均在 CASSCF/6-31G*/~(12,12) 水平下进行优化,并约束对称性。}
%    \label{fig:dhr_optimized_geometries}
%\end{figure}
%
%\begin{figure}[h]
%    \centering
%    \includegraphics[width=0.9\textwidth]{figure_S14.pdf} % 需替换为实际图片路径
%    \caption{DHR 参与跃迁的前线轨道电子云密度分布。展示了 HOMO、HOMO-1、LUMO 和 LUMO+1 轨道的电子云分布特征,不同颜色代表电子云密度高低~(通常红色为高电子密度,蓝色为低电子密度)。}
%    \label{fig:dhr_frontier_orbitals}
%\end{figure}

\begin{table}[h]
    \centering
    \begin{tabular}{ccccc}
        \hline
        结构 & 态 & 能量 ~(eV) & 振子强度 & 跃迁特性 \\
        \hline
        \multirow{2}{*}{S$_0$-min~(吸收)} & S$_1$ ~(Bu) & 1.99 & 0.129 & HOMO → LUMO ~(44\%), HOMO-1 → LUMO ~(16\%), HOMO → LUMO+1 ~(10\%) \\
        & S$_2$ ~(Bu) & 3.05 & 0.200 & HOMO → LUMO ~(36\%), HOMO-1 → LUMO ~(16\%), HOMO → LUMO+1 ~(18\%) \\
        \multirow{2}{*}{S$_1$-min~(发射)} & S$_1$ ~(Bu) & 1.69 & 0.159 & HOMO → LUMO ~(67\%), HOMO-1 → LUMO ~(10\%) \\
        & S$_2$ ~(Bu) & 2.60 & 0.096 & HOMO → LUMO ~(10\%), HOMO-1 → LUMO ~(40\%), HOMO → LUMO+1 ~(16\%) \\
        \multirow{2}{*}{S$_2$-min~(发射)} & S$_1$ ~(Bu) & 1.78 & 0.185 & HOMO → LUMO ~(60\%), HOMO-1 → LUMO ~(11\%) \\
        & S$_2$ ~(Bu) & 2.93 & 0.116 & HOMO → LUMO ~(19\%), HOMO-1 → LUMO ~(29\%), HOMO → LUMO+1 ~(22\%) \\
        \hline
    \end{tabular}
    \caption{DHR 的计算激发能、振子强度和跃迁特性。激发态能量和振子强度基于 CASSCF 轨道在 CASPT2/6-31G*/~(12,12) 水平下计算。}
    \label{tab:dhr_excitation_properties}
\end{table}

\subsection{8.2 Simulated emission}
%\begin{figure}[h]
%    \centering
%    \includegraphics[width=0.9\textwidth]{figure_S15.pdf} % 需替换为实际图片路径
%    \caption{DHR 在 THF 溶液~(1.0×10$^{-7}$ mol/L)中 305 nm 激发下的实验发射光谱~(黑色曲线)与气相中的模拟发射光谱~(绿色曲线:~S$_2$→S$_0$ 跃迁;红色曲线:~S$_1$→S$_0$ 跃迁)。插图 4-1 和 4-2 分别展示了分子中键振动对 S$_1$→S$_0$ 和 S$_2$→S$_0$ 发射的模拟结构贡献,颜色越深的位置表示重组能越大,贡献越显著。横坐标为波长~(nm),纵坐标为发射强度~(Intensity)。}
%    \label{fig:dhr_emission_spectra}
%\end{figure}

\section*{9. Cyclic voltammetry and differential pulse voltammetry measurements for DHR}
%\begin{figure}[h]
%    \centering
%    \includegraphics[width=0.8\textwidth]{figure_S16.pdf} % 需替换为实际图片路径
%    \caption{DHR~(8.9×10$^{-4}$ M)在 DCM/o-DCB~(1:1)混合溶剂中的循环伏安图~(CV,黑色曲线)和差分脉冲伏安图~(DPV,红色曲线)。测试条件:~60 °C,支持电解质为 0.1 M Bu$_4$NPF$_6$,工作电极为玻碳电极,对电极为铂丝,参比电极为 Ag/AgCl,扫描速率为 100 mV/s。二茂铁~(Fc)用作外部参比,电位以 E$_{1/2}$~(Fc/Fc$^+$) 为基准表示。}
%    \label{fig:dhr_cv_dpv}
%\end{figure}

\section*{10. Absorption and ESR spectra of DHR after chemical oxidation}
%\begin{figure}[h]
%    \centering
%    \includegraphics[width=0.8\textwidth]{figure_S17.pdf} % 需替换为实际图片路径
%    \caption{DHR~(1.0×10$^{-3}$ M)在 CF$_3$COOH 中分别加入 1 当量和 6 当量 NOBF$_4$ 氧化后的吸收光谱。横坐标为波长~(nm),纵坐标为吸光度~(Absorption),不同曲线对应不同氧化当量。}
%    \label{fig:dhr_oxidation_absorption}
%\end{figure}
%
%\begin{figure}[h]
%    \centering
%    \includegraphics[width=0.8\textwidth]{figure_S18.pdf} % 需替换为实际图片路径
%    \caption{DHR 加入不同量 NOBF$_4$ 氧化后的 ESR 光谱。横坐标为磁场强度~(G),纵坐标为信号强度~(Intensity),不同曲线对应不同氧化剂用量。}
%    \label{fig:dhr_oxidation_esr}
%\end{figure}

\section*{11. OFET devices}
\subsection{11.1 Substrate treatment}
\begin{verbatim}
基底依次用纯水、食人鱼溶液~(H₂SO₄:H₂O₂ = 3:7)、纯水、异丙醇清洗,然后进行氧等离子体处理~(5 min,100 W)。采用气相沉积法用十八烷基三氯硅烷~(OTS)对 Si/SiO₂ 晶片进行修饰:~将清洁后的 Si/SiO₂ 晶片在 90 °C 真空下干燥 90 min 以去除水分,冷却至室温后滴加少量 OTS,在真空条件下加热至 120 °C 并保持 2 h。
\end{verbatim}

\subsection{11.2 Crystal growth}
\begin{verbatim}
采用双区水平炉通过物理气相传输~(PVT)法生长 DHR 微晶:~将装有 DHR 粉末的石英舟置于高温区~(200-210 °C),OTS/Si/SiO₂ 晶片置于结晶区,以高纯度 Ar 为载气~(流速 30-50 sccm),系统压力维持在 21 Pa,生长 5 h 后,微晶条带沉积在 OTS/Si/SiO₂ 基底上。
\end{verbatim}

\subsection{11.3 OFETs fabrication and measurement}
\begin{verbatim}
基于 OTS 修饰的 Si/SiO₂~(300 nm)基底,采用“条带掩模技术”制备底栅顶接触~(BGTC)结构的 OFET 器件。重掺杂 n 型 Si 晶片用作栅电极,为增强载流子注入,通过热蒸发~(速率 0.1 Å/s)在 DHR 微晶与漏/源电极之间插入 1.5 nm 厚的 MoO₃ 缓冲层,随后以 0.1 Å/s 的低升华速率热蒸发金膜,通过“条带掩模技术”在 DHR 微单晶上沉积 30 nm 厚的漏/源电极。
\end{verbatim}

\subsection{11.4 Characterization}
%\begin{figure}[h]
%    \centering
%    \subfigure[DHR 微晶的光学显微镜图]{
%        \includegraphics[width=0.45\textwidth]{figure_S19a.pdf} % 需替换为实际图片路径
%    }
%    \subfigure[FET 器件的光学显微镜图]{
%        \includegraphics[width=0.45\textwidth]{figure_S19b.pdf} % 需替换为实际图片路径
%    }
%    \subfigure[TEM 图像及对应的 SAED 图谱]{
%        \includegraphics[width=0.45\textwidth]{figure_S19c.pdf} % 需替换为实际图片路径
%    }
%    \subfigure[晶体的 XRD 图谱]{
%        \includegraphics[width=0.45\textwidth]{figure_S19d.pdf} % 需替换为实际图片路径
%    }
%    \caption{DHR 微晶及 OFET 器件的表征:~~(a) 微晶光学显微镜图~(标尺:~40 μm);~(b) 器件光学显微镜图~(标尺:~20 μm);~(c) TEM 图像~(标尺:~50 nm)及 SAED 图谱;~(d) XRD 图谱~(横坐标:~2θ ~(°),纵坐标:~强度 ~(a.u.))。}
%    \label{fig:dhr_crystal_device_char}
%\end{figure}
%
%\begin{figure}[h]
%    \centering
%    \subfigure[转移特性曲线]{
%        \includegraphics[width=0.45\textwidth]{figure_S20a.pdf} % 需替换为实际图片路径
%    }
%    \subfigure[输出特性曲线]{
%        \includegraphics[width=0.45\textwidth]{figure_S20b.pdf} % 需替换为实际图片路径
%    }
%    \caption{DHR 晶体 OFET 器件的电学性能:~~(a) 转移曲线~(横坐标:~栅极电压 V$_G$ ~(V),纵坐标:~漏极电流 I$_{DS}$ ~(A));~(b) 输出曲线~(横坐标:~漏源电压 V$_{DS}$ ~(V),纵坐标:~漏极电流 I$_{DS}$ ~(A))。迁移率通过饱和区公式 I$_{DS} = ~(W/~(2L))C_i\mu~(V_G - V_T)^2$ 提取,其中 W 为沟道宽度,L 为沟道长度,C_i 为栅介质电容,μ 为迁移率,V_T 为阈值电压。}
%    \label{fig:dhr_ofet_electrical}
%\end{figure}

\section*{12. Calculations of the reorganization energy and transfer integrals}
%\begin{figure}[h]
%    \centering
%    \includegraphics[width=0.7\textwidth]{figure_S21.pdf} % 需替换为实际图片路径
%    \caption{DHR 分子前线轨道的计算能级图。标注了 HOMO、HOMO-1、LUMO、LUMO+1 等轨道的能级位置~(单位:~eV),并给出计算得到的第一、第二和第三电离势分别为 4.96、5.44、6.87 eV。}
%    \label{fig:dhr_frontier_energy_levels}
%\end{figure}
%
%\begin{figure}[h]
%    \centering
%    \includegraphics[width=0.8\textwidth]{figure_S22.pdf} % 需替换为实际图片路径
%    \caption{DHR 的晶体结构及用于转移积分计算的分子对示意图。考虑晶体无序性,构象体 i 和构象体 ii 分别用灰色和橙色标记,虚线框标出选定的分子对。}
%    \label{fig:dhr_crystal_transfer_integral}
%\end{figure}

\begin{table}[h]
    \centering
    \begin{tabular}{ccccccccc}
        \hline
        作用模式 & \multicolumn{4}{c}{人字形~(herringbone)} & \multicolumn{4}{c}{平行~(parallel)} \\
        \cline{2-9}
        分子对 & ①i--②i & ①ii--②ii & ①i--②ii & ①ii--①i & ②i--③i & ②ii--③ii & ②i--③ii & ②ii--③i \\
        \hline
        空穴转移积分 t$_h$ ~(meV) & 2.9 & 0.4 & 3.9 & 1.9 & 75.2 & 110.6 & 13.6 & 13.6 \\
        \hline
    \end{tabular}
    \caption{从晶体结构中选取的分子对的空穴转移积分计算结果。所有计算在 DFT-ωB97XD/6-31G** 水平下结合片段轨道法和基组正交化程序进行。}
    \label{tab:dhr_transfer_integrals}
\end{table}

\begin{verbatim}
基于 DHR 基态中性态和离子态的优化几何结构,采用四点势能面法计算空穴转移的重组能为 251 meV。结合极化连续介质模型~(PCM)进一步计算第一、第二和第三电离势,介电常数 ε 设为 20 以考虑溶剂化效应。所有 DFT 计算均使用 Gaussian 16 程序包完成。
\end{verbatim}

\section*{13. NMR Spectra}
\subsection{13.1 NMR of DHR}
%\begin{figure}[h]
%    \centering
%    \includegraphics[width=0.9\textwidth]{figure_S23.pdf} % 需替换为实际图片路径
%    \caption{DHR 在 100℃ 下于 $CDCl_2CDCl_2$ 中的 $^1H$ NMR 谱图~(500 MHz)}
%    \label{fig:dhr_1hnmr_100c}
%\end{figure}
%
%\begin{figure}[h]
%    \centering
%    \includegraphics[width=0.9\textwidth]{figure_S24.pdf} % 需替换为实际图片路径
%    \caption{DHR 在 100℃ 下于 $CDCl_2CDCl_2$ 中的 $^{13}C$ NMR 谱图~(126 MHz)}
%    \label{fig:dhr_13cnmr_100c}
%\end{figure}
%
%\begin{figure}[h]
%    \centering
%    \includegraphics[width=0.9\textwidth]{figure_S25.pdf} % 需替换为实际图片路径
%    \caption{DHR 在 25℃ 下于 $d_8$-THF 中的 $^1H$ NMR 谱图~(400 MHz)。★ 代表 $d_8$-THF 的溶剂残留峰,+ 代表 $H_2O$ 峰,* 代表旋转边带。}
%    \label{fig:dhr_1hnmr_d8thf}
%\end{figure}
%
%\begin{figure}[h]
%    \centering
%    \includegraphics[width=0.9\textwidth]{figure_S26.pdf} % 需替换为实际图片路径
%    \caption{25℃ 下 $d_8$-THF 的 $^1H$ NMR 谱图~(400 MHz)。★ 代表 $d_8$-THF 的溶剂残留峰,+ 代表 $H_2O$ 峰。}
%    \label{fig:d8thf_1hnmr}
%\end{figure}
%
%\begin{figure}[h]
%    \centering
%    \includegraphics[width=0.9\textwidth]{figure_S27.pdf} % 需替换为实际图片路径
%    \caption{DHR 在 $d_8$-THF 中的 $^1H$ NMR 谱图归属~(400 MHz,25℃)。七元环 D/G 中的质子~(H5 和 H4)相较于分子中其他质子呈高场位移。}
%    \label{fig:dhr_1hnmr_assignment}
%\end{figure}
%
%\begin{figure}[h]
%    \centering
%    \includegraphics[width=0.9\textwidth]{figure_S28.pdf} % 需替换为实际图片路径
%    \caption{DHR 在 25℃ 下于 $d_8$-THF 中的 $^{13}C$ NMR 谱图~(239 MHz)。★ 代表 $d_8$-THF 的溶剂残留峰。}
%    \label{fig:dhr_13cnmr_d8thf}
%\end{figure}
%
%\begin{figure}[h]
%    \centering
%    \includegraphics[width=0.9\textwidth]{figure_S29.pdf} % 需替换为实际图片路径
%    \caption{DHR 的固体 $^{13}C$ NMR 谱图~(100 MHz)}
%    \label{fig:dhr_solid_13cnmr}
%\end{figure}
%
%\begin{figure}[h]
%    \centering
%    \includegraphics[width=0.9\textwidth]{figure_S30.pdf} % 需替换为实际图片路径
%    \caption{DHR 在 $d_8$-THF 中的 COSY 谱图~(500 MHz,25℃)}
%    \label{fig:dhr_cosy}
%\end{figure}
%
%\begin{figure}[h]
%    \centering
%    \includegraphics[width=0.9\textwidth]{figure_S31.pdf} % 需替换为实际图片路径
%    \caption{DHR 在 $d_8$-THF 中的 NOESY 谱图~(500 MHz,25℃)}
%    \label{fig:dhr_noesy}
%\end{figure}
%
%\begin{figure}[h]
%    \centering
%    \includegraphics[width=0.9\textwidth]{figure_S32.pdf} % 需替换为实际图片路径
%    \caption{DHR 在 $d_8$-THF 中的 HSQC NMR 谱图~(500 MHz,60℃)}
%    \label{fig:dhr_hsqc}
%\end{figure}
%
%\begin{figure}[h]
%    \centering
%    \includegraphics[width=0.9\textwidth]{figure_S33.pdf} % 需替换为实际图片路径
%    \caption{DHR 在 $d_8$-THF 中的 HMBC NMR 谱图~(950 MHz,25℃)}
%    \label{fig:dhr_hmbc}
%\end{figure}

\subsection{13.2 NMR spectra of compound 1, 2 and 3}
\subsubsection{13.2.1 NMR spectra of compound 1}
%\begin{figure}[h]
%    \centering
%    \includegraphics[width=0.9\textwidth]{figure_S34.pdf} % 需替换为实际图片路径
%    \caption{化合物 1 在 25℃ 下于 $CDCl_3$ 中的 $^1H$ NMR 谱图~(400 MHz)。★ 代表 $CDCl_3$ 的溶剂残留峰,+ 代表 $H_2O$ 峰,# 代表内标 $Me_4Si$ 峰。}
%    \label{fig:compound1_1hnmr}
%\end{figure}
%
%\begin{figure}[h]
%    \centering
%    \includegraphics[width=0.9\textwidth]{figure_S35.pdf} % 需替换为实际图片路径
%    \caption{化合物 1 在 25℃ 下于 $CDCl_3$ 中的 $^{13}C$ NMR 谱图~(100 MHz)。★ 代表 $CDCl_3$ 的溶剂残留峰。}
%    \label{fig:compound1_13cnmr}
%\end{figure}
%
%\begin{figure}[h]
%    \centering
%    \includegraphics[width=0.9\textwidth]{figure_S36.pdf} % 需替换为实际图片路径
%    \caption{化合物 1 在 $d_8$-THF 中的 COSY NMR 谱图~(500 MHz,25℃)}
%    \label{fig:compound1_cosy}
%\end{figure}
%
%\begin{figure}[h]
%    \centering
%    \includegraphics[width=0.9\textwidth]{figure_S37.pdf} % 需替换为实际图片路径
%    \caption{化合物 1 在 $d_8$-THF 中的 NOESY NMR 谱图~(500 MHz,25℃):~全谱图展示}
%    \label{fig:compound1_noesy_full}
%\end{figure}
%
%\begin{figure}[h]
%    \centering
%    \includegraphics[width=0.9\textwidth]{figure_S38.pdf} % 需替换为实际图片路径
%    \caption{化合物 1 在 $d_8$-THF 中的 NOESY NMR 谱图~(500 MHz,25℃):~芳香区展示~(V 为 H9-H8 相关峰)}
%    \label{fig:compound1_noesy_aromatic}
%\end{figure}
%
%\begin{figure}[h]
%    \centering
%    \includegraphics[width=0.9\textwidth]{figure_S39.pdf} % 需替换为实际图片路径
%    \caption{化合物 1 在 $d_8$-THF 中的 NOESY NMR 谱图~(500 MHz,25℃):~烷基区~(H4 和 H5)展示~(VI 为 H3-H4 相关峰,VII 为 H6-H5 相关峰)}
%    \label{fig:compound1_noesy_alkyl}
%\end{figure}
%
%\begin{figure}[h]
%    \centering
%    \includegraphics[width=0.9\textwidth]{figure_S40.pdf} % 需替换为实际图片路径
%    \caption{化合物 1 在 $CDCl_3$ 中的 HSQC NMR 谱图~(600 MHz,25℃)}
%    \label{fig:compound1_hsqc}
%\end{figure}
%
%\begin{figure}[h]
%    \centering
%    \includegraphics[width=0.9\textwidth]{figure_S41.pdf} % 需替换为实际图片路径
%    \caption{化合物 1 在 $CDCl_3$ 中的 HSQC NMR 谱图~(600 MHz,25℃):~扩展图}
%    \label{fig:compound1_hsqc_expanded}
%\end{figure}
%
%\begin{figure}[h]
%    \centering
%    \includegraphics[width=0.9\textwidth]{figure_S42.pdf} % 需替换为实际图片路径
%    \caption{化合物 1 在 $CDCl_3$ 中的 HMBC NMR 谱图~(600 MHz,25℃)}
%    \label{fig:compound1_hmbc}
%\end{figure}
%
%\begin{figure}[h]
%    \centering
%    \includegraphics[width=0.9\textwidth]{figure_S43.pdf} % 需替换为实际图片路径
%    \caption{化合物 1 在 $CDCl_3$ 中的 HMBC NMR 谱图~(600 MHz,25℃):~扩展图}
%    \label{fig:compound1_hmbc_expanded}
%\end{figure}

\subsubsection{13.2.2 NMR spectra of compound 2}
%\begin{figure}[h]
%    \centering
%    \includegraphics[width=0.9\textwidth]{figure_S44.pdf} % 需替换为实际图片路径
%    \caption{化合物 2 在 25℃ 下于 $CDCl_3$ 中的 $^1H$ NMR 谱图~(400 MHz)。★ 代表 $CDCl_3$ 的溶剂残留峰,+ 代表 $H_2O$ 峰,# 代表内标 $Me_4Si$ 峰,* 代表 $CH_2Cl_2$ 溶剂残留峰。}
%    \label{fig:compound2_1hnmr}
%\end{figure}
%
%\begin{figure}[h]
%    \centering
%    \includegraphics[width=0.9\textwidth]{figure_S45.pdf} % 需替换为实际图片路径
%    \caption{化合物 2 在 25℃ 下于 $CDCl_3$ 中的 $^{13}C$ NMR 谱图~(100 MHz)。★ 代表 $CDCl_3$ 的溶剂残留峰。}
%    \label{fig:compound2_13cnmr}
%\end{figure}
%
%\begin{figure}[h]
%    \centering
%    \includegraphics[width=0.9\textwidth]{figure_S46.pdf} % 需替换为实际图片路径
%    \caption{化合物 2 在 $d_8$-THF 中的 COSY NMR 谱图~(500 MHz,25℃)}
%    \label{fig:compound2_cosy}
%\end{figure}
%
%\begin{figure}[h]
%    \centering
%    \includegraphics[width=0.9\textwidth]{figure_S47.pdf} % 需替换为实际图片路径
%       \caption{化合物 2 在 $d_8$-THF 中的 NOESY NMR 谱图~(500 MHz,25℃)~(IV 为 H8-H9 相关峰,VI 为 H3-H4、H5-H6 相关峰)}
%    \label{fig:compound2_noesy}
%\end{figure}
%
%\begin{figure}[h]
%    \centering
%    \includegraphics[width=0.9\textwidth]{figure_S48.pdf} % 需替换为实际图片路径
%    \caption{化合物 2 在 $d_8$-THF 中的 HSQC NMR 谱图~(500 MHz,25℃)}
%    \label{fig:compound2_hsqc}
%\end{figure}
%
%\begin{figure}[h]
%    \centering
%    \includegraphics[width=0.9\textwidth]{figure_S49.pdf} % 需替换为实际图片路径
%    \caption{化合物 2 在 $d_8$-THF 中的 HMBC NMR 谱图~(500 MHz,25℃)}
%    \label{fig:compound2_hmbc}
%\end{figure}

\subsubsection{13.2.3 NMR spectra of compound 3}
%\begin{figure}[h]
%    \centering
%    \includegraphics[width=0.9\textwidth]{figure_S50.pdf} % 需替换为实际图片路径
%    \caption{化合物 3 在 25℃ 下于 $CDCl_3$ 中的 $^1H$ NMR 谱图~(400 MHz)。★ 代表 $CDCl_3$ 的溶剂残留峰,+ 代表 $H_2O$ 峰。}
%    \label{fig:compound3_1hnmr}
%\end{figure}
%
%\begin{figure}[h]
%    \centering
%    \includegraphics[width=0.9\textwidth]{figure_S51.pdf} % 需替换为实际图片路径
%    \caption{化合物 3 在 25℃ 下于 $CDCl_3$ 中的 $^{13}C$ NMR 谱图~(100 MHz)。★ 代表 $CDCl_3$ 的溶剂残留峰。}
%    \label{fig:compound3_13cnmr}
%\end{figure}
%
%\begin{figure}[h]
%    \centering
%    \includegraphics[width=0.9\textwidth]{figure_S52.pdf} % 需替换为实际图片路径
%    \caption{化合物 3 在 $d_8$-THF 中的 COSY NMR 谱图~(500 MHz,25℃)~(IV 为 H1-H7 相关峰)}
%    \label{fig:compound3_cosy}
%\end{figure}
%
%\begin{figure}[h]
%    \centering
%    \includegraphics[width=0.9\textwidth]{figure_S53.pdf} % 需替换为实际图片路径
%    \caption{化合物 3 在 $d_8$-THF 中的 NOESY NMR 谱图~(500 MHz,25℃)}
%    \label{fig:compound3_noesy}
%\end{figure}
%
%\begin{figure}[h]
%    \centering
%    \includegraphics[width=0.9\textwidth]{figure_S54.pdf} % 需替换为实际图片路径
%    \caption{化合物 3 在 $CDCl_3$ 中的 HSQC NMR 谱图~(600 MHz,25℃)}
%    \label{fig:compound3_hsqc}
%\end{figure}
%
%\begin{figure}[h]
%    \centering
%    \includegraphics[width=0.9\textwidth]{figure_S55.pdf} % 需替换为实际图片路径
%    \caption{化合物 3 在 $CDCl_3$ 中的 HSQC NMR 谱图~(600 MHz,25℃):~扩展图}
%    \label{fig:compound3_hsqc_expanded}
%\end{figure}
%
%\begin{figure}[h]
%    \centering
%    \includegraphics[width=0.9\textwidth]{figure_S56.pdf} % 需替换为实际图片路径
%    \caption{化合物 3 在 $CDCl_3$ 中的 HMBC NMR 谱图~(600 MHz,25℃)}
%    \label{fig:compound3_hmbc}
%\end{figure}
%
%\begin{figure}[h]
%    \centering
%    \includegraphics[width=0.9\textwidth]{figure_S57.pdf} % 需替换为实际图片路径
%    \caption{化合物 3 在 $CDCl_3$ 中的 HMBC NMR 谱图~(600 MHz,25℃):~扩展图}
%    \label{fig:compound3_hmbc_expanded}
%\end{figure}

\section*{14. Mass Spectra}
%\begin{figure}[h]
%    \centering
%    \includegraphics[width=0.9\textwidth]{figure_S58.pdf} % 需替换为实际图片路径
%    \caption{DHR 的 MALDI-TOF 质谱图}
%    \label{fig:dhr_maldi_tof_ms}
%\end{figure}
%
%\begin{figure}[h]
%    \centering
%    \includegraphics[width=0.9\textwidth]{figure_S59.pdf} % 需替换为实际图片路径
%    \caption{化合物 1 的 MALDI-TOF 质谱图}
%    \label{fig:compound1_maldi_tof_ms}
%\end{figure}
%
%\begin{figure}[h]
%    \centering
%    \includegraphics[width=0.9\textwidth]{figure_S60.pdf} % 需替换为实际图片路径
%    \caption{化合物 2 的 MALDI-TOF 质谱图}
%    \label{fig:compound2_maldi_tof_ms}
%\end{figure}
%
%\begin{figure}[h]
%    \centering
%    \includegraphics[width=0.9\textwidth]{figure_S61.pdf} % 需替换为实际图片路径
%    \caption{化合物 3 的 MALDI-TOF 质谱图}
%    \label{fig:compound3_maldi_tof_ms}
%\end{figure}
