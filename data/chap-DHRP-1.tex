\chapter{\rm{DHR}的合成与表征}

% DHR的合成
%DHR的合成以化合物1为起始原料,化合物1通过四氯化钛(TiCl₄)促进二苯并环庚酮二聚的改进方法制备(见支持信息中的反应式S1)。化合物1经两步反应转化为DHR(方案1):在0℃下,化合物1与12当量的三氯化铁(FeCl₃)在二氯甲烷(DCM)/硝基甲烷(CH₃NO₂)混合溶剂中发生肖尔反应,同时生成两个五边形,得到化合物3,产率为87%。通过优化后处理步骤(采用萃取/沉淀法替代柱层析),化合物3的合成可实现克级规模,且产率提升至98%。
%
%化合物3与2,3-二氯-5,6-二氰基-1,4-苯醌(DDQ)在二氧六环中发生脱氢反应,生成DHR中的两个七边形。优化反应条件后,使用6.0当量的DDQ时,DHR的产率为87%。此外,本研究还开发了该步骤的无柱后处理方法:通过物理气相传输(PVT)纯化法,可获得克级规模的纯DHR,产率为60%。
%
%![方案1. 二环庚[ijkl,uvwx]红荧烯(DHR)的合成路线](注:原文含合成方案图,此处为方案标题翻译)
%作为对比,我们还尝试了“先脱氢形成两个七边形、后肖尔反应形成两个五边形”的DHR合成路线(方案1):在2.6当量DDQ存在下,化合物1脱氢生成化合物2的反应效率极高,15分钟内即可完成,产率为80%;但在FeCl₃存在下,化合物2经肖尔反应生成DHR的尝试在多种条件下均未成功。
%
%
% DHR的结构表征
% 单晶结构分析
%通过物理气相传输法(280℃,20 Pa)缓慢升华,获得了适用于单晶结构分析的DHR单晶。DHR分子在晶格中存在无序性:两个构象体(i和ii)占据每个晶格位点,占有率分别为46%和54%(见图2a)。构象体i和ii的键长与键角数据见表S1-S4(支持信息)。在构象体i的C-C键中,C2-C3和C9-C10键长最长,达1.487 Å;而C8-C9、C12-C13和C14-C2键长最短,低至1.33 Å;构象体ii表现出类似的键长趋势。与母体薁的C-C键长(1.387-1.427 Å[35])相比,DHR中两个薁单元的C-C键长(构象体i:1.333-1.487 Å;构象体ii:1.354-1.458 Å)离散度更大。
%
%![图2. DHR在晶体中的分子结构ORTEP图(概率水平30%)](注:原文含图,此处为图注翻译)
%a)两个构象体i和ii;b)带原子编号的构象体i;c)构象体i的堆积方式(显示π-π距离及短C∙∙∙H接触)
%
%如图2b和2c所示,DHR分子呈近完全平面构型,具有C₂h对称性。作为对比,我们还获得了前驱体1、2、3的晶体结构,发现它们的中心苯环B与末端苯环E分别形成-54.97°、-47.23°和9.75°的二面角(见图S5,支持信息)。由此推测,“形式薁单元”中五边形与七边形的同时存在,是DHR实现平面结构的关键因素。
%
%如图2c及图S9(支持信息)所示,DHR分子以“人字形”(herringbone)方式堆积。由于晶格中分子的无序性,分子间π-π堆积距离(3.438-3.465 Å)和C∙∙∙H原子间接触距离(2.684-3.194 Å)存在轻微差异。这种人字形分子间排列理论上有利于电荷传输,但如后文所述,分子无序性会导致电荷传输性能下降。
%
%
% 稳定性分析
%通过¹H NMR、¹³C NMR、高分辨质谱(HRMS)及元素分析,确认了DHR的化学结构。热重分析(TGA)数据显示,DHR具有优异的热稳定性:温度升至400℃时,重量损失仍低于5%(见图S1,支持信息)。DHR在室温空气中存放超过6个月,结构无变化;即使在空气中200℃加热1小时,其¹H NMR和质谱谱图仍保持不变(见图S3-S4,支持信息)。因此,DHR的稳定性与Mastalerz团队报道的含两个嵌入式薁单元的多环芳烃[33]相近,而不同于Müllen和Feng近期报道的含两个五边形和两个七边形的纳米石墨烯(空气稳定性差[32])。
%
%
% 芳香性与电子离域性计算
%计算了DHR中不同环的Z轴1Å处各向同性化学屏蔽表面(ICSS(1)zz)及核独立化学位移(NICS(1))值。需注意,ICSS(1)zz值越正,表明化学屏蔽作用越强,芳香性越高。如图3a所示,化学屏蔽强度顺序为:环B > 环E/H > 环A/C > 环F/I > 环D/G。此外,NICS(1)计算结果为:环B(-25.7)、环E/H(-22.4)、环A/C(-18.8)、环F/I(-4.3)、环D/G(12.6)(见图S10,支持信息),表明六元环具有芳香性,五元环呈弱芳香性,七元环呈弱反芳香性。这与¹H NMR谱中“六元环上质子的化学位移比七元环上质子更显著低场位移”的现象一致(见图S27,支持信息)。
%
%π电子定域化轨道指示函数(LOL-π)计算显示,E/F/B/I/H环内的电子离域性更强(图3b);相比之下,D环中C8-C9键和A环中C2-C14键的电子主要呈定域状态,这与这些环中键长的交替性特征一致(见表S1-S4,支持信息)。
%
%![图3. DHR的ICSS(1)zz图(a)与LOL-π图(b)](注:原文含图,此处为图注翻译)
%a)Z轴1Å处各向同性化学屏蔽表面(ICSS(1)zz):深橙色区域表示芳香性强;b)π电子定域化轨道指示函数(LOL-π):橙色表示键内π电子呈定域状态
%
%
% DHR的光物理与电化学性质
% 紫外-可见吸收与荧光发射
%图4a显示了DHR在四氢呋喃(THF)中的紫外-可见吸收光谱。DHR在紫外区(305-345 nm)和可见光区(610-666 nm)均有吸收。密度泛函理论(DFT)计算表明,610-666 nm的吸收源于S₀→S₁跃迁,而紫外区的吸收源于S₀→S₂跃迁(图4a)。DHR溶液在670 nm和400 nm处有发射峰(见图S15,支持信息),结合DFT计算可将其分别归为S₁→S₀跃迁和反常反卡莎规则S₂→S₀跃迁。有趣的是,轨道分析显示,DHR中的“形式薁单元”主要对S₂→S₀跃迁有贡献[36]。
%
%
% 循环伏安法与差分脉冲伏安法
%%在二氯甲烷(DCM)与邻二氯苯(o-DCB)的1:1混合溶剂中,对DHR进行了循环伏安法(CV)和差分脉冲伏安法(DPV)测试。如图4b所示,DHR表现出四个氧化峰,对应电位分别为Eₚ^ox1=0.22 V、Eₚ^ox2=0.52 V、Eₚ^ox3=0.96 V、Eₚ^ox4=1.06 V(相对于Fc/Fc⁺),以及两个还原峰,对应电位为Eₚ^red1=-1.74 V、Eₚ^red2=-2.19 V(相对于Fc/Fc⁺);其中第三和第四个氧化峰间距较近,推测可能是由于溶液中DHR形成分子聚集体所致。这一假设得到了以下实验的支持:当DHR浓度降至8.9×10⁻⁴ mol/L,并在60℃下测试时,第三和第四个氧化峰合并为一个峰(见图S16,支持信息)。
%
%基于起始氧化电位(0.18 V vs Fc/Fc⁺)和起始还原电位(-1.64 V vs Fc/Fc⁺),估算出DHR的最高占据分子轨道(HOMO)能量为-4.98 eV,最低未占据分子轨道(LUMO)能量为-3.16 eV,进而计算出其带隙为1.82 eV。
%
%
% 化学氧化行为
%研究了DHR经四氟硼酸硝鎓(NOBF₄)化学氧化后的吸收光谱变化。氧化产物在THF等常规有机溶剂中的溶解度极低,因此选用三氟乙酸(TFA)作为溶剂进行测试。如图S17(支持信息)所示,加入1当量NOBF₄后,体系在900-1400 nm处出现新的吸收峰;电子顺磁共振(ESR)测试显示,经NOBF₄处理后的DHR溶液出现强ESR信号(见图S18,支持信息),表明900-1400 nm的新吸收峰可归为DHR自由基阳离子的形成。当NOBF₄用量增加至6当量时,溶液的ESR信号消失,同时在950 nm处出现新的吸收峰,这可能是由于形成了DHR二价阳离子。
%
%
% DHR的半导体性能
%通过物理气相传输(PVT)法生长DHR晶体,研究其固态半导体性能。将晶体置于十八烷基三氯硅烷(OTS)修饰的Si/SiO₂衬底上,制备底栅顶接触(BGTC)结构的场效应晶体管(FET)器件:以晶体为传输沟道,采用掩模板遮挡,随后沉积15 nm厚的三氧化钼(MoO₃)作为修饰层,并沉积30 nm厚的金(Au)作为顶源极和漏极。在空气中测试器件的转移曲线和输出曲线(见图S20,支持信息),结果表明DHR晶体表现出典型的p型半导体行为。
%
%基于20个器件的转移曲线,提取出DHR的空穴迁移率:最高空穴迁移率可达0.082 cm²·V⁻¹·s⁻¹,开关比为5.45×10⁴;平均空穴迁移率为0.049 cm²·V⁻¹·s⁻¹。
%
%通过优化DHR的中性基态和离子态几何结构,计算出其空穴传输的重组能为251 meV(见支持信息)。较小的重组能与DHR的刚性结构(所有共轭环均稠合)相符。进一步计算了电荷转移积分,结果见表S6(支持信息):分子间π-π相互作用和C∙∙∙H相互作用相关的转移积分受分子无序性影响——构象体i或ii的同型分子对之间的π-π相互作用转移积分,明显大于构象体i与ii的异型分子对。这一计算结果表明,分子无序性不利于高效电荷传输,与DHR晶体表现出较低电荷迁移率的实验事实一致。
%
%
% 结论
%本文报道了一种含两个五边形和两个七边形的非苯型多环芳烃(DHR),通过三步反应可实现克级规模合成。两个“形式薁单元”的引入不仅影响了共轭结构,还使DHR具有平面性、热稳定性和空气稳定性。单晶结构分析与计算研究表明,E/F/B/I/H环内电子离域性更强,而七元环内电子主要呈定域状态。有趣的是,DHR表现出多氧化还原特性和反常的反卡莎规则S₂→S₀发射,这可能与其高度稠合的5/6/7元环体系的独特共轭模式有关。此外,DHR晶体表现为p型半导体,空穴迁移率最高可达0.082 cm²·V⁻¹·s⁻¹。因此,含两个“形式薁单元”的DHR作为新型非苯型共轭骨架,是开发光电子材料的重要模块。目前,关于DHR的进一步功能化及更大扩展共轭体系的构建工作正在进行中。
%
%
% 实验部分
%合成与表征细节、NMR谱图、吸收光谱、计算数据及OFET器件制备方法,均详见支持信息。DHR(CCDC 1971948)、化合物1(CCDC 1953759)、化合物2(CCDC 1953602)及化合物3(CCDC 1953606)的晶体学数据已存入剑桥晶体学数据中心(CCDC),可通过网址www.ccdc.cam.ac.uk/structures免费获取。
