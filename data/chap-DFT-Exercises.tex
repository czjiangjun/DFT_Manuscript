\newpage
\noindent{\heiti 第十二章~课后习题}

{\heiti 1.~简单说明密度泛函理论、\textrm{Kohn-Sham}方程以及交换-相关势与交换-相关泛函的关系.}\\

{\heiti 回答要点:}\\
(1)密度泛函理论的核心:~体系的基态性质完全可以由体系的基态密度确定,体系的基态能量泛函在体系基态密度取变分极小值\\
(2)密度泛函理论将体系的基本变量由波函数转换为密体,描述问题的变量由$3N$维将为$3$维,大大简化了求解难度\\
(3)体系基态能量的密度泛函的精确形式未知,\textrm{Kohn-Sham}引入无相互作用粒子体系,将真实体系与无相互作用体系的差别放入交换-相关泛函中\\
(4)\textrm{Kohn-Sham}方程形式上是单粒子函数的运动方程,它把波函数重新引入作为获得体系基态密度的重要工具(中间桥梁)\\
(5)\textrm{Kohn-Sham}方程中的交换-相关势源自交换-相关能泛函:~$V_{\mathrm{XC}}=\dfrac{\delta E_{\mathrm{XC}}[\rho]}{\delta\rho(\vec r)}$,交换-相关势出现在\textrm{Kohn-Sham}方程中,交换-相关能则用于体系基态总能量的计算.

{\heiti 2.~简单说明交换-相关泛函的分类方式.}\\

{\heiti 回答要点:}\\
(1)局域(自旋)密度近似\textrm{(L(S)DA)}:~最简单的交换-相关泛函形式,泛函仅与电荷密度$\rho(\vec r)$的空间分布有关(如果是\textrm{LSDA}要考虑电子自旋的影响)\\
(2)广义梯度近似\textrm{(GGA):~}交换-相关泛函与电荷密度$\rho(\vec r)$的空间分布及其空间梯度$\nabla\rho(\vec r)$有关,当电荷密度变化剧烈时,\textrm{GGA}比\textrm{L(S)DA接近真实情况}\\
(3)动态广义梯度近似\textrm{(meta-GGA)}:~在\textrm{GGA}基础上考虑密度空间分布的两阶梯度$\nabla^2\rho(\vec r)$对泛函的贡献\\
(4)杂化泛函\textrm{(hybrid-XC)}:~针对密度泛函理论将连续变量密度代替离散变量粒子引入的误差,引入部分\textrm{Hartree-Fock}方法中精确交换作用的贡献,与经典交换泛函杂糅(通过系数调节),得到更高精度的泛函。杂化泛函的计算量一般比较巨大,目前主要大量用于化学分子、化学反应等高精度计算中,周期体系计算中应用较少.

{\heiti 3.~简单说明赝势的构造原则,可分离赝势引入的原因及应用意义.}\\

{\heiti 回答要点:}\\
(1)赝势的构造原则:~兼顾``柔软性''(展开的平面波数目尽可能少)和``可移植性''(使用的范围尽可能的广泛)。这两个原则是彼此矛盾的,实际构造赝势的时候,需要根据需求有针对性地进行优化
(2)第一原理势的构造通过设计赝波函数,逆向求解原子径向方程,获得局域的赝势(仅限于径向部分)\\
(3)考虑角度部分贡献后的赝势只是半局域\textrm{(semi-local)}的,与角动量量子数$l$和磁量子数$m$有关,这样的赝势在计算中应用非常不方便\\
(4)将赝波函数中包含角度部分并引入近似实现角动量量子数$l$和磁量子数$m$的变量分离,可以获得完全局域的赝势,非局域部分贡献则通过近似后的赝波函数构造投影函数得到\\
(5)通过投影函数实现赝势的局域-非局域分离,可以在计算软件中实现赝势的快速构造,大大提升计算效率。但也因此引入计算误差的产生,可以通过扩大变分空间克服.

{\heiti 4.~简要说明\textrm{PAW}方法中的``全电子''计算与\textrm{LAPW}方法中的``全势''计算的联系与区别.}\\

{\heiti 回答要点:}\\
(1)\textrm{PAW}方法中的``全电子''计算强调计算过程中保留了电子波函数在靠近原子核附近的振荡行为,突出对赝势方法的赝波函数部分的改进\\
(2)\textrm{LAPW}方法中的``全势''突出电子在整个空间感受到的势与真实情况相近,突出的是对赝势方法中赝势部分的改进\\
(3)势函数决定电子波函数的行为,从计算精度上说,全势方法的精确程度会更高一点,但计算量也更大\\
(4)两种方法都是在赝势-平面波方法基础上发展起来的,但由于采用了不同的实现路线,所以选择了两个不同的词本质上对同一个问题不同侧面予以强调,核心目标都是为了提高计算电子体系的精度.

{\heiti 5.~简单说明是否考虑电子相关对计算结果可能产生的影响.}\\

{\heiti 回答要点:}\\
(1)由于用联系的粒子密度代替不连续的粒子,密度泛函方法先天就存在对粒子相关问题考虑的不足
(2)对于含有$d$、$f$电子等局域电子体系,电子相关问题对计算影响更显著,甚至会定性错误,所以对这些体系应用密度泛函方法计算一定要注意
(3)电子相关问题可以通过引入多粒子相关函数予以克服,最常用的包括\textrm{Green~Function}方法,化学家也有应用\textrm{CI(Configuration Interaction)}方法,但计算量增加的非常可观
(4)电子相关主要影响的是体系的某些物理性质,如果只和体系总能有关的物理性质(如基态体积、体弹模量等),受电子相关的影响很小.
