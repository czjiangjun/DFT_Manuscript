\section{线性缀加平面波(Linear Augmented Plane Wave, LAPW)方法}
\subsection{APW与LAPW}
APW是Slater提出MT球近似时设计的周期体系的波函数展开方法\cite{PR51-846_1937,PR91-528_1953},如图\ref{Muffin_tin_0}所示。%后来Slater又对APW方法进行“简化”\cite{}%,但实际上原始的APW思想更直接。
\begin{figure}[h!]
\centering
\includegraphics[height=1.30in,width=1.45in,viewport=1 20 485 435,clip]{APW.png}
\caption{\small \textrm{Partitioning of the unit cell into atomic spheres and an interstitial region.}}%(与文献\cite{EPJB33-47_2003}图1对比)
\label{Muffin_tin_0}
\end{figure}
根据MT近似,在WS原胞的每个MT球内,势能具有球对称性$V(r)$,MT球内的基函数表示为:
\begin{equation}
  \varphi(\vec k_i,\vec r)=\sum_{l=0}^{\infty}\sum_{m=-l}^lA_{lm}(\vec k_i)u_l(|\vec r-\vec r_s|)Y_{lm}(\widehat{\vec r-\vec r_s})
  \label{eq:solid-109}
\end{equation}
这里$Y_{lm}$是球谐函数;$\vec k_i=\vec k+\vec G_i$;$\vec G_i$是倒格矢;$\vec r_s$是第$s$个MT球心位置,$A_{lm}$是待定系数;$u_l(\vec r)$是径向Schr\"odinger方程
\begin{equation}
  -\frac1{r^2}\frac d{dr}\left(r^2\frac{du_l}{dr}\right)+\left[\frac{l(l+1)}{r^2}+V(r)\right]u_l=E_l^{\prime}u_l
  \label{eq:solid-110}
\end{equation}
为确定$u_l(r)$,要求$u_l(r)$的边条件$r$=0是非奇异的,但在MT球面$r$=$R_{MT}^s$上,没有对$u_l(r)$加任何限制条件,因此能量式\eqref{eq:solid-110}中的$E_l^{\prime}$可以取任意值。

在MT球外的间隙区,MT势能取为0,间隙区的基函数为:
\begin{equation}
	\varphi(\vec k_i,\vec r)=\exp\mathrm{i}\vec k_i\vec r=\exp\mathrm{i}\vec k_i\cdot(\vec r_s+\vec r)
  \label{eq:solid-111}
\end{equation}
将平面波用球谐函数展开,
\begin{equation}
	\exp\mathrm{i}\vec k_i\cdot\vec r=4\pi\sum_{l=0}^{\infty}\sum_{m=-l}^l\mathrm{i}^lj_l(|\vec k_i|r)Y_{lm}^{\ast}(\hat{\vec k}_i)Y_{lm}(\hat{\vec r})
  \label{eq:solid-112}
\end{equation}
这里$j_l(|\vec k_i|r)$是第$l$阶球Bessel函数;$\hat{\vec k}_i$和$\hat{\vec r}$是矢量$\vec k$和$\vec r$与$z$轴的夹角对应球坐标角度部分$\theta$和$\phi$。

为了使得基函数在球面上连续条件,要求式\eqref{eq:solid-109}和\eqref{eq:solid-111}在球面上数值相等,由此可确定系数$A_{lm}(\vec k_i)$:
$$A_{lm}(\vec k_i)=4\pi e^{\mathrm{i}\vec k_i\cdot\vec r_s}\mathrm{i}^lY_{lm}(\hat{\vec k}_i)j_l(|\vec k_i|R_{MS}^s)/u_l(E,R_{MT}^s)$$
所以APW的基组可以表示为:
\begin{equation}
  \begin{split}
    \varphi&(\vec k_i,\vec r)\\
    &=\left\{\begin{aligned}
	    &e^{\mathrm{i}\vec k_i\cdot\vec r_s}e^{\mathrm{i}\vec k_i\cdot\vec r},&|\vec r|>R_{MT}^s\\
    &4\pi e^{\mathrm{i}\vec k_i\cdot\vec r_s}\sum_{lm}\mathrm{i}^lj_l(|\vec k_i|R_{MS}^s)Y_{lm}^{\ast}(\hat{\vec k}_i)Y_{lm}(\hat{\vec r})u_l(r,E^{\prime})/u_l(R_{MT}^s,E^{\prime}),\quad&|\vec r|\leqslant R_{MT}^s
    \end{aligned} \right.
  \end{split}
  \label{eq:solid-113}
\end{equation}

对于基函数在MT球面不连续的情形,用Schlosser和Marcus的能量变分表达式\cite{PR131-2529_1963},
\begin{equation}
  \begin{split}
    E\int_{\mathrm{I+II}}\Psi^{\ast}\Psi dV=&\int_{\mathrm{I+II}}\Psi^{\ast}\mathbf H\Psi dV+\frac12\int_S\left[(\Psi_{\mathrm{II}}-\Psi_{\mathrm I})\frac{\partial}{\partial\rho}\Psi_{\mathrm {II}}^{\ast}+\frac{\partial}{\partial\rho}\Psi_{\mathrm I}^{\ast}\right.\\
    &-(\Psi_{\mathrm {II}}^{\ast}+\Psi_{\mathrm I}^{\ast})\left.\left(\frac{\partial}{\partial\rho}\Psi_{\mathrm{II}}-\frac{\partial}{\partial\rho}\Psi_{\mathrm I}\right)\right]dS
  \end{split}
  \label{eq:solid-114}
\end{equation}
这里I和II分别表示MT球的内(I)外(II)两部分。上式中的前两个积分项表示对整个WS原胞的积分,第三个积分项是对MT球的球面S的积分。$\partial/\partial\rho$沿球面正方向对区域I的偏导。采用连续基函数,则\eqref{eq:solid-114}可以简化为:
\begin{equation}
  \begin{split}
    E\int_{\mathrm{I+II}}\Psi^{\ast}\Psi dV=&\int_{\mathrm{I+II}}\Psi^{\ast}\mathbf H\Psi dV\\
    &-\frac12\int_S(\Psi_{\mathrm {II}}^{\ast}+\Psi_{\mathrm I}^{\ast})\left(\frac{\partial}{\partial\rho}\Psi_{\mathrm{II}}-\frac{\partial}{\partial\rho}\Psi_{\mathrm I}\right)dS
  \end{split}
  \label{eq:solid-115}
\end{equation}
这里的球面积分是考虑波函数$\Psi$在MT球面上导数不连续。

因为APW基函数径向部分与能量参数$E_l$有关,因此矩阵元与$E_l$有关。为了获得能量本征值$E_l$,必须求解$\vec k$-空间中每个点的能量$E_l$高阶行列式,该过程非常麻烦。为了克服基函数对能量$E_l$的相关,人们提出了多种线性化思想\cite{PRB2-3098_1970,PRB2-290_1970,JPF9-661_1979,PRB19-6094_1979}。线性化缀加平面波(linearized augmented plane-wave, LAPW)方法的思想是在某个给定值$E_{l0}$附近对能量作Taylor展开到一阶\cite{PRB12-3060_1975}。由于波函数的线性化引入的误差,比因为对势能近似引入的误差小得多。换句话说,通过引入线性化,可使得固体中价电子径向波函数与能量参数无关。由此得到Hamiltonian和重叠矩阵与能量参数无关。Koelling和Arbman将LAPW方法应用于Cu的计算\cite{JPF5-2041_1975}。Smr\v cka提出了二次APW方法\cite{Smrcka}。
与APW方法相似,LAPW方法的基函数取为:
\begin{equation}
  \varphi(\vec k_j,\vec r)=\left\{
  \begin{aligned}
	  &\Omega_0^{-1/2}\exp{\mathrm{i}\vec k_j\cdot\vec r},&r>R_{MT}^s\\
    &\sum_{lm}[A_{lm}u_l(E_l,r)+B_{lm}\dot u_l(E_l,r)]Y_{lm}(\hat{\vec r}),\quad&r\leqslant R_{MT}^s
  \end{aligned}\right.
  \label{eq:LAPW-basis}
\end{equation}
这里$R_{MT}^s$是MT球半径;$\vec k_j=\vec k+\vec G_j$,$\vec k$是不可约Brillouin区中的波矢;$\vec G_j$是倒空间格矢;$u_l(E_l,r)$是满足能量为$E_l$的径向Schr\"odinger方程的解波函数;$\dot u_l(E_l,r)$是解波函数对能量$E_l$的一阶导数;$Y_{lm}$是球谐函数;$\Omega_0$是WS原胞体积。与APW不同,这里$E_l$是指定范围内的能量参数,而非变量。不同的角动量$l$可指定不同的能量。关于APW/LAPW方法基函数(径向部分)在球面上的衔接如图\ref{Muffin_tin}所示。
\begin{figure}[h!]
\centering
\includegraphics[height=1.30in,width=1.98in,viewport=1 20 585 435,clip]{WIEN2k-LAPW.png}
\caption{\small \textrm{Partitioning of the unit cell into atomic spheres(I) and an interstitial region(II)}}%(与文献\cite{EPJB33-47_2003}图1对比)
\label{Muffin_tin}
\end{figure}

为了提高LAPW方法的线性化程度(即提高基组的变分自由度),在同一能量范围处理半芯态(接近价态的能量较高的芯态)和价态,可采用外加基函数(与$\vec k$无关)方案,这部分外加基函数称为局域轨道(local orbitals, LO)\cite{PRB43-6388_1991,Singh}。由此构造的基函数包含两个指定能量的径向波函数和其中一个能量导数,这样的基函数即LAPW-LO:
\begin{equation}
  \phi_{lm}^{LO}(\vec r)=[A_{lm}u_l(E_{1,l},r)+B_{lm}\dot u_l(E_{1,l},r)+C_{lm}u_l(E_{2,l},r)]Y_{lm}(\hat{\vec r})
  \label{eq:LAPW-LO}
\end{equation}
根据条件$\phi_{lm}^{LO}$在MT球面上数值为零且保持一阶导数连续,并要求$\phi_{lm}^{LO}$在MT球内归一化,可以确定系数$A_{lm}$,$B_{lm}$,$C_{lm}$。

Sj\"ostedt等\cite{SSC114-15_2000}的计算表明,在标准LAPW方法中,将平面波展开使之在MT球面上与球内函数数值和一阶导数连续,这并非是实现Slater型APW方法最有效的线性化方法。采用指定能量参数$E_l$的APW形式的径向波函数[式\eqref{eq:APW-basis}],外加APW型局域轨道(local orbit, lo)扩展基函数,是更有效的方案,称为APW+lo。
\begin{equation}
  \varphi_{\vec k_i,\vec r}=\sum_{lm}[A_{lm}(\vec k_i)u_l(E_l,r)]Y_{lm}(\hat{\vec r})
  \label{eq:APW-basis}
\end{equation}
\begin{equation}
  \phi_{lm}^{lo}=[A_{lm}u_l(E_{1,l})+B_{lm}\dot u_l(E_{1,l})]Y_{lm}(\hat{\vec r})
  \label{eq:APW-lo}
\end{equation}
APW+lo基函数式\eqref{eq:APW-lo}形式上与标准LAPW式\eqref{eq:LAPW-basis}形式非常相似,但这里系数$A_{lm}$和$B_{lm}$与$\vec k$无关,是根据式\eqref{eq:APW-lo}在MT球边界上数值为零并且在MT球内归一化确定。这样构造的APW+lo基函数,总的波函数在MT球面上平滑且一阶可导的,但是根据式\eqref{eq:solid-115},在MT球面上有动能部分对Hamiltonian的贡献。为收敛到相同的结果,采用APW+lo基组比起标准LAPW基组小得多\cite{PRB64-195134_2001}。因此选择APW+lo基组比标准LAPW基组的计算效率要高。

%WIEN2K程序包建议\cite{CPC59-399_1990,WIEN2K-UG_2001,CPC147-71_2002}一般平面波基函数收敛缓慢的轨道(比如过渡金属的3d态波函数)或MT球半径特别小的体系用APW+lo基组展开,其余价电子轨道用LAPW基组展开,此外对必要的半芯层可以用LO基组展开。用这样的方式可以同时考虑价态和半芯态。

\subsection{FP-LAPW的总能计算}
根据\textrm{APW}方法的基本思想,原子的芯电子的运动限于\textrm{MT}球内,价电子的运动范围遍及\textrm{MT}球区和间隙区。材料的物理和化学性质主要取决于体系的价电子状态,\textrm{WIEN2k}程序处理价电子时,对应的基函数在空间的不同位置采用不同的表示。

所谓全势(\textrm{Full-potential})方法的思想,简言之,就是无论价电子在\textrm{MT}球区,还是在间隙区,感受到的都是包括芯层电子在内的其他电子的排斥势和各格点原子核的吸引势。原则上,对于周期性体系,已知体系的电荷密度分布,由\textrm{Poisson}方程,可以直接求得体系的\textrm{Coulomb}势。众所周知,由于靠近原子核的区域,电荷密度(及相应的\textrm{Coulomb}势)变化剧烈,因此直接的\textrm{Fourier}级数展开不易收敛。为了克服这一困难,\textrm{Weinert}\cite{JMP22-2433_1981}注意到间隙区的电子\textrm{Coulomb}相互作用只和间隙区的电荷密度及\textrm{MT}球区的电荷多极矩(\textrm{multipole moments})有关。它的基本思想是利用电荷多极矩和球形边界\textrm{Dirichlet}边值问题求解\textrm{MT}球内的\textrm{Coulomb}势分布,进而得到间隙区\textrm{Coulomb}势的分布。

注意到\textrm{MT}球内电荷对间隙区势能的贡献\cite{Jackson}:
\begin{equation}
	V(r)=\sum_{l=0}^{\infty}\sum_{m=-l}^{l}\dfrac{4\pi}{2l+1}q_{lm}\dfrac{Y_{lm}(\hat r)}{r^{l+1}}
	\label{eq_V-multi-mom}
\end{equation}
这里,$q_{lm}$是\textrm{MT}球内的电荷多极矩。
%\begin{equation}
%	\rho_I(\vec r)=\sum_{\vec K}\rho_I(\vec K)\mathrm{e}^{\mathrm{i}\vec K\cdot\vec r}
%	\label{eq_rho_i_four}
%\end{equation}

在实际计算中,\textrm{MT}区球内的电荷多极矩$q_{lm}$由下式给出:
\begin{equation}
	q_{lm}=\int_{S}Y_{lm}^{\ast}(\hat r)r^l\rho(r)\mathrm{d}^3r
	\label{eq_multi-mom}
\end{equation}
因此有可能重构\textrm{MT}球区的电荷密度分布,使它既有与实际电荷密度分布相同的电荷多极矩,又能有较好的\textrm{Fourier}收敛行为。
\begin{equation}
	\rho(\vec r)\rightarrow\tilde\rho(\vec r)=\rho_I(\vec r)\theta(\vec r\in I)+\sum_{spheres}\tilde\rho_i(\vec r)\theta(\vec r\in S_i)
	\label{eq_pseudo-rho}
\end{equation}
间隙区电荷密度$\rho_I(\vec r)$变化平缓,可用方便地作\textrm{Fourier}级数展开:
\begin{equation}
	\rho_I(\vec r)=\sum_{\vec K}\rho_I(\vec K)\mathrm{e}^{\mathrm{i}\vec K\cdot\vec r}
	\label{eq_rho_i_four}
\end{equation}

如果将间隙区的电荷密度延伸到\textrm{MT}球内,于是体系的电荷密度表示为:
\begin{equation}
	\rho(\vec r)=\rho_I(\vec r)+\sum_{spheres}\big[\rho_i(\vec r)-\rho_I(\vec r)\big]\theta(\vec r\in S_i) 
	\label{eq_rho_rean}
\end{equation}
此时$\rho_I(\vec r)$可在整个空间作\textrm{Fourier}展开。其中延伸到\textrm{MT}球内(球心位置记作$\xi_i$)的间隙区电荷密度表示为:
\begin{equation}
	\rho_I(\vec r)=\sum_{\vec K}\rho_I(\vec K)\mathrm{e}^{\mathrm{i}\vec K\cdot\vec\xi_i}\mathrm{e}^{\mathrm{i}\vec K\cdot\vec r_i} 
	\label{eq_rho_i_four_i}
\end{equation}
这里$\vec r_i=\vec r-\xi_i$。

如果已知\textrm{MT}球内的真实电荷密度分布,第$i$个\textrm{MT}球内多极矩记作$q_{lm}^i$。
%\begin{equation}
%	q_{lm}^{Ii}=\biggl\{
%	\begin{aligned}
%		&\dfrac{\sqrt{4\pi}}3R_i^3\rho_I(K=0)\delta_{l0}\quadd \\
%		&4\pi\mathrm{i}^l\sum_{\vec K\neq 0}\rho_I(\vec K)\dfrac{R_i^{l+3}j_{l+1}(\vec K\cdot\vec R)}{KR_i}\mathrm{e}^{\mathrm{i}\vec K\cdot\xi_i}Y_{lm}^{\ast}(\hat K)
%	\end{aligned}\biggr.
%	\label{eq_multi_PW}
%\end{equation}
延伸到球内的间隙区电荷对球内多极矩的贡献即为:
\begin{equation}
	q_{lm}^{Ii}=\dfrac{\sqrt{4\pi}}3R_i^3\rho_I(K=0)\delta_{l0}+4\pi\mathrm{i}^l\sum_{\vec K\neq 0}\rho_I(\vec K)\dfrac{R_i^{l+3}j_{l+1}(\vec K\cdot\vec R)}{KR_i}\mathrm{e}^{\mathrm{i}\vec K\cdot\xi_i}Y_{lm}^{\ast}(\hat K)
	\label{eq_multi_PW}
\end{equation}
这里利用了平面波在球心$\xi_i$处的球谐函数展开关系:
\begin{equation}
	\mathrm{e}^{\mathrm{i}\vec K\cdot \vec r}=\sum_{lm}4\pi\mathrm{i}^lj_l(\vec K\cdot\vec r)Y_{lm}^{\ast}(\hat{\vec K})Y_{lm}(\hat{\vec r})
	\label{eq_PW_Bessel}
\end{equation}
和积分表达式:
\begin{equation}
	\int_0^R r^{l+2}j_l(\vec K\cdot\vec r)\mathrm{d}r=\left\{
	\begin{aligned}
		&\dfrac{R^3}3\delta_{l,0}\quad &\vec K=0\\ 
		&\dfrac{R^{l+3}j_{l+1}(\vec K\cdot\vec R)}{KR} &\vec K\neq 0
	\end{aligned}\right.
	\label{eq_int_r_j}
\end{equation}

因此,重构方程\eqref{eq_rho_rean}在\textrm{MT}球内的电荷密度分布,其多极矩满足:
\begin{equation}
	\tilde q_{lm}^i=q_{lm}^i-q_{lm}^{Ii}
	\label{eq_psmult}
\end{equation}
根据文献\onlinecite{JMP22-2433_1981},可以得到重构的\textrm{MT}球内的赝电荷密度的\textrm{Fourier}展开系数为:
\begin{equation}
	\tilde\rho_s(\vec K)=\dfrac{4\pi}{\Omega}\sum_{lm,i}\dfrac{(-\mathrm{i})^l(2l+2n+3)!!}{(2l+1)!!}\times\dfrac{j_{l+n+1}(\vec K\cdot\vec R_i)}{(KR_i)^{n+1}}\tilde q_{lm}^i\mathrm{e}^{-\mathrm{i}\vec K\cdot\vec\xi_i}Y_{lm}(\hat K)
	\label{eq_psrho}
\end{equation}
这里$n$是整数。$\tilde\rho_s(\vec K)$可以视为$n$的函数,文献\onlinecite{JMP22-2433_1981}讨论了对于不同的$l$和$(\vec K\cdot\vec R_i)_{max}$,优化$n$值得到最佳收敛的$\tilde\rho_s(\vec K)$的情况。一般取$n=\big[\dfrac12(\vec K\cdot\vec R_i)_{max}\big]$,\textrm{WIEN2k}程序中取$n=9$。

当$\vec K=0$,有
$$\tilde\rho_s(\vec K=0)=\dfrac{\sqrt{4\pi}}{\Omega}\sum\limits_i\tilde q_{00}^i$$
因此,用于计算间隙区\textrm{Coulomb}势的赝电荷密度(较平缓)可写成\textrm{Fourier}展开形式:
\begin{equation}
	\tilde\rho(\vec r)=\sum_{\vec K}\big[\rho_I(\vec K)+\tilde\rho_s(\vec K)\big]\mathrm{e}^{\mathrm{i}\vec K\cdot\vec r}
	\label{eq_psrho_four}
\end{equation}
求解\textrm{Poisson}方程,间隙区的\textrm{Coulomb}势表达式为:
\begin{equation}
	V_I(\vec r)=\sum_{\vec K}\dfrac{4\pi}{K^2}\big[\rho_I(\vec K)+\tilde\rho_s(\vec K)\big]\mathrm{e}^{\mathrm{i}\vec K\cdot\vec r}
	\label{eq_Vcou_I}
\end{equation}
与此同时,根据文献\onlinecite{JMP22-2433_1981},已知球面上的\textrm{Coulomb}势,根据\textrm{Dirichlet}的球形边值问题\cite{Jackson},可知\textrm{MT}球内的\textrm{Coulomb}势的表达式\cite{Singh,Nemoshkalenko-Antonov}为:
\begin{equation}
	\begin{aligned}
		V_s(\vec r)=&\sum_{lm}Y_{lm}(\hat r)\left[\dfrac{4\pi}{2l+1}\left\{\dfrac1{r^{l+1}}\int_0^r\mathrm{d}r^{\prime}(r^{\prime})^{l+2}\rho_{lm}(r^{\prime})+r^l\int_r^{R_i}\mathrm{d}r^{\prime}(r^{\prime})^{1-l}\rho_{lm}(r^{\prime})\right\}\right.\\
		&\left.+\left(\dfrac r{R_i}\right)^l4\pi\mathrm{i}^l\sum_{\vec K\neq0}\dfrac{4\pi}{K^2}\tilde\rho_s(\vec K)Y_{lm}^{\ast}(\vec K)\dfrac{\vec K\cdot\vec R_ij_{l-1}(\vec K\cdot\vec R_i)}{2l+1}\right]
	\end{aligned}
	\label{eq_Vcou_s}
\end{equation}
%球面$R_i$上的\textrm{Coulomb}势可以表达为
%\begin{equation}
%	V_{l=0}(R_i)=4\pi\sum_{\vec K\neq0}\dfrac{\vec K\cdot\vec R_i\tilde\rho_s(\vec K)}{K^2}Y_{00}^{\ast}(\vec K)j_0(\vec K\cdot\vec R_i)
%	\label{eq_Vcou_S}
%\end{equation}

注意到由于\textrm{MT}球的存在,倒空间不连续,即
\begin{equation}
	V_I(\vec r)=\left\{\sum_{\vec K}\dfrac{4\pi}{K^2}\bigg[\tilde\rho_I(\vec K)+\tilde\rho_s(\vec K)\bigg]\mathrm{e}^{\mathrm{i}\vec K\cdot\vec r}\right\}\theta(r)\qquad(r\in \mathrm{Interstitial})
	\label{eq_Vcou_I_theta}
\end{equation}
%由于间隙区的势能变化平缓,基函数是平面波,因此可以\textbf{重新定义能量零点,先将$V_C(\vec K=0)$置为零。}
倒空间中的势能表示为:
\begin{equation}
	V_I(\vec K)=\left\{
	\begin{aligned}
		&\mathrm{divergence} \qquad &\vec K=0 \\
		\dfrac{(4\pi)^2}{K^3}\bigg[\rho_I(\vec K)+&\tilde\rho_s(\vec K)\bigg]\sum_i\dfrac{R_i^2j_1(\vec K\cdot\vec R_i)}{\Omega}\mathrm{e}^{-\mathrm{i}\vec K\cdot\vec\xi_i}\quad &\vec K\neq0
	\end{aligned}\right.
	\label{eq_Vcou_KI_theta}
\end{equation}
显然,由于$\vec K=0$项的存在,使得间隙区的电子\textrm{Coulomb}势具有发散性。
%在\textrm{MT}球内,按照原子\textrm{Coulomb}相互作用表达式可以参考文献\onlinecite{Singh,Nemoshkalenko-Antonov}。

综上所述,\textrm{LAPW}方法中在\textrm{MT}球内和间隙区分别采用\eqref{eq_Vcou_s}和\eqref{eq_Vcou_I_theta}的势函数形式,实现全势(\textrm{full potential})\cite{PRB26-4571_1982,PRB33-823_1986}计算。
%一般是在每个\textrm{MT}球内,势函数用球谐函数(或者是满足晶体对称性的球谐函数)展开,MT球外的势函数用\textrm{Fourier}级数展开\cite{PRB13-5362_1976}

引入\textrm{Madelung}势,完成总能量的计算过程\cite{JMP22-2433_1981}:
\begin{equation}
	\begin{aligned}
	E_T[\rho]=&T_s[\rho]+U[\rho]+E_{xc}[\rho]\\
	=&T_s[\rho]+\dfrac12\biggl[\int\dfrac{\rho(\vec r)\rho(\vec r^{\prime}){}}{|\vec r-\vec r^{\prime}|}\mathrm{d}\vec r\mathrm{d}\vec r^{\prime}-2\sum_{\alpha}Z_{\alpha}\int\dfrac{\rho(\vec r)\mathrm{d}\vec r}{|\vec r-\vec R_{\alpha}|}+\sideset{}{^\prime}\sum_{\alpha,\beta}\dfrac{Z_{\alpha}Z_{\beta}}{|\vec R_{\alpha}-\vec R_{\beta}|}\biggr]\\
	&+E_{xc}[\rho]\\
	=&T_s[\rho]+\dfrac{N}2\bigg[\int_{\Omega}\rho(\vec r)V_c(\vec r)\mathrm{d}\vec r-\sum_{\nu}Z_{\nu}V_M(\vec\gamma_{\nu})\biggr]+E_{xc}[\rho]
	\end{aligned}
	\label{eq_E_total}
\end{equation}
这里$V_c(\vec r)$是\textrm{Coulomb}势:
\begin{equation}
	V_c(\vec r)=\int\dfrac{\rho(\vec r^{\prime})}{|\vec r-\vec r^{\prime}|}\mathrm{d}\vec r^{\prime}-\sum_{\alpha}\dfrac{Z_{\alpha}}{|\vec r-\vec R_{\alpha}|}
	\label{eq_Vcou_WIEN}
\end{equation}
$V_M(\vec\gamma_{\nu})$是\textrm{Madelung}势:
\begin{equation}
	V_M(\vec\gamma_{\nu})=\int\dfrac{\rho(\vec r)\mathrm{d}\vec r}{|\vec r-\vec\gamma_{\nu}|}-\sideset{}{^\prime}\sum_{\alpha}\dfrac{Z_{\alpha}}{|\vec R_{\alpha}-\vec\gamma_{\nu}|}
	\label{eq_Vmadelung}
\end{equation}
即将离子相互作用、电子间相互作用和离子-电子相互作用分为电子感受的\textrm{Coulomb}势和原子核位置的\textrm{Madelung}势贡献。因此利用电子的\textrm{Coulomb}势与\textrm{Madelung}势在球心处的贡献的球形部分彼此抵消来排除奇点:

球心位于$\vec\gamma_{\nu}$、半径为$R_{\nu}$的\textrm{MT}球球面上的势能平均值$S_0(R)$可由式\eqref{eq_Vcou_s}计算,因此除去球心核电荷以外所有电荷在MT球面上产生的势能平均为:
\begin{equation}
  S(R_{\nu})\equiv S_0(R_{\nu})+Z_{\nu}/R_{\nu}
  \label{eq:solid-79}
\end{equation}
因此球心$\vec\gamma_{\nu}$处的\textrm{Madelung}势的球对称部分贡献为\cite{PRB26-4571_1982}:
\begin{equation}
  \begin{split}
	  V_M(\vec\gamma_{\nu})&=\frac1{R_{\nu}}\big[R_{\nu}S_0(R_{\nu})+Z_{\nu}-Q_{\nu}\big]+\sqrt{4\pi}\int_0^{R_{\nu}}\mathrm{d}rr\rho_{00}(r)\\
	   &=\frac1{R_{\nu}}\big[R_{\nu}S_0(R_{\nu})+Z_{\nu}-Q_{\nu}\big]+\bigg\langle\frac1r\rho(\vec r)\bigg\rangle_{\nu}
  \end{split}
   \label{eq:Madelung}
\end{equation}
其中$Q_{\nu}$表示\textrm{MT}球内的电子电荷之和。$\rho(\vec r)$是\textrm{MT}球内的电荷密度,可以用球谐函数表示为:
\begin{equation}
  \rho(\vec r_{\nu})=\sum_{lm}\rho_{lm}(r_{\nu})Y_{lm}(\hat r_{\nu})
  \label{eq:solid-80}
\end{equation}
因此\textrm{MT}球内的总能量贡献为:
\begin{equation}
  \begin{split}
	  E_T=\sum_i\varepsilon_i&-\frac12\int_{\Omega}\rho(\vec r)\big[V_c(\vec r)+\mu_{xc}(\vec r)\big]\mathrm{d}\vec r-\frac12\sum_{\nu}Z_{\nu}V_M(\vec\gamma_{\nu})+E_{xc}[\rho] \\
   =\sum_i\varepsilon_i&-\frac12\left(\int_{\Omega}\rho(\vec r)V_c(\vec r)\mathrm{d}\vec r+\sum_{\nu}Z_{\nu}\bigg\langle\frac1r\rho(\vec r)\bigg\rangle_{\nu}\right)-\dfrac12\int_{\Omega}\rho(\vec r)\mu_{xc}(\vec r)\mathrm{d}\vec r\\
   &-\frac12\sum_{\nu}\frac{Z_{\nu}}{R_{\nu}}\big[R_{\nu}S_0(R_{\nu})+Z_{\nu}-Q_{\nu}\big]+E_{xc}[\rho]
  \end{split}
  \label{eq:solid-81}
\end{equation}
这里$V_c(\vec r)=\displaystyle\int\dfrac{\rho(\vec r\,^\prime)}{|\vec r-\vec r\,^\prime|}\mathrm{d}\vec r\,^\prime-\sum\limits_{\alpha}\dfrac{Z_{\alpha}}{|\vec r-\vec R_{\alpha}|}$。总能量写成这样的形式,原子核位置的\textrm{Coulomb}势奇点可以排除。将势能和电荷密度在各原子核附近作球谐展开,在原子核附近,有
\begin{equation}
  \begin{split}
	  &\int_{\Omega}\rho(\vec r)V_c(\vec r)\mathrm{d}\vec r+Z_{\nu}\sqrt{4\pi}\int_0^{R_{\nu}}\mathrm{d}rr^2\frac{\rho_{00}(r)}r\\
	  =&\sqrt{4\pi}\int\mathrm{d}rr^2\rho_{00}(r)\left[V_{00}(r)Y_{00}(\hat r)+\frac{Z_{\nu}}r\right]+\sum_{lm>0}\int\mathrm{d}rr^2\rho_{lm}(r)V_{lm}(r)
  \end{split}
\end{equation}
\textrm{Coulomb}势的奇点只出现在$V_{00}(r)$中,将$V_{00}(r)$写成核的点电荷势与源于电子的平滑势两部分之和,有
$$V_{00}(r)=-\sqrt{4\pi}\frac{Z_{\nu}}r+\hat V_{00}(r)$$
通过这样的方式,可以将总能量中的奇点排除,但同样单独\textrm{Coulomb}势和\textrm{Madelung}势每一项在原子核位置能量仍然是发散的。

%如果采用标准的\textrm{LDA}近似,式\eqref{eq:solid-81}交换-相关能可以写成:
%\begin{equation}
%  E_{xc}[\rho]\approx\int_{\Omega}\rho(\vec r)\varepsilon_{xc}(\vec r)d\vec r
%  \label{eq:solid-82}
%\end{equation}
基于\textrm{LAPW}方法的总能量计算,在空间中不同的区域采用不同的表示。%间隙区(\textrm{I})的电荷密度用平面波展开,有
%$$\rho(\vec r)=\sum_{\vec G}\rho(\vec G)e^{i\vec G\cdot\vec r},\quad \vec r\in\mathrm{I}$$
%在\textrm{MT}球内,电荷密度用球谐函数展开,在动量空间中的展开形式为:
%$$\bar\rho_{lm}(r_{\nu})=4\pi i^l\sum_{\vec G}\rho(\vec G)e^{i\vec G\cdot\vec r_{\nu}}j_l(Gr_{\nu})Y_{lm}^{\ast}(\vec G)$$ 
首先,将间隙区$V_C(\vec K=0)$置零,排除间隙区的\textrm{Coulomb}势$V_I(\vec K)$的奇点:
\begin{equation}
	\tilde V_I(\vec K)=\left\{
	\begin{aligned}
		&0 \qquad &\vec K=0 \\
		\dfrac{(4\pi)^2}{K^3}\big[\rho_I(\vec K)+&\tilde\rho_s(\vec K)\big]\sum_i\dfrac{R_i^2j_1(\vec K\cdot\vec R_i)}{\Omega}\mathrm{e}^{\mathrm{i}\vec K\cdot(\vec r-\vec\xi_i)}\quad &\vec K\neq0
	\end{aligned}\right.
	\label{eq_Vcou_KI_theta_2}
\end{equation}
%在总能量计算中补偿这一平移。
\textrm{MT}球内对总能量的贡献部分在正空间中按上述式\eqref{eq:solid-81}计算。

因此\textrm{LAPW}方法的原胞内的晶体总能量可以写成:
\begin{equation}
  \begin{split}
	  E=&\sum_i\varepsilon_i-\sum_{\vec K}\Omega\rho_I(\vec K)\tilde V_I^{\ast}(\vec K)-\frac12\sum_{\nu}\dfrac{Z_{\nu}}{R_{\nu}}[Z_{\nu}-Q_{\nu}+R_{\nu}S_0(R_{\nu})]\\
    &-\sum_{\nu}\sum_{lm}\int_0^{R_{\nu}}\mathrm{d}rr^2\left[\rho_{lm}(r_{\nu})\left(\tilde V_{lm}^{\ast}(r_{\nu})+\dfrac{\sqrt{4\pi}}{2r_{\nu}}Z_{\nu}\delta_{l0}\right)-\bar\rho_{lm}(\vec r_{\nu})\bar V_{lm}^{\ast}(\vec r_{\nu})\right]
  \end{split}
  \label{eq:solid-83}
\end{equation}
这里$\tilde V(\vec r)$和$\bar V_{lm}(\vec r)$根据都按下式计算:
$$\tilde V(\vec r)=\frac12V_c(\vec r)-\varepsilon_{xc}(\vec r)+\mu_{xc}(\vec r)$$

%\textrm{WIEN2k}程序中,\textbf{程序\textrm{lapw0}部分完成式\eqref{eq:solid-83}中除$\sum\limits_i\varepsilon_i$之外的所有各项的计算}。
%\subsection{LDA+U和GW近似}
%\subsubsection{GW近似}
%Landau的Fermi液体理论是研究群体激发和多体Fermi子体系物理性质的重要方法\cite{Landau}。Fermi液体的特征是准粒子分布$\varepsilon=\varepsilon(\vec k)$由体系总能量$E$对分布函数的变分,即
%\begin{equation}
%  \frac{\delta E}{\delta n(\vec k)}=\varepsilon(\vec k)
%  \label{eq:solid-219}
%\end{equation}
%关联函数$f(\vec k,\vec k^\prime)$由准粒子能量对整个$\vec k$空间粒子分布变分得到
%\begin{equation}
%  \frac{\delta\varepsilon(\vec k)}{\delta n(\vec k^\prime)}=\frac{\delta^2E}{\delta n(\vec k)\delta n(\vec k^\prime)}=f(\vec k,\vec k^\prime)
%  \label{eq:solid-220}
%\end{equation}
%考虑准粒子间相互作用,体系激发能记作
%\begin{equation}
%  W=\sum_{\vec k}\varepsilon(\vec k)\delta n(\vec k)+\frac12\sum_{\vec k}\sum_{\vec k^\prime}f(\vec k,\vec k^\prime)\delta n(\vec k)\delta(\veck^\prime)
%  \label{eq:solid-221}
%\end{equation}
%
%Green函数是研究Fermi液体的重要数学工具。对宏观体系,Green函数定义为\cite{Lifshitz}
%\begin{equation}
%  \tilde G(x,x^\prime)=-i\langle T\hat\Psi(x)\hat\Psi^{\ast}(x^\prime)\rangle
%  \label{eq:solid-222}
%\end{equation}
%这里$x$表示时间$t$和位置$\vec r$,$\langle\cdots\rangle$表示对体系基态求平均,$T$表示按时间先后乘积算符。$\hat\Psi$是Heisenberg算符,对式\eqref{eq:solid-222}进行Fourier变换,得到以$\vec r$,$E$为表象的Green函数$\tilde G(\vec r,\vec r\,^\prime;E)$是Dyson方程\cite{Lifshitz}
%\begin{equation}
%  (\nabla^2+E)\tilde G(\vec r,\vec r\,^\prime;E)-\int d\vec r^{\prime\prime}\Sigma(\vec r,\vec r^{\prime\prime};E)\tilde G(\vec r^{\prime\prime},\vec r\,^\prime;E)=\delta(\vec r-\vec r\,^\prime)
%  \label{eq:solid-223}
%\end{equation}
%的解。这里$\Sigma(\vec r,\vec r\,^\prime;E)$是描述交换和相关效应的质量或自能算符,这是个与能量有关的非定域的非-Hermitian算符,考虑到粒子与体系中其他粒子的相互作用。晶体中的质量具有平移对称性,
%\begin{equation}
%  \Sigma(\vec r+\vec a,\vec r+\vec a;E)=\Sigma(\vec r,\vec r\,^\prime;E),
%  \label{eq:solid:224}
%\end{equation}
%这里$\vec a$是晶体平移格矢。
%
%临近Green函数的极点,式\eqref{eq:solid-223}右面的$\delta$函数为0,所得积分-微分方程的本征值确定体系激发能谱\cite{Lifshitz,PR145-561_1966}
%\begin{equation}
%  -\delta^2\Phi_{\vec k}(\vec r,E)+\int d\vec r\,^\prime\Sigma(\vec r,\vec r\,^\prime;E)\Phi_{\vec k}(\vec r\,^\prime,E)=\varepsilon_{\vec k}\Phi_{\vec k}(\vec r,E) 
%  \label{eq:solid-225}
%\end{equation}
%这里函数$\Phi_{\vec k}(\vec r,E)$类似于周期场中的单电子Bloch波函数。对金属中的Fermi电子液体中用式\eqref{eq:solid-225}替代一般Schr\"odinger方程。与一般的Schr\"odinger方程不同,式\eqref{eq:solid-225}的能量本征值是复数,因为质量算符$\Sigma(\vec r,\vec r\,^\prime;E)$是复数。
%
%由式(\ref{eq:solid-223},\ref{eq:solid-225})得到Green函数
%\begin{equation}
%  \tilde G(\vec r,\vec r\,^\prime;E)=\sum_{\vec k}\frac{\Phi_{\vec k}(\vec r,E)\Phi_{\vec k}^{\ast}(\vec r\,^\prime,E)}{E-\varepsilon_{\vec k}(E)+i0\mathrm{sign}E}
%  \label{eq:solid-226}
%\end{equation}
%$\varepsilon_{\vec k}(E)$是体系中加入一个粒子引起的能量改变。如果将单个准粒子改变引起得能量变化$\varepsilon_{\vec k}(E)$定义为式\eqref{eq:solid-219}中的准粒子能量,对临近Fermi面的态,函数$\tilde G(\vec r,\vec r\,^\prime;E)$在$E=\varepsilon_{\vec k}(E)$有极值。因此Green函数极值确定了多Fermi子体系的元激发能谱。一般说,因为和其他粒子相互作用,准粒子能量为复数。复数能量使得体系激发态的寿命有限$(\tau\sim1/|\mathrm{Im}\varepsilon|)$\cite{Landau-Lifshitz}。能量宽度由$\mathrm{Im}\varepsilon$确定。
%
%%临近Fermi能级,$\mathrm{Im}\varepsilon_{\vec k}(E)\rightarrow0$,相应的$\mathrm{Im}\Sigma\rightarrow0$,于是解方程\eqref{eq:solid-225}得到$\mathrm{Re}\varepsilon_{\vec k}(E)$和实数形式的$\Sigma(\vec r,\vec r\,^\prime;E)$和$\epsilon_{\vec k}(E)$。
%
%根据Hohenberg-Kohn定理,本征值$\Sigma(\vec r,\vec r\,^\prime;E)$由体系基态确定,它也是电子密度的函数。Sham和Kohn建议可用局域电子密度近似表示$\Sigma(\vec r,\vec r\,^\prime;E)$\cite{PR145-561_1966}:
%\begin{equation}
%  \Sigma(\vec r,\vec r\,^\prime;E)=V_C(\vec r)\delta(\vec r-\vec r\,^\prime)+\Sigma_0(\vec r-\vec r\,^\prime;E-V_C(\vec r_0);\rho(\vec r_0))
%  \label{eq:solid-227}
%\end{equation}
%这里$\Sigma_0$是密度为$\rho$的无相互作用电子气的本征能量,$V_C(\vec r_0)$是位于$\vec r_0=(\vec r+\vec r\,^\prime)/2$的静电Coulomb势。此外式\eqref{eq:solid-227}已将能量$\Sigma$中的局域部分以Hartree势的形式分离出来,因此可以利用$\Sigma_0$的短程效应。
%
%对本征函数$\Phi_{\vec k}$作近似
%$$\Phi_{\vec k}(\vec r,E)=A(\vec k)\exp[i\vec p(\vec r)\vec r]$$
%并认为$A$和电子动量与$\vec r$无关,将式\eqref{eq:solid-225}称为类似Kohn-Sham方程的表达式\cite{JPC4-2064_1971},
%\begin{equation}
%  [-\nabla^2+V_C(\vec r)+\Sigma_{xc}(\rho(\vec r),E)]\Phi_{\vec k}(\vec r,E)=\varepsilon_{\vec k}(\vec r,E)
%  \label{eq:solid-228}
%\end{equation}
%若$E=\mu$,$\Sigma_{xc}(\rho(\vec r),\mu)\equiv\mu_{xc}(\rho(\vec r))$我们因此得到符合局域密度近似的DFT方程,两者的交换-相关势$\mu_{xc}$是相同的。因此对于低激发能,可以使用近似$\Sigma_{xc}(\rho(\vec r),E)\simeq\mu_{xc}$,此结果对应于忽略准粒子间的相互作用,即Landau函数$f(\vec k,\vec k^\prime)=0$。
%
%因此基于DFT的能带计算,如果充分考虑交换-相关效应,可以计算多电子体系的元激发能量。注意,采用单电子近似,要求能带比较宽,即中心位于不同格点的波函数有较大的重叠,电子间相互作用不是很强。
%
%Hedin和Lundqvist详细回顾了用Green函数技术解决电子相关的问题\cite{Hedin-Lundqvist}。在单粒子近似下的Green函数,准粒子与谱函数的峰联系在一起。如果峰足够尖锐,表明存在一个明确的准粒子能量,对一般的非均匀体系,准粒子能量和波函数可以通过解Dyson方程\eqref{eq:solid-225}求得。对于准粒子问题,核心问题是对自能算符$\Sigma(\vec r,\vec r\,^\prime;E)$的足够好的近似。常用的方法是GW近似\cite{PR139-A796_1965},自能用屏蔽相互作用$W$计算到最低阶。
%\begin{equation}
%  \Sigma(\vec r,\vec r\,^\prime;E)=\frac i{2\pi}\int_{-\infty}^{\infty}dE^\prime\tilde G(\vec r,\vec r\,^\prime;E+E^\prime)W(\vec r,\vec r\,^\prime;E^\prime)
%  \label{eq:solid-229}
%\end{equation}
%Green函数$\tilde G$由准粒子的波函数和能量表示,屏蔽Coulomb作用$W$
%\begin{equation}
%  W(\vec r,\vec r\,^\prime;E)=\frac1{\Omega}\int d^3r^{\prime\prime}\varepsilon^{-1}(\vec r,\vec r^{\prime\prime};E)V(\vec r^{\prime\prime}-\vec r\,^\prime)
%  \label{eq:solid-230}
%\end{equation}
%这里$V$是未屏蔽的Coulomb势,$\varepsilon^{-1}$是介电函数矩阵的逆阵,
%\begin{equation}
%  \varepsilon^{-1}(\vec r,\vec r\,^\prime;E)=\delta(\vec r-\vec r\,^\prime)+\int d^3r^{\prime\prime} V(\vec r\,^\prime-\vec r^{\prime\prime})P(\vec r^{\prime\prime},\vec r\,^\prime;E)
%  \label{eq:solid-231}
%\end{equation}
%这里$P$是完全响应函数,于是
%\begin{equation}
%  W(\vec r,\vec r\,^\prime;E)=V(\vec r-\vec r\,^\prime)+W_c(\vec r,\vec r\,^\prime;E)
%  \label{eq:solid-232}
%\end{equation}
%这里
%\begin{equation}
%  W_c(\vec r,\vec r;E)=\int d^3r_1d^3r_2V(\vec r\,^\prime-\vec r_1)P(\vec r_1,\vec r_2;E)V(\vec r_2-\vec r\,^\prime)
%  \label{eq:solid-233}
%\end{equation}
%Green函数可以写成谱表示
%\begin{equation}
%  G(\vec r,\vec r\,^\prime;E)=\int_{-\infty}^{\mu}dE^\prime\frac{A(\vec r,\vec r\,^\prime;E^\prime)}{E-E^\prime-i\delta}+\int_{\mu}^{\infty}dE^\prime\frac{A(\vec r,\vec r\,^\prime;E^\prime)}{E-E^\prime+i\delta}
%  \label{eq:solid-234}
%\end{equation}
%这里$A(\vec r,\vec r\,^\prime;E)=-\frac1{\pi}\mathrm{Im}G(\vec r,\vec r\,^\prime;E)\mathrm{sgn}(E-\mu)$
%
%实际计算中,取零阶Green函数,有
%\begin{equation}
%  A(\vec r,\vec r\,^\prime;E)=\sum_{kn}\psi_{kn}(\vec r)\psi_{kn}^{\ast}(\vec r\,^\prime)\delta(E-E_{kn})
%  \label{eq:solid-235}
%\end{equation}
%于是自能可以写成
%\begin{equation}
%  \Sigma(\vec r,\vec r\,^\prime;E)=\Sigma_x(\vec r,\vec r\,^\prime)+\Sigma_c(\vec r,\vec r\,^\prime;E)
%  \label{eq:solid-236}
%\end{equation}
%这里$\Sigma_x$是净的交换势,
%\begin{equation}
%  \Sigma_x(\vec r,\vec r\,^\prime)=-\sum_{kn}^{occ}\psi_{kn}(\vec r)\psi_{kn}^{\ast}(\vec r\,^\prime)V(\vec r-\vec r\,^\prime)
%  \label{eq:solid-254}
%\end{equation}
%$E_c$自能的相关部分,
%\begin{equation}
%  \begin{split}
%    \Sigma_c(\vec r,\vec r\,^\prime;E)=&\sum_{kn}^{occ}\psi_{kn}(\vec r)\psi_{kn}^{\ast}(\vec r\,^\prime)W_c^-(\vec r,\vec r\,^\prime;E-E_{kn})\\
%    &+\sum_{kn}^{occ}\psi_{kn}(\vec r)\psi_{\vec r}^{\ast}(\vec r\,^\prime)W_c^+(\vec r,\vec r\,^\prime;E-E_{kn})
%  \end{split}
%  \label{eq:solid-237}
%\end{equation}
%其中$W_c^{\pm}(\vec r,\vec r\,^\prime;E)=\dfrac i{2\pi}\displaystyle\int_{-\infty}^{+\infty}dE^\prime\frac{W_c(\vec r,\vec r\,^\prime;E^\prime)}{E+E^\prime\pm i\delta}$。于是$\Sigma(\vec r,\vec r\,^\prime;E)$可以记作\cite{JPCM9-767_1997},
%\begin{displaymath}
%  \Sigma(\vec r,\vec r\,^\prime;E)=-\sum_{kn}\psi_{kn}(\vec r)\psi_{kn}^{\ast}(\vec r\,^\prime)W_0(\vec r,\vec r\,^\prime;E-E_{kn})
%\end{displaymath}
%其中
%\begin{displaymath}
%  \begin{split}
%    W_0(\vec r,\vec r\,^\prime;E-E_{kn})\equiv&[V(\vec r-\vec r\,^\prime)-W_c^-(\vec r,\vec r\,^\prime;E-E_{kn})]\theta(\mu-E_{kn})\\
%    &-W_c^+(\vec r,\vec r\,^\prime;E-E_{kn})\theta(E_{kn}-\mu)
%  \end{split}
%\end{displaymath}
%这样的GWA近似的自能与Hartree-Fock方法具有相同的形式,但是自能是能量的函数且因为包含相关作用,因此也依赖于未占据态。GWA可以看作是包含动态屏蔽Coulomb势$W_0$的推广Hartree-Fock方法,注意这里的$W_0$与动态屏蔽势$W$不同。
%
%在各种能带计算方法中引入GW近似,包括赝势方法\cite{PRB34-5390_1986},LMTO-TB方法\cite{PRL74-3221_1995}。GWA校正主要应用于简单金属和过渡金属体系,但由于计算过程比较复杂所以目前还没有广泛应用到复杂体系的计算中。GWA校正的另一个问题是计算屏蔽相互作用所需的响应函数,要借助LDA得到的波函数和能带来计算得到\cite{JPCM9-767_1997}。但是这样的方法只适用于电子相关较小的体系(比如绝缘体或者半导体);对电子强相关体系,则需要采用比LDA近似更好的Hamiltonian,一般可以通过自洽迭代来计算自能\cite{PRL74-3221_1995}。
%
%\subsubsection{LDA+U和GWA校正的关系}
%尽管GWA是由多体微扰理论导出的最简单的自能近似,但是计算量已经很大。GWA和LDA+U可以分别看作包含依赖于频率和轨道屏蔽Coulomb作用的Hartree-Fock方法。至少对含有局域的{\it d}\,或{\it f}\,轨道的过渡金属和稀土金属离子,LDA+U可以看作是对GWA的近似\cite{JPCM9-767_1997}。
%
%LDA+U是针对包含在离域态中的定域态轨道的自能校正,定域态的强Coulomb相关用参数$U$校正,而离域态可以用LDA很好的描述。为了确定LDA+U和GWA之间的关系,对态$\psi_d$,考虑GWA中自能的相关部分
%\begin{equation}
%  \begin{split}
%    \langle\psi_d|\Sigma_c(E_d)|\psi_d\rangle=&\langle\psi_d\psi_d|W_c^-|\psi_d\psi_d\rangle\\
%    &+\sum_{kn\neq d}^{occ}\langle\psi_d\psi_{kn}|W_c^-(E_d-E_{kn})|\psi_{kn}\psi_d\rangle\\
%    &+\sum_{kn}^{unocc}\langle\psi_d\psi_{kn}|W_c^+(E_d-E_{kn})|\psi_{kn}\psi_d\rangle
%  \end{split}
%  \label{eq:solid-238}
%\end{equation}
%严格地说,自能应该是$\tilde E_d=E_d+$自能校正。如果$\psi_d$是局域的而且能量与其他态很好的分离,则式\eqref{eq:solid-238}第一项比剩下的其余项大得多,最后一项含未占据的$\psi_d$态,但因为这些项与占据态正交,因此这一项比第一项小得多。可作近似\cite{JPCM9-767_1997}
%\begin{equation}
%  \langle\psi_d|\Sigma_c(E_d)|\psi_d\rangle\approx\langle\psi_d\psi_d|W_c^-(0)|\psi_d\psi_d\rangle=-\frac12\langle\psi_d\psi_d|W_c(0)|\psi_d\psi_d\rangle
%  \label{eq:solid-239}
%\end{equation}
%将屏蔽势关联部分写成谱函数表象,
%\begin{equation}
%  W_c(E)=\int_{-\infty}^0dE^\prime\frac{B(E^\prime)}{E-E^\prime-i\delta}+\int_0^{\infty}dE^\prime\frac{B(E^\prime)}{E-E^\prime+i\delta}
%  \label{eq:solid-240}
%\end{equation}
%这里$B(E)=-\dfrac1{\pi}W_c(E)\mathrm{sgn}(E)$。
%$W_c$是$E$的偶函数,因此$B(E)$是奇函数,因此$W_c^-(0)=-1/2W_c(0)$,类似的对未占据的{\it d}\,态,有$+1/2\langle\psi_d\psi_d|W_c(0)|\psi_d\psi_d\rangle$,因此能量差为
%\begin{equation}
%  \begin{split}
%    \Delta&=E_2^{HF}-E_1^{HF}+\langle\psi_d\psi_d|W_c(0)|\psi_d\psi_d\rangle\\
%    &=\langle\psi_d\psi_d|V|\psi_d\psi_d\rangle+\langle\psi_d\psi_d|W_c(0)|\psi_d\psi_d\rangle\\
%    &=\langle\psi_d\psi_d|W(0)|\psi_d\psi_d\rangle
%  \end{split}
%  \label{eq:solid-241}
%\end{equation}
%这符合屏蔽Coulomb相互作用$\Delta=U\approx W(0)$。
%
%上述近似中,局域态的GW自能为
%\begin{equation}
%  \Sigma(\vec r,\vec r\,^\prime;E_d)=\Sigma_x(\vec r,\vec r\,^\prime)+\sum_{kn=d}\psi(\vec r)\psi_{kn}^{\ast}(\vec r\,^\prime)W_c^0(\vec r,\vec r\,^\prime;E_d)
%  \label{eq:solid-242}
%\end{equation}
%这里$$W_c^0(\vec r,\vec r\,^\prime;E)=-\frac12W_c(\vec r,\vec r\,^\prime;0)[\theta(\mu-E_d)-\theta(E_d-\mu)]$$
%LDA的自能校正
%\begin{equation}
%  \Delta\Sigma(\vec r,\vec r\,^\prime;E_d)=\Sigma(\vec r,\vec r\,^\prime;E_d)-V_{xc}^{LDA}(\vec r)\delta(\vec r-\vec r\,^\prime)
%  \label{eq:solid-243}
%\end{equation}
%应该与LDA+U方法的$U$值相等。按照LDA+U思想,将空间分为定域部分$\phi_m$(一般是{\it d}\,和{\it f}\,态)和离域部分$\psi_{kn}$
%$$\delta(\vec r-\vec r\,^\prime)=\sum_m\phi_m(\vec r)\phi_m^{\ast}(\vec r\,^\prime)+\sum_{kn}\psi_{kn}(\vec r)\psi_{kn}^{\ast}(\vec r\,^\prime)$$
%自能校正可以写成
%\begin{equation}
%  \begin{split}
%    \Delta(\vec r,\vec r\,^\prime;E_d)=&\sum_{mm^\prime}\phi_m(\vec r)\Delta\Sigma_{mm^\prime}(E_d)\phi_{m^\prime}^{\ast}(\vec r\,^\prime)+\sum_{knn^\prime}\psi_{kn}(\vec r)\Delta\Sigma_{nn^\prime}(E_d)\psi_{kn}^{\ast}(\vec r\,^\prime)\\
%    &+\sum_{knm}\psi_{kn}(\vec r)\delta\Sigma_{nm}(\vec k,E_d)\phi_m^{\ast}(\vec r\,^\prime)+\sum_{kmn}\phi_m(\vec r)\Delta\Sigma_{mn}(\vec k,E_d)\psi_{kn}^{\ast}(\vec r\,^\prime)
%  \end{split}
%  \label{eq:solid-244}
%\end{equation}
%其中第一项是主要的,有近似
%$$\Delta\Sigma(\vec r,\vec r;E_d)\approx\sum_{mm^\prime}\phi_m(\vec r)\Delta\Sigma_{mm^\prime}(E_d)\phi_{m^\prime}^{\ast}(\vec r\,^\prime)$$
%这里$$\Delta\Sigma_{mm^\prime}(E_d)=\langle\phi_m|\Sigma_x-V_{xc}|\phi_m\rangle+\sum_{m^{\prime\prime}m^{\prime\prime\prime}}\langle m,m^{\prime\prime}|W_c^0|m^{\prime\prime\prime},m^\prime\rangle n_{m^{\prime\prime},m^{\prime\prime\prime}}$$
%这里$$n_{m^{\prime\prime},m^{\prime\prime\prime}}=\sum_{kn=d}\langle\phi_{m^{\prime\prime}}|\psi_{kn}\rangle\langle\psi_{kn}|\phi_{m^{\prime\prime\prime}}\rangle$$
%注意到选择的$\phi_m$是原子内的局域轨道,剩下的自能很小,可以包含在单电子项中。
%
%假设只有一个{\it d}\,轨道$\psi_{m\sigma}$的{\it d}\,离子,根据上述近似,局域态的GWA自能为
%\begin{equation}
%  \Sigma(\vec r,\vec r\,^\prime;E_{m\sigma})=\Sigma_x(\vec r,\vec r\,^\prime)+\sum_{m^\prime\omega^\prime}\psi_{m^\prime\omega^\prime}(\vec r)\psi_{m^\prime\sigma^\prime}^{\ast}(\vec r\,^\prime)W_c^0(\vec r,\vec r\,^\prime;E_{m\sigma})
%  \label{eq:solid-245}
%\end{equation}
%这里$$W_c^0(\vec r,\vec r\,^\prime;E_{m\sigma})=-\frac12W_c(\vec r,\vec r\,^\prime;0)[\theta(\mu-E_{m\omega})-\theta(E_{m\sigma}-\mu)]$$
%GWA中的电子-电子相互作用的总势能的矩阵元可以写成
%\begin{equation}
%  \begin{split}
%    \langle\psi_{m\sigma}&|V_{Hartree}+\Sigma_x+\Sigma_c|\psi_{m\sigma}\rangle\\
%    =&\sum_{m^\prime\sigma^\prime}^{occup}\iint d\vec rd\vec r\,^\prime\psi_{m\omega}^{\ast}(\vec r)\psi_{m\omega}(\vec r)V(\vec r-\vec r\,^\prime)\psi_{m^\prime\sigma^\prime}^{\ast}(\vec r\,^\prime)\psi_{m^\prime\sigma^\prime}(\vec r\,^\prime)\\
%    &-\sum_{m^\prime}^{occup}\iint d\vec rd\vec r\,^\prime\psi_{m\omega}^{\ast}(\vec r)\psi_{m^\prime\omega^\prime}(\vec r)V(\vec r-\vec r\,^\prime)\psi_{m\sigma}^{\ast}(\vec r\,^\prime)\psi_{m^\prime\sigma^\prime}(\vec r\,^\prime)\\
%    &+\left(\frac12-n_{m\sigma}\right)\sum_{m^\prime}\iint d\vec rd\vec r\,^\prime\psi_{m\omega}^{\ast}(\vec r)\psi_{m^\prime\omega^\prime}(\vec r)W_c(\vec r,\vec r\,^\prime;0)\psi_{m\sigma}^{\ast}(\vec r\,^\prime)\psi_{m^\prime\sigma^\prime}(\vec r\,^\prime)
%  \end{split}
%  \label{eq:solid-246}
%\end{equation}
%这里$n_{m\sigma}$是$m\sigma$轨道占据状态,如$\mu-E_{m\sigma}>0$则$n_{m\sigma}=1$;$\mu-E_{m\sigma}<0$,有$n_{m\sigma}=0$。上述矩阵元可以写成
%$$V_{m\sigma}^{GWA}=\sum_{m^\prime\sigma^\prime}U_{mm^\prime}^0n_{m^\prime\sigma^\prime}-U_{mm}^0n_{m\sigma}-\sum_{m^\prime\neq m}J_{mm^\prime}n_{m^\prime\sigma}+\left(\frac12-n_{m\sigma}\right)\sum_{m^\prime}W_{mm^\prime}$$
%这里$U_{mm^\prime}^0$是未屏蔽的Coulomb相互作用矩阵元。$J_{mm^\prime}$是交换矩阵,$W_{mm^\prime}$是交换势$W_c(\vec r,\vec r\,^\prime;0)$矩阵元。将屏蔽参数定义为$W=-\sum\limits_{m^\prime}W_{mm^\prime}$,最终GWA的势能矩阵元表达式为\cite{JPCM9-767_1997}
%\begin{equation}
%  V_{m\sigma}^{GWA}=\sum_{m^\prime\sigma^\prime}U_{mm^\prime}^0n_{m^\prime\sigma^\prime}-(U_{mm}^0-W)n_{m\sigma}-\sum_{m^\prime\neq m}J_{mm^\prime}n_{m^\prime\sigma}-\frac12W
%  \label{eq:solid-247}
%\end{equation}
%为了将LDA的校正写成GWA形式,必须将LSDA的势能矩阵元写成上述相似的形式,因为LSDA并非由轨道-轨道相互作用导出,而由类似于均匀电子气的处理方式,用与Coulomb相互作用能有关的电荷密度计算得到的与轨道无关的有效局域势,无法严格处理。{\it d}\,电子的相互作用能作为总的{\it d}\,电子数$N$的函数,$E_{LSDA}[\rho(\vec r)]=E_{LSDA}[N|\psi_{m\sigma}(\vec r)|^2]$。已知LSDA中单电子本征能不是很准确,但是总能量比较准确,于是假设Hartree-Fock计算是好的近似
%\begin{equation}
%  \begin{split}
%   E_{LSDA}[\rho_{\sigma}(\vec r)]&=E_{LSDA}[N_{\sigma}|\psi_{m\sigma}(\vec r)|^2]\\
%   &=\frac12F^0N(N-1)-\frac14JN(N-2)\frac14J(N_{\uparrow}-N_{\downarrow})^2
% \end{split}
%  \label{eq:solid-248}
%\end{equation}
%这里$F^0$是第一Slater积分,$J$是交换能参数,$N_{\sigma}=\sum\nolimits_mn_{m\sigma}$,$N=N_{\uparrow}+N_{\downarrow}$
%
%LSDA的电子相互作用势能是总能量对电荷密度$\rho(\vec r)$的变分导数,相互作用能作为总的{\it d}\,电子总数$N_{\sigma}$的变分导数为:
%\begin{equation}
%  \begin{split}
%    \frac{\partial E_{LSDA}[N_{\sigma}|\psi_{m\sigma}(\vec r)|^2]}{\partial N_{\sigma}}&=\int d\vec r\frac{\delta E_{LSDA}[\rho(\vec r)]}{\delta\rho_{\sigma}(\vec r)}\frac{\partial\rho_{\sigma}(\vec r)}{\partial N_{\sigma}}\\
%    &=\int d\vec rV_{LSDA}^{\sigma}(\rho(\vec r))|\psi_{m\sigma}(\vec r)|^2\\
%    &=F^0N-\frac12(F^0-J)-JN_{\sigma}
%  \end{split}
%  \label{eq:solid-257}
%\end{equation}
%由此可有LSDA的势能矩阵元为$V_{m\sigma}^{LSDA}=F^0N-\frac12(F^0-J)-JN_{\sigma}$。
%GWA对LSDA的势能校正为\cite{JPCM9-767_1997}
%\begin{equation}
%  \begin{split}
%    \delta V_{m\sigma}=&V_{m\sigma}^{GWA}-V_{m\sigma}^{LSDA}\\
%    =&\sum_{m^\prime\sigma^\prime}U_{mm^\prime}^0n_{m^\prime\sigma^\prime}-(U_{mm}^0-W)n_{m\sigma}-\sum_{m^\prime\neq m}j_{mm^\prime}n_{m^\prime\sigma}-\frac12W\\
%    &-F^0\sum_{m^\prime\sigma^\prime}n_{m^\prime\sigma^\prime}+J\sum_mn_{m\sigma}+\frac12(F^0-J)\\
%    =&\sum_{m^\prime\sigma^\prime}(U_{mm^\prime}^0-F^0)n_{m^\prime\sigma^\prime}-(U_{mm}^0-W)n_{m\sigma}-\sum_{m^\prime\neq m}j_{mm^\prime}n_{m^\prime\sigma}\\
%    &-\frac12W+J\sum_mn_{n\sigma}+\frac12(F^0-J)
%  \end{split}
%  \label{eq:solid-249}
%\end{equation}
%差值$U_{mm^\prime}^0-F^0$与Slater积分$F^0$无关(只与Slater积分$F^k$且$k\neq0$有关),而且有$U_{mm^\prime}^0-F^0=U_{mm^\prime}-U$,这里$U=F^0-W$是屏蔽Coulomb参数,$U_{mm^\prime}$是屏蔽Coulomb矩阵元。
%\begin{equation}
%  \begin{split}
%    \delta V_{m\sigma}=&V_{m\sigma}^{GWA}-V_{m\sigma}^{LSDA}\\
%    =&\sum_{m^\prime}U_{mm^\prime}n_{m^\prime-\sigma}+\sum_{m^\prime\neq m}(U_{mm^\prime}-J_{mm^\prime})n_{m^\prime\sigma}\\
%    &-U(N-\frac12)+J(N_{\sigma}-\frac12)
%  \end{split}
%  \label{eq:solid-250}
%\end{equation}
%如果占据矩阵是对角化的,式\eqref{eq:solid-250}等价于LDA+U势校正\eqref{eq:solid-215}。GWA和LDA+U的本质差别在于计算屏蔽Coulomb势参数$U$,在LDA+U中,$U$是通过构造LSDA超晶胞计算的,在GWA中则是通过计算响应函数得到的。
%%\newpage
%\bibliographystyle{mythesis}
%%  \phantomsection\addcontentsline{toc}{section}{bibliography}
%  {\small\bibliography{bib/Myref}}
%%  \nocite{*}

