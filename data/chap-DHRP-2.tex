\chapter{不同外电场下\textrm{DHR}的分子结构与激发特性}

%## 摘要

%## 关键词
%双环庚基红荧烯(DHR);外电场;分子结构;激发特性

## 2 计算方法
外电场作用下分子体系的哈密顿量H可表示为[8-11]:
\[H=H_{0}+H_{int}\]
其中,\(H_{0}\)为无电场时的哈密顿量,\(H_{int}\)为外电场与分子体系相互作用时的哈密顿量。在偶极近似下,\(H_{int}\)可表示为[12,13]:
\[H_{int}=\mu\cdot F\]
式中,μ为分子电偶极矩,F为偶极电场强度。

本文采用B3LYP/6-311G(d,p)方法对DHR分子进行结构优化和频率计算。在优化基础上,运用相同的方法和基组,计算沿分子X轴施加不同电场(0-0.025原子单位)对DHR分子几何结构、能量、偶极矩、前线轨道能级及红外光谱的影响。最后,采用含时密度泛函理论(TD-DFT)中的WB97XD/DEF2-TZVP方法,计算不同电场下DHR分子的前20个激发态,并对主要激发态进行空穴-电子分析,以明确外电场下DHR分子的激发特性。上述所有计算与分析均通过Gaussian 09软件包和Multiwfn3.7多功能波函数软件完成[14,15]。

## 3 计算结果与分析
### 3.1 DHR分子基态的几何结构
DHR由碳元素和氢元素组成,分子式为C₃₀H₁₄。采用DFT/B3LYP方法在6-311G(d,p)基组水平下对DHR分子的几何结构进行优化,结果收敛且无虚频,表明优化后的结构合理。DHR基态优化结构如图1所示,该分子为平面结构,由两个五元环、两个七元环和四个六元环构成。

### 3.2 外电场对DHR基态性质的影响
在无电场优化的基础上,采用相同方法和基组沿分子X轴施加不同强度电场(0-0.025原子单位),探究DHR分子结构及其特性的变化。DHR分子键长随电场强度的变化如表1所示。由于DHR分子的键长和键角数量较多,本文仅选取主要键长和键角进行分析。DHR分子为极性分子,随着外电场强度的增大,键长和键角对电场强度表现出明显的依赖性,具体如图2所示。表2为无外电场时DHR分子的原子电荷分布,表3和图3分别为不同外电场下DHR分子的键角数据及相应关系图。

由图2可知,在0-0.025原子单位的外电场作用下,C-C键(4,7)、C-C键(10,11)、C-C键(13,14)、C-H键(26,41)、C-H键(1,40)、C-H键(17,20)、C-H键(18,21)和C-H键(33,43)的键长随外电场强度增强而增大,其中C-C键(10,11)、C-H键(26,41)和C-H键(33,43)的键长随外电场强度增大而急剧增加;C=C键(23,24)的键长随外电场强度增大无明显变化;C-C键(3,6)、C-C键(4,5)、C-C键(24,25)和C-H键(32,44)的键长随外电场强度增大而减小,其中C-C键(3,6)和C-C键(4,5)的键长减小更为显著。这些现象可通过外电场与分子内电场的叠加效应进行定性解释[18]。

随着外电场强度增大,局域电子转移使C4-C7、C7-C8、C10-C11、C26-H41、C1-C40、C17-H20、C18-H21和C33-H43之间的电场减弱,进而导致键长增大:由表2可知,C10-C11中的C11电负性较强,具有较强的吸电子能力,使得分子内电场方向从C11指向C10。分子内电场方向与外电场方向的夹角为钝角,导致叠加电场减弱;C26-H41和C33-H43中C原子的电负性较强,原本偏向C原子的电子云向H原子偏移,使得分子内电场方向与外电场方向相反,因此外电场对分子内电场具有较强的削弱作用,最终导致C10-C11、C26-H41和C33-H43的键长显著增加。

相反,外电场与分子内电场的叠加使C-C键(3,6)、C-C键(4,5)、C-C键(24,25)和C-H键(32,44)之间的电场增强,进而导致键长减小:C3-C6和C4-C5中的C6和C4电负性较强,在外电场作用下,原本偏向C6和C4的电子云向C3和C5移动。此时外电场方向与电子云偏移方向一致,使得原子间电场显著增强,键长急剧缩短。

由表3和图3可知,DHR分子的键角随电场强度变化而改变。在外电场作用下,随着外电场强度增大,分子的键角和键长均发生变化,导致分子发生弯曲和折叠。造成这一现象的原因主要有两点:一是体系的诱导偶极矩与外电场相互作用,使分子能量降低并产生畸变;二是在外电场作用下,分子内原子电荷重新分布,导致靠近电场区域的电场力远大于远离电场区域的电场力,分子受力不均匀,最终引发分子畸变。

由表4和图4可知,随着外电场强度增大,DHR分子的偶极矩逐渐增大。这是因为DHR分子属于大共轭芳香体系,芳环中π电子的极化程度极高。外电场的引入使原本杂乱的偶极矩趋向于电场方向。在外电场作用下,电子弛豫会产生更大的诱导偶极矩,从而使总偶极矩增大[19-21]。DHR分子的总能量随外电场强度增大而降低,且降低趋势逐渐变大,这是由于诱导偶极矩与外电场的相互作用使分子结构更加紧密、稳定,进而导致分子能量降低。

### 3.3 外电场对DHR红外光谱的影响
在不同外电场(0-0.025原子单位)下DHR分子优化和频率计算的基础上,采用Multiwfn3.7软件进行频率校正,得到不同电场强度下DHR分子的红外光谱,如图5所示。由于光谱峰数量较多,本文仅选取5个较强的特征峰进行分析。在无电场(F=0原子单位)时,光谱图中3185 cm⁻¹的吸收峰主要由含三个C-H键的六元环上C-H键的伸缩振动产生;1445 cm⁻¹的吸收峰主要由分子中C-H键的面内弯曲振动和C-C骨架振动产生;1230 cm⁻¹的吸收峰主要由分子中C-H键的面内弯曲振动产生;864 cm⁻¹的吸收峰主要由含两个C-H键的六元环上C-H键的面外弯曲振动产生;780 cm⁻¹的吸收峰主要由两个七元环上C-H键的面外弯曲振动产生。

DHR分子在不同外电场作用下表现出强烈的振动斯塔克效应[22-24],其中由C-H键伸缩振动引起的3185 cm⁻¹特征吸收峰和由C-H键面内弯曲振动及C-C骨架振动引起的1445 cm⁻¹特征吸收峰的变化最为明显。在0-0.025原子单位的电场范围内,3185 cm⁻¹特征吸收峰出现明显红移,这是因为随着外电场强度增大,C原子和H原子的电荷分布发生变化,导致分子中C-H键的键长逐渐增加。在0-0.015原子单位的外电场作用下,1445 cm⁻¹特征吸收峰出现明显红移,这是由于随着电场强度增大,C原子和H原子的电荷分布逐渐减少,分子骨架上C-H键的键长逐渐增加。当外电场强度大于0.015原子单位时,外电场改变了分子的键角和键长,使分子发生弯曲,振动所需能量增加,相应基团的键能增大,导致分子在1445 cm⁻¹处的特征吸收峰出现轻微蓝移。

对比发现,在无外电场作用时,分子的伸缩振动和弯曲振动具有一定对称性;而在外电场作用时,大部分伸缩振动和弯曲振动的对称性消失。例如,318 cm⁻¹特征吸收峰对应两个不相关六元环上方C-H键的伸缩振动,施加外电场后,该特征吸收峰仅对应其中一个六元环上C-H键的伸缩振动,这可能是由偶极矩变化引起的。

DHR分子的摩尔吸光系数随外电场强度增大而增大,这是因为外电场使分子内部电子发生整体转移,导致分子内部电子密度降低,原子间相互作用减弱,外层电子更易被激发到高能轨道,从而使跃迁电子数量增加。此外,从占据轨道和非占据轨道(如图6所示)也可证明,外电场强度增强使跃迁电子数量增加,进而导致摩尔吸光系数随外电场强度增大而增大。

### 3.4 外电场对DHR激发态的影响
在DHR分子基态几何结构优化的基础上,采用TD-WB97XD/DEF2-TZVP[25]方法计算不同电场下DHR分子的前20个激发态(振子强度阈值设为0.45),并利用Multiwfn软件和Origin软件绘制分子的紫外吸收光谱,如图7所示(图中S0表示基态,Sx表示第x个激发态)。

在DHR分子的紫外吸收光谱中,存在三个紫外吸收峰,分别位于265.94 nm、394.73 nm和568.43 nm处。265.94 nm处的吸收峰摩尔吸光系数为79565.4747 L·mol⁻¹·cm⁻¹,对应基态到第9、10、13和15激发态的跃迁,贡献率分别为13.86%、24.72%、37.55%和19.99%,是共轭体系π→π*跃迁产生的特征吸收,属于K带;568.43 nm处的吸收峰摩尔吸光系数为3832.4605 L·mol⁻¹·cm⁻¹,对应基态到第一激发态的跃迁,跃迁贡献率为99.58%,是芳环上π→π*跃迁产生的特征吸收,属于B带;394.73 nm处的吸收峰强度极弱,故不做分析。

为探究不同外电场下DHR分子的激发特性,本文对不同电场下主要激发态的轨道跃迁和空穴-电子进行了分析。受篇幅限制,本文主要选取不同电场下S0-S1和S0-S13激发态的跃迁作为研究对象。DHR分子的轨道跃迁情况如表5所示,利用Multiwfn软件绘制的轨道波函数等值面图如图8所示,下文将选取主要部分分析其激发特性。

在轨道波函数等值面图中,绿色等值面代表负相位,蓝色等值面代表正相位,等值面数值为0.025。由表5可知,在同一激发态下,不同电场强度对应的轨道跃迁不同,其贡献率也随之变化。

分析表明,当电场强度增大至0.025原子单位时,基态S0到第一激发态S1和第13激发态S13的跃迁特性发生改变。其中,94、96和97轨道均表现出π轨道特性,98轨道表现出σ*轨道特性,100、101和103轨道也表现出σ*轨道特性。因此,当F=0.025原子单位时,S0→S1和S0→S13的激发特性均为π→σ*跃迁;而当F=0原子单位、0.005原子单位、0.010原子单位、0.015原子单位和0.020原子单位时,S0→S1和S0→S13的主要激发特性均为π→π*跃迁。由此可见,外电场可改变轨道的激发特性。

为明确不同电场下DHR分子的电子激发类型,本文列出了DHR分子S0-S1和S0-S13激发态的空穴-电子相关参数,如表6所示。这些参数包括空穴和电子在整个空间的分布指数(Sr,取值范围为[0,1])、空穴与电子的质心距离指数(D)、电子和空穴的整体平均分布宽度(H)、空穴与电子的分离程度指数(t)、空穴离域指数(HDI)和电子离域指数(EDI)[26]。这些参数是衡量电子激发模式的定量指标,常被用作判断电子激发类型的重要依据。图9为利用Multiwfn软件绘制的空穴-电子图及Chole&Cele图,其中绿色代表电子分布,蓝色代表空穴分布。

由表6可知,在S0-S1激发态下,当F=0原子单位时,D指数为0,t指数小于0,结合空穴-电子图和Chole&Cele图可观察到,空穴与电子的分离程度较小,几乎无分离,因此可推断在无外电场作用时,S0-S1激发态为环结构上高度局域化的π→π*激发。当F=0.025原子单位时,D值达到最大,为6.229 Å,结合空穴-电子图和Chole&Cele图可观察到,空穴与电子发生明显分离,且t指数为3.804,明显大于0,表明空穴与电子分布分离显著,由此可推断当F=0.025原子单位时,S0-S1激发态为π→σ*电荷转移激发。

在S0-S13激发态下,当F=0原子单位时,t值小于0,表明电子与空穴未完全分离;D指数为0,表明空穴与电子质心重合;Sr指数高达0.88308,表明空穴与电子的空间分布均匀,无明显分离。结合空穴-电子图和Chole&Cele图可观察到,空穴与电子的重叠程度较大,因此可推断该激发态为环结构上高度局域化的π→π*激发。当F=0.025原子单位时,D值达到最大,为4.535 Å,t指数为2.070 Å,大于0,表明空穴与电子发生明显分离,结合空穴-电子图和Chole&Cele图也可观察到空穴与电子的分离现象,因此可推断当F=0.025原子单位时,S0-S13激发态同样为π→σ*电荷转移激发。

同理可得出,在0-0.020原子单位的外电场作用下,S0-S1和S0-S13激发态均为π→π*局域激发;当外电场强度大于0.020原子单位时,S0-S1和S0-S13激发态转变为π→σ*电荷转移激发。

综上分析,外电场可改变DHR分子的电子激发特性与激发类型,当F=0.025原子单位时,S0-S1和S0-S13激发态由高度局域化的π→π*激发转变为π→σ*电荷转移激发。DHR分子作为半导体材料,电子激发跃迁可通过空穴-电子对的形式完成。在不同外电场作用下,随着电场强度增大,同一电子激发态表现出不同的激发特性,这对该材料未来在光电化学领域的应用具有重要意义。

### 3.5 外电场对DHR分子空穴迁移率的影响
DHR分子可用于有机场效应晶体管,通过影响有机半导体的分子结构改变分子前线轨道能级,进而影响晶体管的传输性能[27,28]。最高占据分子轨道-最低未占据分子轨道(HOMO-LUMO)能隙是分子的重要特性,与分子导电性密切相关,常被用于衡量分子导电性,能隙越小,导电性越好。对于P型半导体,空穴迁移率与HOMO能级呈负相关,其HOMO能级范围为-4.8 eV至-5.5 eV[29,30]。

在Zhang等人的研究中,DHR分子可用作有机P型半导体,其HOMO能级为-4.98 eV,空穴迁移率为0.082 cm²·V⁻¹·s⁻¹。为探究外电场是否可通过改变分子HOMO能级来改变其空穴迁移率,本文计算了DHR分子的轨道能级,结果如表7和图10所示。

由图10可知,在0-0.020原子单位的电场范围内,DHR分子的HOMO轨道能级波动极小,在0.020-0.025原子单位的电场范围内急剧下降;LUMO轨道能级和HOMO-LUMO能隙在0-0.020原子单位的电场范围内逐渐减小,在0.020-0.025原子单位的电场范围内也急剧下降。当电场强度达到0.020原子单位时,随着外电场强度继续增大,LUMO几乎完全脱离分子,体系无法再稳定结合多余电子[31],分子轨道能级显著降低。此外,随着电场强度增大,电子云迁移使DHR分子芳环的大π键减弱,π→π*跃迁能量降低,导致分子的HOMO-LUMO能隙急剧减小。

根据前线轨道理论[32],分子的HOMO能级越高,电子的活性和亲电性越强;LUMO能级越低,越容易吸引电子,亲核性越强;分子能隙大小决定了分子参与化学反应的能力。随着外电场强度增大,DHR分子的HOMO能级降低,电子活性减弱;LUMO轨道能级降低,分子更易获得电子,亲核性增强;分子能隙减小,表明电子从HOMO跃迁到LUMO所需能量减少,电子更易从HOMO跃迁到LUMO,使HOMO产生空穴。

HOMO轨道能级的变化会改变轨道的空穴迁移率,进而影响P型半导体的性能。当F<0.015原子单位时,HOMO能级随电场强度增大而升高,导致DHR的空穴迁移率降低;当F>0.015原子单位时,HOMO能级急剧下降,使DHR的空穴迁移率升高。这是因为当电场强度小于0.015原子单位时,电子跃迁能力较弱,分子相对稳定;而当分子受到较强外电场作用时,分子活性增强,易激发电子从HOMO轨道跃迁到最高未占据分子轨道(HUMO)。当F=0.025原子单位时,DHR的HOMO能级为-5.0660 eV,小于-4.98 eV,表明其空穴迁移率大于0.082 cm²·V⁻¹·s⁻¹。因此,可通过调节外电场强度改变DHR的HOMO能级,进而改变DHR P型半导体的空穴迁移率,改善其半导体性能。

## 4 结论
本文采用B3LYP/6-311G(d,p)方法对DHR分子进行结构优化,在此基础上计算了不同外电场对DHR分子几何结构、偶极矩、能量、前线轨道能级及红外光谱的影响,并采用TD-DFT WB97XD/DEF2-TZVP方法分析了不同外电场下DHR分子的激发态。结果如下:
1. 随着外电场强度增大,原子周围的电子向与电场方向相反的方向重新分布,分子内电场方向随电子云变化而改变。当外电场方向与电子云方向相反时,原子间电场减弱,键长增大;当外电场方向与电子云方向一致时,原子间电场增强,键长减小。在外电场作用下,分子电荷重新分布导致分子受力不均匀,进而使分子发生弯曲变形。
2. DHR分子在不同外电场作用下表现出强烈的振动斯塔克效应。随着外电场强度增大,当分子振动频率降低、键长增大时,出现红移现象;当振动频率升高、键长减小时,出现蓝移现象。同时,分子的摩尔吸光系数随外电场强度增大而增大。
3. 外电场可改变DHR分子的激发特性与激发类型。当外电场强度小于0.025原子单位时,DHR分子的激发态为π→π*局域激发;当F=0.025原子单位时,分子激发态由高度局域化的π→π*激发转变为π→σ*电荷转移激发。
4. 随着外电场强度增大,DHR分子的亲电性减弱,亲核性增强。外电场可通过改变DHR分子的HOMO能级来改变其空穴转移速率。

综上所述,在不同外电场作用下,DHR分子的几何结构、轨道类型、红外光谱及激发态性质均会发生变化。此外,通过施加外电场降低HOMO能级,可提高DHR分子的空穴迁移率。这些结果为未来DHR分子作为有机P型半导体材料在光电化学领域的研究提供了理论依据。

