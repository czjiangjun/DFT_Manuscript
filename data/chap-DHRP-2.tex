\chapter{不同外电场下DHR的化学结构与激发特性}
\section{DHR及前驱体前驱体化合物的晶体结构}
根据文献\cite{ACIE59-3529_2020},DHR和它的前驱体结构分别列于表\ref{Tab:DRH-structure},\ref{Tab:DRH-1-structure},\ref{Tab:DRH-2-structure},\ref{Tab:DRH-3-structure}。
%\subsection{DHR的X射线晶体学数据}
\begin{table}[h!]
    \centering
    \begin{tabular}{ll}
        \hline
	实验式 & \ch{C30H14} \\
        \hline
        分子量 & 374.41 \\
        \hline
        温度/K & 169.99~(11) \\
        \hline
        晶系 & 单斜晶系 \\
        \hline
        空间群 & $P2_1/c$ \\
        \hline
	晶胞参数 & a = 10.5545~(5)\AA~~~$\alpha = 90^{\circ}$ \\
		&b = 3.9469~(2)\AA~~~$\beta = 97.509~(5)^{\circ}$  \\
		&c = 20.5542~(10)\AA~~~$\gamma = 90^{\circ}$\\
        \hline
	体积/$\text{\AA}^3$ & 848.89~(7) \\
        \hline
        Z值 & 2 \\
        \hline
	计算密度$\rho_{\mathrm{calc}}$/$\mathrm{g}\cdot\mathrm{cm}^{-3}$ & 1.465 \\
        \hline
	吸收系数$\mu$/$\mathrm{mm}^{-1}$ & 0.638 \\
        \hline
        F~(000) & 388.0 \\
        \hline
	晶体尺寸/$\mathrm{mm}^3$ & $0.1\times0.05\times0.01$ \\
        \hline
	辐射源 & Cu-$\mathrm{K}_{\alpha}$~($\lambda=1.54184$) \\
        \hline
	数据收集的$2\theta$范围/$^{\circ}$ & 8.678至131.95 \\
        \hline
	指标范围 & $-12\leqslant h\leqslant12, -4\leqslant k\leqslant4, -24\leqslant l\leqslant16$ \\
        \hline
        收集的反射点数 & 7988 \\
        \hline
	独立反射点数 & 1486 $[R_{\mathrm{int}}=0.0485, R_{\mathrm{sigma}}=0.0429]$ \\
        \hline
        数据/限制条件/参数 & 1486/67/272 \\
        \hline
	拟合优度~($\mathrm{F}^2$) & 1.043 \\
        \hline
        最终R指数$[I\geqslant2\sigma~(I)]$ & $R_1=0.0405$,$wR_2=0.0959$ \\
        \hline
        最终R指数~(所有数据) & $R_1=0.0680$,$wR_2=0.1123$ \\
        \hline
	最大差值峰/孔/$\mathrm{e}\cdot\text{\AA}^{-3}$ & 0.10/-0.14 \\
        \hline
    \end{tabular}
    \caption{DHR的X射线晶体学数据}
    \label{Tab:DRH-structure}
\end{table}


%\subsection{前驱体化合物1的X射线晶体学数据}
\begin{table}[h!]
    \centering
    \begin{tabular}{ll}
        \hline
	实验式 & \ch{C30H22} \\
        \hline
        分子量 & 382.47 \\
        \hline
        温度/K & 173.15 \\
        \hline
        波长/\AA & 0.71073 \\
        \hline
        晶系 & 单斜晶系 \\
        \hline
        空间群 & $P2_1/c$ \\
        \hline
	晶胞参数 & a = 7.6867~(15)\AA~~~$\alpha = 90^{\circ}$ \\
		&b = 10.954~(2)\AA~~~$\beta = 94.39~(3)^{\circ}$  \\
		&c = 23.080~(5)\AA~~~$\gamma = 90^{\circ}$\\
        \hline
	体积/$\text{\AA}^3$ & 1937.5~(7) \\
        \hline
        Z值 & 4 \\
        \hline
        计算密度$\rho_{\mathrm{calc}}$/$\mathrm{g}\cdot\mathrm{cm}^{-3}$ & 1.311 \\
        \hline
	吸收系数/$\mathrm{mm}^{-1}$ & 0.074 \\
        \hline
        F~(000) & 808 \\
        \hline
	晶体尺寸/$\mathrm{mm}^3$ & $0.337\times0.291\times0.035$ \\
        \hline
	数据收集的$\theta$范围/$^{\circ}$ & 2.059至27.489 \\
        \hline
	指标范围 & $-9\leqslant h\leqslant 9, -14\leqslant k\leqslant 14, -26\leqslant l\leqslant 29$ \\
        \hline
        收集的反射点数 & 22413 \\
        \hline
	独立反射点数 & 4406 $[\mathrm{R}_{\mathrm{int}}=0.0514]$ \\
        \hline
	$\theta=25.242^{\circ}$时的完整性 & 99.5\% \\
        \hline
        吸收校正 & 基于等效反射的半经验校正 \\
        \hline
        最大/最小透射率 & 1.00000/0.86939 \\
        \hline
        精修方法 & 基于$F^2$的全矩阵最小二乘法 \\
        \hline
        数据/限制条件/参数 & 4406/0/271 \\
        \hline
        拟合优度~($F^2$) & 1.288 \\
        \hline
        最终R指数$[I\geqslant 2\sigma~(I)]$ & $\mathrm{R}_1=0.0663, \mathrm{wR}_2=0.1281$ \\
        \hline
	最终R指数~(所有数据) & $\mathrm{R}_1=0.0695, \mathrm{wR}_2=0.1296$ \\
        \hline
        消光系数 & 无 \\
        \hline
	最大差值峰/孔/$\mathrm{e}\cdot\text{\AA}^{-3}$ & 0.235/-0.158 \\
        \hline
    \end{tabular}
    \caption{前驱体化合物1的X射线晶体学数据}
    \label{Tab:DRH-1-structure}
\end{table}

%\subsection{前驱体化合物2的X射线晶体学数据}
\begin{table}[h!]
    \centering
    \begin{tabular}{ll}
        \hline
	实验式 & \ch{C30H18} \\
        \hline
        分子量 & 378.44 \\
        \hline
        温度/K & 173.15 \\
        \hline
        波长/\AA & 0.71073 \\
        \hline
        晶系 & 单斜晶系 \\
        \hline
        空间群 & $P2_1/c$ \\
        \hline
	晶胞参数 & a = 10.141~(3)\AA~~~$\alpha = 90^{\circ}$ \\
		&b = 13.470~(4)\AA~~~$\beta = 96.330~(3)^{\circ}$  \\
		&c = 13.809~(4)\AA~~~$\gamma = 90^{\circ}$\\
		体积/$\text{ÅA}^3$ & 1874.6~(9) \\
        \hline
        Z值 & 4 \\
        \hline
        计算密度$\rho_{\mathrm{calc}}$/$\mathrm{g}\cdot\mathrm{cm}^{-3}$ & 1.341 \\
        \hline
	吸收系数/$\mathrm{mm}^{-1}$ & 0.076 \\
        \hline
        F~(000) & 792 \\
        \hline
	晶体尺寸/$\mathrm{mm}^3$ & $0.352\times0.261\times0.109$ \\
        \hline
	数据收集的θ范围/$^{\circ}$ & 3.039至27.479 \\
        \hline
	指标范围 & $-13\leqslant h\leqslant12, -17\leqslant k\leqslant 16, -17\leqslant l\leqslant 17$ \\
        \hline
        收集的反射点数 & 15054 \\
        \hline
	独立反射点数 & 4273 $[\mathrm{R}_{\mathrm{int}}=0.0527]$ \\
        \hline
	$\theta=25.242^{\circ}$时的完整性 & 99.5\% \\
        \hline
        吸收校正 & 基于等效反射的半经验校正 \\
        \hline
        最大/最小透射率 & 1.00000/0.75431 \\
        \hline
        精修方法 & 基于$F^2$的全矩阵最小二乘法 \\
        \hline
        数据/限制条件/参数 & 4273/0/271 \\
        \hline
        拟合优度~($F^2$) & 1.164 \\
        \hline
	最终R指数$[I\geqslant 2\sigma~(I)]$ & $\mathrm{R}_1=0.0581, \mathrm{wR}_2=0.1197$ \\
        \hline
	最终R指数~(所有数据) & $\mathrm{R}_1=0.0662, \mathrm{wR}_2=0.1238$ \\
        \hline
        消光系数 & 无 \\
        \hline
	最大差值峰/孔/$\mathrm{e}\cdot\text{\AA}^{-3}$ & 0.271/-0.180 \\
        \hline
    \end{tabular}
    \caption{前驱体化合物2的X射线晶体学数据}
    \label{Tab:DRH-2-structure}
\end{table}

%\subsection{前驱体化合物3的X射线晶体学数据}
\begin{table}[h]
    \centering
    \begin{tabular}{ll}
        \hline
        实验式 & $\mathrm{C}_{15}\mathrm{H}_{9}$ \\
	\hline
        式量 & 189.22 \\
	\hline
        温度/K & 173.15 \\
	\hline
        波长 & 0.71073 \AA \\
	\hline
        晶体系统 & 单斜晶系 \\
	\hline
        空间群 & $P2_1/n$ \\
	\hline
	晶胞参数 & a = 10.012~(3)\AA~~~$\alpha = 90^{\circ}$ \\
		&b = 8.313~(3)\AA~~~$\beta = 110.939~(6)^{\circ}$  \\
		&c = 11.611~(4)\AA~~~$\gamma = 90^{\circ}$\\
		\hline
		体积 & $902.6~(5) \text{\AA}^3$ \\
	\hline
        $Z$ 值 & 4 \\
	\hline
        计算密度$\rho_{\mathrm{calc}}$/$\mathrm{g}\cdot\mathrm{cm}^{-3}$ & 1.393  \\
	\hline
	吸收系数/$\mathrm{mm}^{-1}$ & 0.079  \\
	\hline
        $F~(000)$ & 396 \\
	\hline
        晶体尺寸/$\mathrm{mm}^3$ & $0.522 \times 0.315 \times 0.159$ \\
	\hline
	数据收集的 $\theta$ 范围/$^{\circ}$& 3.347 - 27.495 \\
	\hline
        指标范围 & $-12 \leqslant h \leqslant 12$, $-10 \leqslant k \leqslant 9$, $-15 \leqslant l \leqslant 15$ \\
	\hline
        收集到的反射点 & 9824 \\
	\hline
        独立反射点 & 2055 [$\mathrm{R_{int}} = 0.0409$] \\
	\hline
        对 $\theta = 25.242^\circ$ 的完整性 & 99.3\% \\
	\hline
        吸收校正 & 基于等效反射的半经验法 \\
	\hline
        最大和最小透射率 & 1.00000 和 0.79415 \\
	\hline
        精修方法 & 基于 $F^2$ 的全矩阵最小二乘法 \\
	\hline
        数据/约束/参数 & 2055 / 0 / 136 \\
	\hline
        拟合优度 $\mathrm{GOF}$ on $F^2$ & 1.142 \\
	\hline
	最终 $\mathrm{R}$ 指数 $[\mathrm{I}>2\sigma~(\mathrm{I})]$ & $\mathrm{R_1} = 0.0495$, $\mathrm{wR_2} = 0.1197$ \\
	\hline
        最终 $\mathrm{R}$ 指数 [所有数据] & $\mathrm{R_1} = 0.0514$, $\mathrm{wR_2} = 0.1223$ \\
	\hline
        消光系数 & 无 \\
	\hline
	最大差值峰/孔/$\mathrm{e}\cdot \text{\AA}^{-3}$  & 0.232 和 -0.160 \\
        \hline
    \end{tabular}
    \caption{化合物 3 的 X 射线晶体学数据}
    \label{Tab:DRH-3-structure}
\end{table}
%\subsection{6.5 5/7元环对分子的平面化作用}
\newpage
\subsection{DHR的结构分析}
根据文献\cite{ACIE59-3529_2020},构象体$i$的键长、键角可以分别表示为图\ref{Fig:DHR_Bond-Angle}和图\ref{Fig:DHR_Bond-Angle-1},数据列于表\ref{tab:dhr_conformer_i_analysis}。
%\subsubsection{Table S3  DHR 构象体 $i$ 的选定键长和键角总结与分析}
\begin{figure}[h!]
\centering
\vspace*{-0.1in}
\includegraphics[height=4.0in]{Figures/DHR_DFT-bond_angle.png}
\caption{DHR 构象体 i 的环结构键长与键角结构.}%
\label{Fig:DHR_Bond-Angle}
\end{figure}
\begin{table}[h]
    \centering
    \begin{tabular}{cccccc}
        \hline
	环 & 平均键长 ~(\AA) & 最长键长 ~(\AA) & 最短键长 ~(\AA) & 最大键角 ~($^{\circ}$) & 最小键角 ~($\circ$) \\
        \hline
        环 A ~(C) & 1.403 & 1.436  & 1.381  & 125.02 & 115.15\\
	(六元环)  &  &(C7'-C6')  &(C14-C2)  &($\angle$C14C15C7') & ($\angle$C15C7'C6') \\
	\hline
        环 B & 1.403 & 1.410 & 1.399 & 123.45 & 116.53 \\
        环 (六元环) &  & (C6'-C5') & (C1-C6') & ($\angle$C5C1C6') & ($\angle$C5C6C1') \\
	\hline
        环 E ~(H) & 1.391 & 1.407  & 1.371 & 122.33 & 116.12  \\
        环 E (六元环) & & (C11-C12) & (C12-C13) & ($\angle$C3C4C10) & ($\angle$C4C10C11) \\
	\hline
        环 D ~(G) & 1.427 & 1.487 & 1.333 & 134.11 & 123.29  \\
        环 (七元环) & & (C9-C10) & (C8-C9) & ($\angle$C4C5C6) & ($\angle$C4C10C9) \\
	\hline
        环 F ~(I) & 1.427 & 1.479 & 1.394 & 112.55 & 104.65 \\
        环 (五元环) & & (C2-C3) & (C2-C1) & ($\angle$C5C1C2) & ($\angle$C1C5C4) \\
        \hline
    \end{tabular}
    \caption{DHR 构象体 i 的环结构键长与键角数据}
    \label{tab:dhr_conformer_i_analysis}
\end{table}
\begin{figure}[h!]
\centering
\vspace*{-0.1in}
\includegraphics[height=4.0in]{Figures/DHR_DFT-bond_angle-1.png}
\caption{DHR 构象体 i 中选定的键角~(单位:~°)。采用线框顶视图展示,氢原子已省略以清晰呈现环结构。关键键角数值标注于对应位置。}
\label{Fig:DHR_Bond-Angle-1}
\end{figure}

%\begin{figure}[h]
%    \centering
%    \includegraphics[width=0.9\textwidth]{figure_S6.pdf} % 需替换为实际图片路径
%    \caption{DHR 构象体 i 中选定的键角~(单位:~°)。采用线框顶视图展示,氢原子已省略以清晰呈现环结构。关键键角数值标注于对应位置。}
%    \label{fig:dhr_conformer_i_bond_angles}
%\end{figure}

\subsubsection{Table S4  DHR 构象体 $ii$ 的选定键长和键角总结与分析}
\begin{figure}[h!]
\centering
\vspace*{-0.1in}
\includegraphics[height=4.0in]{Figures/DHR_DFT-bond_angle-2.png}
\caption{DHR 构象体 ii 的环结构键长与键角结构.}%
\label{Fig:DHR_Bond-Angle-2}
\end{figure}
类似地,构象体$ii$的键长、键角可以分别表示为图\ref{Fig:DHR_Bond-Angle-2}和图\ref{Fig:DHR_Bond-Angle-3},数据列于表\ref{tab:dhr_conformer_ii_analysis}。
\begin{table}[h]
    \centering
    \begin{tabular}{cccccc}
        \hline
        环 & 平均键长 ~(Å) & 最长键长 ~(Å) & 最短键长 ~(Å) & 最大键角 ~(°) & 最小键角 ~(°) \\
        \hline
        环 A ~(C) & 1.407 & 1.444 & 1.362 & 123.98 & 115.69 \\
        环 (六元环) & &(C7'-C6') & (C14-C2) & ($\angle$C2C1C6') & ($\angle$C15C7'C6') \\
        \hline
        环 B & 1.400 & 1.412 & 1.392 & 124.61 & 115.48 \\
        环 (六元环) & & (C6'-C5') & (C1-C5) & ($\angle$C5C1C6') & ($\angle$C1C6'C5') \\
        \hline
        环 E ~(H) & 1.397 & 1.422 & 1.379 & 122.70 & 116.22 \\
        环 (六元环) & & (C10-C11) & (C3-C13) & ($\angle$C3C4C10) & ($\angle$C4C10C11) \\
        \hline
        环 D ~(G) & 1.431 & 1.458 & 1.354 & 134.33 & 123.77 \\
        环 (七元环) & & (C9-C10) & (C8-C9) & ($\angle$C4C5C6) & ($\angle$C4C10C9) \\
        \hline
        环 F ~(I) & 1.426 & 1.457 & 1.392 & 111.36 & 105.12 \\
        环 (五元环) & & (C2-C3) & (C1-C5) & ($\angle$C2C1C5) & ($\angle$C1C5C4) \\
        \hline
    \end{tabular}
    \caption{DHR 构象体 ii 的环结构键长与键角分析}
    \label{tab:dhr_conformer_ii_analysis}
\end{table}
\begin{figure}[h!]
\centering
\vspace*{-0.1in}
\includegraphics[height=4.0in]{Figures/DHR_DFT-bond_angle-3.png}
\caption{DHR 构象体 ii 中选定的键角~(单位:~°)。采用线框顶视图展示,氢原子已省略以清晰呈现环结构。关键键角数值标注于对应位置。}
\label{Fig:DHR_Bond-Angle-3}
\end{figure}

\begin{figure}[h!]
\centering
\vspace*{-0.1in}
\includegraphics[height=3.0in]{Figures/DHR_DFT-packing.png}
\caption{DHR 不同构象体~(构象体 i 和构象体 ii)之间的选定堆积模式。图中标记\textcircled{1}和\textcircled{2}的分子分别代表构象体 i 或 ii,展示了单胞中两种分子间的相互作用。关键分子间距离~(单位:~\AA)标注为 d:~d=3.002~(a)、3.153~(b)、3.194~(c)、2.705~(d)、2.685~(e)、2.824~(f)、2.684~(g)、2.880~(h)。}
\label{Fig:DHR_Packing}
\end{figure}
%\begin{figure}[h]
%    \centering
%    \includegraphics[width=0.9\textwidth]{figure_S7.pdf} % 需替换为实际图片路径
%    \caption{DHR 构象体 ii 中选定的键角~(单位:~°)。采用线框顶视图展示,氢原子已省略以清晰呈现环结构。关键键角数值标注于对应位置。}
%    \label{fig:dhr_conformer_ii_bond_angles}
%\end{figure}
%
%\begin{figure}[h]
%    \centering
%    \includegraphics[width=0.6\textwidth]{figure_S8.pdf} % 需替换为实际图片路径
%    \caption{薁~(azulene)的选定键长和键角总结。数据来源于已报道文献\cite{ref14},关键键长~(单位:~Å)和键角~(单位:~°)标注于对应结构位置。}
%    \label{fig:azulene_bond_data}
%\end{figure}
%
%\begin{figure}[h]
%    \centering
%    \includegraphics[width=0.8\textwidth]{figure_S9.pdf} % 需替换为实际图片路径
%    \caption{DHR 不同构象体~(构象体 i 和构象体 ii)之间的选定堆积模式。图中标记①和②的分子分别代表构象体 i 或 ii,展示了单胞中两种分子间的相互作用。关键分子间距离~(单位:~Å)标注为 d:~d=3.002~(a)、3.153~(b)、3.194~(c)、2.705~(d)、2.685~(e)、2.824~(f)、2.684~(g)、2.880~(h)。}
%    \label{fig:dhr_packing_modes}
%\end{figure}
%
%\begin{figure}[h]
%    \centering
%    \includegraphics[width=0.7\textwidth]{figure_S10.pdf} % 需替换为实际图片路径
%    \caption{DHR 中不同环的计算 NICS~(1)~(核独立化学位移)值。各环~(A、B、D、E、F、G、H)对应的 NICS~(1) 数值标注于环中心位置,单位:~ppm。}
%    \label{fig:dhr_nics_values}
%\end{figure}


\section*{12. Calculations of the reorganization energy and transfer integrals}
%\begin{figure}[h]
%    \centering
%    \includegraphics[width=0.7\textwidth]{figure_S21.pdf} % 需替换为实际图片路径
%    \caption{DHR 分子前线轨道的计算能级图。标注了 HOMO、HOMO-1、LUMO、LUMO+1 等轨道的能级位置~(单位:~eV),并给出计算得到的第一、第二和第三电离势分别为 4.96、5.44、6.87 eV。}
%    \label{fig:dhr_frontier_energy_levels}
%\end{figure}
%
%\begin{figure}[h]
%    \centering
%    \includegraphics[width=0.8\textwidth]{figure_S22.pdf} % 需替换为实际图片路径
%    \caption{DHR 的晶体结构及用于转移积分计算的分子对示意图。考虑晶体无序性,构象体 i 和构象体 ii 分别用灰色和橙色标记,虚线框标出选定的分子对。}
%    \label{fig:dhr_crystal_transfer_integral}
%\end{figure}

\begin{table}[h]
    \centering
    \begin{tabular}{ccccccccc}
        \hline
        作用模式 & \multicolumn{4}{c}{人字形~(herringbone)} & \multicolumn{4}{c}{平行~(parallel)} \\
        \cline{2-9}
	分子对 & \textcircled{1}i--\textcircled{2}i & \textcircled{1}ii--\textcircled{2}ii & \textcircled{1}i--\textcircled{2}ii & \textcircled{1}ii--\textcircled{2}i & \textcircled{2}--\textcircled{3}i & \textcircled{2}ii--\textcircled{3}ii & \textcircled{2}i--\textcircled{3}ii & \textcircled{2}ii--\textcircled{3}i \\
        \hline
        t$_h$ ~(meV) & 2.9 & 0.4 & 3.9 & 1.9 & 75.2 & 110.6 & 13.6 & 13.6 \\
        \hline
    \end{tabular}
    \caption{从晶体结构中选取的分子对的空穴转移积分计算结果。所有计算在 DFT-ωB97XD/6-31G** 水平下结合片段轨道法和基组正交化程序进行。}
    \label{tab:dhr_transfer_integrals}
\end{table}

\begin{verbatim}
基于 DHR 基态中性态和离子态的优化几何结构,采用四点势能面法计算空穴转移的重组能为 251 meV。结合极化连续介质模型~(PCM)进一步计算第一、第二和第三电离势,介电常数 ε 设为 20 以考虑溶剂化效应。所有 DFT 计算均使用 Gaussian 16 程序包完成。
\end{verbatim}

%
\section*{8. Simulated absorptions and emissions of DHR}
\subsection{8.1 The optimized geometries and electronic structures of frontier orbitals of DHR}
%\begin{figure}[h]
%    \centering
%    \includegraphics[width=0.8\textwidth]{figure_S13.pdf} % 需替换为实际图片路径
%    \caption{DHR 在 S$_0$、S$_1$ 和 S$_2$ 态下的优化几何结构。所有结构均在 CASSCF/6-31G*/~(12,12) 水平下进行优化,并约束对称性。}
%    \label{fig:dhr_optimized_geometries}
%\end{figure}
%
%\begin{figure}[h]
%    \centering
%    \includegraphics[width=0.9\textwidth]{figure_S14.pdf} % 需替换为实际图片路径
%    \caption{DHR 参与跃迁的前线轨道电子云密度分布。展示了 HOMO、HOMO-1、LUMO 和 LUMO+1 轨道的电子云分布特征,不同颜色代表电子云密度高低~(通常红色为高电子密度,蓝色为低电子密度)。}
%    \label{fig:dhr_frontier_orbitals}
%\end{figure}

\begin{table}[h]
    \centering
    \begin{tabular}{ccccc}
        \hline
        结构 & 态 & 能量 ~(eV) & 振子强度 & 跃迁特性 \\
        \hline
        \multirow{2}{*}{S$_0$-min} & S$_1$ ~(Bu) & 1.99 & 0.129 & HOMO → LUMO ~(44\%), HOMO-1 → LUMO ~(16\%), HOMO → LUMO+1 ~(10\%) \\
        (吸收) & S$_1$ ~(Bu) & 1.99 & 0.129 & HOMO → LUMO ~(44\%), HOMO-1 → LUMO ~(16\%), HOMO → LUMO+1 ~(10\%) \\
        & S$_1$ ~(Bu) & 1.99 & 0.129 & HOMO → LUMO ~(44\%), HOMO-1 → LUMO ~(16\%), HOMO → LUMO+1 ~(10\%) \\
        & S$_2$ ~(Bu) & 3.05 & 0.200 & HOMO → LUMO ~(36\%), HOMO-1 → LUMO ~(16\%), HOMO → LUMO+1 ~(18\%) \\
        & S$_2$ ~(Bu) & 3.05 & 0.200 & HOMO → LUMO ~(36\%), HOMO-1 → LUMO ~(16\%), HOMO → LUMO+1 ~(18\%) \\
        & S$_2$ ~(Bu) & 3.05 & 0.200 & HOMO → LUMO ~(36\%), HOMO-1 → LUMO ~(16\%), HOMO → LUMO+1 ~(18\%) \\
        \multirow{2}{*}{S$_1$-min~(发射)} & S$_1$ ~(Bu) & 1.69 & 0.159 & HOMO → LUMO ~(67\%), HOMO-1 → LUMO ~(10\%) \\
        \multirow{2}{*}{S$_1$-min~(发射)} & S$_1$ ~(Bu) & 1.69 & 0.159 & HOMO → LUMO ~(67\%), HOMO-1 → LUMO ~(10\%) \\
        \multirow{2}{*}{S$_1$-min~(发射)} & S$_1$ ~(Bu) & 1.69 & 0.159 & HOMO → LUMO ~(67\%), HOMO-1 → LUMO ~(10\%) \\
        & S$_2$ ~(Bu) & 2.60 & 0.096 & HOMO → LUMO ~(10\%), HOMO-1 → LUMO ~(40\%), HOMO → LUMO+1 ~(16\%) \\
        & S$_2$ ~(Bu) & 2.60 & 0.096 & HOMO → LUMO ~(10\%), HOMO-1 → LUMO ~(40\%), HOMO → LUMO+1 ~(16\%) \\
        & S$_2$ ~(Bu) & 2.60 & 0.096 & HOMO → LUMO ~(10\%), HOMO-1 → LUMO ~(40\%), HOMO → LUMO+1 ~(16\%) \\
        \multirow{2}{*}{S$_2$-min~(发射)} & S$_1$ ~(Bu) & 1.78 & 0.185 & HOMO → LUMO ~(60\%), HOMO-1 → LUMO ~(11\%) \\
        \multirow{2}{*}{S$_2$-min~(发射)} & S$_1$ ~(Bu) & 1.78 & 0.185 & HOMO → LUMO ~(60\%), HOMO-1 → LUMO ~(11\%) \\
        \multirow{2}{*}{S$_2$-min~(发射)} & S$_1$ ~(Bu) & 1.78 & 0.185 & HOMO → LUMO ~(60\%), HOMO-1 → LUMO ~(11\%) \\
        & S$_2$ ~(Bu) & 2.93 & 0.116 & HOMO → LUMO ~(19\%), HOMO-1 → LUMO ~(29\%), HOMO → LUMO+1 ~(22\%) \\
        & S$_2$ ~(Bu) & 2.93 & 0.116 & HOMO → LUMO ~(19\%), HOMO-1 → LUMO ~(29\%), HOMO → LUMO+1 ~(22\%) \\
        & S$_2$ ~(Bu) & 2.93 & 0.116 & HOMO → LUMO ~(19\%), HOMO-1 → LUMO ~(29\%), HOMO → LUMO+1 ~(22\%) \\
        \hline
    \end{tabular}
    \caption{DHR 的计算激发能、振子强度和跃迁特性。激发态能量和振子强度基于 CASSCF 轨道在 CASPT2/6-31G*/~(12,12) 水平下计算。}
    \label{tab:dhr_excitation_properties}
\end{table}

\subsection{8.2 Simulated emission}
%\begin{figure}[h]



分子的最高占据分子轨道~(HOMO)和最低未占据分子轨道~(LUMO)能量通过以下公式估算:~
\[
\text{HOMO} = -~(E^{\text{ox}}_{\text{onset}} + 4.8)\text{eV}
\]
\[
\text{LUMO} = -~(E^{\text{red}}_{\text{onset}} + 4.8)\text{eV}
\]

通过优化DHR的中性基态和离子态几何结构,计算出其空穴传输的重组能为251 meV。较小的重组能与DHR的刚性结构~(所有共轭环均稠合)相符。进一步计算了电荷转移积分,结果见表S6~(支持信息):~分子间π-π相互作用和C∙∙∙H相互作用相关的转移积分受分子无序性影响——构象体i或ii的同型分子对之间的π-π相互作用转移积分,明显大于构象体i与ii的异型分子对。这一计算结果表明,分子无序性不利于高效电荷传输,与DHR晶体表现出较低电荷迁移率的实验事实一致。

 芳香性与电子离域性计算
计算了DHR中不同环的Z轴1Å处各向同性化学屏蔽表面~(ICSS~(1)zz)及核独立化学位移~(NICS~(1))值。需注意,ICSS~(1)zz值越正,表明化学屏蔽作用越强,芳香性越高。如图3a所示,化学屏蔽强度顺序为:~环B > 环E/H > 环A/C > 环F/I > 环D/G。此外,NICS~(1)计算结果为:~环B~(-25.7)、环E/H~(-22.4)、环A/C~(-18.8)、环F/I~(-4.3)、环D/G~(12.6)~(见图S10,支持信息),表明六元环具有芳香性,五元环呈弱芳香性,七元环呈弱反芳香性。这与¹H NMR谱中“六元环上质子的化学位移比七元环上质子更显著低场位移”的现象一致~(见图S27,支持信息)。

π电子定域化轨道指示函数~(LOL-π)计算显示,E/F/B/I/H环内的电子离域性更强~(图3b);相比之下,D环中C8-C9键和A环中C2-C14键的电子主要呈定域状态,这与这些环中键长的交替性特征一致~(见表S1-S4,支持信息)。

[图3. DHR的ICSS~(1)zz图~(a)与LOL-π图~(b)]~(注:~原文含图,此处为图注翻译)
a)Z轴1Å处各向同性化学屏蔽表面~(ICSS~(1)zz):~深橙色区域表示芳香性强;b)π电子定域化轨道指示函数~(LOL-π):~橙色表示键内π电子呈定域状态

%## 摘要

%## 关键词
%双环庚基红荧烯(DHR);外电场;分子结构;激发特性

%## 2 计算方法
外电场作用下分子体系的哈密顿量H可表示为[8-11]:
\[H=H_{0}+H_{int}\]
其中,\(H_{0}\)为无电场时的哈密顿量,\(H_{int}\)为外电场与分子体系相互作用时的哈密顿量。在偶极近似下,\(H_{int}\)可表示为[12,13]:
\[H_{int}=\mu\cdot F\]
式中,μ为分子电偶极矩,F为偶极电场强度。

本文采用B3LYP/6-311G(d,p)方法对DHR分子进行结构优化和频率计算。在优化基础上,运用相同的方法和基组,计算沿分子X轴施加不同电场(0-0.025原子单位)对DHR分子几何结构、能量、偶极矩、前线轨道能级及红外光谱的影响。最后,采用含时密度泛函理论(TD-DFT)中的WB97XD/DEF2-TZVP方法,计算不同电场下DHR分子的前20个激发态,并对主要激发态进行空穴-电子分析,以明确外电场下DHR分子的激发特性。上述所有计算与分析均通过Gaussian 09软件包和Multiwfn3.7多功能波函数软件完成[14,15]。

%## 3 计算结果与分析
%### 3.1 DHR分子基态的几何结构
DHR由碳元素和氢元素组成,分子式为C₃₀H₁₄。采用DFT/B3LYP方法在6-311G(d,p)基组水平下对DHR分子的几何结构进行优化,结果收敛且无虚频,表明优化后的结构合理。DHR基态优化结构如图1所示,该分子为平面结构,由两个五元环、两个七元环和四个六元环构成。

%### 3.2 外电场对DHR基态性质的影响
在无电场优化的基础上,采用相同方法和基组沿分子X轴施加不同强度电场(0-0.025原子单位),探究DHR分子结构及其特性的变化。DHR分子键长随电场强度的变化如表1所示。由于DHR分子的键长和键角数量较多,本文仅选取主要键长和键角进行分析。DHR分子为极性分子,随着外电场强度的增大,键长和键角对电场强度表现出明显的依赖性,具体如图2所示。表2为无外电场时DHR分子的原子电荷分布,表3和图3分别为不同外电场下DHR分子的键角数据及相应关系图。

由图2可知,在0-0.025原子单位的外电场作用下,C-C键(4,7)、C-C键(10,11)、C-C键(13,14)、C-H键(26,41)、C-H键(1,40)、C-H键(17,20)、C-H键(18,21)和C-H键(33,43)的键长随外电场强度增强而增大,其中C-C键(10,11)、C-H键(26,41)和C-H键(33,43)的键长随外电场强度增大而急剧增加;C=C键(23,24)的键长随外电场强度增大无明显变化;C-C键(3,6)、C-C键(4,5)、C-C键(24,25)和C-H键(32,44)的键长随外电场强度增大而减小,其中C-C键(3,6)和C-C键(4,5)的键长减小更为显著。这些现象可通过外电场与分子内电场的叠加效应进行定性解释[18]。

随着外电场强度增大,局域电子转移使C4-C7、C7-C8、C10-C11、C26-H41、C1-C40、C17-H20、C18-H21和C33-H43之间的电场减弱,进而导致键长增大:由表2可知,C10-C11中的C11电负性较强,具有较强的吸电子能力,使得分子内电场方向从C11指向C10。分子内电场方向与外电场方向的夹角为钝角,导致叠加电场减弱;C26-H41和C33-H43中C原子的电负性较强,原本偏向C原子的电子云向H原子偏移,使得分子内电场方向与外电场方向相反,因此外电场对分子内电场具有较强的削弱作用,最终导致C10-C11、C26-H41和C33-H43的键长显著增加。

相反,外电场与分子内电场的叠加使C-C键(3,6)、C-C键(4,5)、C-C键(24,25)和C-H键(32,44)之间的电场增强,进而导致键长减小:C3-C6和C4-C5中的C6和C4电负性较强,在外电场作用下,原本偏向C6和C4的电子云向C3和C5移动。此时外电场方向与电子云偏移方向一致,使得原子间电场显著增强,键长急剧缩短。

由表3和图3可知,DHR分子的键角随电场强度变化而改变。在外电场作用下,随着外电场强度增大,分子的键角和键长均发生变化,导致分子发生弯曲和折叠。造成这一现象的原因主要有两点:一是体系的诱导偶极矩与外电场相互作用,使分子能量降低并产生畸变;二是在外电场作用下,分子内原子电荷重新分布,导致靠近电场区域的电场力远大于远离电场区域的电场力,分子受力不均匀,最终引发分子畸变。

由表4和图4可知,随着外电场强度增大,DHR分子的偶极矩逐渐增大。这是因为DHR分子属于大共轭芳香体系,芳环中π电子的极化程度极高。外电场的引入使原本杂乱的偶极矩趋向于电场方向。在外电场作用下,电子弛豫会产生更大的诱导偶极矩,从而使总偶极矩增大[19-21]。DHR分子的总能量随外电场强度增大而降低,且降低趋势逐渐变大,这是由于诱导偶极矩与外电场的相互作用使分子结构更加紧密、稳定,进而导致分子能量降低。

%### 3.3 外电场对DHR红外光谱的影响
在不同外电场(0-0.025原子单位)下DHR分子优化和频率计算的基础上,采用Multiwfn3.7软件进行频率校正,得到不同电场强度下DHR分子的红外光谱,如图5所示。由于光谱峰数量较多,本文仅选取5个较强的特征峰进行分析。在无电场(F=0原子单位)时,光谱图中3185 cm⁻¹的吸收峰主要由含三个C-H键的六元环上C-H键的伸缩振动产生;1445 cm⁻¹的吸收峰主要由分子中C-H键的面内弯曲振动和C-C骨架振动产生;1230 cm⁻¹的吸收峰主要由分子中C-H键的面内弯曲振动产生;864 cm⁻¹的吸收峰主要由含两个C-H键的六元环上C-H键的面外弯曲振动产生;780 cm⁻¹的吸收峰主要由两个七元环上C-H键的面外弯曲振动产生。

DHR分子在不同外电场作用下表现出强烈的振动斯塔克效应[22-24],其中由C-H键伸缩振动引起的3185 cm⁻¹特征吸收峰和由C-H键面内弯曲振动及C-C骨架振动引起的1445 cm⁻¹特征吸收峰的变化最为明显。在0-0.025原子单位的电场范围内,3185 cm⁻¹特征吸收峰出现明显红移,这是因为随着外电场强度增大,C原子和H原子的电荷分布发生变化,导致分子中C-H键的键长逐渐增加。在0-0.015原子单位的外电场作用下,1445 cm⁻¹特征吸收峰出现明显红移,这是由于随着电场强度增大,C原子和H原子的电荷分布逐渐减少,分子骨架上C-H键的键长逐渐增加。当外电场强度大于0.015原子单位时,外电场改变了分子的键角和键长,使分子发生弯曲,振动所需能量增加,相应基团的键能增大,导致分子在1445 cm⁻¹处的特征吸收峰出现轻微蓝移。

对比发现,在无外电场作用时,分子的伸缩振动和弯曲振动具有一定对称性;而在外电场作用时,大部分伸缩振动和弯曲振动的对称性消失。例如,318 cm⁻¹特征吸收峰对应两个不相关六元环上方C-H键的伸缩振动,施加外电场后,该特征吸收峰仅对应其中一个六元环上C-H键的伸缩振动,这可能是由偶极矩变化引起的。

DHR分子的摩尔吸光系数随外电场强度增大而增大,这是因为外电场使分子内部电子发生整体转移,导致分子内部电子密度降低,原子间相互作用减弱,外层电子更易被激发到高能轨道,从而使跃迁电子数量增加。此外,从占据轨道和非占据轨道(如图6所示)也可证明,外电场强度增强使跃迁电子数量增加,进而导致摩尔吸光系数随外电场强度增大而增大。

%### 3.4 外电场对DHR激发态的影响
在DHR分子基态几何结构优化的基础上,采用TD-WB97XD/DEF2-TZVP[25]方法计算不同电场下DHR分子的前20个激发态(振子强度阈值设为0.45),并利用Multiwfn软件和Origin软件绘制分子的紫外吸收光谱,如图7所示(图中S0表示基态,Sx表示第x个激发态)。

在DHR分子的紫外吸收光谱中,存在三个紫外吸收峰,分别位于265.94 nm、394.73 nm和568.43 nm处。265.94 nm处的吸收峰摩尔吸光系数为79565.4747 L·mol⁻¹·cm⁻¹,对应基态到第9、10、13和15激发态的跃迁,贡献率分别为13.86%、24.72%、37.55%和19.99%,是共轭体系π→π*跃迁产生的特征吸收,属于K带;568.43 nm处的吸收峰摩尔吸光系数为3832.4605 L·mol⁻¹·cm⁻¹,对应基态到第一激发态的跃迁,跃迁贡献率为99.58%,是芳环上π→π*跃迁产生的特征吸收,属于B带;394.73 nm处的吸收峰强度极弱,故不做分析。

为探究不同外电场下DHR分子的激发特性,本文对不同电场下主要激发态的轨道跃迁和空穴-电子进行了分析。受篇幅限制,本文主要选取不同电场下S0-S1和S0-S13激发态的跃迁作为研究对象。DHR分子的轨道跃迁情况如表5所示,利用Multiwfn软件绘制的轨道波函数等值面图如图8所示,下文将选取主要部分分析其激发特性。

在轨道波函数等值面图中,绿色等值面代表负相位,蓝色等值面代表正相位,等值面数值为0.025。由表5可知,在同一激发态下,不同电场强度对应的轨道跃迁不同,其贡献率也随之变化。

分析表明,当电场强度增大至0.025原子单位时,基态S0到第一激发态S1和第13激发态S13的跃迁特性发生改变。其中,94、96和97轨道均表现出π轨道特性,98轨道表现出σ*轨道特性,100、101和103轨道也表现出σ*轨道特性。因此,当F=0.025原子单位时,S0→S1和S0→S13的激发特性均为π→σ*跃迁;而当F=0原子单位、0.005原子单位、0.010原子单位、0.015原子单位和0.020原子单位时,S0→S1和S0→S13的主要激发特性均为π→π*跃迁。由此可见,外电场可改变轨道的激发特性。

为明确不同电场下DHR分子的电子激发类型,本文列出了DHR分子S0-S1和S0-S13激发态的空穴-电子相关参数,如表6所示。这些参数包括空穴和电子在整个空间的分布指数(Sr,取值范围为[0,1])、空穴与电子的质心距离指数(D)、电子和空穴的整体平均分布宽度(H)、空穴与电子的分离程度指数(t)、空穴离域指数(HDI)和电子离域指数(EDI)[26]。这些参数是衡量电子激发模式的定量指标,常被用作判断电子激发类型的重要依据。图9为利用Multiwfn软件绘制的空穴-电子图及Chole\&Cele图,其中绿色代表电子分布,蓝色代表空穴分布。

由表6可知,在S0-S1激发态下,当F=0原子单位时,D指数为0,t指数小于0,结合空穴-电子图和Chole\&Cele图可观察到,空穴与电子的分离程度较小,几乎无分离,因此可推断在无外电场作用时,S0-S1激发态为环结构上高度局域化的π→π*激发。当F=0.025原子单位时,D值达到最大,为6.229 Å,结合空穴-电子图和Chole\&Cele图可观察到,空穴与电子发生明显分离,且t指数为3.804,明显大于0,表明空穴与电子分布分离显著,由此可推断当F=0.025原子单位时,S0-S1激发态为π→σ*电荷转移激发。

在S0-S13激发态下,当F=0原子单位时,t值小于0,表明电子与空穴未完全分离;D指数为0,表明空穴与电子质心重合;Sr指数高达0.88308,表明空穴与电子的空间分布均匀,无明显分离。结合空穴-电子图和Chole\&Cele图可观察到,空穴与电子的重叠程度较大,因此可推断该激发态为环结构上高度局域化的π→π*激发。当F=0.025原子单位时,D值达到最大,为4.535 Å,t指数为2.070 Å,大于0,表明空穴与电子发生明显分离,结合空穴-电子图和Chole\&Cele图也可观察到空穴与电子的分离现象,因此可推断当F=0.025原子单位时,S0-S13激发态同样为π→σ*电荷转移激发。

同理可得出,在0-0.020原子单位的外电场作用下,S0-S1和S0-S13激发态均为π→π*局域激发;当外电场强度大于0.020原子单位时,S0-S1和S0-S13激发态转变为π→σ*电荷转移激发。

综上分析,外电场可改变DHR分子的电子激发特性与激发类型,当F=0.025原子单位时,S0-S1和S0-S13激发态由高度局域化的π→π*激发转变为π→σ*电荷转移激发。DHR分子作为半导体材料,电子激发跃迁可通过空穴-电子对的形式完成。在不同外电场作用下,随着电场强度增大,同一电子激发态表现出不同的激发特性,这对该材料未来在光电化学领域的应用具有重要意义。

%### 3.5 外电场对DHR分子空穴迁移率的影响
DHR分子可用于有机场效应晶体管,通过影响有机半导体的分子结构改变分子前线轨道能级,进而影响晶体管的传输性能[27,28]。最高占据分子轨道-最低未占据分子轨道(HOMO-LUMO)能隙是分子的重要特性,与分子导电性密切相关,常被用于衡量分子导电性,能隙越小,导电性越好。对于P型半导体,空穴迁移率与HOMO能级呈负相关,其HOMO能级范围为-4.8 eV至-5.5 eV[29,30]。

在Zhang等人的研究中,DHR分子可用作有机P型半导体,其HOMO能级为-4.98 eV,空穴迁移率为0.082 cm²·V⁻¹·s⁻¹。为探究外电场是否可通过改变分子HOMO能级来改变其空穴迁移率,本文计算了DHR分子的轨道能级,结果如表7和图10所示。

由图10可知,在0-0.020原子单位的电场范围内,DHR分子的HOMO轨道能级波动极小,在0.020-0.025原子单位的电场范围内急剧下降;LUMO轨道能级和HOMO-LUMO能隙在0-0.020原子单位的电场范围内逐渐减小,在0.020-0.025原子单位的电场范围内也急剧下降。当电场强度达到0.020原子单位时,随着外电场强度继续增大,LUMO几乎完全脱离分子,体系无法再稳定结合多余电子[31],分子轨道能级显著降低。此外,随着电场强度增大,电子云迁移使DHR分子芳环的大π键减弱,π→π*跃迁能量降低,导致分子的HOMO-LUMO能隙急剧减小。

根据前线轨道理论[32],分子的HOMO能级越高,电子的活性和亲电性越强;LUMO能级越低,越容易吸引电子,亲核性越强;分子能隙大小决定了分子参与化学反应的能力。随着外电场强度增大,DHR分子的HOMO能级降低,电子活性减弱;LUMO轨道能级降低,分子更易获得电子,亲核性增强;分子能隙减小,表明电子从HOMO跃迁到LUMO所需能量减少,电子更易从HOMO跃迁到LUMO,使HOMO产生空穴。

HOMO轨道能级的变化会改变轨道的空穴迁移率,进而影响P型半导体的性能。当F<0.015原子单位时,HOMO能级随电场强度增大而升高,导致DHR的空穴迁移率降低;当F>0.015原子单位时,HOMO能级急剧下降,使DHR的空穴迁移率升高。这是因为当电场强度小于0.015原子单位时,电子跃迁能力较弱,分子相对稳定;而当分子受到较强外电场作用时,分子活性增强,易激发电子从HOMO轨道跃迁到最高未占据分子轨道(HUMO)。当F=0.025原子单位时,DHR的HOMO能级为-5.0660 eV,小于-4.98 eV,表明其空穴迁移率大于0.082 cm²·V⁻¹·s⁻¹。因此,可通过调节外电场强度改变DHR的HOMO能级,进而改变DHR P型半导体的空穴迁移率,改善其半导体性能。

%    \centering
%    \includegraphics[width=0.9\textwidth]{figure_S15.pdf} % 需替换为实际图片路径
%    \caption{DHR 在 THF 溶液~(1.0×10$^{-7}$ mol/L)中 305 nm 激发下的实验发射光谱~(黑色曲线)与气相中的模拟发射光谱~(绿色曲线:~S$_2$→S$_0$ 跃迁;红色曲线:~S$_1$→S$_0$ 跃迁)。插图 4-1 和 4-2 分别展示了分子中键振动对 S$_1$→S$_0$ 和 S$_2$→S$_0$ 发射的模拟结构贡献,颜色越深的位置表示重组能越大,贡献越显著。横坐标为波长~(nm),纵坐标为发射强度~(Intensity)。}
%    \label{fig:dhr_emission_spectra}
%\end{figure}

