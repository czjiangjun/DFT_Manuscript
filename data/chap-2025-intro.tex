\chapter{研究背景}
\textrm{Dicycloheptarubicene~(简称DHR)}是一种同时含有两个五元环、两个七元环和四个六元环为基本单元的平面型二维共轭有机分子,研究发现,\textrm{DHR}具有\textrm{P-}型半导体性质和独特的电子结构。如图\ref{Fig:DHRP-structure}所示,\textrm{DHR}含有\textrm{5/7/5}元环系的共轭分子,与\textrm{6}元环系的结构类似,具有特定的多层堆积性质。%\textrm{DHR}由两个五元环(薁环)和两个七元环组成,整体呈平面结构。
\begin{figure}[h!]
\centering
\vspace*{-0.1in}
\includegraphics[height=1.8in]{Figures/DHRP-strucure.png}
%\caption{\fontsize{7.2pt}{4.2pt}\selectfont{含有\textrm{5/7/5}环系的共轭分子模型示意图,具有特定的多层堆积特性.}}%
\caption{含有\textrm{5/7/5}环系的共轭分子模型示意图,具有特定的多层堆积特性.}%
\label{Fig:DHRP-structure}
\end{figure}
近年来非苯型烃类化合物(如薁、茚(苯并二戊环烯)、苯并二庚轮烯稠环芳烃,见图\ref{Fig:DHRP-mult-structure})因其电子结构和结构特征与对应的苯型化合物存在显著差异,受到越来越多的关注。含有五边形[6]或七边形[7]的纳米石墨烯[5](如碗烯衍生物[8-9]、苏门答腊烯衍生物[10]、茚并芴衍生物[11]、红荧烯衍生物[12]及[7]环烯衍生物[13-15])已被成功合成并研究。在多环芳烃中引入五边形或七边形,可改变其电子结构和构象。事实上,大多数含五边形或七边形的多环芳烃呈非平面结构(如碗状或负曲率结构),且在固态下表现出独特的电子结构和分子间堆积方式。
%
\begin{figure}[h!]
\centering
\vspace*{-0.1in}
\includegraphics[width=5.8in]{Figures/Non-Benzenoid-polycyclic.png}
%\caption{\fontsize{7.2pt}{4.2pt}\selectfont{含有\textrm{5/7/5}环系的共轭分子模型示意图,具有特定的多层堆积特性.}}%
\caption{非苯型稠环芳烃类化合物(a)薁\textrm{(Azulene)},(b)茚\textrm{(Indacene)},(c)苯并二庚轮烯.}%
\label{Fig:DHRP-mult-structure}
\end{figure}
%![图1. 代表性非苯型多环芳烃(a)及非苯型二维纳米石墨烯二环庚红荧烯的设计思路(b)](注:原文含图,此处为图注翻译)
%a)代表性非苯型多环芳烃(如薁);b)同时含五元和七元环的二维纳米石墨烯

%值得注意的是,同时包含五边形和七边形的多环芳烃相对稀缺[17]。目前已合成的此类化合物包括薁并菲[18]、二萘并薁[19]、二环庚[cd,g]茚[20]、戊搭七元烯[21]、薁并芘[22]、二环庚[cd,gh]戊搭烯[23]、萘并二薁[24]等,它们的五边形和七边形均形成“形式薁单元”。然而,直到近期,在合成含多个“形式薁单元”的大π扩展共轭分子方面才取得重大进展[25-28]。例如,Ie、Aso及其团队报道了具有反芳香性的环庚三烯双并茚并芴的合成[29];Konishi、Yasuda及其团队近期报道了二芴并七元烯的合成与表征,该化合物含两个五边形和两个七边形,其七元烯核心赋予了分子反芳香性和开壳层特性[30];同时,Müllen、Feng及其团队通过表面化学[31]和溶液合成[32]两种方法,成功将两个“薁单元”嵌入纳米石墨烯中,但这两种扩展共轭分子因独特的电子结构,表现出开壳层特性且空气稳定性较差。最近,Mastalerz及其团队报道了含15个稠合芳环和两个嵌入式薁单元的更大扭曲型可溶性多环芳烃的合成[33],单晶结构分析明确证实了其结构稳定性,并研究了扭曲程度对芳香性和共轭性的影响。
%
%本文报道了含两个“形式薁单元”的二环庚[ijkl,uvwx]红荧烯(DHR,见图1b)的合成、结构表征及半导体性能。这种新型非苯型纳米石墨烯含两个五边形和两个七边形,可视为二苯并[bc,kl]蔻的同分异构体。与常见的含五/七元环的非平面纳米石墨烯[16b,34]不同,单晶结构分析表明DHR呈完全平面结构,为其在有机光电子材料中的应用奠定了基础。此外,DHR的溶液和固体样品在空气中可稳定存放数月,其晶体表现为p型半导体,平均迁移率为4.94×10⁻² cm²·V⁻¹·s⁻¹。

双环庚基红荧烯(DHR)是一种新型有机P型半导体材料。2020年,张德庆实验室研究团队成功研发出一种新型非苯类二维有机共轭分子,该分子包含两个五元环和两个七元环(薁单元)。DHR属于多环芳烃,由六元环、五元环和七元环构成。DHR分子的薄膜和晶体具有P型半导体特性,且分子呈平面结构,同时还具有特殊的反卡莎(anti-Kasha)吸收和发射现象,在多波长发光材料调制领域具有潜在应用前景[1]。近年来,多环芳烃化合物(PAHs)因其独特的分子结构和优异的电子性能,已被广泛应用于发光二极管、有机半导体材料、太阳能电池、有机场效应晶体管、新型分子电子器件、传感器及近红外材料等领域[2]。

众所周知,外电场对化学体系具有显著影响,这一直是研究热点领域[3]。在外电场作用下,分子会产生新的物理现象和化学变化,例如分子结构改变、化学键断裂、激发态变化及分子极性改变等[4]。李等人[5]研究了外电场下C₅F₁₀O的结构与激发特性;Tuchin[6]探究了外电场下C₆₀富勒烯的振动光谱与电子结构;李等人[7]研究了外电场下环状C₁₈的基态性质与激发特性,这些研究均表明外电场对分子结构具有显著影响。DHR分子可作为有机半导体材料,其分子结构会影响相应器件的性能。因此,研究不同外电场下DHR分子的结构变化,对于理解DHR分子的激发特性及器件性能具有重要意义。目前,关于外电场下DHR分子结构与性能的研究尚未见报道。本文尝试通过外电场调控DHR的分子结构、前线轨道能级及激发特性,以提高DHR P型半导体的空穴转移速率,为未来将DHR分子设计为有机半导体材料提供理论依据。


与传统含五元-七元环结构的分子不同,\textrm{DHR}的分子式为\ch{C30H14},其单晶结构表明,所有共轭环共平面。非苯型的共轭分子\textrm{DHRP}骨架与相应的苯类化合物相比,具有不同的电子和结构特征,展现特异的光、电、磁等性质。含五元环或七元环的共轭分子受到了越来越多的关注并被合成和研究。含有非苯型的五元环或七元环的大共轭分子可能作为含有“缺陷”的新型碳的同素异形体,目前此类化合物研究主要存在稠合模式单一、合成复杂、结构不稳定、分子不平面、缺乏功能研究等问题。 
