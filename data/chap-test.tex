\chapter{理想晶体中的电子态}\label{ideal_electron}
固体能带结构的确定是一个多体问题,其需要针对大量原子核和电子求解薛定谔方程。即使我们设法求解上述方程并根据所有原子核和电子的位置找到晶体的完整的波函数,我们也还要面临如何将该函数应用于物理可观测值计算的复杂问题。因此,多体问题的精确解既不可能,也完全没有必要。若要在理论上对物理兴趣量进行描述,只需要知道依赖于几个变量的能量谱和多个关联函数(电子密度,偶关联函数等)。

由于只有较低的晶体能谱激发分支对于我们的讨论具有重要意义,这样我们可以引入准粒子概念作为系统的元激发。因此,我们的问题缩减到确定准粒子的色散曲线并分析它们的相互作用。已知存在两种类型的准粒子,即费米子和玻色子。在晶体中,只有电子是费米子,声子和磁子均为玻色子。

由此确立的问题仍然相当复杂,还需要进一步简化。第一种简化是假设形成晶格的离子质量大大超出电子质量,。这种巨大的质量差导致其速度存在很大的差异,同时可做出以下假设:任何原子核的浓度(甚至非平衡的原子核)均能够合理地与电子的准平衡配置相关联,其几乎不存在原子核运动的惯性。因此,我们可以认为电子处于几乎静止的原子核内,也就是说电子和声子的能量可以很容易分开。这种近似就是Born-Oppenheimer绝热近似。

虽然经验表明电子和声子之间的相互作用对电子能量和费米面的形状略有影响,但还存在许多其他性质需要在第一近似中考虑电子-声子相互作用。事实上,在某些情况下,如果不是由于上述的相互作用,那么所考虑的性质将不复存在。这些性质包括诸如所有的传输性质和超导现象。



1
2






在本书中,我们将使用绝热近似并且仅考虑电子子系统。对晶体中电子-声子相互作用感兴趣的读者可参考[23]。我们还使用了理想晶格的近似,这就意味着形成晶格的离子以严格的周期顺序排列。因此,与杂质、无序晶体和表面现象的晶体电子态相关的问题也超出了本书的范围。

在这些近似中,晶体中多电子系统的非相对论哈密顿量为1

其中第一项为各个单个电子的动能之和,第二项定义了此类电子中的每一个与原子核所产生电位的相互作用,最后一项包含电子对之间的排斥库仑相互作用能。

应该指出的是我们电子子系统的两个重要特性。首先,所有金属的电子密度均导致系统中一个电子的平均体积与成比例。可以看出,该值在数量级上与粒子势能与其平均动能的比率相一致。因此,金属中的传导电子不是电子气,而是量子费米液体(这些量子效应相当可观的电子简并温度是大约为至)。

其次,金属中半径小于晶格常数的电子被屏蔽。在Bohm和Pines[24,25]、Hubbard[26]、Gell-Mann和Brueckner[27]的论文发表后,库仑相互作用的长程部分对于集体运动的作用愈加明确。这种集体运动(等离子体振荡)具有足够高的激发能量,可以避免在系统基态附近激发等离子体振荡。因此,电子的单个运动可以很容易从具有相当小作用半径(某些情况下小到1)的屏蔽库仑相互作用中确定。在较大的相互作用距离,只能观察到电子之间的某种平均相互作用。

第一个性质不允许我们引入小参数。因此,我们无法使用标准形式的扰动理论。这使得针对金中的电子子系统的理论分析更加困难,并且让某些近似值难以控制。因此,理论估计与实验数据的比较至关重要。


1除非另有说明,否则我们所使用原子单位假定普朗克常数等于统一,使用波尔半径=0.529177作为长度单位,使用电子质量m的两倍作为质量单位,以及Rydbergs中的氢原子电离能(1Ry=13.6049eV)作为能量单位。


3



子系统的第二个性质允许我们引入弱相互作用准粒子的概念,因此,利用Landau的观点[28],任何宏观多费米子系统的弱激发均表现出单个粒子行为。显然,对于其他系统而言,存在长寿命弱相互作用粒子的能量范围将有所区别。金属中的这一范围相当显著,达到了接近费米能级的。这使得基于单个粒子概念的能带理论可以成功应用于金属的电子特性分析。

本章第1.1节介绍晶体中电子系统的基态。任何多电子系统的基态特性均根据密度泛函理论(DFT)进行讨论。根据该理论[6-8],此类系统基态的所有特性均可通过与电子密度相关的某种函数式得到很好的描述。到目前为止,明确定义该函数式仍然存在难度,问题尚未解决。然而,对于均匀和非均匀系统则存在着相当不错的近似方法。

第1.1节还介绍了Hartree和Fock[5]先前提出的电子系统理论分析方法。该方法使用一组哈密顿量,每个哈密顿量均为一个电子坐标的函数。薛定谔方程中的变量是分离的,并对电子进行独立分析。这种单电子方法导致费米能级[29]金属中电子态密度的消失,这肯定与金属热容的实验数据相冲突。Slater[5]在他的交换相关电位近似中找到了解决这一争议的方法,该方法现已在能带理论中得到广泛使用。

在1.2节中,我们讨论计算多电子系统基本激发的方法。构造晶体电位的方法在1.3节给出,该节介绍了晶体能带结构中交换相关电位的各种近似。

1.1 多电子系统的基态
电子在凝聚态介质中的运动是高度相关的。乍一看会得出这样的结论,即,不可能用独立粒子近似描述此类系统。但是,我们可以使用非相互作用粒子模型系统,其中总能量  和电子密度 与真实系统的类似函数相匹配,并可通过外部场描述电子之间相互作用的所有效应。这是密度泛函理论(DFT)的本质。在描述DFT之前,我们先简要介绍一下Hartree、Fock和Slater提出的自洽场方法。

1.1.1 Hartree-Fock方程

Hartree的主要贡献是提出一种近似方法,其中每个电子的运动独立于核外其他电子,并且所有其他电子与给定电子之间的相互作用可以被此类电子状态平均电荷密度产生的静电场作用所取代。这样,每个电子均被分配一个单独函数 ,亦即,一个轨道,而电子系统的整个波函数表示为所有轨道的乘积。因此轨道的单电子方程为[5]




4




(1.1)
.(1.2)

此处,  为动能运算符;  为处于点电子的库仑势,其由与原子核的相互作用以及给定系统所有电子电荷的总密度得出。为了补偿中的自相互作用,我们需要能量

,(1.3)

其中  为轨道中的电子数
当为带有核电荷的原子时,库仑势为

.(1.4)

因此,在一个包含电子的系统中,每个电子均受到原子核和电子所产生场的作用。

Hartree方程可写出利用函数确定的能量函数式。函数 作为的一个自洽解,必须最小化这一函数。上述解的构造被Fock所使用,并由Slater独立用于得到所谓的Hartree-Fock方程。

根据泡利原理,系统的总波函数相对于电子对的交换应该是反对称的。Hartree-Fock方法考虑了泡利原理,并通过以下行列式近似所有多电子波函数
.(1.5)
当我们将这种多电子波函数代入薛定谔方程时,利用Ritz变分原理,
(1.6)
并考虑轨道正交化的所有条件,我们得到Hartree-Fock方程[5]
.(1.7)



5


该方程的求和是平行自旋电子状态的和。(1.7)中的第一项与Hartrec方程 中的第一项相同。(1.7)左侧的第一项描述了交换相互作用,其亦为库仑相互作用,并且与使用形式波函数得出的电子运动相关性有关。注意,不是对求和,如果交换项中仅有分量,那么该项将与Hartree方程(1.3)中的相同。该项描述了电子与其自身的相互作用。因此,很明显,Hartree-Fock方法中的交换校正包括描述自相互作用的项以及 项 ,即交换项。

Hartree-Fock方程采用(1.1)形式最为方便。现在,我们得到运算符[5]
(1.8)
其中
.(1.9)

交换电位 是从其周围空穴中去除一个电子电荷的电位。因此,Hartree-Fock方程可以被解释为Hartree方程,即,单独电子在原子核和整个电子电荷的场内中移动,并且该场受到量级与位于给定电子周围费米空穴中电子电荷相同的交换电荷场作用而衰减。

运算符 与轨道相关,这让Hartree-Fock方程的解复杂化。因此,后者与通常通过迭代技术求解的非线性积分微分方程有关。首先选中初始的一组轨道 并代入(1.8)。然后,解出特征值问题  1.1)定义另一组轨道,从而实现自洽。

目前,Hartree-Fock方法是最好的单电子方法。并且,其在计算有限多粒子系统(即原子)的电子结构方面也最为有效。

然而,即便在自由电子气情况下,应用于宏观多电子系统也存在困难。下面表明自由电子气的交换电位与函数成比例,该函数从费米分布下方的unity变化为峰值的1/2,在激发态的区域变为零。由于交换能量为负,所以其包含在内时,能级数和费米分布下方的态密度大幅度降低。图1.1显示了包含交换校正的自由电子系统计算密度[30]。在不考虑交换的情况下,态密度不再与能量的平方根成正比,而是在费米球上变为零。这与金属低温热容量的实验数据相矛盾。

这种效应与库仑场的远程效应有关。上面已经提到过,金属等离子体在紫外线范围内的激发处于高能级,因此其在常温下不会激发。由于屏蔽效应,电子-电子相互作用能量应由电位定义,其中 为常数,并非由库仑势得出。因此,电位在小于晶格常数的距离处呈指数减小。在较大的电子间距离处,仅存在平均的电子-电子相互作用。 

6



图1.1自由电子气的态密度[30]:1-Sommerfeld理论:2-Hartree-Fock理论。
因此,在Hartree-Fock理论中,当中的交换能量存在强相关性时,通过Hartree-Fock确定的单电子能量对于宏观系统的费米统计是不正确的。 

然而,在交换项(1.9)由仅与  相关的普通交换校正取代的情况下,很多时候可以观察到理论和实验之间存在更好的一致性。这正是Slater所提出  方法的意义。

7


1.1.2  Slater近似

为了得到对于所有电子态均相同的交换电位,Slater建议用对应于点第i个轨道所描述电子存在概率的加权平均值代替交换项(1.9)。由第i个轨道产生的点电荷密度等于,该点电子的电荷密度总和等于。因此,电子在处于第i轨道所描述状态的概率为


(1.10)
.

然后,相应的加权平均值为

.(1.11)

该电位现在对于所有电子态均相同。因此,(1.1,1.8)成为

.(1.12)

需要再次注意的是,平均交换(1.11)不仅作为Hartree-Fock方法的简化引入,而且还解决了Hartree-Fock方法在确定固体能带能量方面的困难。因此,(1.12)的重要特性是其波函数相互正交,并且电子在由所有其他电子产生的平均场中移动。自由电子气的交换电位(1.9,1.11)可以实现精确计算。在这种情况下,Hartree-Fock波函数为平面波。动量空间中自由电子气的基态为半径的费米球。具有波数状态的交换电位由方程[5]定义
(1.13)
其中和

.(1.14)

与中出现的对数奇点是由费米能量的电子态密度等于零引起的,如上所示,由此导致Hartree-Fock理论与晶体的电子热容和磁化率实验数据之间产生分歧。为了得到自由电子气的电位(1.11),我们必须在所有以内的占据态上对其进行平均。由于和之间的电子态数量与成正比,我们有



8

现在我们得到
(1.15)
平均交换电位取决于电子电荷密度,而在自由电子气情况下,电子电荷密度恒定。Slater在[31]中提出,与一般情况类似,当电荷密度不均匀时,平均交换电位可以通过自由电子气的电位进行近似,但具有局部电子密度

(1.16)

这就是所谓的Slater交换电位。由于Slater提出在能带计算中使用交换电位(1.16),因此已经进行过很多系统电子结构和物理性质的计算。在大多数情况下,理论数据和实验数据之间存在良好的一致性,尤其是对于纯过渡金属。注意,该电位方便描述基态的性质(电子密度分布、总能量、晶格参数、费米面拓扑)和金属激发态(-射线发射、光谱等)。

但是,我们应该指出,Slater所使用的交换电位平均方法很难称之为严谨。这是因为在量子力学中,作为近似值的平均值仅在物理可观察量被平均时才具有物理意义。因此,Slater的方法是唯象的,并且(1.12)包含交换电位的应被视为单粒子方程而非单电子方程。

后来,密度泛函理论得到发展,并为Hartree、Fock和Slater开发的自洽场方法提供了理论基础。Slater近似可以作为以下电子密度函数的特例得到
,(1.17)
其中为同质系统的交换能量。因此,如果我们使用仅具有交换效应的DFT并忽略相关性,那么我们将得到具有因子的Slater电位。这就是Gaspar-Kohn-Sham交换电位[7,32]。该差异可以解释如下。在得出Hartree-Fock方程后,Slater从总能量的行列式表达式开始。在该表达式中,轨道的变化让总能量最小化。该过程导致(1.8)。然后,交换电位(1.9)用平均表达式(1.il)代替,然后再用统计等价物(1.16)代替。与此相反的是,Gaspar、Kohn和Sham在统计近似中写出总能量的表达式,然后通过函数上变化得到单电子方程。此处,在(1.12)而非中,该交换分量仅得到2/3的数量。换句话说,通过统计等价物进行总能量变分和交换分量替换的两个运算无法换算。

Gaspar、Kohn和Sham的结果也可以通过用(即在费米面上)真值替换Hartree-Fock交换电位(1.13)得到。

9

在这种情况下。在物理意义上,如果我们注意到电子密度排序是由接近费米能级的电子重新分布决定的,那么就可以理解这一点。因此,Slater方法相当于在占据水平上对交换电位进行平均,而Gaspar、Kohn和Sham选择了对应于费米能量的电位。之后,Slater提出为交换电位(1.16)引入系数,并将其视为拟合参数[5](因此“”交换电位):

.(1.18)

由于方法的系统总能量包含线性依赖于的负交换分量,能量最小化不允许的确定,只要参数越高,总能量就越低。

有几种方法可以找到。一个是,选择从而让总能量恰好等于原子的Hartree-Fock能量。另一个方法是,选择从而让维里定理得到满足。在这个定理中,轨道计算得出的系统势能应该等于带反转符号动能的两倍(假设系统压力等于零)。通过这种方式,Schwarz等人[33,34]得到了大量中性原子的和值。从表1.1中可以看到当-过渡金属的原子序数增加到0.72-0.70时和将会减少,并随着的增加而持续缓慢下降。注意,和的差非常微小。在[35]中,对整个周期系统进行了类似的计算(图1.2)。

现在,Slater电位的成功非常明显。在能带理论发展的早期阶段,计算通常以完全Slater交换的非自洽方式进行。随后,采用电位(1.18)的自洽计算得到了几乎相同的结果。应当注意,在使用变分原理时,借助因自洽问题解决所得到的轨道确定的总能量将小于使用任何其他轨道计算得出的总能量。相比之下,在上得到了比较小更深的势阱(表1.1)。事实上,这两种相反的效果是完全平衡的,至少在过渡金属能带结构的计算中即是如此。 

图1.2.Slater电位参数对于的依赖性


10


表1.1.根据和Hartree-Fock方法中的原子总能量相等条件确定的值以及维里定理条件

z
Atom
C*VT
CCHF
Z
Atom
ffvt
«HF
1
H
0.97804
—
22
Ti
0.71648
0.71695
2
He
0.77236
0.77298
23
V
0.71506
0.71556
3
Li
0.78087
0.78147
24
Cr
0.71296
0.71352
4
Be
0.76756
0.76823
25
Mn
0.71228
0.71279
5
B
0.76452
0.76531
26
Fe
0.71094
0.71151
6
C
0.75847
0.75928
27
Co
0.70966
0.71018
7
N
0.75118
0.75197
28
Ni
0.70843
0.70896
8
0
0.74767
0.74447
29
Cu
0.70635
0.70697
9
F
0.73651
0.73732
30
Zn
0.70619
0.70673
10
Ne
0.72997
0.73081
31
Ga
0.70644
0.70690
11
Nd
0.73044
0.73115
32
Ge
0.70645
0.70684
12
Mg
0.72850
0.72913
33
As
0.70630
0.70665
13
A1
0.72795
0.72853
34
Se
0.70606
0.70638
14
Si
0.72696
0.72751
35
Br
0.70576
0.70606
15
P
0.72569
0.72620
36
Kr
0.70544
0.70574
16
S
0.72426
0.72475
37
Rb
0.70525
0.70553
17
Cl
0.72277
0.72325
38
Sr
0.70480
0.70504
18
Ar
0.72131
0.72177
39
Y
0.70440
0.70465
19
K
0.72072
0.72117
40
Zr
0.70398
0.70424
20
Ca
0.71941
0.71984
41
Nb
0.70357
0.70383
21
Sc
0.71793
0.71841





这种偶然的误差相互消除使得近二十年来能够得到正确的物理结果。但是,对过渡金属化合物并没有观察到完全的补偿,在这种情况下,必须使用电位(1.18)进行自洽计算。

就Hartree-Fock理论而言,针对很多物理现象的描述不够充分可能还有另一个原因,就是关联效应没有得到妥善的处理。在实际系统中,每个电子的运动与其他电子的运动均具有关联性。由于库仑排斥,两个电子彼此接近会消耗能量;在(1.4)中当时。这将允许假设每个电子均被库仑穴所包围。

由于根据泡利原理,不超过一个电子可能会处于一个量子态,因此我们可以通过类比库仑穴讨论费米穴。因此,Hartree-Fock波函数(1.5)允许仅限平行自旋电子的相关性。在其他情况下,反平行自旋电子的相关性也称之为相关性。根据定义,相关能量等于相互作用电子系统的总能量减去Hartree-Fock能量:

(]。19)

此处,包括相对论修正)。
在Hartree-Fock理论中,电子相关性可以包含两种方式,即,通过在系统的总波函数中直接包括电子间距离,或者通过使用配置交互方法。由于成对配置[36]对电子相关能量起主要作用,第一种方式似乎比较合理。因此,试波函数为





其中为Hartree-Fock方法中使用的Slater行列式(1.5)。函数由双电子函数构成
. 
现在需要用到变分原理,同时优化和。但是,因为双电子函数违反了正交性,因此并不容易。简化函数被定义为非相互作用粒子气,即,而写出由此可以忽略对自旋的依赖性。该数学程式已在[37]中用于均相电子气。函数使用以下表达式

其中由数值变分决定。作者与Vashishta和Singwi报告的数据得到很好的吻合[38]。[39]中给出了对这些方法的简要评论。

配置交互方法在原子计算中得到广泛使用。在该方法中,波函数被写为由以下单电子自旋-轨道波函数所构成行列式函数的线性组合

.(1.22)

此处,为由电子所组成系统不同配置的Slater行列式(1.5)。对于(1.22)的主要贡献来自Hartree-Fock方程得出的行列式,即“主”行列式。其他行列式是通过将主行列式的一个或几个轨道替换为与主行列式所有轨道正交的其他轨道而得到的。将(1.22)代入(1.6)将得到系数与能量的无穷线性方程组,当然,其必须截尾。随着配置数量的增加,计算变得更加精确。如果轨道和系数同时变化,则结果更准确。

这种方法应用于原子的严谨公式在[40]中给出。配置交互方法的主要缺点是非常麻烦;为了确保准确性,必须计算很多行列式。

自洽场和配置交互方法都允许电子的相关性并可显著改善Hartree-Fock计算,但其仅适用于诸如自由原子等有限多粒子系统。对于晶体,则必须使用Slater近似。Bohm和Pines[24,25]已经证明,包含电子相关性会导致交换能量对于Hartree-Fock理论中的强依赖性消失。这是近似所独有的成功。

1.1.3  密度泛函理论 

数学程式 DFT基于Hohenberg和Kohn定理[6],因此可以通过引入电子密度的某些函数描述相互作用电子气基态的所有性质。。系统的标准哈密顿量被[7]取代


12



,(1.23)
其中为包含原子核场的外场;函数式包括电子的动能和交换相关能。系统的总能量由函数的极值给出,其中为电荷分布。因此,确定总能量系统我们不需要知道所有电子的波函数,其足以确定某些函数式并确定其最小值。注意具有普遍性,不依赖于任何外场。

这一概念由Sham和Kohn[8]进一步发展,其曾提出一个表格
.(1.24)

此处为密度为的非相互作用电子系统的动能;函数式包含Hartree理论中未包括的多电子效应,即交换和相关性。

让我们把电子密度写为

,(1.25)

其中为电子的数量。在新变量中(根据通常的标准化条件),

.(1.26)

此处,为电荷原子核的位置;为形成单粒子态能谱的拉格朗日乘子。交换相关电位是泛函导数

.(1.27)

从(1.26)中我们可以发现电子密度以及系统基态的总能量。

虽然只有基态才能得到严格证实,对于交换相关能函数目前只有粗略的近似,该理论对实际应用的重要性几乎不可能被高估,原因在于其将多电子问题简化为实质上有效局部电位的单粒子问题



13


显然,(1.26)应有自洽解,因为对于我们正在寻求的轨道具有依赖性。

在给出方程精确值时,只要其能够准确定义电子密度和总能量,方程(1.23-5)就是精确的。因此,使用DFT的核心问题在于定义函数。不均匀电子气的精确表达可以写成围绕交换相关穴的电子和变化密度之间的库仑相互作用[41,42]:

.(1.29)

在(1.29)中,被定义为

,(1.30)

其中为对相关函数;为耦合常数。
该与交换相关穴的实际形状无关。我们可以通过替换变量来显示[43]

(1.31)

并且仅取决于球面平均电荷密度

.(1.32)

这意味着库仑能量仅与距离有关,而与方向无关。此外,空穴电荷密度满足加和规则[43]

(1.33)

这意味着交换相关穴对应于电子较少的电荷密度,其恰好等于电子的电荷。

局部密度近似  
在能带计算中,通常使用的是交换相关电位的某些近似。最简单和最常用的是局部密度近似,其中具有与均匀电子气相似的形式,但是空间内每个点的密度均被电荷密度的局部值代替:

,(1.34)

其中为均匀电子系统的对相关函数。该近似满足加和规则(1.33),这是其主要优势之一。将(1.34)代入(129),我们得到局部电子密度近似[7]:
.(1.35)

14

此处,是对具有密度的均匀相互作用电子气总能量(每个电子)交换和相关性的贡献。这种近似对应于交换相关穴围绕的每个电子,并且当变化缓慢时必定如预期一样非常好。通过多种方法的计算导致结果相差几个百分点[44]。因此,我们可以考虑将数量充分定义。的插值表达式由Hedin和Lundqvist提供[10]。在局部密度近似中,有效电位(1.28)为

(1.36)

其中为具有局部密度的均匀相互作用电子气化学势的交换相关部分,

.(1.37)

对于自旋极化系统,使用局部自旋密度近似[8,45]

.(1.38)

此处,为上自旋态和下自旋态密度分别为和的均匀系统每电子交换相关能量。

注意,局部密度近似和自旋局部密度不包含拟合参数。此外,由于DFT没有小参数,因此几乎不可能对不同近似的精度进行纯粹的理论分析。因此,可以通过计算数据和实验数据之间的可容忍的一致性证明对实际系统交换相关电位使用任何近似的合理性。

计算过渡金属基态时局部电子密度近似的精度如图1.3所示。该图显示了26种过渡金属的结合能、原子密度和体积模量的实验值和理论值[46]。磁性-过渡金属和表明计算与实验之间的最大差异。 进一步的研究[47]表明,这些差异之所以存在主要是由于磁性有序的计算不准确,如果得到纠正,则应会减少固体的总能量。

这个例子非常重要,这是因为图1.3所示的结果是通过仅调整一个参数,即每个原子的电子数得到的。总的来说,采用局部密度近似计算能够确保获得对金属基态的很好描述。

量子力学的多粒子系统存在两种不同类型的问题:宏观多粒子系统和原子系统或多个原子宏观系统簇所包含的粒子以及作用在或尺度的影响小到可以忽略不计。N10到100的原子和簇不可以忽略在和尺度上的特性。此外,在自由原子或簇的边界上可以观察到电子密度的剧烈变化,而金属内原子周边的电子密度是距离缓慢变化的函数。


15


图1.3.使用局部密度近似计算的能带能量、平衡Wigner-Seitz半径和弹性体积模量与实验数据对比[46]。

对于有限系统(原子和簇),通过局部密度近似计算得出的总能量误差通常为5%至8%。即使对于像氢原子这样的简单系统,总能量计算为0.976Ry而非1.0Ry[46]。因此,有限多粒子系统需要使用某些其他方法。

由于金属是宏观多粒子系统,所以采用局部密度近似对于基态能量和电子密度能够获得足够好的结果。

与Hartree-Fock-Slater方法相比,DFT能够以更自然的方式包含交换和相关性效应。此处,交换相关电位可以表示为

,(139)

其中为Gaspar-KohnSham电位,而由以下得出

(1.40)

该参数在数量级上对应于粒子势能与其平均动能的比率。

在(1.39)中,交换效应包含在中,所有相关性效应均包含在与电子密度相关的因子中。Wigner[48]认为中间电子密度的相关能可以通过在电子气的高密度和低密度极限值之间插值得到:


16



.(1.41)

根据,我们得到

.(1.42)

Hedin和Lundqvist[10]使用[38]中给出的结果估算并得到

.(1.43)

该文献提供了用于的另一个插值公式,其通常给出与非常相似的表达式。因此,当存在相关性时,电位(1.18)中的参数与半径有关,并且随着从零增加到内切球半径过渡金属则发生从至的变化。的平均值通常接近于相应的或。

交换相关电位的各种近似对于能带结构的影响将在1.3节详细讨论。

1.1.4  局部密度近似的修正

过去十年的许多计算均表明,局部自旋密度近似(LSDA)能够很好地描述固体的基态特性。LSDA已成为固态物理学第一性原理计算的实用工具,并在微观层面对材料特性的理解方面作出了显著贡献。然而,在使用LSDA时发现存在某些系统误差,例如对几乎所有元素固体均存在内聚能过高的估算,并且在许多情况下对晶格参数估算过低。并且LSDA也无法描述诸如莫特绝缘体和某些能带材料等高度相关系统的特性。最后,LSDA错误地预测对于而言,结构的总能量低于结构。

Hohenberg、Kohn和Sham在早期研究中引入了局部密度近似,但指出其用在密度不均匀系统时需要进行修正。Hohenberg和Kohn[6]提出的一个修正为近似

(144) 

其中核心与均匀介质的介电函数有关。这种近似在以下较弱密度变化的极限上是精确的
(145)
其中但真实系统的结果并不尽人意。对于自由原子而言,能量是无限的,表明无法满足加和规则(1.35)。

改善LSD近似存在几种方法,即基于精确方程的近似、梯度校正、自相互作用校正、LDA+U和轨道极化校正。





基于精确方程的近似   
交换相关能量(1.31)的方程表明,精确交换穴和近似交换穴之间的差异主要在于穴的非球形分量。由于这些对于没有贡献,即使在密度分布远非均匀的系统中,总能量和总能量差也可以相当不错。在中我们假设交换相关穴仅取决于电子的电荷密度。假设[49,50]取决于适当的平均将更为合适,

.(1.46)

可以选择确定的权重函数,这样函数在几乎恒定的密度极限内降低到精确的结果。近似(1.46)满足加和规则(1.11)。在[49,50]中提出了略有不同的权重函数方程式。该近似给出了原子总能量的改进结果。

如果我们在方程(1.30)中保留适当的前置因子,则可以获得另一种近似,
最终得到所谓的加权密度近似:

,(1.47)

其中选择满足加和规则(1.33)[50-52]。针对已经提出了不同的形式。Gunnarsson和Jones[43]提出一种的分析形式。

他们认为

,(1.48)

其中和为所要确定的参数。函数在较大距离上像一样,而为获得图像电位需要较大的距离。对于具有密度的均匀系统,我们要求模型函数均应满足的加和规则并可给出精确的交换相关能量。其将获得两个方程式:

,(1.49)
(1.50)

该方程足以确定两个参数和。

这一函数只有在以下多个限制条件下才是准确的:(1)均匀系统;(2)对于如氢原子的单电子系统,其可精确地消除电子自相互作用;(3)对于原子,其给出远离原子核的交换相关能量密度的正确行为,;(4)对于远在表面之外,其可给出图像电位。LSDA为条件(3)和(4)给出了定性错误答案,而对于条件(2)仅为近似消除。由于(2)得到满足,这种近似提供了我们将在下面讨论“自相互作用校正”。Barstel等[53]和Przybylski以及Barstel[54,55]在和V研究中使用了近似的变分。对于,他们发现近似能够正确将未占据能带向上移动。


18


对于他们得到了-能带和费米面的改进描述,而对于,费米面LSDA中的误差实际上被过度校正。对于半导体而言,发现能带隙的LSDA只有少量(Si,[56])或不存在改善。

梯度校正 梯度展开近似是改善LSDA的早期尝试。然而,原子[58,59]和凝胶表面[60]的计算显示,如果使用梯度校正计算得出的ab initio系数,则不会改善LSDA。中的误差已经由Langreth和Perdew[60,63]以及Perdew[64]进行过研究。据此分析,Langreth和Mehl[65]和和Langreth[66],以及Perdew[67]和Perdew和Wang[68]和Becke[69]已经实现了广义梯度近似。 

交换相关函数可以写为



其中为近似中的交换相关能量,其使用RPA电子气数据、和自自旋密度、和,与。通过在(1.51)中带入(或),我们可以恢复正常的梯度展开。

交换函数写为

(1.52)

其中,以及,与,和。对于自旋极化系统,可以从中得到交换能量。相关能量由下式给出

,(1.53)

其中为Ceperly-AIder参数化的相关能量[70],

(1.54)

利用,成为梯度展开的系数,以及

I 
与。






和函数已经在多种情况下进行了测试,并在这些情况下获得基态属性的一些重要结果。对于原子而言,可以发现与LSDA相比,总能量和解离能量均在函数中得到改善[65,66]。函数进一步改善了原子的总能量[67,68]。两个函数还改善了第一行双原子分子的结合能[72,73]。在一项关于和能带结构的研究中,Norman和Koelling[74]发现电位能够让费米面的而非获得改善。并利用[75]中的和梯度展开函数计算得出第三行元素的内聚能、晶格参数和体积模量。最后发现函数相比函数更好一些,且两者均被发现在近似中通常会去除一半的误差,而的结果相比局部密度近似更差一些。对于,和函数正确预测了铁磁性基态,而LSDA和梯度展开则预测了非磁性基态。Fe的基态也在[76,77]中通过成为铁磁性获得正确的预测。另一方面,没有解决过渡金属一氧化物和中遇到的问题。通过氧化物获得的磁力矩和能带结构与通过LSDA所获得的实际上存在同样的问题。

最近Engel和Vosko[78]分析了各种版本的交换电位,并认为,由于其形式特别简单,GGA不能同时复现交换能量和交换电位。他们表明的成功主要在于导致的被积函数误差消除,而不是复现准确的交换电位。为此他们构建了一种新的函数形式并试图更好地复现原子中的,甚至以更少为代价。¥¥¥¥¥该函数已应用于各种固体[79]并将和描述为反铁磁绝缘体,而的版本则产生金属基态。Engel和Vosko函数有利于磁性,例如,自旋磁化率增加,但是仍然保持与实验一致的非磁性。另一方面,取决于平衡体积和体积模量等准确描述的数量与实验不一致,例如,对于被测系统,晶格常数太大(1.5-9.6%),而体积模量太小(高达56%)。

自相互作用校正  
在密度表示型中,每个电子均通过库仑静电能进行自相互作用。这种非物理的相互作用将被精确表示型中的交换相关能量的贡献所相消。在近似中这种相消并不完美,但在数值上相当不错。近似函数对自相互作用的错误处理导致许多人考虑自相互作用校正(SIC)函数[80]。 这种校正在Thomas Fermi近似[81]、Hartree近似[82]、Hartree-Slater近似[83]以及近似[84-86]的背景下进行了研究。在LSD近似内,SIC函数采用了以下形式[80]


20



(1.56)
其中为对应于方程的解的电荷密度,而为具有自旋密度和均匀系统的交换相关(xc)能量。第二项减去电子自身的非物理库仑相互作用以及相应的xc能量。自旋轨道相应的xc电位为[80]

(1.57)

其中为个xc电位。SIC电位的一个重要特性是其轨道依赖性。
其将导致状态相关电位,因此该解并非为自动正交。因此有必要引入拉格朗日参数实现正交性

,(1.58)

其中为进入正常计算的有效电位,而为拉格朗日参数。

SIC消除了占据电子态的非物理自相互作用并降低了占据轨道能量。多位作者已经进行了原子计算[84-91]。总交换和相关能量中的误差远小于通过近似得到的误差。Perdew和Zunger[86]也表明孤立原子的最高占据轨道与实验电离能吻合得更好。

对于固体的应用存在严重的问题,原因是在占据轨道的幺正变换下,能量泛函不是不变的,我们可以在中构造出多个解。如果我们选择Bloch轨道,轨道电荷密度会在无穷体积限内消失。因此,对于这样的轨道,SIC能量恰好为零。这并不意味着SIC对于固体是不必要的,因为我们可以采用原子轨道或构造具有有限SIC能量的局部Wannier轨道。在很多固体计算中,SIC适用于局部轨道,这是在某些物理假设下选择的。这些方法部分成功地改善了宽间隙绝缘体的电子结构[92,93]。其能带隙明显好于LSD近似。对于LiCl,能带隙为10.6(SIC)、6.0(LSDA)和9.4-9.9eV(exp);对于Ar,其为13.5(SIC)、7.9()、和14.2eV(exp)[92,93]。在这些系统中,能带隙的大幅改进与相应自由原子特征值的改善有关。


21


表示型中一个长期存在的问题是对局部化的描述,例如,在莫特绝缘体中或在中因近藤效应导致的过渡中。由于这些能带隙的争议性质,针对绝缘反铁磁性过渡金属(TM)氧化物已经进行了数十年的深入研究。该近似归因于损失的某些方面对于内聚能贡献的损失[94],但能带隙太小或为零,磁矩在某些情况下也太小[95]。

最近,Svane和Gunnarsson以及Szotek等[97]对TM单氧化物在内进行了自洽计算,得到了能隙和磁矩,这与实验非常一致。他们没有强加任何物理假设,而是通过比较总能量来选择解。所选择的解由过渡金属能带的局部轨道以及氧能带扩展Bloch轨道组成。换句话说,SIC仅对过渡金属轨道有效,而氧轨道不受SIC直接影响。TM单氧化物的电子结构也通过[98]中的计算。作者仔细研究了选择轨道的标准,并尝试了局部和扩展氧轨道的两种解。该解表示为Muffin-Tin轨道(LMTO)的线性组合。结果表明,这些解的总能量很大程度上不受交换相关能量函数选择的影响。或者,如果通过选择解让所有轨道均被定位为Wannier函数,则能隙被高估。但是在这些解中,占据过渡金属能带和氧能带的相对位置与通过聚类配置-相互作用(CI)理论[99,100]得到的光电发射光谱分析一致。

近似中形成反铁磁矩的趋势在某些情况下被严重低估。其中一个例子为一维Habbard模型,其精确解已知[101]。能带隙、总能量、局部力矩和动量分布通过SIC近似得到明显优于的描述[102]。另一个例子为来自于高超导体,其中Stoner参数至少比使用近似要小2-3倍[103]。Svane和Gunnarsson[104]对的一个简单模型进行了计算,其中包括重要的 轨道以及指向原子的氧轨道。结果发现,与近似相比,SIC的反铁磁性趋势大大增强,SIC甚至可能高估了实验力矩。最近已经计算得出SIC近似中的电子结构[105]。该近似中复现了正确的反铁磁和半导体基态。与磁矩与能隙和其他电子激发能量与实验的良好定量一致性被发现。



22


 LDA + U方法  
与准确密度泛函之间的主要区别主要在于后者中的电位必须随电子数增加按整数值[106]以不连续方式跳跃,而前者中的电位是的连续函数。不存在电位跳跃(似为准确密度泛函)是因为无法描述过渡金属和稀土化合物等莫特绝缘体的能带隙。Gunnarsson和Schonhammer[107]表明单电子势的不连续性可以对能带隙产生很大的贡献。第二个重要的事实是,尽管轨道能量[为轨道占据数总能量的导数]通常与实验或更严谨的计算非常不一致但总能量通常相当不错。一个很好的示例就是氢原子的轨道能量为Ry(而非-1.0Ry)但总能量非常接近-1.0Ry[102]。在参考文献[108]中建议通过添加电位轨道相关校正(所谓的方法)克服这种缺陷。方法的主要思想与Anderson杂质模型[109]相同:将电子分离成两个子系统-局部化或电子,为此在模型哈密顿量(通过单点库仑相互作用)中应考虑库仑或相互作用),以及非局部化和电子,其可通过使用轨道无关单电子势描述。

的含义已经由Herring[110]仔细讨论过。例如,在每个原子存在n电子的电子系统中,定义为反应的能量成本

,(1.59)

即,在两个分别初始具有个电子的原子之间移动电子的能量成本。

让我们将离子视为一个具有电子波动数量开放系统。库仑能量的正确公式交互作为数量的函数电子由...给出应该。¥¥¥¥¥如果我们从总能量函数中减去这个表达式并添加类似Hubbard的项(忽略一下交换和非球形)我们将获得以下函数:

.(1.60)

轨道能量为(1.60)轨道占据的导数:


.(1.61)

这个简单的公式给出了轨道能量的转变-对于占据轨道的和非占据轨道的。对于轨道相关电位发现了类似的公式[其中变差并非取决于总电荷密度而是取决于特定第i轨道的电荷密度]:

(1.62)

表达式(1.62)恢复了精确密度泛函理论的单电子势不连续行为。

函数(1.60)忽略了库仑相互作用的交换和非球面性。如果我们将交换纳入考虑,那么对于具有相同自旋投影的电子,相互作用能量将为,为交换参数,并且它仍为不同的自旋:



23



.(I.63)
在中,部分考虑交换的方式是让具有不同自旋投影的电子数量相等。这将获得以下相互作用的库仑能量表达式作为电子总数的函数:



最后,我们可以考虑库仑的非球面性和交换相互作用,即依赖于哪些特定轨道和被占据,通过引入矩阵和:

,(1.64)
,(1.65)
(1.66)
(1.67)

为Slater积分和三个球谐函数积的积分

我们现在可以用[111]形式写出总能量函数:


(1.68)


(1.68)超过轨道占据的导数给出了轨道相关单电子势[111]的表达式:

,(169)
其中



24

为了计算矩阵和,应该知道电子的Slater积分)。屏蔽库仑和交换参数和可以在参考文献中所述的超晶胞近似中自洽计算。[112]。库仑参数可以用Slater积分确认。如果我们用所有可能的对平均矩阵和,我们应该可得到表达式和(1.63)。利用Clebsch-Gordan系数的特性可以证明该平均可给出[111]:

,(1.70)
,   (1.71)
.(1.72)

为了定义和所有三个Slater积分,只需要知道比例。在参考文献[113]中,和为所有金属的列表。所有离子的比例在0.62和0.63之间。因此,如果我们将此比例的值设定为0.625,则Slater积分的表达式为

,(1.73)
(1.74)

表达式(1.64-1.74)定义了所谓的方法[108,111]。当电子数量随着整数值增加时,函数最重要的特性是电位的不连续性以及最大占据轨道能量,就能带隙而言,不存在与精确密度函数[106]相比局部密度近似的主要缺陷。¥¥¥¥

应该提到的是,由于SIC方程是用齐次电子气理论的框架推导出来的,因此这种方法是的一个逻辑延伸,其显然不是方法。后一种方法存在与平均场(Hartree-Fock)方法相同的缺陷。方程(1.69)中的轨道相关单电子势为投影算符形式。 这意味着方法实际上取决于该算符中局部轨道集的选择。这是方法的类似基本Anderson模型观念的结果:将总变分空间分离成局部轨道子空间,它们之间的库仑相互作用采用哈密顿量中的哈伯德类项处理,而所有其他状态的子空间对于库仑相互作用的局部密度近似据认为是足够充分的。局部轨道选择的随意性并非像预期的那样关键。库仑相关效应对其非常重要的轨道确实很好地局限在空间内并保持其在固体中的原子特性。在各种电子结构计算方案中使用LDA+U近似的经验表明,结果对于局部轨道的特定形式并不敏感。


25


使用LDA+U方法发现[108]所有后3d过渡金属一氧化物以及高TC化合物的母系化合物均为电荷转移型大间隙磁绝缘体。此外,该方法预测到是一种低自旋铁磁体以及局部力矩型金属。该方法还成功应用于光电发射(X射线光电子能谱)和韧致辐射(bremsstahlung)等色谱的计算[111]。方法的优点是其能够在同一计算方案中同时处理离域导带电子和局部化电子。对于这种方法,重要的是确保这两种类型能带相对能量位置的正确再现。的例子让我们对此充满信心[114]。通常作为LSDA给出因占据和未占据能带自旋极化分裂的正确电子结构示例(LSDA在所有其他稀土金属中给出非物理费米能量峰值)。在LSDA中,能带之间的能量分离不仅被严重低估交换分裂仅为5而非实验值)而且非占据能带非常接近费米能量,因此强烈影响费米面和磁性基态性质(在LSDA计算中,反铁磁状态的总能量低于铁磁性状态,与实验相矛盾)。针对采用方法可获得计算和实验光谱之间的良好一致性,不仅适用于能带之间的分离,而且也适用于相对于费米能的峰值[114]。

在该方法的框架中,已经完成和MO Kerr光谱的研究[115]。对于,预测到关于的庞极性KelT自旋,与最近发现在中的最大可能自旋一致。

方法被证明是一种用于计算系统电子结构非常有效和可靠的工具,其中的库仑相互作用强到足以引起电子的局域化。其不仅适用于稀土离子的类核轨道(其中无限慢局部化轨道和无限快流动轨道子空间中的电子态分离是有效的),而且也适用于过渡金属氧化物等系统(其中轨道与氧轨道混成相当强烈)[108]。鉴于方法为通常不足以描述金属-绝缘体过渡和强相关金属的平均场近似,在某些情况下,诸如金属-绝缘体过渡和计算通过深入了解这些过渡属性获得重要信息[116]。

方法已成功应用于-化合物,但-态的局部化程度并非如同-金属那么清晰。在中已经得出异常磁相变理论[117]。 

