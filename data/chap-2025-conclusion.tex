\chapter{结论}
本文报道了一种含两个五边形和两个七边形的非苯型多环芳烃(DHR),通过三步反应可实现克级规模合成。两个“形式薁单元”的引入不仅影响了共轭结构,还使DHR具有平面性、热稳定性和空气稳定性。单晶结构分析与计算研究表明,E/F/B/I/H环内电子离域性更强,而七元环内电子主要呈定域状态。有趣的是,DHR表现出多氧化还原特性和反常的反卡莎规则$\mathrm{S}_2\rightarrow\mathrm{S}_0$发射,这可能与其高度稠合的5/6/7元环体系的独特共轭模式有关。此外,DHR晶体表现为p型半导体,空穴迁移率最高可达$0.082 \mathrm{cm}^2\cdot\mathrm{V}^{-1}\cdot\mathrm{s}^{-1}$。因此,含两个“形式薁单元”的DHR作为新型非苯型共轭骨架,是开发光电子材料的重要模块。目前,关于DHR的进一步功能化及更大扩展共轭体系的构建工作正在进行中。

%实验部分
%合成与表征细节、NMR谱图、吸收光谱、计算数据及OFET器件制备方法,均详见支持信息。DHR(CCDC 1971948)、化合物1(CCDC 1953759)、化合物2(CCDC 1953602)及化合物3(CCDC 1953606)的晶体学数据已存入剑桥晶体学数据中心(CCDC),可通过网址www.ccdc.cam.ac.uk/structures免费获取。

采用B3LYP/6-311G(d,p)方法对DHR分子进行结构优化,在此基础上计算了不同外电场对DHR分子几何结构、偶极矩、能量、前线轨道能级及红外光谱的影响,并采用TD-DFT WB97XD/DEF2-TZVP方法分析了不同外电场下DHR分子的激发态。结果如下:
\begin{enumerate}
	\item 随着外电场强度增大,原子周围的电子向与电场方向相反的方向重新分布,分子内电场方向随电子云变化而改变。当外电场方向与电子云方向相反时,原子间电场减弱,键长增大;当外电场方向与电子云方向一致时,原子间电场增强,键长减小。在外电场作用下,分子电荷重新分布导致分子受力不均匀,进而使分子发生弯曲变形。
		\item DHR分子在不同外电场作用下表现出强烈的振动斯塔克效应。随着外电场强度增大,当分子振动频率降低、键长增大时,出现红移现象;当振动频率升高、键长减小时,出现蓝移现象。同时,分子的摩尔吸光系数随外电场强度增大而增大。
		\item 外电场可改变DHR分子的激发特性与激发类型。当外电场强度小于0.025原子单位时,DHR分子的激发态为π$\rightarrow$π*局域激发;当F=0.025原子单位时,分子激发态由高度局域化的π$\rightarrow$π*激发转变为π$\rightarrow$σ*电荷转移激发。
		\item 随着外电场强度增大,DHR分子的亲电性减弱,亲核性增强。外电场可通过改变DHR分子的HOMO能级来改变其空穴转移速率。
\end{enumerate}

综上所述,在不同外电场作用下,DHR分子的几何结构、轨道类型、红外光谱及激发态性质均会发生变化。此外,通过施加外电场降低HOMO能级,可提高DHR分子的空穴迁移率。这些结果为未来DHR分子作为有机P型半导体材料在光电化学领域的研究提供了理论依据。

%\appendix
\section*{附录:~DHR及前驱体前驱体化合物的晶体结构与物理性质数据}
根据文献\cite{ACIE59-3529_2020},DHR和它的前驱体结构分别列于表\ref{Tab:DRH-structure},\ref{Tab:DRH-1-structure},\ref{Tab:DRH-2-structure},\ref{Tab:DRH-3-structure}。
%\subsection{DHR的X射线晶体学数据}
\begin{table}[h!]
    \centering
    \begin{tabular}{ll}
        \hline
	实验式 & \ch{C30H14} \\
        \hline
        分子量 & 374.41 \\
        \hline
        温度/K & 169.99~(11) \\
        \hline
        晶系 & 单斜晶系 \\
        \hline
        空间群 & $P2_1/c$ \\
        \hline
	晶胞参数 & a = 10.5545~(5)\AA~~~$\alpha = 90^{\circ}$ \\
		&b = 3.9469~(2)\AA~~~$\beta = 97.509~(5)^{\circ}$  \\
		&c = 20.5542~(10)\AA~~~$\gamma = 90^{\circ}$\\
        \hline
	体积/$\text{\AA}^3$ & 848.89~(7) \\
        \hline
        Z值 & 2 \\
        \hline
	计算密度$\rho_{\mathrm{calc}}$/$\mathrm{g}\cdot\mathrm{cm}^{-3}$ & 1.465 \\
        \hline
	吸收系数$\mu$/$\mathrm{mm}^{-1}$ & 0.638 \\
        \hline
        F~(000) & 388.0 \\
        \hline
	晶体尺寸/$\mathrm{mm}^3$ & $0.1\times0.05\times0.01$ \\
        \hline
	辐射源 & Cu-$\mathrm{K}_{\alpha}$~($\lambda=1.54184$) \\
        \hline
	数据收集的$2\theta$范围/$^{\circ}$ & 8.678至131.95 \\
        \hline
	指标范围 & $-12\leqslant h\leqslant12, -4\leqslant k\leqslant4, -24\leqslant l\leqslant16$ \\
        \hline
        收集的反射点数 & 7988 \\
        \hline
	独立反射点数 & 1486 $[R_{\mathrm{int}}=0.0485, R_{\mathrm{sigma}}=0.0429]$ \\
        \hline
        数据/限制条件/参数 & 1486/67/272 \\
        \hline
	拟合优度~($\mathrm{F}^2$) & 1.043 \\
        \hline
        最终R指数$[I\geqslant2\sigma~(I)]$ & $R_1=0.0405$,$wR_2=0.0959$ \\
        \hline
        最终R指数~(所有数据) & $R_1=0.0680$,$wR_2=0.1123$ \\
        \hline
	最大差值峰/孔/$\mathrm{e}\cdot\text{\AA}^{-3}$ & 0.10/-0.14 \\
        \hline
    \end{tabular}
    \caption{DHR的X射线晶体学数据}
    \label{Tab:DRH-structure}
\end{table}

%\subsection{前驱体化合物1的X射线晶体学数据}
\begin{table}[h!]
    \centering
    \begin{tabular}{ll}
        \hline
	实验式 & \ch{C30H22} \\
        \hline
        分子量 & 382.47 \\
        \hline
        温度/K & 173.15 \\
        \hline
        波长/\AA & 0.71073 \\
        \hline
        晶系 & 单斜晶系 \\
        \hline
        空间群 & $P2_1/c$ \\
        \hline
	晶胞参数 & a = 7.6867~(15)\AA~~~$\alpha = 90^{\circ}$ \\
		&b = 10.954~(2)\AA~~~$\beta = 94.39~(3)^{\circ}$  \\
		&c = 23.080~(5)\AA~~~$\gamma = 90^{\circ}$\\
        \hline
	体积/$\text{\AA}^3$ & 1937.5~(7) \\
        \hline
        Z值 & 4 \\
        \hline
        计算密度$\rho_{\mathrm{calc}}$/$\mathrm{g}\cdot\mathrm{cm}^{-3}$ & 1.311 \\
        \hline
	吸收系数/$\mathrm{mm}^{-1}$ & 0.074 \\
        \hline
        F~(000) & 808 \\
        \hline
	晶体尺寸/$\mathrm{mm}^3$ & $0.337\times0.291\times0.035$ \\
        \hline
	数据收集的$\theta$范围/$^{\circ}$ & 2.059至27.489 \\
        \hline
	指标范围 & $-9\leqslant h\leqslant 9, -14\leqslant k\leqslant 14, -26\leqslant l\leqslant 29$ \\
        \hline
        收集的反射点数 & 22413 \\
        \hline
	独立反射点数 & 4406 $[\mathrm{R}_{\mathrm{int}}=0.0514]$ \\
        \hline
	$\theta=25.242^{\circ}$时的完整性 & 99.5\% \\
        \hline
        吸收校正 & 基于等效反射的半经验校正 \\
        \hline
        最大/最小透射率 & 1.00000/0.86939 \\
        \hline
        精修方法 & 基于$F^2$的全矩阵最小二乘法 \\
        \hline
        数据/限制条件/参数 & 4406/0/271 \\
        \hline
        拟合优度~($F^2$) & 1.288 \\
        \hline
        最终R指数$[I\geqslant 2\sigma~(I)]$ & $\mathrm{R}_1=0.0663, \mathrm{wR}_2=0.1281$ \\
        \hline
	最终R指数~(所有数据) & $\mathrm{R}_1=0.0695, \mathrm{wR}_2=0.1296$ \\
        \hline
        消光系数 & 无 \\
        \hline
	最大差值峰/孔/$\mathrm{e}\cdot\text{\AA}^{-3}$ & 0.235/-0.158 \\
        \hline
    \end{tabular}
    \caption{前驱体化合物1的X射线晶体学数据}
    \label{Tab:DRH-1-structure}
\end{table}

%\subsection{前驱体化合物2的X射线晶体学数据}
\begin{table}[h!]
    \centering
    \begin{tabular}{ll}
        \hline
	实验式 & \ch{C30H18} \\
        \hline
        分子量 & 378.44 \\
        \hline
        温度/K & 173.15 \\
        \hline
        波长/\AA & 0.71073 \\
        \hline
        晶系 & 单斜晶系 \\
        \hline
        空间群 & $P2_1/c$ \\
        \hline
	晶胞参数 & a = 10.141~(3)\AA~~~$\alpha = 90^{\circ}$ \\
		&b = 13.470~(4)\AA~~~$\beta = 96.330~(3)^{\circ}$  \\
		&c = 13.809~(4)\AA~~~$\gamma = 90^{\circ}$\\
        \hline
		体积/$\text{\AA}^3$ & 1874.6~(9) \\
        \hline
        Z值 & 4 \\
        \hline
        计算密度$\rho_{\mathrm{calc}}$/$\mathrm{g}\cdot\mathrm{cm}^{-3}$ & 1.341 \\
        \hline
	吸收系数/$\mathrm{mm}^{-1}$ & 0.076 \\
        \hline
        F~(000) & 792 \\
        \hline
	晶体尺寸/$\mathrm{mm}^3$ & $0.352\times0.261\times0.109$ \\
        \hline
	数据收集的θ范围/$^{\circ}$ & 3.039至27.479 \\
        \hline
	指标范围 & $-13\leqslant h\leqslant12, -17\leqslant k\leqslant 16, -17\leqslant l\leqslant 17$ \\
        \hline
        收集的反射点数 & 15054 \\
        \hline
	独立反射点数 & 4273 $[\mathrm{R}_{\mathrm{int}}=0.0527]$ \\
        \hline
	$\theta=25.242^{\circ}$时的完整性 & 99.5\% \\
        \hline
        吸收校正 & 基于等效反射的半经验校正 \\
        \hline
        最大/最小透射率 & 1.00000/0.75431 \\
        \hline
        精修方法 & 基于$F^2$的全矩阵最小二乘法 \\
        \hline
        数据/限制条件/参数 & 4273/0/271 \\
        \hline
        拟合优度~($F^2$) & 1.164 \\
        \hline
	最终R指数$[I\geqslant 2\sigma~(I)]$ & $\mathrm{R}_1=0.0581, \mathrm{wR}_2=0.1197$ \\
        \hline
	最终R指数~(所有数据) & $\mathrm{R}_1=0.0662, \mathrm{wR}_2=0.1238$ \\
        \hline
        消光系数 & 无 \\
        \hline
	最大差值峰/孔/$\mathrm{e}\cdot\text{\AA}^{-3}$ & 0.271/-0.180 \\
        \hline
    \end{tabular}
    \caption{前驱体化合物2的X射线晶体学数据}
    \label{Tab:DRH-2-structure}
\end{table}

%\subsection{前驱体化合物3的X射线晶体学数据}
\begin{table}[h]
    \centering
    \begin{tabular}{ll}
        \hline
        实验式 & $\mathrm{C}_{15}\mathrm{H}_{9}$ \\
	\hline
        式量 & 189.22 \\
	\hline
        温度/K & 173.15 \\
	\hline
        波长 & 0.71073 \AA \\
	\hline
        晶体系统 & 单斜晶系 \\
	\hline
        空间群 & $P2_1/n$ \\
	\hline
	晶胞参数 & a = 10.012~(3)\AA~~~$\alpha = 90^{\circ}$ \\
		&b = 8.313~(3)\AA~~~$\beta = 110.939~(6)^{\circ}$  \\
		&c = 11.611~(4)\AA~~~$\gamma = 90^{\circ}$\\
		\hline
		体积/$\text{\AA}^3$ & 902.6~(5)  \\
	\hline
        $Z$ 值 & 4 \\
	\hline
        计算密度$\rho_{\mathrm{calc}}$/$\mathrm{g}\cdot\mathrm{cm}^{-3}$ & 1.393  \\
	\hline
	吸收系数/$\mathrm{mm}^{-1}$ & 0.079  \\
	\hline
        $F~(000)$ & 396 \\
	\hline
        晶体尺寸/$\mathrm{mm}^3$ & $0.522 \times 0.315 \times 0.159$ \\
	\hline
	数据收集的 $\theta$ 范围/$^{\circ}$& 3.347 - 27.495 \\
	\hline
        指标范围 & $-12 \leqslant h \leqslant 12$, $-10 \leqslant k \leqslant 9$, $-15 \leqslant l \leqslant 15$ \\
	\hline
        收集到的反射点 & 9824 \\
	\hline
        独立反射点 & 2055 [$\mathrm{R_{int}} = 0.0409$] \\
	\hline
        对 $\theta = 25.242^\circ$ 的完整性 & 99.3\% \\
	\hline
        吸收校正 & 基于等效反射的半经验法 \\
	\hline
        最大和最小透射率 & 1.00000 和 0.79415 \\
	\hline
        精修方法 & 基于 $F^2$ 的全矩阵最小二乘法 \\
	\hline
        数据/约束/参数 & 2055 / 0 / 136 \\
	\hline
        拟合优度 $\mathrm{GOF}$ on $F^2$ & 1.142 \\
	\hline
	最终 $\mathrm{R}$ 指数 $[\mathrm{I}>2\sigma~(\mathrm{I})]$ & $\mathrm{R_1} = 0.0495$, $\mathrm{wR_2} = 0.1197$ \\
	\hline
        最终 $\mathrm{R}$ 指数 [所有数据] & $\mathrm{R_1} = 0.0514$, $\mathrm{wR_2} = 0.1223$ \\
	\hline
        消光系数 & 无 \\
	\hline
	最大差值峰/孔/$\mathrm{e}\cdot \text{\AA}^{-3}$  & 0.232 和 -0.160 \\
        \hline
    \end{tabular}
    \caption{化合物 3 的 X 射线晶体学数据}
    \label{Tab:DRH-3-structure}
\end{table}
