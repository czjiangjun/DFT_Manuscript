\chapter{结论}
本文报道了一种含两个五边形和两个七边形的非苯型多环芳烃(DHR),通过三步反应可实现克级规模合成。两个“形式薁单元”的引入不仅影响了共轭结构,还使DHR具有平面性、热稳定性和空气稳定性。单晶结构分析与计算研究表明,E/F/B/I/H环内电子离域性更强,而七元环内电子主要呈定域状态。有趣的是,DHR表现出多氧化还原特性和反常的反卡莎规则S₂→S₀发射,这可能与其高度稠合的5/6/7元环体系的独特共轭模式有关。此外,DHR晶体表现为p型半导体,空穴迁移率最高可达0.082 cm²·V⁻¹·s⁻¹。因此,含两个“形式薁单元”的DHR作为新型非苯型共轭骨架,是开发光电子材料的重要模块。目前,关于DHR的进一步功能化及更大扩展共轭体系的构建工作正在进行中。

%实验部分
%合成与表征细节、NMR谱图、吸收光谱、计算数据及OFET器件制备方法,均详见支持信息。DHR(CCDC 1971948)、化合物1(CCDC 1953759)、化合物2(CCDC 1953602)及化合物3(CCDC 1953606)的晶体学数据已存入剑桥晶体学数据中心(CCDC),可通过网址www.ccdc.cam.ac.uk/structures免费获取。

采用B3LYP/6-311G(d,p)方法对DHR分子进行结构优化,在此基础上计算了不同外电场对DHR分子几何结构、偶极矩、能量、前线轨道能级及红外光谱的影响,并采用TD-DFT WB97XD/DEF2-TZVP方法分析了不同外电场下DHR分子的激发态。结果如下:
\begin{enumerate}
	\item 随着外电场强度增大,原子周围的电子向与电场方向相反的方向重新分布,分子内电场方向随电子云变化而改变。当外电场方向与电子云方向相反时,原子间电场减弱,键长增大;当外电场方向与电子云方向一致时,原子间电场增强,键长减小。在外电场作用下,分子电荷重新分布导致分子受力不均匀,进而使分子发生弯曲变形。
		\item DHR分子在不同外电场作用下表现出强烈的振动斯塔克效应。随着外电场强度增大,当分子振动频率降低、键长增大时,出现红移现象;当振动频率升高、键长减小时,出现蓝移现象。同时,分子的摩尔吸光系数随外电场强度增大而增大。
		\item 外电场可改变DHR分子的激发特性与激发类型。当外电场强度小于0.025原子单位时,DHR分子的激发态为π→π*局域激发;当F=0.025原子单位时,分子激发态由高度局域化的π→π*激发转变为π→σ*电荷转移激发。
		\item 随着外电场强度增大,DHR分子的亲电性减弱,亲核性增强。外电场可通过改变DHR分子的HOMO能级来改变其空穴转移速率。
\end{enumerate}

综上所述,在不同外电场作用下,DHR分子的几何结构、轨道类型、红外光谱及激发态性质均会发生变化。此外,通过施加外电场降低HOMO能级,可提高DHR分子的空穴迁移率。这些结果为未来DHR分子作为有机P型半导体材料在光电化学领域的研究提供了理论依据。

