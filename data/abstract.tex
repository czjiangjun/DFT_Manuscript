% !TeX root = ../thuthesis-example.tex

% 中英文摘要和关键字

\begin{abstract}
	近年来,含六元环/五元环或六元环/七元环的稠环芳烃\textrm{(PAHs)}已得到广泛研究,因为同时包含五元-六元-七元环的稠环芳烃较为罕见,且具有较为特殊的属性。本文研究的是一种非苯型纳米石墨烯——双七元环\textrm{[ijkl,uvwx]rubicene}\footnote{\textrm{rubicene有时译作``玉红省''。}}~\textrm{(DHR)},它是二苯并[bc,kl]蔻的同分异构体,分子结构中含有两个五元环和两个七元环。通过\textrm{Scholl}反应\textrm{(Scholl reaction)}和脱氢反应,研究发现了一种高效、可规模化的\textrm{DHR}合成方法。\textrm{DHR}的晶体结构表明,其苯环、两个五元环和两个七元环呈共平面排列。对\textrm{DHR}的键长分析、\textrm{Z}轴\textrm{1Å}处各向同性化学屏蔽表面\textrm{(ICSS(1)zz)}及$π$电子定域化轨道指示函数\textrm{(LOL-π)}计算结果显示,两个表观的五元环和七元环将会对共轭结构产生影响,使苯环和五边形环的$π$电子离域性更强。循环伏安法研究表明,\textrm{DHR}具有多个氧化和还原电位。有意思的是,\textrm{DHR}也表现出特殊的$\mathrm{S}_0\rightarrow\mathrm{S}_2$吸收以及反常的反\textrm{Kasha}规则\textrm{(anti-Kasha)}~$\mathrm{S}_2\rightarrow\mathrm{S}_0$发射。此外,\textrm{DHR}晶体具有半导体特性,空穴迁移率最高可达$0.082~\mathrm{cm}^2\cdot\mathrm{V}^{-1}\cdot \mathrm{s}^{-1}$。

为了加深对%双环庚基红荧烯(DHR)是
\textrm{DHR}的\textrm{P}型半导体材料的电子结构的认知,本文采用密度泛函理论\textrm{(DFT)}中的\textrm{B3LYP/6-311G~(\textit{d},~\textit{p})}方法对\textrm{DHR}分子的几何结构进行优化。在此基础上,运用相同的方法和基组,计算并分析了沿$x$轴施加不同电场\textrm{(0-0.025原子单位)}对\textrm{DHR}分子的几何结构、能量、偶极矩及分子红外光谱的影响。随后,采用含时密度泛函理论\textrm{(TD-DFT)}中的\textrm{WB97XD/DEF2-TZVP}方法,计算了不同电场下\textrm{DHR}分子的前20个激发态,并通过空穴-电子分析对主要激发态进行了研究。最后,通过计算前线轨道能级,探究了外电场对\textrm{DHR}空穴迁移率的影响。结果表明:~\textrm{DHR}分子在不同外电场作用下表现出强烈的振动\textrm{Stack}效应,随着红移或蓝移的出现,分子的摩尔吸光系数也发生重新分布;~外电场可改变DHR分子的电子激发特性与激发类型,当电场强度\textrm{F=0.025}原子单位时,$\mathrm{S}_0\rightarrow\mathrm{S}_1$和$\mathrm{S}_0\rightarrow\mathrm{S}_{13}$激发态由高度局域化的$π\rightarrow π^{\ast}$激发转变为$π\rightarrow\sigma^{\ast}$电荷转移激发;~外电场可通过改变\textrm{DHR}分子的最高占据分子轨道\textrm{(HOMO)}能级来改变其空穴转移速率,当$F>0.025$原子单位时,\textrm{DHR}分子的空穴转移速率将超过$0.082~\mathrm{cm}^2\cdot\mathrm{V}^{-1}\cdot\mathrm{s}^{-1}$。该研究对后续将\textrm{DHR}分子设计为有机半导体材料提供了理论依据。
  \thusetup{
	  keywords = {\textrm{DHR}, 外电场, 晶体结构, 分子与电子结构模拟, 空穴迁移, 激发特性},
  }
\end{abstract}

%\begin{abstract*}
%  An abstract of a dissertation is a summary and extraction of research work and contributions.
%  Included in an abstract should be description of research topic and research objective, brief introduction to methodology and research process, and summary of conclusion and contributions of the research.
%  An abstract should be characterized by independence and clarity and carry identical information with the dissertation.
%  It should be such that the general idea and major contributions of the dissertation are conveyed without reading the dissertation.
%
%  An abstract should be concise and to the point.
%  It is a misunderstanding to make an abstract an outline of the dissertation and words “the first chapter”, “the second chapter” and the like should be avoided in the abstract.
%
%  Keywords are terms used in a dissertation for indexing, reflecting core information of the dissertation.
%  An abstract may contain a maximum of 5 keywords, with semi-colons used in between to separate one another.

  % Use comma as separator when inputting
%  \thusetup{
%    keywords* = {keyword 1, keyword 2, keyword 3, keyword 4, keyword 5},
%  }
%\end{abstract*}
