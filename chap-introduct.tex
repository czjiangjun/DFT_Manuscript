\chapter{引言} \label{chap:intro}
\section{项目研究的意义}
二十世纪七十年代以来,随着计算机科学和密度泛函理论\textrm{(Density Functional Theory, DFT)}\cite{PRB136-864_1964,PRA140-1133_1965,Parr-Yang,CR91-651_1991}、分子动力学\textrm{(Molecular Dynamics, MD)}的发展和融合,计算材料科学\textrm{(computational materials science)}与物理、化学、工程力学以及应用数学等诸多基础和应用学科交叉日益紧密,逐渐成为一门新兴的独立学科。\cite{Nat-Mat3-429_2004,App-CataA5-254_2003,JACS125-4306_2003,JCC5-472_2003,Mater_Sci-Tech16-1_2005,Nature392-694_1998}材料计算模拟研究的不断深入,也推动了材料模拟核心计算软件日益成熟。核心计算软件解决的是物质运动基本方程求解和基本物性计算的问题,除此之外,完整的材料模拟过程还包括“计算前处理”的问题建模、计算参数选择到“计算后处理”的数据分析,几乎每一部分都有大量具体的工作有赖人工,频繁的人机交互严重影响了模拟计算流程的顺畅与效率。进入二十一世纪,大规模高性能计算的兴起,逐渐变革了材料计算的研发模式。通过计算流程设计,衔接不同尺度的计算过程,降低人机交互频次,优化模拟流程中的任务调度,实现材料计算流程的自动化与跨尺度物性模拟,形成完善的材料数据库,已经成为材料计算软件发展的新趋势。

通过理论计算模拟不同尺度下的材料物性,不但可以节约新材料研制环节中的设计、试验和制造成本,缩短研发周期,还可能提供实验过程难以获得的信息。以燃烧催化材料的研究为例,在催化剂作用下,多相催化-氧化反应和燃烧过程的自由基反应同时发生,\cite{ACSSym-Seri12-495_1992}这使得催化燃烧反应的研究变得非常困难,因此严重制约了燃烧催化剂研发。模拟计算可以不受燃烧实验的条件限制,对催化燃烧的微观过程给出理性的分析,为实验制备与合成催化剂研究提供理论依据。\cite{PCC118-1999_2014,Nat-Commun8-14621_2017,CES56-2659_2001}但是催化燃烧过程是典型的跨尺度问题,对模拟计算软件和计算流程都提出了具体的需求,主要的研究难点是:
\begin{enumerate}
	\item 催化反应发生在可燃物-金属表面或缺陷(包括空位、掺杂等)-氧气分子之间,这种异相界面结构的建模,比均匀的体相材料更为复杂;除了分子-表面作用,还要考虑催化燃烧过程中产生的大量活性自由基存在对反应过程的影响。
	\item 分子-金属表面或缺陷的物理作用和化学反应过程模拟,涉及大量的电子结构中间态,计算这些物理、化学过程的热力学函数,核心软件的迭代收敛非常困难。
	\item 燃烧过程产生海量的自由基,大部分自由基反应属于基元反应,活化能低,反应速度快;只有少数活化能高的基元反应对反应速度影响大(称为决速步),是确定催化反应动力学的关键,需要精确的电子结构计算。但是决速步反应在自由基反应中仅占1\%,这使得异相界面催化反应过程比一般化学反应的动力学模拟计算复杂。
\end{enumerate}
因此在模拟计算中需要考虑\textrm{DFT-MD}的耦合迭代,如何从海量基元反应中快速高效地确认决速步成为动力学机理研究的重点内容。提升核心计算软件的并行能力,毫无疑问将是有效手段之一。但是由于催化燃烧模拟的基元反应的量级为$10^3\sim10^4$,即使在高性能计算系统中,自动流程通过作业管理系统提交基元反应的~\textrm{Kohn-Sham}~方程求解,依然会遇到严重的 排队问题。通过计算流程层面的算法设计,实现海量基元反应的~\textrm{Kohn-Sham}~方程并发提交和筛选,无疑将对快速确定反应决速步有重要的意义。催化燃烧研究计算流程中引入合理的并发、筛选和调度机制,不但有望降低基元反应Konh-Sham方程求解的单位时间,还将使得DFT-MD计算耦合更紧密,提高迭代收敛的稳定性,节约模拟全过程的时间。

作为催化燃烧研究的典型代表,研发适合天然气燃烧的催化剂一直是能源利用领域的重要课题。\cite{GasHeat22-12_2002,GasHeat22-523_2002,GasHeat21-61_2001}天然气的主要成分是甲烷($\mathrm{CH}_4$),在没有催化剂的条件下,甲烷在空气中直接燃烧,温度高达1600$^{\circ}\mathrm{C}$左右,并生成氮氧化物($\mathrm{NO}_x$)等污染物质。使用催化剂,不仅可以降低$\mathrm{CH}_4$ 的起燃温度和燃烧峰值温度,减少污染物生成;并且燃烧利用率可以达到99.9\%,接近完全氧化,基本不会形成\textrm{CO}和碳氢化合物,因此$\mathrm{CH}_4$ 催化燃烧可以达到近零污染排放。

2014年底,国务院颁布的 《能源发展战略行动计划2014-2020》指出,我国优化能源结构的路径是:降低煤炭消费比重,提高天然气消费比重,大力发展风电、 太阳能、 地热能等可再生能源,安全发展核电。提高清洁能源比重,是我国能源结构调整的必由之路。结合我国目前的能源使用方式,天然气是短中期替代煤炭的最佳选择。天然气作为清洁的化石能源,具备资源丰富、性价比高等优势,但是由于$\mathrm{CH}_4$催化燃烧过程复杂性的制约,直到现在,催化剂与$\mathrm{CH}_4$相互作用的微观机理仍不清楚,开发适合天然气在$377\sim877^{\circ}\mathrm{C}$范围燃烧的催化剂,仍是研究的重点和难点。\cite{ProcChem15-242_2003}因此加快$\mathrm{CH}_4$燃烧机理研究,对于加快我国能源消费结构调整,减少空气污染具有很重要的科学价值。

综上所述,对于异相界面催化反应机理的理论研究,需要从跨尺度\textrm{DFT-MD}约化耦合算法、高并发基元反应筛选算法和复杂的催化燃烧自动流程实现等方面作出新的探索和发展。本项目立足现有的\textrm{DFT-MD}计算流程,面向加快$\mathrm{CH}_4$ 燃烧催化材料开发的迫切需求,研究跨尺度\textrm{DFT-MD}约化耦合算法、高并发\textrm{Kohn-Sham}方程任务筛选与调度平衡算法,支持催化燃烧过程涉及的复杂第一原理-反应动力学的机理研究。通过本项目的研究成果,为推动我国能源消费结构调整,加快天然气资源利用的步伐,降低天然气燃烧的污染问题,提供科学依据。

%天然气具有储量丰富、热效率高、价格低廉、污染较小等优点,许多专家和学者认为在21世纪天然气的能源地位将不断提高。但是由于天然气的主要成分甲烷(\textrm{CH}$_4$)的直接燃烧温度高达1600$^{\circ}\mathrm{C}$左右,天然气在空气中燃烧生成氮氧化物(\textrm{NO}$_x$)等物质,对环境造成一定的污染。甲烷催化燃烧可以通过催化作用,降低燃料的起燃温度和燃烧的峰值温度,提高甲烷燃烧利用率,减少大气污染物的生成。因此,甲烷催化燃烧技术一直是能源利用领域的热点课题。
%属催化剂和非贵金属催化剂,其中非贵金属催化剂又包括钙钛矿型金属氧化物催化剂、六铝酸盐类催化剂、过渡金属复合氧化物催化剂。在没有催化剂的情况下,甲烷在空气中燃烧时,自由基反应剧烈,反应温度急剧上升; 

\section{研究内容和研究目标}
\subsection{研究内容}
开发适合催化材料微观机理模拟的\textrm{DFT-MD}自动流程软件,研究动力学约化耦合算法、高通量基元反应\textrm{Kohn-Sham}方程筛选与调度平衡算法,以$\mathrm{CH}_4$催化燃烧反应研究为牵引,面向微观尺度下异相界面催化燃烧反应动力学机理模拟计算的高效率实现。具体内容如下:
\begin{itemize}
	\item 适应催化反应动力学的\textrm{DFT-MD}的约化耦合程序流程设计。主要包括:
		\begin{enumerate}
			\item 研究并设计面向化学反应的自适应高通量作业调度与负载平衡算法。化学反应是典型的\textrm{DFT-MD}跨尺度模拟,核心计算软件本身可以实现并行计算,主要依赖于系统的作业调度和管理系统;而反应势能面的确定,则要大量的人机交互,针对含有多决速步的\textrm{DFT-MD}计算流程,有必要研究自动流程与调度算法。在计算资源有限的情况下,优化并发计算流程,保证负载均衡。
			\item 研究自由基反应中的活化能特征,确定影响反应进程的决速步反应的活化能范围,在自动流程中设计算法筛选掉大量不重要的自由基反应,提升\textrm{DFT}计算确定活化能的效率,预计可将自由基反应计算数目由$10^3\sim10^4$降为$10\sim10^2$数量级。
			\item 针对催化反应动力学中\textrm{DFT-MD}耦合算法研究。催化反应模拟与一般化学反应计算的显著不同是,反应过程由于催化剂的参与形成异质界面。因此在分子动力学计算中必须考虑多决速步引起的多尺度动力学计算;通过研究\textrm{Verlet-SHAKE}算法,特别是多决速步的存在对\textrm{SHAKE}迭代的影响,确定用\textrm{DFT-MD}耦合模拟催化反应动力学的算法,提高反应动力学的\textrm{Verlet}算法稳定性。
		\end{enumerate}

	\item 集成上述研究成果,研制适应多决速步化学反应动力学机理研究的自动计算流程软件框架,支撑典型催化燃烧的复杂\textrm{DFT-MD}模拟 。
%Fig.1 集成ASE-MP软件自动流程功能模块后开发的多决速步化学反应动力学自动流程软件框架.
\end{itemize}
\subsection{研究目标}
\begin{enumerate}
	\item 提出适应于异相界面催化反应动力学反应的\textrm{DFT-MD}的约化耦合算法,通过基元反应活化能确定影响进程的决速步,筛选对反应机理干扰的大量自由基反应;优化决速步反应得到的势能面(\textrm{DFT-MD}耦合的关键)提升分子反应动力学计算的迭代稳定性。
	\item 基于任务管理系统设计调度计算流程的高通量\textrm{Kohn-Sham}方程求解的初级调度算法,以此为共性特征形成含有多决速步或非关联第一原理计算流程作业框架,结合具体的燃烧反应动力学求解流程实例化,得到适用于催化反应动力学研究的流程版本,经优化的并发DFT求解流程可将电子步计算自动流程的计算效率提高20\%左右。
	\item 研制适应多决速步化学反应的复杂反应动力学机理研究的自动流程软件,并在商业计算机和国产高性能超级计算机上成功测试运行,成为可跨平台典型催化燃烧反应动力学模拟的自动流程软件。
\end{enumerate}

\section{研究方案与技术路线}
对于适合异相界面催化反应动力学反应的DFT-MD的约化耦合算法研究,可行的方法按如下两步完成:
\begin{enumerate}
	\item 通过自由基反应的活化能筛选决速步:参与活化能筛选的自由基反应数量约为$10^3$量级,在计算流程中设计自由基反应的反应物-产物结构批量产生算法,要求单次产生的反应结构模型在$10^2$量级,通过计算流程提交自由基反应并求解\textrm{Kohn-Sham}方程,得到各自由基反应的活化能。该步的关键技术为流程中有自由基参与分子-表面团簇结构批量产生,一般可结合分子动力学或\textrm{Monte-Carlo}方法的\textrm{Markov chain}思想得到所需要的计算模型。
	\item \textrm{DFT-MD}的约化耦合算法的实现。根据决速步反应的活化能和原子-表面的团簇结构,利用机器学习优化算法,包括\textrm{Gaussian}回归、\textrm{Bayesian}优化、\textrm{ANN}方法等,得到反应过程(反应通道)的势函数,将该势函数用于分子动力学迭代计算,根据\textrm{Verlet-SHAKE}算法,可设计\textrm{DFT-MD}迭代流程,使反应动力学计算过程稳定收敛。流程设计的关键技术是在自动流程中的引入机器学习模块优化反应势能面,以及合适的机器学习方法的对比与选择。\textrm{DFT-MD}约化耦合计算比普通的\textrm{DFT-MD}流程更稳定,保证动力学计算过程的快速收敛。
	\item 对于决速步的自适应高通量\textrm{Kohn-Sham}方程求解流程的并发算法,由于不同自由基反应或决速步反应的\textrm{Kohn-Sham}方程迭代次数差别较大,必须经过测试后才能配置比较合理的作业并发算法。一种可行的方案是首先确定不同决速步\textrm{Kohn-Sham}方程迭代计算时长差别,通过计算流程获取作业管理系统分配的可用计算资源,在计算负载均衡要求前提下,通过优化调度算法对不同的决速步分配不同的计算资源。该步的关键技术是自动流程对计算资源的优化配置算法,通过合理分配资源,提高计算流程中的迭代计算效率。
	\item 适应催化燃烧反应动力学的高通量\textrm{DFT-MD}耦合的自动流程实现,关键是设计出\textrm{DFT-MD}迭代耦合与并发式作业提交的协同算法,提高反应动力学求解的收敛稳定性。对此考虑的实现思路是在现有自动流程基础上,将\textrm{Verlet-SHAKE}迭代算法求解\textrm{DFT-MD}约化耦合的自动流程分解成数据库元素,利用数据库支持的\textrm{Hadoop}协同调度和推荐算法维护和管理;并发式\textrm{Kohn-Sham}方程求解阶段,主要是作业管理系统分配的计算资源与\textrm{DFT}方程任务的匹配优化;决速步反应计算得到的活化能,经机器学习方法优化后生成原子间相互作用函数(势函数),用于\textrm{MD}模拟后返回决速步\textrm{DFT}计算并迭代循环,最后得到稳定收敛的催化燃烧反应动力学计算流程。%本软件设计的自动流程的技术设计路线示意图见Fig. 2。
\end{enumerate}
