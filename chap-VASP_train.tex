\setcounter{chapter}{10}
\chapter{VASP的安装与应用示例}\label{chap:VASP_train}
\textrm{VASP}是目前最流行的第一原理材料模拟软件包之一,由维也纳大学\textrm{(Universit\"at~Wien)}的\textrm{G.~Kresse}等开发。\cite{VASP_manual}\textrm{VASP}以其优异的计算性能,成为\textrm{DFT}计算的执牛耳者。\cite{CPC177-6_2007}只要几十个\textrm{CPU}的并行规模,即可处理上百个原子的体系。\textrm{VASP}能在众多材料模拟软件中脱颖而出,究其原因,一方面是因为\textrm{VASP}采用的\textrm{PAW~(Projector Augmented-Wave)}方法\cite{PRB50-17953_1994,PRB59-1758_1999},平衡了传统赝势方法和全电子计算优点,保证了计算的精度;~另一方面是在计算过程中采用实空间优化的投影函数\textrm{(Projector)}的思想,将主要的计算任务变换到实空间完成,大大节省了基组的维度,并能兼顾计算过程中的计算精度和效率。在此基础上,\textrm{VASP}通过引入多种优化算法,有效提升了矩阵对角化和电荷密度搜索的效率。此外特别值得一提的是,程序的并行实现中,重点解决了快速\textrm{Fourier}变换\textrm{(Fast Fourier Transformation,~FFT)}网格与并行计算的计算格点的分配平衡问题,大大提升了软件的并行效率。相比于其他第一原理计算软件,\textrm{VASP}从物理思想与方法、优化算法和并行计算实现等多个方面都有出色的性能展示度。关于\textrm{VASP}软件实现的文献,主要如下:~
\begin{enumerate}
	\item 物理思想与方法:~\textrm{PAW}赝势方法:~参阅文献\inlinecite{PRB50-17953_1994,PRB59-1758_1999};~投影函数的实空间优化,可参阅文献\inlinecite{JPCM6-8245_1994,PRB44-13063_1991,PRB44-8503_1991}
	\item 优化算法:~主要包括准\textrm{Newton}方法、\textrm{Davidson}方法、共轭梯度(\textrm{Conjugate Gradient, CG})方法和\textrm{RM-DIIS~(Residual minimization scheme, direct inversion in the iterative subspace)}方法,可参阅文献\inlinecite{CMS6-15_1996,PRB54-11169_1996}
	\item 高效的并行计算:~ \textrm{FFT}网格与计算格点的分配:~\textrm{VASP}中应用了大量的\textrm{FFT}变换,为了提高计算效率,\textrm{VASP}在实程序实现时建立了\textrm{FFT}网格和并行计算网格的映射关系,实现\textrm{FFT}网格计算的并行化。 %可参阅\textrm{VASP}源代码的\textrm{mgrid.F}
\end{enumerate}

\section{VASP概览}
\textrm{VASP}软件基于\textrm{DFT}计算材料的基本物性,是新材料研发中的有力工具。\textrm{VASP}的计算流程如图\ref{Fig:VASP_procedure}所示。
\begin{enumerate}
	\item \textrm{VASP}计算条件与控制
		\begin{itemize}
			\item 计算前提,需要提供每个元素的赝势,可以是超软赝势(\textrm{Ultra Soft Pseudopotential, USPP}),也可以是\textrm{PAW}赝势。
			\item 计算中可以选择使用的泛函,包括\textrm{LDA}、\textrm{PW91}和\textrm{PBE},以及一些杂化泛函。
			\item 通过\textrm{FFT}计算正空间和倒空间中的电荷密度和\textrm{Kohn-Sham~Hamilitonian}的能量项。
			\item \textrm{Kohn-Sham}方程求解:~矩阵迭代对角化和电荷密度混合。
		\end{itemize}
	\item \textrm{VASP}计算内容
		\begin{itemize}
			\item 电子和离子能量的最小化(结构弛豫或优化)直至能量收敛。
			\item 基本性质计算:~总能,能带结构(\textrm{Band structure}),态密度(\textrm{Density of State, DOS}),声子谱等。
			\item 可用于计算原子、分子、体相固体、表面和团簇等。
		\end{itemize}
	\item \textrm{VASP}的计算规模和实用性
		\begin{itemize}
			\item 并行计算的\textrm{VASP},可以最多处理约4000价电子的体系。
		\end{itemize}
\end{enumerate}

\textrm{VASP}为各类计算和控制参数提供了默认值,对于一般的简单体系,这些参数默认值基本能满足计算要求。这样的设计,极大地方便了初学者快速上手。
\begin{figure}[h!]
\centering
\includegraphics[height=4.5in,width=3.6in,viewport=0 0 480 630,clip]{VASP_procedure.png}
%\includegraphics[height=1.8in,width=4.in,viewport=30 210 570 440,clip]{PAW_projector.eps}
%\caption{\small \textrm{The Flow of calculation for KS-ground state in VASP.}}%(与文献\cite{EPJB33-47_2003}图1对比)
\caption{\small \textrm{VASP}中根据\textrm{DFT}计算体系基态能量的基本流程.}%(与文献\cite{EPJB33-47_2003}图1对比)
\label{Fig:VASP_procedure}
\end{figure}

\section{VASP的安装}
\subsection{软件编译的一般过程}\label{SubSec:Compilation}
高级语言编写的程序,从源代码变成可执行的文件,必须经过编译\textrm{compilation}。编译的功能类似于翻译,把人类编写的符合高级语言语法规则的语句,翻译成计算机可以执行的指令。完整的编译过程包括四个步骤,如图\ref{Fig:Compiler}所示。
\begin{figure}[h!]
\centering
\includegraphics[height=3.0in,viewport=0 0 470 560,clip]{compiler_procedure.png}
%\includegraphics[height=1.8in,width=4.in,viewport=30 210 570 440,clip]{PAW_projector.eps}
\caption{\small \textrm{软件编译的步骤示意(预编译-编译-汇编-链接).}}%(与文献\cite{EPJB33-47_2003}图1对比)
\label{Fig:Compiler}
\end{figure}
\begin{itemize}
	\item 预处理\textrm{(Pre-Processing)}:~用预处理器\textrm{cpp}将根据源代码中以字符\#开头的指令要求,修改原始程序。如\textrm{C}程序中\textcolor{purple}{\textrm{\#include <stdio.h>}}指令告诉预处理器读系统头文件\textrm{stdio.h}的内容,并把它直接插入到程序文本中去。结果就得到另外一个\textrm{C}程序。%通常是 以.i作为文件扩展名的。
	\item 编译\textrm{(Compiling)}:~在编译阶段,编译器\textrm{compiler}会检查代码的规范性,是否存在拼写和语法错误等,确定代码实际执行的具体任务和指令。检查无误后,编译器把代码翻译成汇编语言。汇编语言非常有用,它为不同高级语言不同编译器提供了通用的语言。比如\textrm{C}编译器和\textrm{Fortran}编译器产生的输出文件用的都是一样的汇编语言。用户可以使用\textrm{-S}选项来进行查看,该选项只完成编译过程而不进行汇编,生成汇编代码。
	\item 汇编\textrm{(Assembling)}:~汇编阶段是把编译阶段生成的\textrm{.s}文件(汇编代码)转成目标文件。用户可使用\textrm{-c}选项就可看到汇编代码将转化为\textrm{.o}的二进制目标代码。
	\item 链接\textrm{(Linking)}:~汇编生成二进制代码之后,就进入了链接阶段。用户采用高级语言编写的程序中,有相当一部分功能是由外部软件函数库或其他目标代码提供,无需用户自行编写(如输入输出命令、复杂函数的解法器等),因此程序执行时就需要和这些函数库的支持。函数库一般分为静态库(后缀名一般为\textrm{.a})和动态库(后缀名一般为\textrm{.so})两种
		\begin{itemize}
			\item 静态库是指编译链接时,把库文件的代码全部加入到可执行文件中,因此生成的文件比较大,但在运行时就不再需要库文件支持
			\item 动态库在编译链接时并没有把库文件的代码加入到可执行文件中,而在程序执行时由运行时链接文件加载库,这样可以节省系统的开销
		\end{itemize}
完成了链接之后,编译器就可以生成最终可执行文件。
\end{itemize}
\subsection{VASP的编译}
本讲义将以\textrm{VASP~5.X}为例介绍\textrm{VASP}的编译过程,关于更高版本的\textrm{VASP~6.X}软件的编译,用户可参阅~\url{https://www.vasp.at/wiki/index.php/Installing_VASP.6.X.X}的介绍。\textrm{VASP~5.X}软件代码结构如图\ref{Fig:VASP_build}所示:~
\begin{figure}[h!]
\centering
\includegraphics[height=1.5in,viewport=0 0 490 140,clip]{VASP_build.png}
%\includegraphics[height=1.8in,width=4.in,viewport=30 210 570 440,clip]{PAW_projector.eps}
\caption{\small \textrm{VASP}软件的代码结构.}%(与文献\cite{EPJB33-47_2003}图1对比)
\label{Fig:VASP_build}
\end{figure}
\begin{itemize}
	\item \textcolor{purple}{root/}:~是\textrm{VASP}代码的总目录,其中存放总编译安装脚本(\textrm{Makefile})和各种子目录
	\item \textcolor{purple}{root/src}:~存放\textrm{VASP}的全部源代码和一些底层的\textrm{makefile}
	\item \textcolor{purple}{root/src/lib}:~存放\textrm{VASP}自带库函数的源代码(用于\textrm{vasp.X.lib})和库函数编译的\textrm{makefile}
	\item \textcolor{purple}{root/src/parser}:~存放\textrm{VASP}的\textrm{LOCPROJ}解析器的源代码和编译用的\textrm{makefile}(\textrm{VASP}版本\textrm{~>=5.4.4~}才会有该目录)
	\item \textcolor{purple}{root/src/CUDA}:~存放\textrm{GPU}版本\textrm{VASP}中由\textrm{GPU}执行的\textrm{cuda}源代码
	\item \textcolor{purple}{root/arch}:~存放针对各类不同编译环境的\textrm{makefile.include.arch}文件
	\item \textcolor{purple}{root/build}:~\textrm{VASP}编译过程中,不同类型(如\textrm{standard}/\textrm{gamma-only}/\textrm{non-collinear})的代码在编译过程中产生的中间文件和目标文件,会保存到该目录下的相应子目录中
	\item \textcolor{purple}{root/bin}:~用于存放编译生成的二进制文件的目录
\end{itemize}

\textrm{VASP}的安装就是编译并产生可执行文件的过程。
根据\ref{SubSec:Compilation}的介绍,编译软件最重要的因素是编译器、编译选项和库函数。具体到\textrm{VASP}编译安装,考虑到\textrm{VASP}的源代码主要由\textrm{Fortran}语言编写,另有少量\textrm{C}语言代码,因此当前系统环境下必须正确安装如下应用(注意相应的版本号):~
\begin{itemize}
	\item \textrm{Fortran}语言编译器和\textrm{C}语言编译器
	\item \textrm{MPI~(Message Passing Interface)}支持(用于编译并行版\textrm{VASP})
	\item 各类数值函数求解库,如\textrm{BLAS},~\textrm{LAPACK},~\textrm{ScaLAPACK},~\textrm{FFTW}等
\end{itemize}
根据确定的编译器和库函数,选择\textrm{root/arch}中最吻合的\textrm{makefile.include.arch}文件,复制到根目录下。比如系统有\textrm{intel}编译器(这也是最常见的情况):~\\
\textcolor{magenta}{\textrm{cp~arch/makefile.include.linux\_intel~~./makefile.include}}\\
一般地,\textrm{makefile.include}的内容,特别是编译选项的设置,需要根据具体的系统环境修改,留待后面作单独介绍。如果所有选项设置完毕,可执行命令:\\
\textcolor{magenta}{\textrm{make~all}}\\
依次生成\textrm{standard}/\textrm{gamma-only}/\textrm{non-collinear}三种类型的可执行文件,或分别执行命令:\\
\textcolor{magenta}{\textrm{make~std}}\\
\textcolor{magenta}{\textrm{make~gam}}\\
\textcolor{magenta}{\textrm{make~nlc}}\\
也将得到这三种类型可执行文件。

如果要编译支持\textrm{GPU}计算的\textrm{VASP},执行如下命令(仍以\textrm{intel}编译器为例):\\
\textcolor{magenta}{\textrm{cp~arch/makefile.include.linux\_intel\_cuda~~./makefile.include}}\\
修改编译参数后,分别执行命令:\\
\textcolor{magenta}{\textrm{make~gpu}}\\
\textcolor{magenta}{\textrm{make~gpu\_nlc}}\\
得打支持\textrm{GPU}计算的\textrm{VASP}。目前的\textrm{VASP}的\textrm{gamma}版不支持\textrm{GPU}计算。

\subsection{VASP的编译选项}
熟悉软件编译的用户一定知道,编译选项的选择非常依赖编译器和库函数的版本,因此准确地确定编译选项几乎是软件编译安装中最重要的环节,非常依赖用户的个人经验。这里就\textrm{VASP}编译中的编译选项作一般性介绍:~
\subsubsection{\rm{预编译}}
\textrm{makefile.include}中有关预编译的变量主要为\\
\textrm{CPP}:~指定使用的预编译器
\begin{itemize}
	\item 使用\textrm{intel}的\textrm{Fortran}预编译器:\\
		\textcolor{brown}{\textrm{CPP~=~fpp~-f\_com=no~-free~-w0~\$*\$(FUFFIX)~\$*\$(SUFFIX) \$(CPP\_OPTIONS)}}
	\item 使用\textrm{cpp}预编译器:\\
		\textcolor{brown}{\textrm{CPP~=~/usr/bin/cpp~-P~-C~-traditional~\$*\$(FUFFIX)~>~\$*\$(SUFFIX)~\$(CPP\_OPTIONS)}}
\end{itemize}
\textrm{CPP\_OPTIONS}:~针对不同类型和不同条件(如内存使用上限)的预处理控制参数\\
	\textcolor{brown}{\textrm{CPP\_OPTIONS~= ~[-Dflag1~[-Dflag2]~ $\cdots$~ ]}}
\subsubsection{\rm{编译选项}}
\textrm{Fortran}编译器编译时添加编译选项格式为:\\
\textcolor{brown}{\textrm{\$(FC)~~\$(FREE)~~\$(FFLAGS)~~\$(OFLAG)~~\$(INCS)}}\\
\begin{itemize}
	\item \textrm{Free}:~是指定\textrm{Fortran}编译器接受某些自由格式的代码,不受源代码行长度限制
		\begin{itemize}
			\item 采用\textrm{intel}编译器时:\\
				\textcolor{brown}{\textrm{FREE=-free~-names~lowercase}}
			\item 采用\textrm{gfortran}编译器时:\\
				\textcolor{brown}{\textrm{FREE=-ffree-form~-ffree-line-length-none}}
		\end{itemize}
	\item \textrm{FC}:~指定当前可选择的\textrm{Fortran}编译器(比如\textrm{gfortran/ifort/mpif90/mpiifort}等等)
	\item \textrm{FCL}:~编译器带链接库函数方式,比如采用\textrm{intel}的\textrm{MKL}库时:\\
		\textcolor{brown}{\textrm{FCL=\$(FC)}~-mkl}
	\item \textrm{OFLAG}:~编译的优化选项(默认为:~\textcolor{brown}{\textrm{OFLAG=-O2}})
	\item \textrm{FFLAGS}:~更多许可的编译选项。比如为方便程序查错(\textrm{debug}),可以添加选项:\\
		\textcolor{brown}{\textrm{FFLAG+=-g}}
	\item \textrm{OFLAG\_IN}:~绝大部分情况下设置:\\
		\textcolor{brown}{\textrm{OFLAG\_IN=\$(OFLAG)}}
	\item \textrm{DEBUG}:~编译时主程序\textrm{main.F}的优化选项,一般设置:\\
		\textcolor{brown}{\textrm{DEBUG=-O0}}
	\item \textrm{INCS}:~指定包含在其中的对象(\textrm{objects}):\\
		\textcolor{brown}{\textrm{INCS=-Idirectory-with-files-to-be-included}}
\end{itemize}
\subsubsection{\rm{链接库函数}}
链接库函数形式为:\\
\textcolor{brown}{\textrm{\$(FCL)~~-o~~vasp~~..objects..\$(LLIBS)~~\$(LINK)}}
\begin{itemize}
	\item \textrm{LLIBS}:~指定链接的库或对象,一般格式为:\\
		\textcolor{brown}{\textrm{LLIBS~=~[-Ldirectory~-llibrary]~[path/library.a]~[path/object.o]}}
\end{itemize}
一般用户需要指定多个数学库(如\textrm{BLAS/LAPACK/scaLAPACK}等),如最常用的\textrm{intel-MKL}库时,设置为:\\
\textcolor{brown}{\textrm{
	MKL\_PATH~~ = ~~\$(MKLROOT)/lib/intel64\\
BLACS~~ = ~~ -lmkl\_blacs\_openmpi\_lp64\\
SCALAPACK ~~ = ~~  \$(MKL\_PATH)/libmkl\_scalapack\_lp64.a ~\$(BLACS)\\
LLIBS ~~ = ~~ \$(SCALAPACK) ~\$(LAPACK) }}\\
如果使用其它的编译器和库函数,请参考\textrm{root/arch/}所提供的相应的\textrm{makefile.include.arch}中的编译选项调整和选择相应的设置。

$\ast$注意,如果选用\textrm{LAPACK~>=3.6},直接链接\textrm{LAPACK3.6}或更高版本时,编译时会出错,这是因为子程序\textrm{DGEGV}已经被\textrm{DGGEV}替代。解决的步骤为:~
\begin{enumerate}
	\item 在\textrm{makefie.include}预编译处添加:\\
		\textcolor{brown}{\textrm{CPP\_OPTIONS~+= ~-DLPACK36}}
	\item 在源代码\textrm{./src/symbol.inc}中添加:\\
		\textcolor{brown}{\textrm{
			\!~ routines ~replaced~ in~ LAPACK~ >=3.6\\
\#ifdef~ LAPACK36\\
\#define~ DGEGV~ DGGEV\\
\#endif}}
\end{enumerate}
这样就会在预编译时,将所有代码张调用的\textrm{DGEGV}替换为\textrm{DGGEV}。

\subsubsection{\rm{链接对象}}
编译\textrm{VASP}的标准配置的链接对象(子程序)列表见\textrm{root/src/.objects}文件中的\textrm{SOURCE}变量,此外需要的链接对象由\textrm{makefile.include}配置(比如\textrm{FFT~(Fast-Fourier-Transforms)}库生成的对象等):\\
\textcolor{brown}{\textrm{OBJECTS~= ~$\cdots$ objects. $\cdots$}}

\subsubsection{\rm{FFT}库}
\begin{itemize}
	\item \textrm{OBJECTS}:~指定\textrm{VASP}编译需要的\textrm{FFTW}的一些用于编译的对象或静态库,形式为:\\
\textcolor{brown}{\textrm{OBJECTS~= ~$\cdots$~fftw3d.o~~ fftmpiw.o~ $\cdots$}}
	\item \textrm{INCS}:~指定\textrm{FFTW}-库的对象,形式为\\
\textcolor{brown}{\textrm{INCS=-Idirectory-that-holds-fftw3f}}
\end{itemize}
具体应用中
\begin{itemize}
	\item 采用\textrm{MKL}封装的\textrm{FFTW}(预编译设置包含\textcolor{brown}{\textrm{CPP\_OPIONS~=~$\cdots$~-DMPI~$\cdots$}}):\\
		\textcolor{brown}{\textrm{
			OBJECTS~= ~fftmpiw.o~~ fftmpi\_map.o~~ fftw3d.o ~~fft3dlib.o $\setminus$ \\
         \$(MKLROOT)/interfaces/fftw3xf/libfftw3xf\_intel.a\\
INCS~=~ -I\$(MKLROOT)/include/fftw }}
	\item 使用\textrm{Juergen~Furtmuller}的\textrm{FFT}变换,设置为:\\
		\textcolor{brown}{\textrm{
			OBJECTS~= ~fftmpiw.o~~ fftmpi\_map.o~~ fftw3dfurth.o ~~fft3dlib.o \\
INCS~= }}
\end{itemize}
类似地,如果使用其它\textrm{FFTW}的库函数和链接,请参考\textrm{root/arch/}所提供的相应的\textrm{makefile.include.arch}中的编译选项调整和选择相应的设置。
\subsubsection{\rm{源代码部分编译规则}}
\begin{itemize}
	\item \textrm{FFT}对象有关规则:\\
		\textcolor{brown}{\textrm{
			OBJECTS\_O1 ~+= ~fft3dfurth.o ~fftw3d.o ~fftmpi.o ~fftmpiw.o\\
OBJECTS\_O2 ~+= ~fft3dlib.o }}
	\item 源代码部分的特定编译规则:~在\textrm{root/src/makefile}文件中有\textrm{VASP}源代码的一些特定编译规则,其形式为:\\
		\textcolor{brown}{\textrm{\$(FC) ~\$(FREE) ~\$(FFLAGS\_x) ~\$(OFLAG\_x) ~\$(INCS\_x)}}\\
		这里\textrm{x}代表$1/2/3/\mathrm{IN}$
		\begin{itemize}
			\item \textrm{FFLGS\_x}:~默认值:~\textcolor{brown}{\textrm{FFLAGS\_x~=~\$(FFLAGS)}}, $\mathrm{x}=1, 2, 3, \mathrm{IN}$
			\item \textrm{OFLGS\_x}:~默认值:~\textcolor{brown}{\textrm{OFLAGS\_x~=~-Ox}}, $\mathrm{x}=1, 2, 3, ~\mathrm{OFLG\_IN}$~=~-O2
			\item \textrm{INCS\_x}:~默认值:~\textcolor{brown}{\textrm{INCS\_x~=~\$(INCS)}}, $\mathrm{x}=1, 2, 3, \mathrm{IN}$
			\item \textrm{OFLGS\_x}:~默认值:~\textcolor{brown}{\textrm{OFLAGS\_x~=~-Ox}}, $\mathrm{x}=1, 2, 3, ~\mathrm{OFLG\_IN}$~=~-O2
		\end{itemize}
		采用这些特定规则编译生成的对象的指定方式是:~
		\begin{itemize}
			\item \textrm{OBJECTS\_O1, OBJECTS\_O2, OBJECTS\_O3, OBJECTS\_IN}:~一般针对默认的编译优化选项为\textrm{-O1}和\textrm{-O2}
			\item \textrm{SOURCE\_O1, SOURCE\_O2, OBJECTS\_IN}:~在\textrm{.object}文件中用来指定需要特别编译的子程序
		\end{itemize}
		比如要改掉默认编译选项\textrm{-O1},可以按照如下方式指定编译规则:~\\
		\textcolor{brown}{\textrm{
			SOURCE\_O1=\\
OBJECTS\_O1~= ~$\cdots$ objects-list $\cdots$ }}
\end{itemize}
\subsubsection{\rm{VASP}自带的函数库}
针对\textrm{VASP}自带函数库的编译规则
\begin{itemize}
	\item \textrm{CPP\_LIB}:~指定预编译器\\
		\textcolor{brown}{\textrm{CPP\_LIB=\$(CPP)}}
	\item \textrm{FP\_LIB}:~指定\textrm{Fortran}编译器\\
		\textcolor{brown}{\textrm{FC\_LIB=\$(FC)}}\\
		注意\textrm{VASP}的库函数可以不用\textrm{MPI}支持,因此,如果\textrm{FC=mpif90},\textrm{FC\_LIB}可以指定另外的\textrm{MPI}编译器,比如\textrm{FC\_LIB=ifort}
	\item \textrm{FFLAGS\_LIB}:~指定\textrm{Fortran}编译选项,包括优化等级,最常用的设置是:\\
		\textcolor{brown}{\textrm{FFLAGS\_LIB=-O1}}
	\item \textrm{FREE\_LIB}:~指定\textrm{Fortran}编译的源代码中的指令不受行长度限制,最常用的设置是:\\
		\textcolor{brown}{\textrm{FREE\_LIB=\$(FREE)}}
	\item \textrm{CC\_LIB}:~指定\textrm{C}编译器,可以不用\textrm{MPI}支持,一般编译器为\textrm{gcc,icc}等
	\item \textrm{CFLAGS\_LIB}:~指定\textrm{C}编译选项,包括优化等级,最常用的设置是:\\
		\textcolor{brown}{\textrm{CFLAGS\_LIB=-O}}
	\item \textrm{OBJECTS\_LIB}:~指定\textrm{VASP}自带库的非标准编译对象:\\
		\textcolor{brown}{\textrm{OBJECTS\_LIB=linpack\_double.o}}\\
		注意:~如果编译\textrm{VASP}时,预编译选项有\textrm{-Duse\_shmem},则编译选项必须加上\textrm{getshmem.o},即:\\
		\textcolor{brown}{\textrm{OBJECTS\_LIB=$\cdots$getshmem.o $\cdots$}}
\end{itemize}
\subsubsection{\rm{LOCPROJ}的解析代码编译选项}
\begin{itemize}
	\item \textrm{CXX\_PARS}:~指定\textrm{C++}编译器(如:~\textrm{g++,icpc})
	\item 对编译选项增加内容:~\textcolor{brown}{\textrm{LIBS~+=~ parser}}
	\item 对链接对象增加:~\textcolor{brown}{\textrm{LLIBS~+=~ -Lparser ~-lparser ~-lstdc++}}
\end{itemize}
\subsubsection{\rm{Wannier90}接口编译选项}
\begin{itemize}
	\item 用户下载\textrm{Wannier90}后,编译出库函数\textrm{libwannier.a}
	\item 编译\textrm{Wannier90}接口(预编译选项\textrm{-DVASP2WANNIER90}或\textrm{-DVASP2WANNIER90v2}),编译选项增加:\\
		\textcolor{brown}{\textrm{LLIBS~+=~ /your-wannier90-directory/libwannier.a}}
\end{itemize}
\subsubsection{\rm{libbeef}(备选)}
\begin{itemize}
	\item 用户下载\textrm{libbeef}并编译出动态库
	\item 编译\textrm{VASP}用的\textrm{BEEF~van-der-Waals}泛函(预编译选项\textrm{-Dlibbeef}),编译选项增加:\\
		\textcolor{brown}{\textrm{LLIBS~+=~ -Lyour-libbeef-directory~ -lbeef }}
\end{itemize}
\subsubsection{\rm{VASP的GPU计算部分的编译选项}}
\begin{itemize}
	\item \textrm{CUDA\_ROOT}:~指定\textrm{CUDA}的安装位置,例如:\\
\textcolor{brown}{\textrm{CUDA\_ROOT~ :=~ /opt/cuda }}
	\item \textrm{CUDA\_LIB}:~指定\textrm{CUDA}的有关的库函数,例如:\\
\textcolor{brown}{\textrm{CUDA\_LIB~ :=~ -L\$(CUDA\_ROOT)/lib64 ~-lnvToolsExt ~-lcuda ~-lcufft ~-lcublas}}
	\item \textrm{NVCC}:~指定\textrm{CUDA}的编译器和选项,例如:\\
		\textcolor{brown}{\textrm{NVCC~ :=~ \$(CUDA\_ROOT)/bin/nvcc ~-g }}
	\item \textrm{OBJECTS\_GPU}:~指定\textrm{FFT}编译的对象和链接(含静态库),例如,对\textrm{FFTW}:\\
\textcolor{brown}{\textrm{OBJECTS\_GPU~ =~ fftmpiw.o ~fftmpi\_map.o ~fft3lib.o ~fftw3d\_gpu.o ~fftmpiw\_gpu.o}}
%	\item \textrm{GENCODE\_ARCH}:~针对当前\textrm{GPU}架构的编译器选项:~
%		\begin{itemize}
%			\item 对\textrm{Kepler}架构:\\
%\textcolor{brown}{\textrm{GENCODE\_ARCH~ :=~ gencode=arch=compute\_35, ~code= $\setminus$\`\` ~sm\_35, ~compute\_35 $\setminus$~\'\'}}
%			\item 对\textrm{Maxwell}架构:\\
%\textcolor{brown}{\textrm{GENCODE\_ARCH~ :=~ gencode=arch=compute\_53, ~code= $\setminus$\`\` ~sm\_53,~compute\_53$\setminus$~\'\'}}
%\end{itemize}
可以通过多个\textrm{\`~gencode~\'}声明来编译跨平台的可执行文件。
	\item \textrm{MPI\_INC}:~指定\textrm{MPI}的\textrm{include}目录,方便\textrm{CUDA}找到该目录,例如:\\
\textcolor{brown}{\textrm{MPI\_INC~ :=~ /opt/openmpi/include}}\\
该选项一般可以通过\textrm{mpicc ~~\-\-show}查看
	\item \textrm{CPP\_GPU}:~\textrm{GPU}编译的预编译选项:
		\begin{itemize}
			\item 标准选项:\\
				\textcolor{brown}{\textrm{-DCUDA\_GPU to build cross-platform sources for GPU,~-DUSE\_PINNED\_MEMORY to use pinned memory for transfer buffers, and~-DRPROMU\_CPROJ\_OVERLAP to overlap communication and computation in RPROJ\_MU.}}
			\item 可选选项:\\
				\textcolor{brown}{\textrm{set -DCUFFT\_MIN=N to intercept any FFT calls of size greater than $\mathrm{N}^3$ and evaluate on GPU}}
			\item 实验选项:\\
				\textcolor{brown}{\textrm{set -DUSE\_MAGMA to use MAGMA for LAPACK-like calls on the GPU}}
		\end{itemize}
		典型的\textrm{GPU}预编译选项:\\
		\textcolor{brown}{\textrm{CPP\_GPU~=~-DCUDA\_GPU~-DRPROMU\_CPROJ\_OVERLAP~-DUSE\_PINNED\_MEMORY~-DCUFFT\_MIN=28}}
\end{itemize}
\subsubsection{\rm{MAGMAN\_ROOT}的编译选项}
如果有\textrm{MAGMA}实验的支持,指定\textrm{MAGMA-1.6}函数库的路径:\\
\textcolor{brown}{\textrm{MAGMA\_ROOT~ :=~ /opt/magma/lib }}


\newpage
\textbf{本讲义以下算例主要来自文献\inlinecite{J.-G._Lee}。由于\textrm{VASP}版本的不同,以及用户用于计算的\textrm{POTCAR}数据差异,所以重复讲义中的算例时,计算过程和结果的数据,绝对数值可能会有一些出入,但应该不会差别太大。特此说明。}

\section{原子/分子计算}\label{Sec:atom-Pt}
真空中的孤立原子的基态能量是\textrm{VASP}中最简单的算例,通过学习金属\textrm{Pt}原子基态能量的计算,可以掌握典型的\textrm{VASP}的主体流程\footnote{在所有计算之前,请确认\textrm{VASP}软件已经正确安装。},了解体系基态能量最小化的基本算法,并熟悉基本的输入/输出文件的内容。此外,还可以了解如何在已完成计算的基础上,进行计算精度提升或完成后续计算等一系列处理方式。
\subsection{输入文件}
\textrm{VASP}要求所有的计算都在指定的计算目录下进行(比如针对当前的算例,可以建目录\textrm{Pt\_atom})。首先准备计算所必需的四个输入文件,文件名必须是\textrm{INCAR}、\textrm{KPOINTS}、\textrm{POSCAR}和\textrm{POTCAR}。前三个文件,既可以手动输入也可以通过修改其余计算中的对应文件得到;~\textrm{POTCAR}文件存储的是计算对象构成元素的原子数据,包括具体的原子分波、赝原子分波以及赝势和构造赝势使用的交换-相关泛函等信息。\textrm{VASP}开发组在发布软件时会一并提供全部元素的\textrm{POTCAR}。\footnote{由于历史的原因,习惯上将\textrm{POTCAR}称为赝势文件,但实际上\textrm{POTCAR}文件中包含的内容不仅是原子赝势,还包括原子价电子分波函数和赝分波函数,以及补偿电荷等和赝势、\textrm{PAW}势相关的诸多信息,因此称“数据集(\textrm{Dataset})”更准确.}
\subsubsection{\rm{INCAR}}
\textrm{INCAR}是\textrm{VASP}计算的核心控制文件。对于\textrm{Pt}的计算,\textrm{INCAR}的设置如图\ref{Pt_atom:INCAR}所示。
\begin{figure}[h!]
\centering
\includegraphics[height=1.00in,width=3.6in,viewport=0 0 480 135,clip]{Pt_atom-INCAR.png}
\caption{\small \textrm{VASP}计算的主要输入控制文件\textrm{INCAR}.}%(与文献\cite{EPJB33-47_2003}图1对比)
\label{Pt_atom:INCAR}
\end{figure}
对于原子/分子计算,只需要确定少量参数,其余的参数都采用软件推荐的默认值。针对本次计算设置的计算参数,说明如下:~
\begin{itemize}
	\item \textit{ISMEAR}:~该参数设定的是\textrm{Fermi~}能级的展宽方法,引入展宽为的是加速收敛。本例中参数确定的展宽方法是\textrm{Gaussian}展宽,该方法适用于单原子和局域体系,避免出现\textrm{Fermi}面附近占据数为负值。展宽值由参数\textit{SIGMA}确定。
	\item \textit{ISPIN}:~因为\textrm{Pt}原子的价电子态是$5\mathit{d}^96\mathit{s}^1$,含有未成对电子,因此计算需要考虑自旋极化,相比于非自旋极化的计算量加倍,即两种自旋(\textrm{spin-up/spin-down})下的电荷密度都要计算。
	\item \textit{ISYM}:~这是控制对称性的计算参数。本次计算中不考虑体系的点群对称性。
\end{itemize}
\subsubsection{\rm{KPOINTS}}
\textrm{KPOINTS}是描述不可约\textrm{Brillouin}区(\textrm{Irreducible Brillouin Zone,~IBZ})的$\vec k$-点分布的文件,见图\ref{Pt_atom:KPOINTS}。因为当前计算的是单原子体系,无需考虑\textrm{Bloch}定理的影响,因此只需要取一个$\vec k$点($\Gamma$点)。
\begin{figure}[h!]
\centering
\vskip -5pt
\includegraphics[height=0.75in, viewport=0 20 750 118,clip]{Pt_atom-KPOINTS.png}
\caption{\small \textrm{VASP}计算的\textrm{IBZ}中的$\vec k$点分布文件\textrm{KPOINTS}.}%(与文献\cite{EPJB33-47_2003}图1对比)
\label{Pt_atom:KPOINTS}
\end{figure}
\subsubsection{\rm{POSCAR}}
\textrm{POSCAR}是描述计算对象结构的文件,见图\ref{Pt_atom:POSCAR}。\textrm{VASP}中所有的计算对象必须是周期体系,因此对于单原子体系,可以将原子置于一个大的超晶胞($10\times10\times10$)中心,以此确保原子间相互作用足够小。
\begin{figure}[h!]
\centering
\includegraphics[height=1.05in,viewport=0 0 850 155,clip]{Pt_atom-POSCAR.png}
\caption{\small \textrm{VASP}计算对象的结构文件\textrm{POSCAR}.}%(与文献\cite{EPJB33-47_2003}图1对比)
\label{Pt_atom:POSCAR}
\end{figure}
\subsubsection{\rm{POTCAR}}
\textrm{POTCAR}是计算对象构成元素的数据集文件,包含了\textrm{PAW}计算所需的原子数据,见图\ref{Pt_atom:POTCAR}。本次计算使用\textrm{PAW\_PBE}数据集,是兼顾精度和效率的考虑。\textrm{POTCAR}文件可通过复制\textrm{VASP}提供的原子数据集库中的对应元素数据得到:\\
\textcolor{magenta}{\textrm{cat~~potcar.PBE.paw/Pt/POTCAR~>~POTCAR}}\\
\begin{figure}[h!]
\centering
\includegraphics[height=3.5in,viewport=0 15 780 528,clip]{Pt_atom-POTCAR.png}
\caption{\small \textrm{VASP}计算的原子数据集文件\textrm{POTCAR}.}%(与文献\cite{EPJB33-47_2003}图1对比)
\label{Pt_atom:POTCAR}
\end{figure}
注意,这里参数\textit{ENMAX}的值230.283\textrm{eV}将作为计算中能量截断参数\textit{ENCUT}的默认值。

\subsection{VASP的运行}
\textrm{VASP}软件安装在\textrm{Linux}系统环境下,因此在运行\textrm{VASP}的时候需要掌握\textrm{Linux}的一些基本命令。当上节中指定的四个输入文件准备完毕后,只要输入命令:\\
\textcolor{magenta}{\textrm{VASP\_RUN\_PATH}}\footnote{假设\textrm{VASP}软件正确安装,\textcolor{magenta}{\textrm{VASP\_RUN\_PATH}}表示其可执行文件.}\\
即可执行,随后屏幕上会有输出,直到计算结束。见图\ref{Pt_atom:runout}。
\begin{figure}[h!]
\centering
%\includegraphics[height=4.5in,width=4.2in,viewport=0 0 780 530,clip]{Pt_atom-runout.png}
\includegraphics[width=5.9in,viewport=880 0 1870 1465,clip]{Pt_atom-runout.png}
\caption{\small \textrm{VASP}运行时的屏幕输出.}%(与文献\cite{EPJB33-47_2003}图1对比)
\label{Pt_atom:runout}
\end{figure}
从输出内容中可以看到警告提示\textrm{wrap around},只要是\textrm{VASP}软件认为用于\textrm{FFT}计算的网格不够密,就会出现该提示。该提示提示用户,由于\textrm{FFT}网格不够密集,会导致电荷密度在\textrm{FFT}处理时存在混叠误差\textrm{(aliasing error)}。如果想避免出现该提示,可以搜索输出文件\textrm{OUTCAR}中的参数\textit{NGX/NGY/NGZ}避免混叠误差的推荐值,并在\textrm{INCAR}文件中设定该参数。不过通常情况下此处警告提示可以忽略,因为混叠误差对结果的影响微乎其微。当看到最后一行输出\textrm{writing wavefunctions},就表明\textrm{VASP}正常结束。
\newpage
\subsection{\rm{VASP}的结果}
正常结束的\textrm{VASP}计算,会产生13个输出文件,见图\ref{Pt_atom:lsout}。
\begin{figure}[h!]
\centering
\vskip -5pt
\includegraphics[height=0.25in,viewport=0 2 750 38,clip]{Pt_atom-lsout.png}
\caption{\small \textrm{VASP}运行结束后的文件.}%(与文献\cite{EPJB33-47_2003}图1对比)
\label{Pt_atom:lsout}
\end{figure}

以下先介绍\textrm{OSZICAR}和\textrm{OUTCAR}这两个文件。
\subsubsection{\rm{OSZICAR}}
\textrm{OSZICAR}主要存储的是\textrm{VASP}迭代循环过程中的能量变化情况,部分内容如图\ref{Pt_atom:OSZICAR}所示。文件最后一行给出的$\mathrm{E}_0$的值($-0.522\mathrm{eV}$)就是当前参数设置下,单个\textrm{Pt}原子在真空中的基态能量。
\begin{figure}[h!]
\centering
\includegraphics[height=1.8in,viewport=0 0 880 290,clip]{Pt_atom-OSZICAR.png}
\caption{\small \textrm{VASP}的输出文件\textrm{OSZICAR}(部分).}%(与文献\cite{EPJB33-47_2003}图1对比)
\label{Pt_atom:OSZICAR}
\end{figure}

\textrm{OSZICAR}的第一列表示计算中采用\textrm{Davidson}迭代收敛方法,第二列为迭代次数,本算例达到默认的收敛条件($\delta\mathrm{E}<1.0\times10^{-4}~\mathrm{eV}$)共迭代了42次。需要指出的是,这里$\mathrm{E}_0$($-.522\mathrm{eV}$)不是原子的绝对能量,而是\textrm{Pt}原子相对于\textrm{PAW}势的能量。\textrm{Pt}的原子构型为([\ch{Xe}]$4\mathit{f}^{14}5\mathit{d}^96\mathit{s}^1$),考虑自旋极化可以算出\textrm{Pt}原子磁矩为$\mathrm{mag}=1.9986\mu_{\mathrm{B}}$(\textrm{Bohr magneton}),表明磁矩来自原子中两个未成对电子:
\begin{displaymath}
	\mathrm{mag}=\mu_{\mathrm{B}}[\rho_{\uparrow}-\rho_{\downarrow}]
\end{displaymath}
\subsubsection{\rm{OUTCAR}}
\textrm{OUTCAR}保存的是\textrm{VASP}运行中最详尽的过程记录。文件不仅包括输入文件的既有信息,还有计算体系的对称性分析,$\vec k$空间布点和具体位置,平面波基信息和最近邻原子的距离等基本信息;~此外记录了每一步离子弛豫和电子弛豫的计算信息。所以\textbf{OUTCAR文件是\textrm{VASP}运行过程中最重要的输出文件},\textrm{OUTCAR}的文件结构为:~
\begin{itemize}
	\item \textrm{VASP}版本和基本计算环境和计算资源信息
	\item 读入\textrm{INCAR}、\textrm{POTCAR}、\textrm{POSCAR}
	\item 最近邻原子和对称性分析信息
	\item 关于计算过程的详尽的控制参数信息(含默认值)
	\item 晶格正空间和$\vec k$-空间信息与原子坐标
	\item 平面波基信息(截断能和平面波数目)
	\item 非局域赝势信息
	\item 每个电子步计算的信息
	\item 每个电子步迭代的时间和能量信息
\begin{figure}[h!]
\centering
\includegraphics[height=2.5in,viewport=0 40 770 555,clip]{VASP_train-OUTCAR-iteration.png}
\caption{\small \textrm{VASP}的输出文件\textrm{OUTCAR}的电子步迭代信息示例.}%(与文献\cite{EPJB33-47_2003}图1对比)
\label{VASP_train-OUTCAR-iteration}
\end{figure}
	\item 能量本征值信息
\begin{figure}[h!]
\centering
\includegraphics[height=1.5in,viewport=0 0 580 235,clip]{VASP_train-OUTCAR-eigenvalue.png}
\caption{\small \textrm{VASP}的输出文件\textrm{OUTCAR}电子步迭代完成能量本征值信息示例.}%(与文献\cite{EPJB33-47_2003}图1对比)
\label{VASP_train-OUTCAR-eigenvalue}
\end{figure}
	\item 原子的受力与应力张量和晶胞信息
\begin{figure}[h!]
\centering
\includegraphics[height=2.5in,viewport=0 0 780 515,clip]{VASP_train-OUTCAR-stress_and_volumn.png}
\caption{\small \textrm{VASP}的输出文件\textrm{OUTCAR}原子受力与应力张量和晶胞信息示例.}%(与文献\cite{EPJB33-47_2003}图1对比)
\label{VASP_train-OUTCAR-stress_and_volumn}
\end{figure}
	\item 体系基态总能和自由能信息
\begin{figure}[h!]
\centering
\includegraphics[height=0.5in,viewport=0 0 720 95,clip]{VASP_train-OUTCAR-free_energy.png}
\caption{\small \textrm{VASP}的输出文件\textrm{OUTCAR}基态总能和自由能信息.}%(与文献\cite{EPJB33-47_2003}图1对比)
\label{VASP_train-OUTCAR-free_energy}
\end{figure}
	\item 完整计算过程的资源和计算时间统计信息
\begin{figure}[h!]
\centering
\includegraphics[height=4.7in,viewport=5 0 680 835,clip]{VASP_train-OUTCAR-time_and_memory.png}
\caption{\small \textrm{VASP}的输出文件\textrm{OUTCAR}的计算资源和时间统计信息.}%(与文献\cite{EPJB33-47_2003}图1对比)
\label{VASP_train-OUTCAR-time_and_memory}
\end{figure}
\end{itemize}

在本算例中,文件末尾记录的全部计算过程运行时间和计算资源使用情况,如图\ref{Pt_atom:runtime}所示:
\begin{figure}[h!]
\centering
\includegraphics[height=1.5in,viewport=0 0 680 215,clip]{Pt_atom-runtime.png}
\caption{\small \textrm{VASP}的输出文件\textrm{OUTCAR}最后部分.}%(与文献\cite{EPJB33-47_2003}图1对比)
\label{Pt_atom:runtime}
\end{figure}

计算结果表明,\textrm{VASP}计算简单原子,只需要几十秒即可完成。
\subsubsection{断点续算}
输出文件中,除了\textrm{OSZICAR}和\textrm{OUTCAR}外,还有三个重要的输出文件:~\textrm{CHGCAR}、\textrm{CONTCAR}和\textrm{WAVECAR},分别存储的是电荷密度、结构弛豫后的原子位置和最终的波函数文件(非格式输出的二进制文件)。\textrm{VASP}计算因故结束后,可以利用这些输出文件重启计算,从中断计算位置继续完成计算,从而计算时间。执行断点续算的时候,除了需要将\textrm{CONTCAR}的内容复制到\textrm{POSCAR}中,还要对\textrm{INCAR}文件作必要的修改并增添如下内容:
\begin{figure}[h!]
\centering
\vskip -5pt
\includegraphics[height=1.2in,viewport=0 0 700 157,clip]{Pt_atom-continurun.png}
\caption{\small \textrm{VASP}断点续算的输入控制文件\textit{INCAR}.}%(与文献\cite{EPJB33-47_2003}图1对比)
\label{Pt_atom:continuruntime}
\end{figure}

再次运行\textrm{VASP},程序除了读取原来四个输入文件的内容,还将读取\textrm{CHGCAR}和\textrm{WAVECAR}的内容,经过约近二十次迭代后,计算结果将进一步优化$\mathrm{E}_0=-.555\mathrm{~eV/atom}$,磁矩为$\mathrm{mag}=2.0000\mu_{\mathrm{B}}$,这次得到的原子基态能量将被用于后续体相\textrm{Pt}的结合能\textrm{(Cohesive energy)}计算。

\section{面心立方Pt晶胞计算}\label{Sec:FCC-Pt}
这里计算的是面心立方\textrm{(Face-centered Cubic, FCC)}结构的\textrm{Pt}晶体,晶胞参数为$a_0=3.975~\mathrm{\AA}$。已知每个面心立方中含有4个\textrm{Pt}原子,结构弛豫(包括离子和电子能量的最小化)时,要求保持面心立方对称性。
\subsection{输入文件}
计算\textrm{FCC-Pt},除了\textrm{POTCAR}与\textrm{Pt}原子相同,其余输入原子依次讨论如下:~
\subsubsection{\rm{INCAR}}
对含有偶数个相同原子的晶格,自旋电子密度趋于相同,计算中不再考虑自旋极化,输入文件\textrm{INCAR}如图\ref{Pt_FCC:INCAR}所示:
\begin{figure}[h!]
\centering
\includegraphics[height=2.2in,viewport=0 0 720 292,clip]{Pt_FCC-INCAR.png}
\caption{\small \textrm{FCC-Pt}的输入控制文件\textrm{INCAR}.}%(与文献\cite{EPJB33-47_2003}图1对比)
\label{Pt_FCC:INCAR}
\end{figure}
\begin{itemize}
	\item \textit{ENCUT}:~该参数在需要手工指定平面波\textrm{(Plane Wave, PW)}基组截断能时使用,一旦该参数被指定,由参数\textit{PREC}确定的截断能不再有效。这里\textit{ENCUT}~=~400\textrm{~eV},远高于\textrm{POTCAR}中推荐的能量截断值(\textit{ENMAX}~=~230.283\textrm{~eV}),这样做主要是考虑到后续模拟\textrm{Pt}表面吸附\textrm{O}时,\textit{ENMAX}为400\textrm{~eV},计算组分体系时,平面波截断能应选为不低于最高的\textit{ENMAX}之值。
	\item \textit{PREC}:~一旦人工指定\textit{ENCUT},参数\textit{PREC}确定的是波函数和电荷密度的\textrm{FFT}变换的网格精度,因此\textit{PREC}也是影响计算精度的参数。绝大多数情况下,参数值\textrm{``normal''}的精度足以满足要求;~如果有更高的精度要求,可以设为\textit{PREC}=\textrm{Accurate}。
	\item \textit{EDIFF}:~该参数确定的是电子步迭代收敛的截断值,\textrm{VASP}计算中,要求最近邻两个电子步的总能(自由能)和\textrm{Kohn-Sham}方程能量本征值都满足收敛标准,程序才会停止运行。因此对一般计算来说,收敛标准的默认值$1.0\times10^{-4}$,已是足够的高精度。
	\item \textit{ALGO}:~该参数确定的是离子步嵌套的电子步矩阵迭代对角化算法,参数值\textrm{``fast''}是最常用的,要求除了在第一个离子步弛豫时,最初五个电子步迭代采用稳定的\textrm{Davidson}算法,后续电子步迭代采用\textrm{RMM-DIIS}算法;~其余离子步弛豫,则只有第一个电子步弛豫采用\textrm{Davidson}算法,其余电子步都采用\textrm{RMM-DIIS}算法。
	\item \textit{EDIFFG}:~该参数确定的是离子步弛豫的收敛截断值。当截断值为正,则要求最近邻两个离子步的总能(自由能)满足收敛标准;~当截断值为负,则要求计算对象中的每个原子上的受力都小于该值的绝对值(因此是更严格的收敛标准),默认值为$0.02\mathrm{eV/\AA}$。在没有达到收敛标准之前,程序将不断移动原子的位置;~在达到收敛标准之后,将不再改变体系中原子的位移。
\end{itemize}
对于一个新的\textrm{VASP}计算任务,初始波函数(也称为初猜波函数)可以通过原子赝势构造,这是符合物理直觉的。从不同截断能或不同结构的计算态出发,无疑将会节省计算时间。其余参数留待后续计算中说明。当需要对比两个体系的总能时,要确保两个体系的精度参数要求相同,这些参数包括:~\textit{ENCUT}、\textit{PREC}、\textit{EDIFF}、\textit{EDIFFG}和\textit{ISMEAR}。
\subsubsection{\rm{KPOINTS}}
$\vec k$空间布点数为$9\times9\times9$,布点方案由\textrm{Monkhorst-Pack}方法生成,这是对于金属和导体非常适用的布点方法。如图\ref{Pt_FCC:KPOINTS}所示。
\begin{figure}[h!]
\centering
\includegraphics[height=0.8in,viewport=0 0 330 118,clip]{Pt_FCC-KPOINTS.png}
\caption{\small \textrm{VASP}计算\textrm{FCC-Pt}的输入文件\textrm{KPOINTS}.}%(与文献\cite{EPJB33-47_2003}图1对比)
\label{Pt_FCC:KPOINTS}
\end{figure}
\subsubsection{\rm{POSCAR}}
面心立方\textrm{Pt}的晶胞中含有4个原子,在\textrm{POSCAR}文件中,四个原子分别置于原点和立方体的三个面心位置。如图\ref{Pt_FCC:structure}所示。
\begin{figure}[h!]
\centering
\includegraphics[height=1.8in,viewport=350 210 840 640,clip]{VASP_train-FCC.png}
\caption{\small \textrm{FCC-Pt}的空间结构.}%(与文献\cite{EPJB33-47_2003}图1对比)
\label{Pt_FCC:structure}
\end{figure}

本算例中,晶胞中第一个\textrm{Pt}原子位置固定,其余三个原子允许在各个维度方向自由移动。注意到第二个\textrm{Pt}原子在$x$方向上偏移面心立方位置0.01(见图\ref{Pt_FCC:POSCAR}),可以预见,经过结构弛豫后,晶胞将回到正常的面心立方位置。
\begin{figure}[h!]
\centering
\includegraphics[height=1.8in,viewport=0 30 740 270,clip]{Pt_FCC-POSCAR.png}
\caption{\small 用于\textrm{VASP}计算的\textrm{FCC-Pt}的结构文件\textrm{POSCAR}.}%(与文献\cite{EPJB33-47_2003}图1对比)
\label{Pt_FCC:POSCAR}
\end{figure}

\subsection{用shell脚本运行VASP计算}
\subsubsection{脚本运行}
这里学习用\textrm{shell}脚本来提交\textrm{VASP}运行任务,后面的练习也将用脚本提交任务。提交\textrm{VASP}计算的\textrm{shell}脚本(文件名为\textcolor{green}{\textrm{run.vasp}})的内容如图\ref{Pt_FCC:run}所示:
\begin{figure}[h!]
\centering
\vskip -12pt
\includegraphics[height=0.3in,viewport=0 0 380 45,clip]{Pt_FCC-run.png}
\caption{\small 提交\textrm{VASP}计算任务\textrm{FCC-Pt}的\textrm{shell}脚本内容.}%(与文献\cite{EPJB33-47_2003}图1对比)
\label{Pt_FCC:run}
\end{figure}
	\begin{itemize}
		\item 命令形式\textcolor{magenta}{~\textrm{nohup}~[执行命令]~\&~}表示将执行命令提交为后台进程,这样即使当前用户在计算执行过程中执行别的计算命令甚至退出,也不会影响计算任务正常的执行。这里\textrm{``nohup''}是\textrm{``no hang-up''}的缩写。
		\item 命令中\textcolor{magenta}{\textrm{mpirun~-np~8~}}表示此次\textrm{VASP}计算用8个\textrm{CPU}并行完成。
		\item \textcolor{magenta}{\$\textrm{VASP\_RUN\_PATH~}}表示\textrm{VASP}的执行文件的路径。
	\end{itemize}
将该执行脚本和\textrm{VASP}计算的四个输入文件放在同一个目录下,加可执行权限:~\\
\textcolor{magenta}{chmod~+x~run.vasp}\\% \# change the file mode to execution mode\\
然后就可以运行如下命令执行\textrm{VASP}计算:\\
	\textcolor{magenta}{./run.vasp }\\
		计算过程中,通过\textrm{Hellman-Feynman}定理计算作用在每个离子步位置上的原子受力,离子步迭代过程直到全部原子的受力都满足收敛标准\textit{EDIFFG}($0.02\mathrm{eV/\AA}$)。在计算过程中,可以通过以下命令监控原子受力情况的变化:\\
		\textcolor{magenta}{grep~\`~drift~\'~ OUTCAR}
\subsubsection{\rm{nohup.out}}
用\textcolor{magenta}{$\mathrm{nohup}~[\cdots]~\&$}提交的任务,计算过程会保存在输出文件\textrm{nohup.out}中,如图\ref{Pt_FCC:nohupout}所示。
%\newpage
\begin{figure}[h!]
\centering
\includegraphics[height=5.5in,viewport=0 20 620 730,clip]{Pt_FCC-nohupout.png}
\caption{\small \textrm{VASP}计算\textrm{FCC-Pt}的输出文件\textrm{nohup.out}.}%(与文献\cite{EPJB33-47_2003}图1对比)
\label{Pt_FCC:nohupout}
\end{figure}

因为参数\textit{ALGO}=\textrm{fast},因此第一次离子步弛豫中,最初5步电子步迭代的矩阵对角化是\textrm{Davidson}方法,然后是\textrm{RMM-DIIS}方法;~电子步迭代结束后,根据原子受力,将原子移动到新的位置(完成离子弛豫),在新的原子构型下开始电子步迭代,循环往复,直到满足收敛标准(\textit{EDIFF}和\textit{EDIFFG})。最后出现的\textrm{``writing wave functions''}表明运行结束。波函数将写入\textrm{WAVECAR}文件中(参照\textrm{INCAR}中的输出设置)。
\subsection{计算结果}
\subsubsection{\rm{CONTCAR}与结构弛豫}
\textrm{CONTCAR}文件记录的是结构弛豫完成时晶体中原子的位置,本算例\textrm{FCC-Pt}弛豫后的\textrm{CONTCAR}内容如图\ref{Pt_FCC:CONTCAR}所示。
\begin{figure}[h!]
\centering
\includegraphics[height=1.8in,viewport=0 0 580 230,clip]{Pt_FCC-CONTCAR.png}
\caption{\small \textrm{VASP}计算中记录\textrm{FCC-Pt}弛豫后结构的文件\textrm{CONTCAR}.}%(与文献\cite{EPJB33-47_2003}图1对比)
\label{Pt_FCC:CONTCAR}
\end{figure}

显然,结构弛豫完成后,第二个原子由起始位置\textrm{(0.510000,~0.500000,~0.00000)}弛豫到\textrm{FCC}结构要求的原子位置(存在一定的数值误差)。
\subsubsection{\rm{OUTCAR}}
为监控离子步弛豫过程中体系总能量收敛的情况,可以使用以下命令检索输出文件\textrm{OUTCAR}:\\
\textcolor{magenta}{\textrm{grep~~\'~energy~without~entropy~\'~~OUTCAR}}\\
结果如图\ref{Pt_FCC:OUTCAR_totene}所示。
\begin{figure}[h!]
\centering
\includegraphics[height=1.2in,viewport=0 0 680 130,clip]{Pt_FCC-OUTCAR_totene.png}
\caption{\small \textrm{VASP}计算\textrm{FCC-Pt}的弛豫过程的基态总能收敛情况.}%(与文献\cite{EPJB33-47_2003}图1对比)
\label{Pt_FCC:OUTCAR_totene}
\end{figure}

不难看出,离子步迭代收敛时的基态能量$\mathrm{E}_0$的值为~--24.387\textrm{~eV},换言之,当面心立方的晶胞参数为$3.975\mathrm{\AA}$时,体系中\textrm{Pt}原子的基态能量为$-24.387/4=-6.097\mathrm{eV/atom}$,比\ref{Sec:atom-Pt}节计算的孤立原子\textrm{Pt}的基态能量要低。

下一节我们将讨论晶胞参数的优化和影响\textrm{VASP}计算精度的重要参数的确定。
\section{收敛测试}\label{Sec:convergence}
对于一个正常的\textrm{DFT}迭代计算过程,随着迭代次数递增,体系总能(或自由能)会快速下降,随后进到逐渐平稳变化的区域,并伴随小的振荡,最终达到基态能量,如图\ref{Fig:convergence}所示,这个过程称为\textbf{收敛}。
\begin{figure}[h!]
\centering
\includegraphics[height=4.1in,viewport=0 33 740 600,clip]{Ab-initio-Ene.png}
\caption{\small 迭代计算中的能量收敛示意图,引自文献\inlinecite{Comput-phys}.}%(与文献\cite{EPJB33-47_2003}图1对比)
\label{Fig:convergence}
\end{figure}

一般地,对于任何研究体系,首先必须保证迭代计算能收敛,然后才谈得上进行有意义的\textrm{DFT}物性计算。同时,面对一个全新的计算体系,合理的收敛测试也是非常必要的,否则难免遭遇为克服迭代计算收敛的困难,浪费过多计算资源和精力的尴尬。考虑到\textrm{DFT}计算的本质是变分过程,因此只有各种初始条件选择都比较合理,计算才可能收敛到正确的结果,否则很可能收敛到错误的结果。在收敛测试中,最重要的参数主要包括:
\begin{itemize}
	\item \textrm{INCAR}中的参数\textit{ENCUT}
	\item \textrm{KPOINTS}中的$\vec k$空间布点数目
\end{itemize}
在正常收敛测试中,两个参数的数值都应该由小到大逐渐增加,这将有助于考察计算体系的总能是否随参数逐渐趋于收敛。
\subsection{能量收敛测试}
体系截断能的定义
\begin{equation}
	E_{\mathrm{cut}}\geqslant\dfrac12|\vec k+\vec G|^2
	\label{eq:Ecut}
\end{equation}
换句话说,截断能确定的是平面波基波矢的上限。一般截断能的取值范围是$150\sim400\mathrm{~eV}$,其默认阈值由计算体系组成元素的\textrm{POTCAR}中\textit{ENMAX}和\textit{ENMIN}中的极大值和极小值确定。但是,在具体计算过程中,合理的截断能应该通过测试确定。本次练习使用的\textrm{shell}脚本中,将包含3个输入文件(\textrm{INCAR},\textrm{KPOINTS}和\textrm{POSCAR})的设置,并且有4个不同的截断能参数\textit{ENCUT}(200,~225,~250和350\textrm{~eV})。计算中$\vec k$空间布点数固定($9\times9\times9$)。由此可以同时确定\textit{ENCUT}的收敛情况和平衡态晶胞参数。
\subsubsection{\rm{shell}脚本\rm{run.lattice}}
这里给出一个\textrm{shell}脚本的范例(文件名\textcolor{green}{\textrm{run.lattice}}),如图\ref{Pt_FCC-runlattice}所示,当前截断能为\textit{ENCUT}=250\textrm{~eV}。
\begin{figure}[h!]
\centering
\hspace*{-20pt}
\includegraphics[height=5.2in,viewport=0 20 860 700,clip]{Pt_FCC-OUTCAR_runlattice.png}
\caption{\small \textrm{FCC-Pt}截断能收敛测试\textrm{shell}脚本\textrm{run.lattice}示例.}%(与文献\cite{EPJB33-47_2003}图1对比)
\label{Pt_FCC-runlattice}
\end{figure}

从该\textrm{shell}脚本内容看,晶胞参数将\textrm{\$a}从$3.90\mathrm{\AA}$到$4.06\mathrm{\AA}$变化,并且每改变一次晶胞参数将会执行一轮\textrm{VASP}晶格弛豫计算流程。因为是收敛测试,而平面波的收敛与自旋无关,因此\textrm{VASP}计算不考虑自旋极化的影响。

有必要指出的是,在\textrm{shell}脚本中,生成\textrm{INCAR}、\textrm{KPOINTS}和\textrm{POSCAR}文件的各命令行必须连续,不能被注释或空行隔断,这一点请初学者务必注意。
\subsubsection{\rm{测试脚本的执行}}
与\ref{Sec:FCC-Pt}的\textrm{shell}脚本执行类似,首先可执行权限:\\
\textcolor{magenta}{\textrm{
chmod~+x~./run.lattice}}\\% \# change the file mode to execution mode\\
再运行:\\
\textcolor{magenta}{\textrm{./run.lattice }}
\subsubsection{\rm{执行结果}}
脚本执行完毕,结果将写入到文件\textrm{Pt-lattice-999-E.dat}中。因此当截断能\textit{ENCUT}=250\textrm{~eV}时,\textrm{FCC-Pt}的不同晶胞参数下得到不同的基态总能,可用命令\\
\textcolor{magenta}{\textrm{cat~Pt-lattice-999-E.dat}}\\
查看,如图\ref{Pt_FCC-energy-con}所示。
\begin{figure}[h!]
\centering
\vskip -8pt
\includegraphics[height=1.3in,viewport=0 0 620 200,clip]{Pt_FCC-energy-con.png}
\caption{\small \textrm{FCC-Pt}截断能收敛测试:~不同晶胞参数下的基态总能.}%(与文献\cite{EPJB33-47_2003}图1对比)
\label{Pt_FCC-energy-con}
\end{figure}

图\ref{Pt_FCC-energy-curve}给出的是四种不同的截断能(分别是200,~225,~250和350\textrm{~eV})下的运行结果,截断能在\textit{ENCUT}=250\textrm{~eV}时,总能达到收敛(差别小于5\textrm{~meV})。平衡态的晶格参数为$3.077\mathrm{\AA}$,该值与\textrm{Bentmann}等的计算值$3.975\mathrm{\AA}$\cite{PRB78-205302_2008}吻合得很好,与实验值$3.928\mathrm{\AA}$\cite{Kittel}也较为接近。根据优化参数得到的体系总能为~--24.2269\textrm{~eV}(--6.057\textrm{~eV/atom})。
\begin{figure}[h!]
\centering
\includegraphics[height=3.3in,viewport=0 11 820 650,clip]{Pt_FCC-Ecut-convergence.png}
\caption{\small \textrm{FCC-Pt}截断能收敛测试:~不同截断能下的基态总能曲线($\vec k$-点数目$9\times9\times9$).}%(与文献\cite{EPJB33-47_2003}图1对比)
\label{Pt_FCC-energy-curve}
\end{figure}

不难预见,随着截断能的增加,计算时长将会增加($\vec k$点增加的情形类似地,见下)。关于\textit{ENCUT}的设置注意以下事项:
\begin{itemize}
	\item 对于多元素组分的体系,一般默认最高的截断能作为体系截断能;
	\item 对于需要比照的体系,计算时应将\textit{ENCUT}和\textrm{KPOINTS}的数值设置成相同;
	\item 对于晶格形状和体积都可以弛豫(\textit{ISIF}=3)的体系,一般\textit{ENCUT}比默认值增大$\sim30\%$,并设置参数\textit{PREC}=\textrm{accurate}
\end{itemize}

由截断能的定理式\eqref{eq:Ecut}可知,随着\textit{ENCUT}的增大,每个\textrm{k}-点上用于展开轨道的平面波数目的相应变化,将会呈现$\mathrm{N}_{\mathrm{PW}}\propto E_{\mathrm{cut}}^{3/2}$的基本规律。这一点可以在\textrm{OUTCAR}中查找平面波数目得到印证。命令为:~\\
\textcolor{magenta}{\textrm{grep~\`~plane waves\`~OUTCAR}}\\
图\ref{Pt_FCC-PW}给出当截断能\textit{ENCUT}=250\textrm{~eV}时,部分$\vec k$点上用于展开波函数的平面波基组的数目。
\begin{figure}[h!]
\centering
\vskip -8pt
\includegraphics[height=0.6in,viewport=0 0 500 80,clip]{Pt_FCC-PW.png}
\caption{\small \textrm{FCC-Pt}截断能收敛测试:~截断能为250\textrm{~eV}时各$\vec k$点的平面波基数目(部分).}%(与文献\cite{EPJB33-47_2003}图1对比)
\label{Pt_FCC-PW}
\end{figure}

此外,在\textrm{OUTCAR}中,\textit{NPLWV}表示\textrm{FFT}变换的总的网格点数($\mathit{NGX}\times\mathit{NGY}\times\mathit{NGZ}$),这里$\mathit{NGX}$, $\mathit{NGY}$和$\mathit{NGZ}$分别对应$x$-,$y$-和$z$-方向的网格点数。如果对比不同晶体结构的\textrm{Pt}体系,如六方密堆积的\textrm{HCP-Pt}和体心立方的\textrm{BCC-Pt}完成相同的结构弛豫,我们将会发现,面心立方结构\textrm{FCC-Pt}具有最低的基态能量,亦即具有是最稳定的基态结构。这部分内容作为课后练习,留给大家自行完成并验证上述结论。

\subsection{$\vec k$点收敛测试}
一旦确定了合适的\textit{ENCUT},可以用类似的方式确定$\vec k$点数目。利用晶体对称性,倒空间中的均匀网格点将被约化到不可约\textrm{Brillouin}区(\textrm{Irreducible Brillouin Zone,~IBZ})中。这些不可约$\vec k$-点作为$\vec k$空间的积分权重点,将会用于体系物性计算。当确定参数\textit{ENCUT}为250\textrm{~eV},平衡态晶格常数为$3.977\mathrm{\AA}$后,依次增加$\vec k$-空间网格布点(由$2\times2\times2$到$10\times10\times10$),直至总能的能量差是收敛到$\sim1\mathrm{meV}$。

输出文件\textrm{IBZKPT}中保存的是\textrm{IBZ}区域的$\vec k$点信息,随着$\vec k$点数的增加,\textrm{IBZ}的网格点将由1增大到35。$\vec k$-点收敛测试与\textit{ENCUT}收敛的情形类似,结果如图\ref{Pt_FCC-kpoint-curve}所示。从图中不难看出,对于当前体系,$9\times9\times9$的网格点能保证足够精度。
\begin{figure}[h!]
\centering
\includegraphics[height=3.3in,viewport=0 11 820 650,clip]{Pt_FCC-kpoint-convergence.png}
\caption{\small \textrm{FCC-Pt}截断能收敛测试:~基态总能随$\vec k$-点变化的曲线(截断能\textit{ENCUT}=250\textrm{~eV}).}%(与文献\cite{EPJB33-47_2003}图1对比)
\label{Pt_FCC-kpoint-curve}
\end{figure}

需要指出的是,从文件\textrm{IBZKPT}中的$\vec k$点分布看,采用\textrm{Monkhorst-Pack}布点方案,将每个维度格点剖分成偶数时,$\vec k$空间的网格点分布会更均匀。

\section{Pt超晶胞计算}\label{Sec:bulk-Pt}
下面我们将学习由32个\textrm{Pt}原子构成的超晶胞(\textrm{FCC}晶胞按$2\times2\times2$堆积得到)的基本的物理性质计算:~重点掌握内聚能\textrm{(cohesive energy)}和空位形成能\textrm{(vavanvy formation energy)}的计算。为简明起见,前面介绍过的参数,今后不再作详细说明。

\subsection{Pt超晶胞的内聚能计算}
为了计算内聚能,首先计算理想超晶胞的\textrm{Pt}基态能量。为了节约计算时间和内存的消耗,计算过程中不再要求输出电荷密度和波函数到\textrm{CHGCAR}和\textrm{WAVECAR}文件。此外还将利用体系的对称性来加速计算,因此计算控制参数中引入\textit{ISYM}=1。输入控制文件\textrm{INCAR}的修改部分如图\ref{Pt_Bulk-INCAR-modified}所示:~
\begin{figure}[h!]
\centering
\includegraphics[height=1.1in,viewport=0 20 340 118,clip]{Pt_Bulk-INCAR.png}
%\includegraphics[height=1.8in,width=4.in,viewport=30 210 570 440,clip]{PAW_projector.eps}
\caption{\small \textrm{计算\textrm{Pt}超晶胞时\textrm{INCAR}文件的修改部分.}}%(与文献\cite{EPJB33-47_2003}图1对比)
\label{Pt_Bulk-INCAR-modified}
\end{figure}

因为超晶胞是由8倍的\textrm{FCC-Pt}构成,$\vec k$-点数约简为$5\times5\times5$,\textrm{KPOINTS}文件如图\ref{Pt_Bulk-KPOINTS}所示:~
\begin{figure}[h!]
\centering
\vskip -3pt
\includegraphics[height=1.0in,viewport=0 20 240 108,clip]{Pt_Bulk-KPOINTS.png}
%\includegraphics[height=1.8in,width=4.in,viewport=30 210 570 440,clip]{PAW_projector.eps}
\caption{\small \textrm{计算\textrm{Pt}超晶胞时的\textrm{KPOINTS}文件.}}%(与文献\cite{EPJB33-47_2003}图1对比)
\label{Pt_Bulk-KPOINTS}
\end{figure}

考虑到\ref{Sec:FCC-Pt}已经得到\textrm{FCC-Pt}结构弛豫后的晶胞,因此$2\times2\times2$堆积的超晶胞中原子起始结构如图\ref{Pt_Bulk-POSCAR}所示。虽然\textrm{INCAR}中的控制参数允许晶胞结构弛豫,超晶胞的\textrm{POSCAR}文件中的原子也仍保持可移动状态,但可以预见,体系的弛豫计算不应该带来很显著的形变。
\begin{figure}[h!]
\centering
\vskip -5pt
\includegraphics[height=2.0in,viewport=0 25 500 270,clip]{Pt_Bulk-POSCAR.png}
%\includegraphics[height=1.8in,width=4.in,viewport=30 210 570 440,clip]{PAW_projector.eps}
\caption{\small \textrm{计算\textrm{Pt}超晶胞时的\textrm{POSCAR}文件(部分).}}%(与文献\cite{EPJB33-47_2003}图1对比)
\label{Pt_Bulk-POSCAR}
\end{figure}

用\ref{Sec:FCC-Pt}的\textrm{shell}脚本\textcolor{green}{\textrm{run.vasp}}提交计算任务至后台运行,计算过程中可以检查输出文件\textrm{nohup.out}或\textrm{OSZICAR}的内容可以监控计算过程,图\ref{Pt_Bulk-OSZICAR}给出\textrm{nohup.out}的部分结果:
\begin{figure}[h!]
\centering
\hspace*{-25pt}
\includegraphics[height=1.5in,viewport=0 0 950 215,clip]{Pt_Bulk-OSZICAR.png}
%\includegraphics[height=1.8in,width=4.in,viewport=30 210 570 440,clip]{PAW_projector.eps}
\caption{\small \textrm{计算\textrm{Pt}超晶胞时的\textrm{nohup}文件(局部).}}%(与文献\cite{EPJB33-47_2003}图1对比)
\label{Pt_Bulk-OSZICAR}
\end{figure}
这里各参数的含义:
\begin{itemize}
	\item \textrm{N}:~统计两个离子步计算之间的电子步迭代次数
	\item \textrm{E}:~当前的基态总能
	\item \textrm{dE}:~两次电子步迭代之间的基态总能变化
	\item \textrm{d~eps}:~两次电子步迭代的能量本征值的改变(势保持不变)
	\item \textrm{ncg}:~完成一次迭代时,\textrm{Hamiltonian}算符\textbf{H}作用于轨道的次数~
	\item \textrm{rms}:~每次迭代开始时全部占据轨道的初始残矢\textrm{(residual vector)}——$\mathbf{R}=(\mathbf{H}-\varepsilon)\mathbf{S}|\phi\rangle$——之和,该值表明轨道的收敛情形的优劣
	\item \textrm{rms(c)}:~一次迭代前后(即输入输出)的电荷密度差
\end{itemize}

\textrm{VASP}程序为了使迭代过程尽可能快速收敛,设定完成最初五次电子步迭代后,才开始电荷密度的迭代,所以最初5次电子步迭代时,没有\textrm{rms(c)}输出。电子步结束时,最后一行的\textrm{F}表示电子步达到收敛的自由能,根据同一行的$\mathrm{E}_0$值(0\textrm{K}时$\sigma\rightarrow0$),可以算出超晶胞中的单个原子\textrm{Pt}的能量为--193.8569/32=--6.058\textrm{~eV/atom}。不难看出,该值比晶格参数为$3.975\mathrm{\AA}$时计算的\textrm{FCC-Pt}的原子能量--6.056\textrm{eV/atom}仅有微小变化。

如果迭代计算过程中出错,\textrm{VASP}将会输出错误或警告提示信息,并终止程序运行。此外也可以通过\textrm{kill}命令终止程序。命令为:~\\
\textcolor{magenta}{\textrm{killall -9 VASP\_pid}}\\
这里\textrm{VASP\_pid}是当前执行的\textrm{VASP}进程号。

如果计算过程中发现有参数或设置问题,可以在计算目录中产生一个\textrm{STOPCAR}文件,并向其中增添一行内容:~\\
\textit{LSTOP}=\textrm{.TRUE.}\\
则正在运行的\textrm{VASP}程序会在当前离子步结束并输出\textrm{WAVECAR}和\textrm{CHGCAR}后,进入下一离子步计算时退出运行。一般用户可以在修正发现的错误后,在现有计算基础上,继续向下完成整个计算任务。

\subsubsection{内聚能}
内聚能\textrm{(Cohesive energy) $E_{\mathrm{coh}}$}定义为体相的平均原子能量和自由原子能量$E_{\mathrm{atom}}$的能量差。内聚能是衡量原子构成固体时原子间相互作用强弱的参数,也是材料的一个基本属性。换句话说,内聚能也可以通过固体原子在平衡态附近的极小值扣除自由原子能量得到。因此可以用以下两个值计算内聚能:~
\begin{displaymath}
	\begin{aligned}
		E_{\mathrm{bulk}}&=-193.85695/32=-6.058~\mathrm{eV/atom}\\
		E_{\mathrm{coh}}&=E_{\mathrm{atom}}-E_{\mathrm{bulk}}=(-0.528)-(-6.058)=5.53\mathrm{eV/atom}
	\end{aligned}
\end{displaymath}
该值与\textrm{Benthmann等}~的计算值5.53\textrm{~eV/atom}\cite{PRB78-205302_2008}一致,与实验值5.45\textrm{~eV/atom}\cite{Landolt-Bornstein}的数值也比较吻合。
\subsection{\rm{Pt}的空位生成能计算}
室温条件下,金属体相内都会存在空位,空位浓度约为$10^{-6}$量级。显然,我们不可能为了模拟一个空位,就用$10^6$量级的超晶胞。合理的做法应该是选择合适尺度的超晶胞并在其中设计原子空位,要求作为重复单元的超晶胞间的这些空位相互作用可以忽略,由此计算晶格中的空位能。因为计算的超晶胞中仅有一个空位,计算时将\textrm{INCAR}中关闭对称性。
\begin{figure}[h!]
\centering
\includegraphics[height=1.1in,viewport=0 15 370 180,clip]{Pt_vacancy-INCAR.png}
%\includegraphics[height=1.8in,width=4.in,viewport=30 210 570 440,clip]{PAW_projector.eps}
\caption{\small \textrm{计算\textrm{FCC-Pt}超晶胞出现空位时\textrm{INCAR}文件的修改部分.}}%(与文献\cite{EPJB33-47_2003}图1对比)
\label{Pt_Bulk-INCAR-modified}
\end{figure}

\textrm{POSCAR}文件可以从之前的超晶胞算例中\textrm{copy}来,将总的原子数减为31,扣除位于(0.5,~0.5,~0.5)处的\textrm{Pt}原子,产生空位,如图\ref{Pt_vacancy-POSCAR}所示:~
\begin{figure}[h!]
\centering
\vskip -3pt
\includegraphics[height=0.85in,viewport=0 15 750 120,clip]{Pt_vacancy-POSCAR.png}
%\includegraphics[height=1.8in,width=4.in,viewport=30 210 570 440,clip]{PAW_projector.eps}
\caption{\small \textrm{模拟\textrm{FCC-Pt}超晶胞中含有一个空位时\textrm{POSCAR}文件的修改分.}}%(与文献\cite{EPJB33-47_2003}图1对比)
\label{Pt_vacancy-POSCAR}
\end{figure}

后续计算与\textrm{Pt}超晶胞计算类似,从\textrm{OSZICAR}文件中看到,含有31个\textrm{Pt}原子和一个空位的超晶胞基态能量是--186.95325\textrm{~eV/atom}。
\subsubsection{空位形成能}
根据空位形成能$E_{\mathrm{v}}^f$的定义可以有:~
\begin{displaymath}
	E_{\mathrm{v}}^f=E_{\mathrm{v}}-\dfrac{N-1}N\times E_{\mathrm{bulk}}=-186.95325-\dfrac{31}{32}(-193.85695)=0.846\mathrm{eV/atom}
\end{displaymath}
这里$E_{\mathrm{v}}$是含有一个空位的超晶胞基态总能,$N$是理想超晶胞的原子个数,$E_{\mathrm{bulk}}$是理想晶体的总能(上个算例中的值是~--193.85695~\textrm{eV})。我们这里的计算结果与其他计算值0.68\textrm{~eV/vacancy}或通过正电子子湮灭测量的值1.35\textrm{~eV/vacancy}\cite{PSSA102-47_1987}有不小的差别。\textrm{Mattsson}等发现,只要引入体系的表面误差校正,就能改善\textrm{Pt}的空位形成能,结果为1.18\textrm{eV/vacancy}。\cite{PRB66-214110_2002}对于其他的材料,比如\textrm{W}\cite{JNM383-244_2009}或\ch{SiC}\cite{JMS44-1828_2009},用\textrm{DFT}计算的缺陷形成能与实验值吻合得比较好。\textrm{CONTCAR}文件中可以看到空位附近原子的弛豫情况。\textrm{WAVECAR}是二进制文件,保存的是最终得到的电子轨道波函数,亦即\textrm{Kohn-Sham}方程的解。在本次算例中,因为\textrm{INCAR}中的设置,则未将波函数写到\textrm{WAVECAR}中。
\subsubsection{\rm{CHGCAR}绘图}
\textrm{VASP}运行结束,会将电荷密度写到\textrm{CHG}和\textrm{CHGCAR}文件,这两个文件的内容基本相同,主要包括:~晶格矢量、原子位置,电荷密度等。表示电荷密度的网格与超晶胞的形态成正比,网格的数目由\textrm{OUTCAR}中\textrm{FFT}变换的网格数\textit{NFXF}、\textit{NGYF}和\textit{NGZF}确定。将\textrm{CHGCAR}或\textrm{CHG}中的电荷密度按超晶胞体积划分,就可以得到电荷密度的轮廓。例如,本次练习得到的\textrm{CHGCAR}文件的内容如图\ref{Pt_vacancy-CHGCAR}所示:~
\begin{figure}[h!]
\centering
\vskip -5pt
\includegraphics[height=2.0in,viewport=0 20 490 240,clip]{Pt_vacancy-CONTCAR.png}
%\includegraphics[height=1.8in,width=4.in,viewport=30 210 570 440,clip]{PAW_projector.eps}
\caption{\small \textrm{计算\textrm{FCC-Pt}超晶胞出现空位时的\textrm{CHGCAR}文件(部分).}}%(与文献\cite{EPJB33-47_2003}图1对比)
\label{Pt_vacancy-CHGCAR}
\end{figure}

用开源软件比如\textrm{VASPview}\footnote{\textrm{VASPview~\url{http://vaspview.sourceforge.net}}}或\textrm{VESTA}\footnote{\textrm{VESTA(ver.2.90.1b),~2011.~\url{http://www.geocities.jp/kmo\_mma/index-en.html}}}可以将\textrm{CHGCAR}文件的电荷密度可视化,效果如图\ref{Pt_vacancy-Density}所示:
\begin{figure}[h!]
\centering
\includegraphics[height=3.0in,viewport=0 0 640 660,clip]{Pt_vacancy-CHGCAR.png}
%\includegraphics[height=1.8in,width=4.in,viewport=30 210 570 440,clip]{PAW_projector.eps}
\caption{\small \textrm{计算\textrm{FCC-Pt}超晶胞出现空位时的图像.}}%(与文献\cite{EPJB33-47_2003}图1对比)
\label{Pt_vacancy-Density}
\end{figure}

注意:~图中空位附近的空白区域(图中只显示了超晶胞的下半区域)和每个原子附近近乎均匀分布的电子分布。

类似地,间隙形成能\textrm{(the formation energies of an interstitial)}$E_{\mathrm{inter}}^f$可以定义为
\begin{displaymath}
	E_{\mathrm{inter}}^f=E_{\mathrm{inter}}^{\mathrm{bulk}}-E^{\mathrm{bulk}}-nE^{\mathrm{atom}}
\end{displaymath}
这里$E_{\mathrm{inter}}^{\mathrm{bulk}}$和$E^{\mathrm{bulk}}$是带有间隙的超晶胞和理想超晶胞的总能,$n$是间隙处的原子数目,$E^{\mathrm{atom}}$是孤立原子的能量。

各种缺陷的形成能都可以用这些简化模型来模拟。用户在构造这种包含缺陷的超晶胞时,必须切记,设计的超晶胞要足够大,确保缺陷间的相互作用足够小。

进一步推广,如果存在两个固相\textrm{A}和{B},它们可以形成\textrm{AB}相,则三个体相的能量能量差
\begin{displaymath}
	\Delta H_{\mathrm{AB}}=E_{\mathrm{AB}}^{\mathrm{bulk}}-E_{\mathrm{A}}^{\mathrm{bulk}}-E_{\mathrm{B}}^{\mathrm{bulk}}
\end{displaymath}
定义为固体生成焓\textrm{(the formation enthalpy)}。
\section{\rm{Pt~(111)}表面的计算}\label{Sec:Surface-Pt}
实际应用中,材料的表面性质和体相性质一样重要。表面的很多基本性质,比如表面能\textrm{(surface energy)}、功函数\textrm{(work function)}、吸附能\textrm{(adsorption energy)}、吸附原子迁移势垒\textrm{(barrier energy for transport of the adatom)}等可用来确定材料的用途。比如表面能是表面形貌学研究(向外/向内弛豫、重构、屈曲分析)和裂纹扩散到断裂研究的重要因素。此外,功函数、吸附能/解吸能和势垒能量是研究表面氧化、薄膜表面和纳米结构的生长和稳定、腐蚀、钝化和催化反应的重要决定因素。只有真正理解了这些表面现象,才有望从设计层面获得更高性能的材料。本节中,我们将学习构建\textrm{Pt}原子构成的表面,并研究上述提到的表面属性的计算。
\subsection{\rm{Pt~(111)}表面}
首先构建\textrm{Pt~(111)}面薄层并计算其基态总能、弛豫能和表面能的计算方案。这些数据将成为后续进一步计算吸附能和迁移势垒能量的参考基准。
\subsubsection{\rm{INCAR}}
\textrm{INCAR}文件的内容如图\ref{Pt_Slab-INCAR}所示:~
\begin{figure}[h!]
\centering
\vskip -5pt
\includegraphics[height=3.3in,viewport=0 10 690 388,clip]{Pt_Slab-INCAR.png}
%\includegraphics[height=1.8in,width=4.in,viewport=30 210 570 440,clip]{PAW_projector.eps}
\caption{\small \textrm{计算\textrm{Pt}表面时\textrm{INCAR}文件.}}%(与文献\cite{EPJB33-47_2003}图1对比)
\label{Pt_Slab-INCAR}
\end{figure}
\subsubsection{\rm{KPOINTS}}
\textrm{KPOINTS}内容如图\ref{Pt_Slab-KPOINTS}所示:~
\begin{figure}[h!]
\centering
\vskip -5pt
\includegraphics[height=1.0in,viewport=0 10 400 98,clip]{Pt_Slab-KPOINTS.png}
%\includegraphics[height=1.8in,width=4.in,viewport=30 210 570 440,clip]{PAW_projector.eps}
\caption{\small \textrm{计算\textrm{Pt}表面时\textrm{KPOINTS}文件.}}%(与文献\cite{EPJB33-47_2003}图1对比)
\label{Pt_Slab-KPOINTS}
\end{figure}
\subsubsection{\rm{POSCAR}}
一般作为表面计算时,是用薄层和外加真空层来代表表面。因此要做两个收敛测试:~即针对薄层厚度和真空厚度选择的测试。本练习中,\textrm{Pt~(111)}表面如图\ref{Pt_surface}所示,最小重复单元由45个原子构成的薄层外加大的真空层($\sim16\mathrm{\AA}$)构成,为的是防止表面与临近的表面间存在相互作用。至于选择\textrm{Pt~(111)}面作为研究对象,是因为这个表面在\textrm{Pt}催化材料中研究得最多。注意,在\textrm{Pt~(111)}表面层,要考虑\textrm{Pt}原子发生重排问题,比如会形成六方密堆积\textrm{(Hexagonal closed-packing, HCP)}构型,如图\ref{Pt_surface}中所示。
\begin{figure}[h!]
\centering
\includegraphics[height=3.1in,viewport=0 0 300 580,clip]{Pt_surface.png}
%\includegraphics[height=1.8in,width=4.in,viewport=30 210 570 440,clip]{PAW_projector.eps}
\caption{\small \textrm{计算\textrm{Pt}薄层的表面与重排.}}%(与文献\cite{EPJB33-47_2003}图1对比)
\label{Pt_surface}
\end{figure}

考虑\textrm{FCC}立方晶格参数$a$与六方晶格\textrm{HCP}参数的关系:~
\begin{displaymath}
	a_{\mathrm{HCP}}=\dfrac{a_{\mathrm{FCC}}}{\sqrt2}=\dfrac{3.977}2=2.812\mathrm{\AA}
\end{displaymath}
这就是\textrm{Pt}原子的最近邻原子间距离。\textrm{POSCAR}中共有45个\textrm{Pt}原子,按\textrm{FCC}结构的\textit{a-b-c-a-b}顺序排列。一般说,差不多当原子堆积厚度达到九层才能使薄层内两侧表层的原子间相互作用忽略不计。不过在本次练习中,我们的模型只使用五层原子,主要是考虑到节省学习过程中的等待时间,同时也因为由此(薄层层数过少)产生误差尚在可接受的范围内。

在\textrm{FCC-Pt}体相划出晶格矢量为
\begin{displaymath}
	(a,0,0),~\bigg(\dfrac a2,\dfrac{\sqrt3a}2,0\bigg),~(0,0,c)
\end{displaymath}
的\textrm{Pt}原子密堆积单胞,这里$a=2.812\mathrm{\AA}$,~$c=\sqrt{8/3}a=1.633a$。用$3\times3\times6$~(含真空层)的单胞堆积得到模拟\textrm{Pt~(111)}表面的超晶胞,构成模拟\textrm{FCC}中\textrm{(111)}表面的最小重复单元。

为了稳定薄层模型,最底层(九个原子)完全固定,而紧邻底层(九个原子)则只在$x$和$y$方向固定。其余三层则允许在各方向弛豫。\textrm{Pt~(111)}表面模拟使用的\textrm{POSCAR}文件如下图\ref{Pt_surface-POSCAR}所示:
\begin{figure}[h!]
\centering
\includegraphics[height=5.1in,viewport=0 5 540 565,clip]{Pt_surface-POSCAR.png}
%\includegraphics[height=1.8in,width=4.in,viewport=30 210 570 440,clip]{PAW_projector.eps}
\caption{\small \textrm{模拟\textrm{Pt~(111)}表面时使用的\textrm{POSCAR}文件.}}%(与文献\cite{EPJB33-47_2003}图1对比)
\label{Pt_surface-POSCAR}
\end{figure}

这样选择的\textrm{Pt~(111)}表面是\textrm{FCC}结构下最密堆积的表面, 比\textrm{(001)}表面的原子密度要高约15\%。计算开始之后,在最初生成的\textrm{OUTCAR}文件中,我们将能看到\textrm{Pt}原子间的最近邻距离是$2.81\mathrm{\AA}$。

\subsubsection{\rm{计算结果}}
\textrm{OSZICAR}文件中记录的是弛豫过程中的\textrm{Pt~(111)}表面体系的基态能量变化情况。弛豫完全结束后,可以通过命令:\\
\textcolor{magenta}{\textrm{grep~\`~E0~\'~OSZICAR}}\\
来查看弛豫过程中的能量变化情况,结果如图\ref{Pt_surface-OSZICAR}所示。
\begin{figure}[h!]
\centering
\includegraphics[height=0.9in,viewport=0 0 640 108,clip]{Pt_surface-OSZICAR.png}
%\includegraphics[height=1.8in,width=4.in,viewport=30 210 570 440,clip]{PAW_projector.eps}
\caption{\small \textrm{\textrm{Pt~(111)}表面计算弛豫过程中的体系能量变化(部分).}}%(与文献\cite{EPJB33-47_2003}图1对比)
\label{Pt_surface-OSZICAR}
\end{figure}

最后一行就是体系弛豫完成后的自由能和基态能量。

在\ref{Sec:bulk-Pt}节,我们得到体相超晶胞中的一个\textrm{Pt}原子的能量是~--6.058\textrm{~eV/atom},因此未弛豫表面能\textrm{(the unrelaxed surface energy)},$\gamma_{\mathrm{unrel}}$可由下式确定:~
\begin{displaymath}
	\begin{aligned}
		\gamma_{\mathrm{s}}^{\mathrm{unrel}}&=\dfrac12\big(E_{\mathrm{slab}}^{\mathrm{unrel}}-NE_{\mathrm{atom}}^{\mathrm{bulk}}\big)=\dfrac12[-261.0906-45(-6.058)]
	&=5.76\mathrm{~eV/system}
	\end{aligned}
\end{displaymath}
表面弛豫能\textrm{(the surface relaxation energy)}可由由体系完全弛豫的能量与初始能量(未经弛豫)差计算,也就是\textrm{OSZICAR}的最后一行与第一行的$E_0$差:~
\begin{displaymath}
	\gamma_{\mathrm{s}}^{\mathrm{rel}}=E_{\mathrm{last}}-E_{\mathrm{1st}}=-261.1438-(-261.0906)=-0.053\mathrm{~eV/system}
\end{displaymath}
注意表面弛豫能(也称为表面重构能)一般都比较非常小(约<1\%的表面能),而考虑底层固定的弛豫表面能\textrm{(the relaxed surface energy)}为:~
\begin{displaymath}
	5.76-0.053=5.707\mathrm{~eV/sytem}~(\mbox{即}~0.09\mathrm{~eV/\AA^2})
\end{displaymath}
该弛豫表面能与\textrm{Getman}等的高精度计算结果$0.09\mathrm{~eV/\AA^2}$\cite{JPC112-9559_2008}完全一致。另一种计算表面能的方法是固定中间层而弛豫上下表面,注意到\textrm{FCC}晶体中不同截面的原子堆垛方式不同,一般表面能按以下顺序递增:~
\begin{displaymath}
	\gamma_{111}<\gamma_{100}<\gamma_{110}
\end{displaymath}
如果对比\textrm{POSCR}和\textrm{CONTCAR}文件的晶胞参数变化,不难发现,表面层的膨胀$\sim1\%$。

在\textrm{Pt~(111)}表面,波函数随着真空层出现,将快速衰减为零。但计算中的平面波基组则遍及真空层和薄层,正因为此,这种表面模拟的平面基数目会比普通体相计算大得多,相应的计算时长也增加得比较多(本算例中,45个原子的薄层体系耗时6344秒),一般比体相计算的计算量大1-2倍。本次计算没有考虑自旋极化效应,因为在超晶胞中,自旋倾向于均匀分布\footnote{所谓“均匀分布”,实质上是未成对自旋取向孤立分布,且又彼此远离,所以很难耦合形成有效磁矩}。对于两种固体构成的表面,如金属-金属、金属-氧化物表面,也可以按类似方式处理。

\subsection{吸附能}
与表面有关的反应过程,如异相催化和化学气相沉积等,很大程度上与吸附表面的表面能有关。对于金属材料,吸附能的大小可以视为描述吸附原子轨道与金属原子的\textit{s}-,\textit{p}-和\textit{d}-电子轨道的相互作用强弱的物理量,吸附能对于确定表面化学反应的反应机理至关重要,这部分练习中,我们以简单的氧原子吸附在\textrm{Pt~(111)}表面为例,计算体系的吸附能。在\textrm{Pt~(111)}面上,有四种可能的吸附位点\textrm{(adsorption site)}:~\textrm{FCC}结构和\textrm{HCP}的间隙位\textrm{(hollow)},\textrm{Pt}原子的顶位\textrm{(top site)}和两个最近邻\textrm{Pt}原子的桥接位\textrm{(bridge site)},如图\ref{Pt_surface-site}所示。考虑到氧原子倾向于优先占据\textrm{FCC}间隙位,因此后面的计算将围绕该吸附位点进行。
\begin{figure}[h!]
\centering
\includegraphics[height=3.8in,viewport=0 0 860 530,clip]{Pt_surface-site.png}
%\includegraphics[height=1.8in,width=4.in,viewport=30 210 570 440,clip]{PAW_projector.eps}
\caption{\small \textrm{Pt~(111)表面俯视图,大球表示上层,小球表示下层。各吸附位见箭头标注.}}%(与文献\cite{EPJB33-47_2003}图1对比)
\label{Pt_surface-site}
\end{figure}

\subsubsection{\rm{POSCAR}}
为计算固体表面吸附的气体原子或分子的吸附能,建模时要确保气体原子或分子与固体表面原子足够近,否则吸附原子或分子极易离开表面。因此在构建\textrm{POSCAR}时,可以通过在前面计算的\textrm{Pt~(111)}表面模型得到的\textrm{CONTCAR}基础上得到吸附表面的\textrm{POSCAR}文件,如图\ref{Pt_surface-adsorption-POSCAR}所示:~
\begin{figure}[h!]
\centering
\includegraphics[height=2.4in,viewport=0 10 520 280,clip]{Pt_surface-adsorption-POSCAR.png}
%\includegraphics[height=1.8in,width=4.in,viewport=30 210 570 440,clip]{PAW_projector.eps}
\caption{\small \textrm{\textrm{Pt~(111)}表面吸附\textrm{O}的结构模型.}}%(与文献\cite{EPJB33-47_2003}图1对比)
\label{Pt_surface-adsorption-POSCAR}
\end{figure}

这里,总的原子数变成46个:~原有的45个\textrm{Pt}原子的薄层模型加上一个吸附的\textrm{O}原子。\textrm{O}原子坐标~(0.55555,~0.55555,~0.37593)~加在文件最后一行,这是\textrm{FCC}结构的间隙位,在空间中与三个\textrm{Pt}原子近邻。\textrm{O}原子的$z$-轴坐标根据文献报道的\textrm{Pt-O}距离确定的。\textrm{KPOINTS}文件与之前\textrm{Pt~(111)}表面计算相同。
\subsubsection{\rm{POTCAR}}
表面吸附计算需要的\textrm{POTCAR}文件由\textrm{Pt}和\textrm{O}各自的\textrm{POTCAR}文件拼接而成,命令如下:~\\
\textcolor{magenta}{\textrm{cat~ POTCAR-Pt~ POTCAR-O ~ > ~POTCAR}}\\
这里选择的是\textrm{PBE}泛函构造的原子数据集,因此用户可以通过关键词\textrm{PBE},获得有关元素和交换-相关泛函信息,如:~\\
\textcolor{magenta}{\textrm{grep ~ \`~PBE~\' ~ POTCAR}}\\
\subsubsection{\rm{计算结果}}
首先检查\textrm{Pt~(111)}表面吸附\textrm{O}原子的模型基态能量,命令为:\\
\textcolor{magenta}{\textcolor{magenta}{\textrm{tail ~ nohup.out}}}\\
结果如图\ref{Pt_surface-adsorption-nohup}所示。
\begin{figure}[h!]
\centering
\vskip -10pt
\includegraphics[height=0.75in,viewport=0 0 940 127,clip]{Pt_surface-adsorption-nohup.png}
%\includegraphics[height=1.8in,width=4.in,viewport=30 210 570 440,clip]{PAW_projector.eps}
\caption{\small \textrm{\textrm{Pt~(111)}表面吸附\textrm{O}原子后的基态能量(部分).}}%(与文献\cite{EPJB33-47_2003}图1对比)
\label{Pt_surface-adsorption-nohup}
\end{figure}

查看弛豫后的\textrm{CONTCAR}文件,命令为:~\\
\textcolor{magenta}{\textrm{cat ~ CONTCAR}}\\
可以看出,吸附在\textrm{FCC}结构间隙位的\textrm{O}原子轻度弛豫后的原子位置为~(0.55555,~0.55555,~0.37598)。结果如图\ref{Pt_surface-adsorption-CONTCAR}所示:~
\begin{figure}[h!]
\centering
\vskip -10pt
\includegraphics[height=0.8in,viewport=0 0 650 95,clip]{Pt_surface-adsorption-CONTCAR.png}
%\includegraphics[height=1.8in,width=4.in,viewport=30 210 570 440,clip]{PAW_projector.eps}
\caption{\small \textrm{\textrm{Pt~(111)}表面吸附\textrm{O}原子弛豫后的\textrm{O}原子位置.}}%(与文献\cite{EPJB33-47_2003}图1对比)
\label{Pt_surface-adsorption-CONTCAR}
\end{figure}

为了计算吸附能,还需要计算孤立\textrm{O}原子能量,计算方法与\ref{Sec:atom-Pt}的\textrm{Pt}原子计算方法一样。计算完毕后,可以确定孤立\textrm{O}原子的基态能量。命令为:~\\
\textcolor{magenta}{\textrm{tail~-5~nohup.out}}\\
结果如图\ref{O_adsorption-OSZICAR}所示:
\begin{figure}[h!]
\centering
\vskip -10pt
\includegraphics[height=0.75in,viewport=0 0 940 120,clip]{O_adsorption-OSZICAR.png}
%\includegraphics[height=1.8in,width=4.in,viewport=30 210 570 440,clip]{PAW_projector.eps}
\caption{\small \textrm{孤立\textrm{O}原子的基态能量计算}.}%(与文献\cite{EPJB33-47_2003}图1对比)
\label{O_adsorption-OSZICAR}
\end{figure}

吸附能$E_{\mathrm{abs}}$可以通过薄层吸附\textrm{O}原子的基态能量和洁净的薄层与孤立原子能量差来计算:~
\begin{displaymath}
	\begin{aligned}
		E_{\mathrm{ads}}=&\dfrac1{N_{\mathrm{O}}^{\mathrm{atom}}}\big(E_{\mathrm{O/Pt~(111)}}^{\mathrm{slab}}-E_{\mathrm{Pt~(111)}}^{\mathrm{slab}}-N_{\mathrm{O}}^{\mathrm{atom}}E_{\mathrm{O}}^{\mathrm{atom}}\big)\\
		=&-267.2488-(-261.1438)-(-1.5514)=-4.55\mathrm{~eV}
	\end{aligned}
\end{displaymath}
需要说明的是,针对突出稳定吸附体系,习惯将吸附能只取其绝对值。本次计算的吸附能数值~--4.55\textrm{~eV}与\textrm{Pang}等的计算值~--4.68\textrm{~eV}\cite{ASS257-3047_2011}吻合得很好。原子在其他的吸附位,如\textrm{HCP}的间隙位、桥接位和紧邻缺陷位,或者别的物质可能的吸附,都可以用这样的方式计算。

很多时候“氧的吸附能”,也会通过自由的气态\ch{O2}分子的吸附给出。在这种情况下,需要考虑到\ch{O2}成键能减半,因此计算式中的吸附能会减小至~--1.42\textrm{~eV},也就是~--3.13\textrm{~eV/O}\cite{Eectrochim52-2219_2007}。

\subsection{功函数的计算与偶极校正}
该练习是利用前面的计算结果来计算校正偶极后的功函数。功函数的定义是将一个电子从固体表面的\textrm{Fermi}能级移到无穷远处(真空中)所做的功。而垂直于表面$z$-轴方向的偶极,$U_{\mathrm{dipole}}$,定义为:~
\begin{displaymath}
	U_{\mathrm{dipole}}(z)=\dfrac1A\iint U_{\mathrm{dipole}}(\vec r)\mathrm{d}x\mathrm{d}y
\end{displaymath}
因此,经偶极校正后的功函数为$-E_{\mathrm{F}}+U_{\mathrm{dipole}}$,它和表面的几何结构和特性密切相关。
\subsubsection\rm{{功函数的计算}}
为了计算功函数,\textrm{INCAR}文件中多五个选项,诸如规定模型中薄层中心的坐标位置。内容如图\ref{Pt_surface-workfunction-INCAR}所示:~
\begin{figure}[h!]
\centering
\includegraphics[height=1.1in,viewport=0 0 680 150,clip]{Pt_surface-workfunction-INCAR.png}
%\includegraphics[height=1.8in,width=4.in,viewport=30 210 570 440,clip]{PAW_projector.eps}
\caption{\small \textrm{计算\textrm{Pt}表面功函数时\textrm{INCAR}文件增加的内容.}}%(与文献\cite{EPJB33-47_2003}图1对比)
\label{Pt_surface-workfunction-INCAR}
\end{figure}

\textrm{KPOINTS}文件、\textrm{POSCAR}和\textrm{POTCAR}与前述表面吸附算例\textrm{Pt~(111)-O}中的相同。

\subsubsection{\rm{计算结果}}
首先用命令查找\textrm{OUTCAR}中的\textrm{Fermi}能级的数值:~\\
\textcolor{magenta}{\textrm{grep ~ fermi ~ OUTCAR}}\\
结果如图\ref{Pt_surface-workfunction-LOCPOT}所示。
\begin{figure}[h!]
\centering
\vskip -15pt
\includegraphics[height=0.3in,viewport=0 20 460 50,clip]{Pt_surface-workfunction-Fermi.png}
%\includegraphics[height=1.8in,width=4.in,viewport=30 210 570 440,clip]{PAW_projector.eps}
\caption{\small \textrm{计算\textrm{Pt}表面功函数时\textrm{Fermi}能级的数值.}}%(与文献\cite{EPJB33-47_2003}图1对比)
\label{Pt_surface-workfunction-LOCPOT}
\end{figure}

计算中产生新的输出文件\textrm{LOCPOT}与\textrm{CHGCAR}文件的保存格式相同,存储的内容是三维空间中的表面平均化的局域静电势\textrm{(the planar-averaged local-electrostatic potential)},但不包含交换-相关势。
\begin{figure}[h!]
\centering
\includegraphics[height=2.1in,viewport=0 10 480 250,clip]{Pt_surface-workfunction-LOCPOT.png}
%\includegraphics[height=1.8in,width=4.in,viewport=30 210 570 440,clip]{PAW_projector.eps}
\caption{\small \textrm{计算\textrm{Pt}表面功函数时\textrm{LOCPOT}文件的内容.}}%(与文献\cite{EPJB33-47_2003}图1对比)
\label{Pt_surface-workfunction-LOCPOT}
\end{figure}

为了确定$z$-方向的势函数的数值分布,可以编写脚本处理\textrm{LOCPOT}的数据,如本讲义附上的\textrm{Python}脚本\textcolor{blue}{\textrm{locpot\_workfunc.py}}\footnote{该软件由韩国汉城大学\textrm{Jinwoo,Park,Lanhee Yang}开发\cite{Park-Yang}},可以将\textrm{LOCPOT}的数值转成$\ast\mathrm{.dat}$格式的文件,运行命令为:\\
\textcolor{blue}{\textrm{python ~ locpot.py ~ LOCPOT ~ --0.3174}}\\

按照一般的习惯,\textrm{Fermi}能级~--0.3174~将会被设为0,脚本执行后的输出文件\textrm{output.dat}将被用于绘制功函数图(如图\ref{Pt_surface-workfunction}所示)。功函数图表明,\textrm{Pt~(111)}层沿$z$-轴方向的相对位置,\textrm{Fermi}能级位置,功函数为\textrm{5.8}\textrm{~eV}。
\begin{figure}[h!]
\centering
\includegraphics[height=3.1in,viewport=0 0 760 550,clip]{Pt_surface-workfunction.png}
%\includegraphics[height=1.8in,width=4.in,viewport=30 210 570 440,clip]{PAW_projector.eps}
\caption{\small \textrm{计算\textrm{Pt~(111)}表面吸附\textrm{O}原子后的势函数和功函数,\textrm{Fermi}能已经置零.}}%(与文献\cite{EPJB33-47_2003}图1对比)
\label{Pt_surface-workfunction}
\end{figure}

\section{NEB方法与反应过渡态搜索}
日常经验告诉我们,两地间行走时,通常会有多条路径可供选择。如果时间有限,为及时赶到,一般会选择最短的路径;~但如果时间充裕,又想体验行走的乐趣,就可以选择更有风味的路线,如图\ref{Pt_NEB-move}所示。原子(或其物质)间因相互作用而运动时,在力驱使下,则一定会沿着能量最低的路径行走,就像俗话说的“水往低处流”。
\begin{figure}[h!]
\centering
\includegraphics[height=2.8in,viewport=0 0 630 420,clip]{Pt_NEB-move.png}
%\includegraphics[height=1.8in,width=4.in,viewport=30 210 570 440,clip]{PAW_projector.eps}
\caption{\small \textrm{两点之间人和原子受力运动的示意图.}}%(与文献\cite{EPJB33-47_2003}图1对比)
\label{Pt_NEB-move}
\end{figure}

材料学研究中的一个基本问题是,体系如何从一个稳定态变化到另一个稳定态,也就是确定变化动力学过程的最小能量路径\textrm{(the minimum energy path, MEP)}。本节练习中,我们学习微动弹性带\textrm{(Nudged elastic band, NEB)~}方法\cite{JCP113-9978_2000,JCP113-9901_2000,JCC32-1769_2011}来确定最小能量路径的计算过程,并确定相应的反应势垒。
\subsection{\rm{NEB}的基本原理}
如图\ref{Pt_NEB}所示的势能面上,始态\textrm{(the initial state)}和终态\textrm{(the finial state)}分别对应势能面上的两个局域极小值。有两个典型的连线可以将这两个态关联起来:~一个是直的点线连接,另一个是最小能量曲线。由箭头所示,我们将从直线出发,学习如何尽可能快速高效地找到最小能量曲线。
\subsection{NEB方法的计算过程}
\subsubsection{\textrm{始和终态}}
首先,一般在电子和离子弛豫的时候,都需要确定始态和终态。在化学反应过程中,这两种构型的能量都应该是位于极小位置,各原子受力(即能量的一阶导数)也都是零。
\subsubsection{\rm{初始能量路线}}
既然最小能量路径是始态向终态变化的某一条能量路线(化学上习惯称反应通道),我们首先用直线能量路径把两个态连接起来,以此作为尝试的能量路线,并在能量路线上等距离地选择一些点(作为假想态,\textrm{images}),如图\ref{Pt_NEB}所示。注意每个假想态表示的是反应过程中的一个特定的中间构型。所需假想态的数目则依赖于能量路线的复杂性,即能量路径曲率。对于绝大部分材料,三到七个假想态已经足够。这种始态和终态连线上的假想态,可能与最小能量路径中原子构型偏差很大。显然,这种构型中的原子将受到极大的外力,计算模拟中的收敛也会很困难,常常需要一些手工的调整,使得假想态接近最小能量路径中的原子构型。
\subsubsection{\rm{推拉能量路径}}
要求初始能量路线(直线)中的各原子构型一点点地朝原子受力为零的构型方向轻微移动(类似于对一个弹性带子施以推力或拉力),就有可能搜索到最小能量路径。为了很好地控制能量路线的移动,假设对能量路线上的等间距分布的假想态施加沿特定方向的弹性应力,确保能量路线连续地朝最小能量路径方向过渡。换句话说,反应通道的搜索过程,就是反应势能面上各原子构型落到局域极小处的过程,即每个假想态都达到最小受力构型,如图\ref{Pt_NEB}所示。这些假想态的关系,类似于穿越沙丘的骆驼队:~每头骆驼都用缰绳彼此串在一起,保证每头骆驼都不会脱离驼队。
\begin{figure}[h!]
\centering
\includegraphics[height=3.5in,viewport=0 0 770 530,clip]{Pt_NEB.png}
%\includegraphics[height=1.8in,width=4.in,viewport=30 210 570 440,clip]{PAW_projector.eps}
\caption{\small \textrm{NEB}方法的基本原理:~起始时将三个假想态用直线能量路径(虚线)串联,然后向最小能带路径(实线)微动,该最小能量路径将通过各势垒的鞍点.}%(与文献\cite{EPJB33-47_2003}图1对比)
\label{Pt_NEB}
\end{figure}
\subsubsection{\rm{受力计算}}
\textrm{NEB}方法采用受力投影(不是能量投影),目的是确保假想态的连线能通过鞍点\textrm{(the saddle points)},达到最小能量路线。初始能量连线上的每个假想态的原子受力移动的力是弹性回复力和原子间力的合力。为了将初始能量路线直接推向最小能量路线,只考虑沿能量路线(该点切线)的弹性回复力投影和垂直能量路线(法线)的原子间作用力投影,而忽略其余力的贡献,因此直线能量路线将会向最小能量路线移动。一般通过\textrm{DFT}计算得到每个微动过程中的原子构型的电子态极小值,并计算每个原子上的受力以确定减小受力的微动方向。重复上述过程,直到每个原子上的受力都小于预设的接近零的小值,得到的能量路线就认为是最小能量路线。由于上述微动是根据受力推测的,因此在计算过程中,原子构型的结构弛豫算法采用\textrm{damped MD}算法(\textit{IBRION}=3)。
\subsubsection{\rm{CI-NEB}方法}
\textrm{CI-NEB}方法是为了确保有一个假想态正好位于鞍点而提出来的。为了让最高能量的假想态爬上鞍点,经过几次沿能量路线切线的微动后,其真实受力将被反向(例如在原来受力方向的反向加两倍的力),于是,出现爬坡的假想态,其能量将会对应沿能量路线上的极大值,同时又是垂直能量路线方向的极小值(即鞍点)。
\subsection{Pt~(111)-O-NEB}
以下算例说明如何用\textrm{NEB}方法确定最小能量路径并计算\textrm{Pt~(111)}面上吸附的\textrm{O}原子由\textrm{HCP}间隙位(如图\ref{Pt_NEB-init-POSCAR}所示)扩散到近邻的\textrm{FCC}间隙位的表面扩散势垒能量。结果如图\ref{Pt_NEB-config}所示。
\begin{figure}[h!]
\centering
\includegraphics[height=2.8in,viewport=0 10 1040 530,clip]{Pt_NEB-config.png}
%\includegraphics[height=1.8in,width=4.in,viewport=30 210 570 440,clip]{PAW_projector.eps}
\caption{\small \textrm{\textrm{O}原子(浅灰色球)在\textrm{Pt~(111)}的\textrm{HCP}间隙位(a)经过中间态(b)扩散到最近临的\textrm{FCC}间隙位(c)时的原子构型变化}.}%(与 文献\cite{EPJB33-47_2003}图1对比)
\label{Pt_NEB-config}
\end{figure} 

\subsubsection{\rm{Pt~(111)-slab-O-HCP}}
采用\textrm{NEB}方法计算,首先要确定始态和终态构型的极小值。在\ref{Sec:Surface-Pt}节我们已经确定表面吸附终态\textrm{Pt~(111)-slab-O-FCC}的构型。这里,我们将用相同的方法确定始态\textrm{Pt~(111)-slab-O-HCP}的构型及能量,只是这里的\textrm{POSCAR}文件中要把\textrm{O}原子放在\textrm{HCP}的间隙位置。如图\ref{Pt_NEB-init-POSCAR}所示。
\begin{figure}[h!]
\centering
\includegraphics[height=4.1in,viewport=0 5 470 500,clip]{Pt_NEB-init-POSCAR.png}
%\includegraphics[height=1.8in,width=4.in,viewport=30 210 570 440,clip]{PAW_projector.eps}
\caption{\small \textrm{\textrm{Pt~(111)}表面\textrm{HCP}间隙位吸附\textrm{O}原子时的\textrm{POSCAR}结构文件}.}%(与文献\cite{EPJB33-47_2003}图1对比)
\label{Pt_NEB-init-POSCAR}
\end{figure}

计算结果表明,收敛的基态能量是~--266.87\textrm{~eV/system}和\textrm{NEB}计算需要的输出文件,如\textrm{CONTCAR}和\textrm{OUTCAR},可以确定始态能量极小值时的构型和原子位置。
\subsubsection{\rm{用VTST脚本运行NEB}}
\textrm{VTST}脚本\footnote{\textrm{VTST:~\url{http://theory.cm.utexas.edu/vtsttools/}}}是一个开源软件,提供有\textrm{VASP}的接口,可以方便地支持\textrm{VASP}开展\textrm{NEB}计算。下载\textrm{VTST}后,将源文件解压到\textrm{VASP}的源代码目录下,并用\textrm{make}命令编译全部$\ast.\mathrm{F}$文件(包括\textrm{neb.F},\textrm{dynamt.F}等)得到新的可执行\textrm{NEB}版\textrm{VASP}。为了实现\textrm{NEB}计算,在\textrm{INCAR}文件中,除了修改必要的离子弛豫参数\textit{IBRON}=3外,额外增加3个参数\textit{IMAGES}=1,\textit{SPRING}=-5.0,\textit{LCLIMB}=\textrm{.TRUE.}。如图\ref{Pt_NEB-INCAR}所示。
\begin{figure}[h!]
\centering
\includegraphics[height=1.8in,viewport=0 0 630 200,clip]{Pt_NEB-INCAR.png}
%\includegraphics[height=1.8in,width=4.in,viewport=30 210 570 440,clip]{PAW_projector.eps}
\caption{\small \textrm{VASP}计算\textrm{NEB}计算时需要新增的\textrm{INCAR}选项.}%(与文献\cite{EPJB33-47_2003}图1对比)
\label{Pt_NEB-INCAR}
\end{figure}

注意假想态的数目要与\textrm{CPU}的数目匹配,因为每个假想态要均衡地分配到\textrm{CPU}上去计算,以\textrm{CPU}数为8为例,假想态数目可选为1,2,4或8。脚本会在计算目录下新建目录,并将前述两类计算得到的\textrm{CONTCAR}文件复制到该目录下,并分别重命名为\textrm{CONTCAR1}(始态,\textrm{O}位于\textrm{HCP}间隙位)和\textrm{CONTCAR}(终态,\textrm{O}位于\textrm{FCC}间隙位)。由于本算例是简单的直接扩散,所以可以只用一个假想态,\textrm{VTST}将采用线性插值为中间假想态生成\textrm{POSCAR}文件。命令为:~\\
\textcolor{magenta}{\textrm{$\sim$/vtstscripts/nebmake.pl~\textrm{CONTCAR1}~\textrm{CONTCAR2}~1}}\\
\textrm{VTST}脚本会产生三个目录,分别为00(始态),01(中间假想态),02(终态)。在\textrm{POSCAR}文件中,原子位置的坐标将都会用正值表示,如$-0.001\rightarrow0.999$。其中目录00中的\textrm{OUTCAR}直接从\textrm{Pt~(111)-slab-O-HCP}中复制,类似地,目录02中的\textrm{OUTCAR}来自\textrm{Pt~(111)-slab-O-FCC}。这几个目录中的\textrm{OUTCAR}中最后得到的基态总能\textit{E0}将是计算扩散势垒的根据。计算中的\textrm{POTCAR}和\textrm{KPOINTS}也都完全相同。计算过程中,可以通过命令监控计算过程中原子受力变化的情况:\\
\textcolor{magenta}{\textrm{grep~\`~max~atom~\'~OUTCAR}}\\

%这里第一个数是当前假想态中受力最大的原子,
当每个假想态上的原子受力低于力标准(本算例为$<0.05\mathrm{eV/\AA}$),即达到收敛。如果起始受力太大($>10\mathrm{eV/\AA}$),就要调整为更合适的起始结构重新来计算。

计算过程可以用脚本\textrm{vtstscript/nebef.pl}来监控,监控信息将会写到文件$\mathbf{neb}\ast\mathbf{.dat}$中。例如,经过19次和30次的离子弛豫后的数据如图\ref{Pt_NEB-VTST-1}所示:
\begin{figure}[h!]
\centering
\includegraphics[height=2.2in,viewport=0 20 300 200,clip]{Pt_NEB-VTST-1.png}
%\includegraphics[height=1.8in,width=4.in,viewport=30 210 570 440,clip]{PAW_projector.eps}
\caption{\small \textrm{经过19次和30次离子弛豫后的数据.}}%(与文献\cite{EPJB33-47_2003}图1对比)
\label{Pt_NEB-VTST-1}
\end{figure}

这里所有数据按假想态的顺序排列,分别是力,基态总能以及基态总能与起始态的能量差。所有的数据文件(如\textrm{OUTCAR}、\textrm{WAVECAR}、\textrm{CHGCAR}等)都写到01目录内。一般一个\textrm{VTST}计算得到稳定的鞍点假想态,需要比较长的时间,因为这个假想态是亚稳态,实际上非常不稳定。
\subsubsection{计算结果}
脚本\textrm{nebbarrier.pl}输出的是反应坐标,即以能量路线上各假想态为横坐标,计算各假想态基态总能的能量差和每个假想态的原子的最大受力。 而脚本\textrm{nebresults.pl}执行后则输出一系列的后处理数据文件。
\begin{figure}[h!]
\centering
\includegraphics[height=1.0in,viewport=0 0 300 80,clip]{Pt_NEB-VTST-2.png}
%\includegraphics[height=1.8in,width=4.in,viewport=30 210 570 440,clip]{PAW_projector.eps}
\caption{\small \textrm{NEB}计算的反应坐标数据.}%(与文献\cite{EPJB33-47_2003}图1对比)
\label{Pt_NEB-VTST-2}
\end{figure}

图\ref{Pt_NEB-energydiff}表明原子从\textrm{Pt~(111)}的\textrm{HCP}的间隙位移动到\textrm{FCC}间隙位时,原子构型对相应基态总能差的曲线(即反应通道)。该反应路径中正向扩散势垒为0.155\textrm{eV},反向扩散,即从\textrm{FCC}间隙位扩散到\textrm{HCP}间隙位,其势垒则为0.53\textrm{eV},这和\textrm{Pang}等的计算值(0.5\textrm{eV})\cite{ASS257-3047_2011}吻合得很好,与实验值(0.43\textrm{eV})\cite{PRL77-123_1996}也比较一致。计算结果表明,中等强度的热扰动就可以将吸附在\textrm{Pt~(111)}面上\textrm{HCP}间隙位的\textrm{O}原子扩散到\textrm{FCC}间隙位上。本次练习的计算流程可以推广到双金属薄层\cite{PRL81-2819_1998}中的芯-壳纳米团簇体系的应力结构研究等相关理论计算中去。
\begin{figure}[h!]
\centering
\includegraphics[height=3.5in,viewport=0 0 790 640,clip]{Pt_NEB-energydiff.png}
%\includegraphics[height=1.8in,width=4.in,viewport=30 210 570 440,clip]{PAW_projector.eps}
\caption{\small \textrm{CI-NEB}方法计算的反应势垒.}%(与文献\cite{EPJB33-47_2003}图1对比)
\label{Pt_NEB-energydiff}
\end{figure}

\section{Pt~(111)表面催化的计算}
\subsection{催化剂}
催化剂是能够改变化学反应进程(一般是加速反应完成,也有少数催化剂是减缓反应过程),但在反应前、后不改变物质组分的物质。如图\ref{Pt_NEB-reaction}所示,对于一个给定化学反应,由于催化剂的存在,降低了活化能,为反应过程提供了新的反应通道,但反应的自由能($\Delta G$)不变,也就是说,催化剂的存在,虽然改变了化学反应的动力学学过程,但并不改变化学反应的热力学过程。图示的催化剂的概念很简单,但是在实际的反应过程,会涉及大量的化学结构和电子态的变化,故此一般催化反应过程都相当复杂。
\begin{figure}[h!]
\centering
\includegraphics[height=3.5in,viewport=0 0 850 660,clip]{Pt_NEB-reaction.png}
%\includegraphics[height=1.8in,width=4.in,viewport=30 210 570 440,clip]{PAW_projector.eps}
\caption{\small 催化剂的存在对反应通道的能量影响(反应通道)的示意图.}%(与文献\cite{EPJB33-47_2003}图1对比)
\label{Pt_NEB-reaction}
\end{figure} 

不难理解,催化剂表面与吸附介质相互作用的细节对于深入认知催化反应过程非常关键。\textrm{DFT}方法已经成为寻找更优异的催化材料的高效手段,并且为解释实验结果,并能提供实验难以或无法提供的细节。本节中通过对态密度\textrm{(Density of States, DOS)}的研究来表征催化表面的电子结构的改变。
\subsection{态密度}
对周期体系,在每个$\vec k$点上都要通过自洽迭代来求解\textrm{Kohn-Sham}方程,得到\textrm{Kohn-Sham}本征态轨道和能量本征值(一般称为单电子本征态)。电子结构的表示有两种重要的方式,也就是能带结构和\textrm{DOS}。一般能带结构都是沿着不可约\textrm{Brillouin}区中的高对称性线方向绘制。而\textrm{DOS}则表示的是整个\textrm{Brillouin}区的电子态的数目。

\textrm{DOS}反映的是电子态在$\vec k$空间的分布状况,定义为通过单位能量范围内的电子态的数目:
\begin{equation}
	\begin{aligned}
		D(\varepsilon)=&\dfrac{\mbox{能量}\varepsilon\mbox{和}\partial\varepsilon\mbox{之间的态的数目}}{\partial\varepsilon}\\
		=&2\sum_{\vec k}\delta[\varepsilon-\varepsilon_n(\vec k)]
	\end{aligned}
	\label{eq:DOS}
\end{equation}
在特定能级附近的高态密度意味着在这个能量附近有很多占据态。如果对态密度积分,积分到\textrm{Fermi}能级,可以得到体系的电子数:~
\begin{equation}
	n=\int_0^{E_{\mathrm{F}}}D(\varepsilon)\mathrm{d}\varepsilon
	\label{eq:DOS-integral}
\end{equation}

有时我们需要将\textrm{DOS}中的部分投影出来,称为投影态密度(或分波态密度)。态密度的投影可以分为针对轨道、原子、元素或薄层的层等等多种投影。在实验技术中,固体的态密度一般通过光电子发射谱测量。
\subsection{Pt~(111)-slab-O-DOS计算}
\subsubsection{\rm{静态计算}}
一旦能量弛豫结束,开始静态计算时,将在\textrm{INCAR}中设参数\textit{NSW}=0,并将\textrm{CONTCAR}和\textrm{CHGCAR}作为静态自洽的结构和初始电荷密度的起点。此外,要把\textrm{KPOINTS}中的$\vec k$点增加为$6\times6\times1$,以保证计算计算精度。
\subsubsection{\rm{DOS}计算}
静态计算之后是根据当前得到的电荷密度\textrm{CHGCAR}进行非自洽计算(物性计算),要将\textrm{INCAR}中设置参数\textit{ICHARG}=11。如图\ref{Pt_surface-DOS-INCAR}所示。
\begin{figure}[h!]
\centering
\includegraphics[height=0.9in,viewport=0 0 500 120,clip]{Pt_surface-DOS-INCAR.png}
%\includegraphics[height=1.8in,width=4.in,viewport=30 210 570 440,clip]{PAW_projector.eps}
\caption{\small \textrm{静态计算最主要的任务之一:~\textrm{DOS}计算时的\textrm{INCAR}设置.}}%(与文献\cite{EPJB33-47_2003}图1对比)
\label{Pt_surface-DOS-INCAR}
\end{figure}

这样的参数设置会在计算过程中将保持基态电荷密度和势不变,最后得到每个$\vec k$点上的电子本征态。
\subsubsection{\rm{计算结果}}
\textrm{DOSCAR}文件给出了体系的态密度和积分态密度\textrm{(the integrated DOS)}(单位:~态数目/\textrm{unit cell}),如图\ref{Pt_PDOS-d}所示。而\textrm{PROCAR}文件给出的是每个能带的轨道按轨道分量~\textit{s}-~/~\textit{p}-~/~\textit{d}-~/~\textit{f}-~投影的情况。
\begin{figure}[h!]
\centering
\includegraphics[height=3.5in,viewport=0 20 830 620,clip]{Pt_PDOS-d.png}
%\includegraphics[height=1.8in,width=4.in,viewport=30 210 570 440,clip]{PAW_projector.eps}
\caption{\small \textrm{\textrm{Pt(111)中的\textrm{Pt}的\textit{d}轨道对\textrm{DOS}的贡献}.}}%(与文献\cite{EPJB33-47_2003}图1对比)
\label{Pt_PDOS-d}
\end{figure}

我们还要将计算文件中的数据经人工提取和必要处理才能进行分析,当然这可以借助脚本完成(比如这里用的就是\textcolor{blue}{\textrm{dos.py}})。处理后得到一列能量数据和对应的每个轨道的分解占据情况。在本算例中,对\textrm{DOS}的投影,就是两个成键原子的贡献,即\textrm{Pt}原子和\textrm{O}原子。


根据数据文件绘制\textrm{DOS}图,还需要知道体系的\textrm{Fermi}能级。命令如下:\\
\textcolor{magenta}{\textrm{grep~fermi~OUTCAR}}\\
将上述分波\textrm{DOS}的数据文件\textrm{pdos-Pt-O-46O.dat}导入\textrm{Xmgrace},\textrm{MS}~\textrm{Excel}或\textrm{ORIGIN}等数据绘图软件中,第一列指定为绘图时的横坐标,并以\textrm{Fermi}能级为能量原点,可以得到效果如图\ref{Pt_surface_PDOS}所示的态密度图。此外还可以画出净的\textrm{Pt}薄层的分态密度。这里将总的态密度投影到各原子上表示为分态密度,并给出\textrm{Pt}的\textit{d}-\textrm{DOS}和\textrm{O}的\textit{p}-DOS。根据该态密度,可以定性地估计两种原子的成键的键强度:~例如,可以明显地看出\textrm{Pt-O}成键主要来自\textrm{Pt}-5\textit{d}和\textrm{O}-2\textit{p}轨道的杂化。
\begin{figure}[h!]
\centering
\includegraphics[height=3.5in,viewport=0 0 820 620,clip]{Pt_surface_PDOS.png}
%\includegraphics[height=1.8in,width=4.in,viewport=30 210 570 440,clip]{PAW_projector.eps}
\caption{\small \textrm{\textrm{Pt(111)-O}体系的\textrm{PDOS}贡献:~来自\textrm{Pt}-5\textit{d}和\textrm{O}-2\textit{p}轨道.}}%(与文献\cite{EPJB33-47_2003}图1对比)
\label{Pt_surface_PDOS}
\end{figure}

\section{Si的能带结构}\label{Sec:Si-band}
在引入能带理论的时候,已知自由电子的能量是$\vec k$的二次函数(开口向上的抛物线),能量极小值位于$\vec k=0$~($\Gamma$点),考虑到平移周期性,可以将能量函数平移到第一\textrm{Brillouin}区表示;~实际的周期体系由于周期性势函数的存在,能量在倒空间中的分布存在带隙\textrm{(band gap)},也不一定关于$\Gamma$点对称,这是和自由电子的最大区别。因为周期势场内的电子既不是完全局域也不是完全的自由的。

本次练习,将学习计算半导体\textrm{Si}的能带结构,我们将看到单个的\textrm{Si}原子堆积成固体\textrm{Si}时,原子轨道如何构成能带。该练习也将说明\textrm{Kohn-Sham}方程的能量$\varepsilon_{\vec k}^n$的表示,除了能带记号$n$,还必须包括波矢量子数$\vec k$。
\subsection{静态计算}
首先根据\ref{Sec:bulk-Pt}的结构弛豫方法计算确定金刚石结构的\textrm{Si}晶胞为$a_0=5.47\mathrm{\AA}$。为了得到\textrm{Si}的电子结构,必须要完成两组连续计算:~即通过静态计算得到迭代收敛的基态电子密度,随后用获得的电子密度,按特定的对称性方向,直接计算(无需自洽迭代)得到能带。静态计算是获得电子基态信息的重要步骤,在结构弛豫后完成,也是材料电子学性质计算的起点。静态计算时,\textrm{INCAR}中设置参数\textit{ENCUT}=250;~\textrm{KPOINTS}的$\vec k$点数目设为$7\times7\times7$。金刚石结构的\textrm{Si}晶胞可以视为两套\textrm{FCC}结构错开($a$/4,$a$/4,$a$/4)排列,这里$a$即立方晶胞参数。所以\textrm{POSCAR}文件可以写成包含两个\textrm{Si}原子的\textrm{FCC}初基原胞\textrm{(primitive unit cell)}:如图\ref{Si_POSCAR}所示。
\begin{figure}[h!]
\centering
\includegraphics[height=1.3in,viewport=0 0 370 150,clip]{Si_POSCAR.png}
%\includegraphics[height=1.8in,width=4.in,viewport=30 210 570 440,clip]{PAW_projector.eps}
\caption{\small \textrm{Si}的初基原胞的\textrm{POSCAR}文件.}%(与文献\cite{EPJB33-47_2003}图1对比)
\label{Si_POSCAR}
\end{figure}

计算需要的\textrm{Si}原子的\textrm{POTCAR}文件可以从\textrm{VASP}的赝势库中知道并保存到当前目录,内容如图\ref{Si_POTCAR}。
\begin{figure}[h!]
\centering
\includegraphics[height=1.5in,viewport=0 0 400 200,clip]{Si_POTCAR.png}
%\includegraphics[height=1.8in,width=4.in,viewport=30 210 570 440,clip]{PAW_projector.eps}
\caption{\small \textrm{Si}的\textrm{POTCAR}.文件(部分)}%(与文献\cite{EPJB33-47_2003}图1对比)
\label{Si_POTCAR}
\end{figure}

静态计算完毕,可以从文件\textrm{OSZICAR}得到体系的基态能量,电荷密度则保存到文件\textrm{CHGCAR}中。
\subsection{Si的能带结构计算}
为了完成能带计算,主要是\textrm{CHGCAR}文件,其余文件改动极少。
\subsubsection{\rm{INCAR}}
能带计算无需自洽迭代,\textrm{INCAR}文件中参数设置如图\ref{Si_Band-INCAR}所示。这里设置\textit{ICHARG}=11,要求电荷密度来自此前计算的\textrm{CHGCAR},并在计算过程中保持不变。
\begin{figure}[h!] 
\centering
\includegraphics[height=0.8in,viewport=0 0 330 80,clip]{Si_Band-INCAR.png}
%\includegraphics[height=1.8in,width=4.in,viewport=30 210 570 440,clip]{PAW_projector.eps}
\caption{\small \textrm{Si}的能带计算中的\textrm{INCAR}设置.}%(与文献\cite{EPJB33-47_2003}图1对比)
\label{Si_Band-INCAR}
\end{figure}
\subsubsection{\rm{KPOINTS}}
一般地,波矢$\vec k $的取值是三维空间(第一\textrm{Brillouin}区)中的矢量。在能带表示中的波矢$\vec k$选择为沿不可约第一\textrm{Brillouin}区的高对称性点连线。表\ref{Tabble-kpath}给出的是\textrm{FCC}结构的第一\textrm{Brillouin}区的高对称性$\vec k$点列表。

\begin{table}[!h]
\tabcolsep 0pt \vspace*{-5pt}
\begin{minipage}{0.88\textwidth}
%\begin{center}
\centering
\caption{\textrm{FCC}的第一\textrm{Brillouin}区中高对称性点的列表}\label{Table-kpath}
\def\temptablewidth{0.85\textwidth}
\renewcommand\arraystretch{0.8} %表格宽度控制(普通表格宽度的两倍)
\rule{\temptablewidth}{1pt}
\begin{tabular*} {\temptablewidth}{@{\extracolsep{\fill}}c@{\extracolsep{\fill}}c@{\extracolsep{\fill}}c}
%-------------------------------------------------------------------------------------------------------------------------
	&\textrm{Reciprocal coordinates} &\textrm{Cartesian coordinates}\\
	\textrm{Points}	&\textrm{(unit of $b_1,b_2,b_3$)} &\textrm{unit of $2\pi/a$} \\\hline
	$\Gamma$ &0~~0~~0 &0~~0~~0 \\
	X &1/2~~0~~1/2 &0~~1~~0 \\
	W &1/2~~1/4~~3/4 &1/2~~1~~0 \\
	L &1/2~~1/2~~1/2 &1/2~~1/2~~1/2 \\
	$\Delta$ &1/4~~0~~1/4 &0~~1/2~~0 \\
	$\Lambda$ &1/ 4~~1/4~~1/4 &1/4~~1/4~~1/4 \\
\end{tabular*}
\rule{\temptablewidth}{1pt}
\end{minipage}
%\end{center}
\end{table}
因此,为了绘制平滑的能带曲线,需要在\textrm{KPOINTS}文件中指定足够多的$\vec k$点,能带计算时,只需针对这些指定的$\vec k$点计算各轨道的能量本征值。%如图\ref{Si_KPOINTS}所示:
\begin{figure}[h!]
\centering 
\includegraphics[height=2.0in,viewport=0 0 540 210,clip]{Si_KPOINTS.png}
%\includegraphics[height=1.8in,width=4.in,viewport=30 210 570 440,clip]{PAW_projector.eps}
\caption{\small \textrm{Si}的能带计算时\textrm{KPOINTS}的设置.}%(与文献\cite{EPJB33-47_2003}图1对比)
\label{Si_KPOINTS}
\end{figure}
图\ref{Si_KPOINTS}给出的是\textrm{VASP}计算沿$\vec k$点路线$\mathrm{W}-\mathrm{L}-\Gamma-\mathrm{X}-\mathrm{W}$共计80个$\vec k$点(20点/连线$\times$4组连线)的\textrm{KPOINTS}文件,对应\textrm{Brillouin}区的$\vec k$点连线如图所示\ref{VASP_train-FCC-BZ}。
\begin{figure}[h!]
\centering 
\includegraphics[height=4.1in,viewport=0 0 680 680,clip]{VASP_train-FCC-BZ.png}
%\includegraphics[height=1.8in,width=4.in,viewport=30 210 570 440,clip]{PAW_projector.eps}
\caption{\small \textrm{Brillouin}区为\textrm{FCC}时的空间结构和$\vec k$-点路径关系.}%(与文献\cite{EPJB33-47_2003}图1对比)
\label{VASP_train-FCC-BZ}
\end{figure}

\subsubsection{\rm{能带中的本征值}}
因为能带计算是非自洽迭代计算,计算出来能量本征值$\varepsilon_{\vec k}^n$写入\textrm{EIGENVAL}文件中,用于绘制能带图。如图\ref{Si_Band-EIGENVAL}所示
\begin{figure}[h!]
\centering
\includegraphics[height=2.8in,viewport=0 0 640 375,clip]{Si_Band-EIGENVAL.png}
%\includegraphics[height=1.8in,width=4.in,viewport=30 210 570 440,clip]{PAW_projector.eps}
\caption{\small \textrm{VASP}计算\textrm{Si}的能量本征值文件\textrm{EiGENVAL}(部分).}%(与文献\cite{EPJB33-47_2003}图1对比)
\label{Si_Band-EIGENVAL}
\end{figure}

文件\textrm{EIGENVAL}中第一行是倒晶格中的$\vec k$点位置,随后8行是对应的8个能带的能量本征值:~其中前4个(\textrm{No.}~1-4)能级表示价带,是0\textrm{K}下的占据态。从\textrm{EIGENVAL}文件中提取出各$\vec k$点的能量本征值$\varepsilon_{\vec k}^n$,并将\textrm{Fermi}能级设置为0~(可以用本讲义提供的脚本\textcolor{blue}{\textrm{vasp\_to\_image.py}}实现)。运行该脚本,将\textrm{Fermi}能级(由\textrm{OUTCAR}可知\textrm{Fermi}能级为5.6980\textrm{eV})设为0~\textrm{eV},可以得到$\vec k$点和能量本征值关系的数据文件\textrm{Si-band.dat},内容如图\ref{Si_Band-data}所示

\begin{figure}[h!]
\centering
\includegraphics[height=1.2in,viewport=0 0 540 140,clip]{Si_Band-data.png}
%\includegraphics[height=1.8in,width=4.in,viewport=30 210 570 440,clip]{PAW_projector.eps}
\caption{\small \textrm{用于绘制\textrm{Si}能带结构的数据.}}%(与文献\cite{EPJB33-47_2003}图1对比)
\label{Si_Band-data}
\end{figure}

图\ref{Si_Band-DOS}给出绘制的能带图,可见按照当前的$\vec k$点路径选择,\textrm{Si}是间接带隙半导体,其带隙为$E_g=0.61\textrm{eV}$。这是$\Gamma$点和$X$点之间的带隙值,同时也是图中所有$\vec k$点之间最小的能量差。该值仅为实验测得的带隙的1/2,这是因为\textrm{DFT}本身存在先天不足,会低估能带的带隙。
\begin{figure}[h!]
\centering
\includegraphics[height=3.5in,viewport=0 10 920 610,clip]{Si_Band-DOS.png}
%\includegraphics[height=1.8in,width=4.in,viewport=30 210 570 440,clip]{PAW_projector.eps}
\caption{\small \textrm{Si}的能带和相应的态密度图,从能带图看出\textrm{Si}是间接带隙半导体.}%(与文献\cite{EPJB33-47_2003}图1对比)
\label{Si_Band-DOS}
\end{figure}

不过除了对带隙的低估,\textrm{DFT}计算的色散关系(能量在倒空间分布关系)是完全正确的,比如\textrm{Fermi}面附近有四个价带,四个导带,能量$\varepsilon_{\vec k}^n$随$\vec k$空间的变化逐渐改变,因此\textrm{DFT}理论的计算定性是完全正确的。实际上,其他物理量(比如波函数和电子密度),也和能量本征值有类似的倒空间分布关系。表明这样的$\vec k$点采样是合理的,能够表现物理量在不可约\textrm{Brillouin}区的定量变化。

\section{Si的声子计算}
声子计算除了\textrm{VASP}软件,还需要用到开源声子计算程序\textrm{phonopy}\cite{Phonopy}。声子是描述晶格振动的,通过波矢$\vec q$在固体中传播,其特征频率$\omega(\vec q)$是量子化的。晶格中的原子有序排列,类似于弹簧串联的小球,一个(或多个)原子偏离平衡位置引起的振动,通过“弹簧”传递,会引起一系列的原子产生类似运动,根据晶格动力学理论就产生了力常数\textrm{(force constants)}、动力学矩阵\textrm{(dynamic matrix)}等物理量。根据这些物理量,我们就可以绘制出声子在第一\textrm{Brillouin}区的频率分布$\omega(\vec q)$在$\vec k$空间的色散关系,对于具体的固体研究,声子色散关系非常重要。这里我们学习以\textrm{Si}为例计算简单的声子方式。实验技术中,一般通过中子散射确定固体中的声子分布。
\subsection{\rm{构建输入文件}}
根据\textrm{Si}的初基原胞,首先构造\textrm{POSCAR}文件如图\ref{Si_phonon-POSCAR_primitve}所示:~
\begin{figure}[h!]
\centering
\includegraphics[height=1.2in,viewport=0 0 520 160,clip]{Si_phonon-POSCAR_primtive.png}
%\includegraphics[height=1.8in,width=4.in,viewport=30 210 570 440,clip]{PAW_projector.eps}
\caption{\small \textrm{用于声子计算的\textrm{Si}的初基原胞的\textrm{POSCAR}文件.}}%(与文献\cite{EPJB33-47_2003}图1对比)
\label{Si_phonon-POSCAR_primitve}
\end{figure}

在此基础上,通过\textrm{phonopy}程序产生若干超晶胞(如$2\times2\times2$超晶胞),%命令如下:~\\
%\textcolor{blue}{\textrm{phonopy~-d~--dim=~\`\`~2~2~2~\'\'}}\\
%该命令
将产生三组文件,包括\textrm{SPOSCAR},\textrm{DISP}和\textrm{POSCAR}-\{\textrm{number}\}。其中\textrm{SPOSCAR}保存的是理想超晶胞结构,\textrm{DISP}记录的是各方向的原子偏离信息,而\textrm{POSCAR}-\{\textrm{number}\}是超晶胞中原子发生偏离后的位置。将初基原胞结构文件\textrm{POSCAR}重命名为\textrm{POSCAR1},而\textrm{SPOSCAR}重命名为\textrm{POSCAR}:~\\
\textcolor{magenta}{\textrm{mv~POSCAR~POSCAR1}}\\
\textcolor{magenta}{\textrm{mv~SPOSCAR~POSCAR}}

为了\textrm{VASP}计算,要求有如下的\textrm{INCAR}文件和\textrm{KPOINTS}文件,分别如图\ref{Si_phonon-INCAR},\ref{Si_phonon-KPOINTS}所示:~
\begin{figure}[h!]
\centering
\includegraphics[height=0.9in,viewport=0 0 640 110,clip]{Si_phonon-INCAR.png}
%\includegraphics[height=1.8in,width=4.in,viewport=30 210 570 440,clip]{PAW_projector.eps}
\caption{\small \textrm{用于声子计算的\textrm{Si}的输入参数控制\textrm{INCAR}文件.}}%(与文献\cite{EPJB33-47_2003}图1对比)
\label{Si_phonon-INCAR}
\end{figure}
\begin{figure}[h!]
\centering
\vskip -20pt
\includegraphics[height=0.9in,viewport=0 20 160 130,clip]{Si_phonon-KPOINTS.png}
%\includegraphics[height=1.8in,width=4.in,viewport=30 210 570 440,clip]{PAW_projector.eps}
\caption{\small \textrm{用于声子计算的\textrm{Si}的$\vec k$空间布点\textrm{KPOINTS}文件.}}%(与文献\cite{EPJB33-47_2003}图1对比)
\label{Si_phonon-KPOINTS}
\end{figure}
根据上述输入文件,可以开启一个标准的\textrm{VASP}静态计算。
\subsection{声子计算}
\textrm{VASP}计算结束,将会产生\textrm{DYNMAT}文件。用以下命令可由该文件提取出力常数数据:\\
\textcolor{blue}{\textrm{phonopy~--fc~vasprun.xml}}\\
将会生成\textrm{FORCE\_CONSTANTS}文件,下一步\textrm{phonopy}的输入文件\textrm{INPHON}文件中会涉及该文件。
\subsubsection{\rm{INPHON}}
\textrm{INPHON}内容如图\ref{Si_phonon-INPHON}所示:~
\begin{figure}[h!]
\centering
\includegraphics[height=1.0in,viewport=0 0 810 140,clip]{Si_phonon-INPHON.png}
%\includegraphics[height=1.8in,width=4.in,viewport=30 210 570 440,clip]{PAW_projector.eps}
\caption{\small \textrm{用于声子计算的\textrm{Si}的控制\textrm{INPHON}文件.}}%(与文献\cite{EPJB33-47_2003}图1对比)
\label{Si_phonon-INPHON}
\end{figure}

将\textrm{POSCAR}重命名为\textrm{POSCAR}$2\times2\times2$,\textrm{POSCAR1}重命名为\textrm{POSCAR},执行声子计算:~\\
\textcolor{blue}{\textrm{phonopy~-p}}\\
该计算会产生绘制声子色散关系图所需的文件\textrm{band.yaml},为了绘制声子色散图,执行以下命令:~\\
\textcolor{blue}{\textrm{bandplot~band.yaml}}\\
注意\textrm{bandplot}命令与\textrm{phonopy}在同一个执行目录\textrm{bin}下,运行后会产生弹出式窗口如图\ref{Fig:Si_pop-out_screen}所示。最后,\textrm{Si}在$2\times2\times2$超晶胞的声子色散分布如图\ref{Fig:Si_phonon}所示,为了表明计算收敛的必要性,$3\times3\times3$和$4\times4\times4$超晶胞的结果也和$2\times2\times2$的结果列在同一张图中,结果表现出了更低的声学支和更高的光学支振动。
\begin{figure}[h!]
\centering
\includegraphics[height=3.5in,viewport=0 0 1050 650,clip]{Si_pop-out_screen.png}
%\includegraphics[height=1.8in,width=4.in,viewport=30 210 570 440,clip]{PAW_projector.eps}
\caption{\small \textrm{运行\textrm{bandplot}时的弹出界面.}}%(与文献\cite{EPJB33-47_2003}图1对比)
\label{Fig:Si_pop-out_screen}
\end{figure}

\begin{figure}[h!]
\centering
\includegraphics[height=3.5in,viewport=0 0 780 600,clip]{Si_phonon}
%\includegraphics[height=1.8in,width=4.in,viewport=30 210 570 440,clip]{PAW_projector.eps}
\caption{\small 不同的超晶胞计算的\textrm{Si}的声子谱曲线.}%(与文献\cite{EPJB33-47_2003}图1对比)
\label{Fig:Si_phonon}
\end{figure}

有了这些声子计算的基本数据,就可以通过计算从理论上估计晶体的熵、自由能和热膨胀等热力学物性数据。当然,这些内容已经超出本讲义的范围,兹不赘言。

\section{其他计算}
关于\textrm{VASP}计算材料的\textrm{LDA}+\textit{U}~计算和光学介电函数、\textit{GW}校正的计算参数设置和相关内容,有兴趣的读者可以参阅:~\url{https://www.vasp.at/wiki/index.php/Category:Examples}.这些计算是在结构弛豫之后,完成静态\textrm{DFT}计算基础上的物理性质模拟。
\subsection{LDA+$U$计算}
计算反铁磁(\textrm{antiferromagnetic})的\ch{NiO}的\textrm{LSDA+$U$}。其中\textrm{POSCAR}文件和$\vec k$空间布点方案($4\times4\times4$网格点),如图\ref{NiO-LDA_U-Input}所示:~
\begin{figure}[h!]
\centering
\includegraphics[height=2.0in,viewport=0 5 210 260,clip]{NiO-LDA_U-POSCAR.png}
\includegraphics[height=0.8in,viewport=0 0 100 110,clip]{NiO-LDA_U-KPOINTS.png}
%\includegraphics[height=1.8in,width=4.in,viewport=30 210 570 440,clip]{PAW_projector.eps}
\caption{\small \textrm{用于\textrm{LDA}+$U$的反铁磁\ch{NiO}的结构\textrm{POSCAR}文件(左)和$\vec k$空间布点\textrm{KPOINTS}(右).}}%(与文献\cite{EPJB33-47_2003}图1对比)
\label{NiO-LDA_U-Input}
\end{figure}

\textrm{INCAR}中的输入控制参数如图\ref{NiO-LDA_U-INCAR}所示。因为是反铁磁计算,初始的原子磁矩(共线磁矩)控制两个\ch{Ni}原子(原子构型为:~\textrm{[Ar]3\textit{d}~$^8$})的磁矩相反(2.0,~-2.0),\ch{O}原子的净磁矩为0。考虑\textrm{LDA}+$U$的类型为\textrm{Dudarev}近似(\textit{LDAUTYPE}=2)。参数\textit{LDAUL}指定对应元素的角动量\textit{l}~需要考虑轨道的在位相互作用,$-1$表示该元素无需考虑在位相互作用。\textit{LDAUU}和\textit{LDAUJ}是对应的$U$和$J$的数值(单位~\textrm{eV})。为了考察不同角动量对磁性的贡献,对电荷密度按轨道角动量分解,因此设\textit{LORBIT}=11。考虑到反铁磁计算对电荷密度精度混合精度要求较高,因此设置电荷密度混合参数\textit{AMIX},\textit{BMIX},\textit{AMIX\_MAG}和\textit{BMIX\_MAG}以及\textit{LMAXMIN}等参数。

\begin{figure}[h!]
\centering
\includegraphics[height=3.0in,viewport=0 0 690 615,clip]{NiO-LDA_U-INCAR.png}
%\includegraphics[height=1.8in,width=4.in,viewport=30 210 570 440,clip]{PAW_projector.eps}
\caption{\small \textrm{用于\textrm{LDA}+$U$的反铁磁\ch{NiO}计算的控制文件\textrm{INCAR}文件.}}%(与文献\cite{EPJB33-47_2003}图1对比)
\label{NiO-LDA_U-INCAR}
\end{figure}

在\textrm{VASP}中,加$U$的方式有三种:~
\begin{itemize}
	\item \textit{LDAUTYPE}=1:~\textrm{Liechtenstein}等引入的旋转不变\textrm{LDA}+$U$方法\cite{PRB52-R5467_1995}:~形式为:
		\begin{displaymath}
			E_{\mathrm{HF}}=\dfrac12\sum_{\gamma}(U_{\gamma_1\gamma_3\gamma_2\gamma_4}-U_{\gamma_1\gamma_3\gamma_4\gamma_2})\hat{n}_{\gamma_1\gamma_2}\hat{n}_{\gamma_3\gamma_4}
		\end{displaymath}
		其中在位占据数由\textrm{PAW}方法确定:~
		\begin{displaymath}
			\hat{n}_{\gamma_1\gamma_2}=\langle\Psi^{s_2}|m_2\rangle\langle m_1|\Psi^{s_1}\rangle
		\end{displaymath}
		电子-电子相互作用由下式计算:
		\begin{displaymath}
			U_{\gamma_1\gamma_3\gamma_2\gamma_4}=\langle m_1m_3|\dfrac1{|\vec r-\vec r^{\prime}|}|m_2m_4\rangle]\delta_{s_1s_2}\delta_{s_3s_4}
		\end{displaymath}
		这里$|m\rangle$表示球谐函数。
	\item \textit{LDAUTYPE}=2:~\textrm{Dudarev}等引入的简化旋转不变\textrm{LDA}+$U$方法\cite{PRB57-1505_1998}
		\begin{displaymath}
			E_{\mathrm{LSDA+}U}=E_{\mathrm{LSDA}}+\dfrac{U-J}2+\sum_{\sigma}\bigg[\bigg(\sum_{m_1}n_{m_1,m_1}^{\sigma}\bigg)-\bigg(\sum_{m_1,m_2}n_{m_1,m_2}^{\sigma}n_{m_2,m_1}^{\sigma}\bigg)\bigg]
		\end{displaymath}
	\item \textit{LDAUTYPE}=4:~与\textit{LDAUTYPE}=1相似,不过用\textrm{LDA}+$U$代替\textrm{LSDA}+$U$(不再考虑\textrm{LSDA}的自旋分裂),因此计算表达式为:~
		\begin{displaymath}
			E_{\mathrm{dc}}(\hat n)=\dfrac{U}2\hat{n}_{\mathrm{tot}}(\hat{n}_{\mathrm{tot}}-1)-\dfrac{J}2\sum_{\sigma}\hat{n}_{\mathrm{tot}}^{\sigma}(\hat{n}_{\mathrm{tot}}^{\sigma}-1)
		\end{displaymath}
\end{itemize}

对比考虑不加$U$时(即$U=0,J=0$)时的\ch{Ni}原子的$d$电子在位占据数(如图\ref{NiO-LDA-onsite_matrix}所示),\textit{d}~电子分波投影标记在图中已经标出。
\begin{figure}[h!]
\centering
\includegraphics[height=3.0in,viewport=0 0 700 515,clip]{NiO-LDA-onsit_matrix.png}
%\includegraphics[height=1.8in,width=4.in,viewport=30 210 570 440,clip]{PAW_projector.eps}
\caption{\small \textrm{不加$U$的反铁磁\ch{NiO}计算的\ch{Ni}的$d$电子在位原子占据数.}}%(与文献\cite{EPJB33-47_2003}图1对比)
\label{NiO-LDA-onsite_matrix}
\end{figure}

加$U$计算得到的\ch{Ni}原子的$d$电子在位占据数如图\ref{NiO-LDA_U-onsite_matrix}所示,,可以看到$U$值对$d$电子数的影响。
\begin{figure}[h!]
\centering
\includegraphics[height=3.0in,viewport=0 0 700 525,clip]{NiO-LDA_U-onsit_matrix.png}
%\includegraphics[height=1.8in,width=4.in,viewport=30 210 570 440,clip]{PAW_projector.eps}
\caption{\small \textrm{用于\textrm{LDA}+$U$的反铁磁\ch{NiO}计算的\ch{Ni}的$d$电子在位原子占据数.}}%(与文献\cite{EPJB33-47_2003}图1对比)
\label{NiO-LDA_U-onsite_matrix}
\end{figure}

类似地,与轨道$l$相关的局域磁矩变化如图\ref{NiO-LDA_U-magnetization}所示:
\begin{figure}[h!]
\centering
\includegraphics[height=2.5in,viewport=0 0 410 415,clip]{NiO-LDA-magnetization.png}
\includegraphics[height=2.5in,viewport=0 0 410 415,clip]{NiO-LDA_U-magnetization.png}
%\includegraphics[height=1.8in,width=4.in,viewport=30 210 570 440,clip]{PAW_projector.eps}
\caption{\small \textrm{加$U$对反铁磁\ch{NiO}计算的\ch{Ni}的$d$电子数和局域磁矩的影响(左:~未考虑加$U$;~右:~考虑加$U$).}}%(与文献\cite{EPJB33-47_2003}图1对比)
\label{NiO-LDA_U-magnetization}
\end{figure}

考虑\textrm{LDA}+$U$的态密度和分波态密度如图\ref{NiO-LDA_U-DOS}所示:
\begin{figure}[h!]
\centering
\includegraphics[height=2.0in,viewport=0 0 2010 680,clip]{NiO-LDA_U-DOS.png}
%\includegraphics[height=1.8in,width=4.in,viewport=30 210 570 440,clip]{PAW_projector.eps}
\caption{\small \textrm{反铁磁\ch{NiO}计算的\ch{Ni}的$d$电子分态密度受$U$的影响(左:~未考虑加$U$;~右:~考虑加$U$).}}%(与文献\cite{EPJB33-47_2003}图1对比)
\label{NiO-LDA_U-DOS}
\end{figure}

考虑\textrm{LDA}+$U$对总能的影响,如图\ref{NiO-LDA_U-tot}所示:~
\begin{figure}[h!]
\centering
\includegraphics[height=1.3in,viewport=6.5 0 836 235,clip]{NiO-LDA-OSZICAR.png}
\includegraphics[height=1.5in,viewport=0 0 830 270,clip]{NiO-LDA_U-OSZICAR.png}
%\includegraphics[height=1.8in,width=4.in,viewport=30 210 570 440,clip]{PAW_projector.eps}
\caption{\small \textrm{加$U$对反铁磁\ch{NiO}计算的\ch{Ni}的$d$对基态总能的影响(上:~未考虑加$U$;~下:~考虑加$U$).}}%(与文献\cite{EPJB33-47_2003}图1对比)
\label{NiO-LDA_U-tot}
\end{figure}
不难看出,对于\textrm{Dudarev}近似,只要$(U-J)>0$,则考虑\textrm{LDA}+$U$的基态总能总会大于不考虑加$U$的情况。因此,如果两次计算中$(U-J)$的数值不等,不能将总能简单对比。
\subsection{光学介电函数的计算}
本部分我们将学习如何利用\textrm{VASP}计算\ch{SiC}的光学介电函数,关于\textrm{VASP}中光学介电函数计算主要参见文献\inlinecite{PRB73-045112_2006}。

计算对象\ch{SiC}的结构和$\vec k$空间布点分别如如图\ref{SiC-optic-Input}所示:
\begin{figure}[h!]
\centering
\includegraphics[height=1.5in,viewport=0 15 186 220,clip]{SiC-optic-POSCAR.png}
\includegraphics[height=0.8in,viewport=0 0 100 110,clip]{SiC-epsilon-KPOINTS.png}
%\includegraphics[height=1.8in,width=4.in,viewport=30 210 570 440,clip]{PAW_projector.eps}
\caption{\small \textrm{计算\ch{SiC}的光学介电函数时的模型(左)和$\vec k$空间布点方案(右).}}%(与文献\cite{EPJB33-47_2003}图1对比)
\label{SiC-optic-Input}
\end{figure}

介电函数包括静态介电函数(\textrm{\textit{LEPSILON}=.TURE.})和动态(频率相关)介电函数(\textrm{\textit{LOPITC}=.TURE.})两类,这里将分别介绍。
\subsubsection{\rm{静态介电函数的计算}}
一般静态介电函数计算是通过密度泛函微扰理论(\textrm{Density Functional Perturbation Theory, DFPT})计算的,同时计算的物理量还包括\textrm{Born}有效电荷(\textrm{the Born effective charges}),铁电张量(\textrm{piezoelectric tensors})等。计算控制参数如图\ref{SiC-Epsilon-INCAR}所示。
\begin{figure}[h!]
\centering
\includegraphics[height=1.2in,viewport=0 380 420 540,clip]{SiC-epsilon-INCAR.png}
%\includegraphics[height=1.8in,width=4.in,viewport=30 210 570 440,clip]{PAW_projector.eps}
\caption{\small \textrm{计算\ch{SiC}的光学介电函数时的计算控制文件\textrm{INCAR}.}}%(与文献\cite{EPJB33-47_2003}图1对比)
\label{SiC-Epsilon-INCAR}
\end{figure}

根据微扰理论,计算静态介电函数时,
\begin{displaymath}
	\nabla_{\vec k}|\tilde u_{n\vec k}\rangle=\sum_{n^{\prime}\neq n}\dfrac{|\tilde u_{n^{\prime}\vec k}\rangle\langle\tilde u_{n^{\prime}\vec k}|\frac{\partial(\mathbf{H}(\vec k)-\varepsilon_{n\vec k}\mathbf{S}(\vec k))}{\partial\vec k}|\tilde u_{n\vec k}\rangle}{\varepsilon_{n\vec k}-\varepsilon_{n^{\prime}\vec k}}
\end{displaymath}
可用线性\textrm{Sternheimer}方程改写
\begin{displaymath}
	(\mathbf{H}(\vec k)-\varepsilon_{n\vec k}\mathbf{S}(\vec k))|\nabla_{\vec k}|\tilde u_{n\vec k}\rangle=-\dfrac{\partial(\mathbf{H}(\vec k)-\varepsilon_{n\vec k}\mathbf{S}(\vec k))}{\partial\vec k}|\tilde u_{n\vec k}\rangle
\end{displaymath}
方程的求解过程和自洽迭代的思想类似。

在输出文件\textrm{OUTCAR}中计算得到的静态光学介电函数等物理量如图\ref{SiC-Epsilon-OUTCAR}为:~
\begin{figure}[h!]
\centering
\includegraphics[height=3.5in,viewport=0 0 1030 845,clip]{SiC-epsilon-OUTCAR.png}
%\includegraphics[height=1.8in,width=4.in,viewport=30 210 570 440,clip]{PAW_projector.eps}
\caption{\small \textrm{计算\ch{SiC}的光学介电函数时的计算结果\textrm{OUTCAR}中有关输出内容.}}%(与文献\cite{EPJB33-47_2003}图1对比)
\label{SiC-Epsilon-OUTCAR}
\end{figure}

如果控制参数中引入\textrm{\textit{LRPA}=.TRUE.},则可以在无规相近似(\textrm{Random-Phase-Approximation, RPA})下,在静态介电函数计算中引入局域\textrm{Hartree}势场效应(\textrm{local field effect on the Hartree level});~类似地,微扰理论也可以通过有限差分方法计算,控制参数为\textrm{\textit{LPEAD}=.TRUE.},这两种计算对应的输入文件分别如图\ref{SiC-epsilon-INCARs}所示:~
\begin{figure}[h!]
\centering
\includegraphics[height=1.5in,viewport=0 310 420 540,clip]{SiC-epsilon-INCAR.png}
\includegraphics[height=1.5in,viewport=0 290 420 520,clip]{SiC-epsilon-INCAR_PEAD.png}
%\includegraphics[height=1.8in,width=4.in,viewport=30 210 570 440,clip]{PAW_projector.eps}
\caption{\small \textrm{计算\ch{SiC}的静态光学介电函数时\textrm{RPA}近似(左)和有限差分方法(右).}}%(与文献\cite{EPJB33-47_2003}图1对比)
\label{SiC-epsilon-INCARs}
\end{figure}
\subsubsection{\rm{频率相关的介电函数}}
\textrm{VASP}中动态介电函数虚部计算公式为:
\begin{displaymath}
	\varepsilon_{\alpha\beta}^{(2)}(\omega)=\dfrac{4\pi^2e^2}{\Omega}\lim_{q\rightarrow 0}\frac1{q^2}\sum_{c,v,\vec k}2w_{\vec k}\delta(\varepsilon_{c,\vec k}-\varepsilon_{v,\vec k}-\omega)\times\langle u_{c,\vec k+\mathbf{e}_{\alpha}q}|u_{v,\vec k}\rangle\langle u_{c,\vec k+\mathbf{e}_{\beta}q}|u_{v,\vec k}\rangle^{\ast}
\end{displaymath}
介电函数虚部和实部满足\textrm{Kramers-Kr\"onig}变换关系:
\begin{displaymath}
	\varepsilon_{\alpha\beta}^{(1)}(\omega)=1+\dfrac2{\pi}\mathcal{P}\int_0^{\infty}\dfrac{\varepsilon_{\alpha\beta}^{(2)}(\omega^{\prime})\omega^{\prime}}{\omega^{\prime2}-\omega^2+\mathrm{i}\eta}\mathrm{d}{\omega^{\prime}}
\end{displaymath}
这里$\mathcal{P}$表示取积分主值。$\eta$对应控制参数\textit{CSHIFT}。

为了计算动态介电函数,首先完成标准\textrm{DFT}计算,控制文件和$\vec k$空间布点方案如图\ref{SiC-optic-Input}所示:~
\begin{figure}[h!]
\centering
\includegraphics[height=0.9in,viewport=0 10 170 90,clip]{SiC-optic-INCAR.png}
\includegraphics[height=0.9in,viewport=0 0 100 106,clip]{SiC-optic-KPOINTS.png}
%\includegraphics[height=1.8in,width=4.in,viewport=30 210 570 440,clip]{PAW_projector.eps}
\caption{\small \textrm{为计算\ch{SiC}的动态光学介电函数,完成标准\textrm{DFT}计算的控制文件(左)和$\vec k$空间布点方案(右).}}%(与文献\cite{EPJB33-47_2003}图1对比)
\label{SiC-optic-Input}
\end{figure}

基于标准\textrm{DFT}计算的\textrm{WAVECAR},计算独立粒子\textrm{(independent-particle, IP)}图像下的动态介电函数时,注意要引入较多的空轨道,计算控制参数如图\ref{SiC-optic-INCAR}所示:
\begin{figure}[h!]
\centering
\includegraphics[height=2.0in,viewport=0 10 500 300,clip]{SiC-optic-INCAR.png}
%\includegraphics[height=1.8in,width=4.in,viewport=30 210 570 440,clip]{PAW_projector.eps}
\caption{\small \textrm{为计算\ch{SiC}的动态光学介电函数的控制文件.}}%(与文献\cite{EPJB33-47_2003}图1对比)
\label{SiC-optic-INCAR}
\end{figure}

在输出文件\textrm{OUTCAR}中,频率相关的介电函数虚部和实部数据如图\ref{SiC-optic-OUTCAR}所示:~
\begin{figure}[h!]
\centering
\includegraphics[height=4.5in,viewport=0 0 990 1000,clip]{SiC-optic-OUTCAR.png}
%\includegraphics[height=1.8in,width=4.in,viewport=30 210 570 440,clip]{PAW_projector.eps}
\caption{\small \textrm{计算\ch{SiC}输出文件中的动态介电函数的虚部和实部.}}%(与文献\cite{EPJB33-47_2003}图1对比)
\label{SiC-optic-OUTCAR}
\end{figure}
与静态介电函数计算类似,可以通过有限差分方法计算$\nabla_{\vec k}\tilde u_{n\vec k}$,进而得到动态介电函数,只要在\textrm{INCAR}中设置控制参数\textrm{\textit{LPEAD}=.TRUE.},这个留待作为练习。

\subsection{GW计算}
本部分我们将学习如何利用\textrm{VASP}完成对\ch{Si}的\textit{GW}计算。为了完成\textit{GW}计算,首先也需要完成标准\textrm{DFT}计算,相关输入文件(\textrm{Si}的结构文件、计算控制文件和$\vec k$空间布点方案)如图\ref{Si-GW-Input}所示:~
\begin{figure}[h!]
\centering
\includegraphics[height=2.0in,viewport=0 10 160 210,clip]{Si-GW-POSCAR.png}
\includegraphics[height=1.2in,viewport=0 10 160 90,clip]{SiC-optic-INCAR.png}
\includegraphics[height=1.2in,viewport=0 0 100 106,clip]{SiC-optic-KPOINTS.png}
%\includegraphics[height=1.8in,width=4.in,viewport=30 210 570 440,clip]{PAW_projector.eps}
\caption{\small \textrm{为实现\ch{Si}的\textit{GW}计算,所需的标准\textrm{DFT}计算的结构文件(左)、计算控制文件(中)和$\vec k$空间布点方案(右).}}%(与文献\cite{EPJB33-47_2003}图1对比)
\label{Si-GW-Input}
\end{figure}

在完成标准\textrm{DFT}计算基础上,\textit{GW}计算还需要\textrm{DFT}计算的波函数\textrm{WAVECAR文件(含有大量的空轨道)}和波函数导数\textrm{WAVEDER}文件,因此后续计算与光学动态介电函数的计算类似,控制参数如图\ref{Si-GW-DFT-INCAR}所示:
\begin{figure}[h!]
\centering
\includegraphics[height=2.0in,viewport=0 10 490 285,clip]{Si-GW-DFT-INCAR.png}
%\includegraphics[height=1.8in,width=4.in,viewport=30 210 570 440,clip]{PAW_projector.eps}
\caption{\small \textrm{为支持\ch{Si}的\textit{GW}计算,标准\textrm{DFT}后续获得\textrm{WAVECAR}和\textrm{WAVEDER}的控制文件.}}%(与文献\cite{EPJB33-47_2003}图1对比)
\label{Si-GW-DFT-INCAR}
\end{figure}

正式的\textit{GW}计算,如果只是单步计算(不迭代)准粒子(\textrm{quasi-particle, QP})能量,称为$G_0W_0$计算,控制参数如图\ref{Si-G0W0-INCAR}所示。
\begin{figure}[h!]
\centering
\includegraphics[height=2.7in,viewport=0 10 595 465,clip]{Si-G0W0-INCAR.png}
%\includegraphics[height=1.8in,width=4.in,viewport=30 210 570 440,clip]{PAW_projector.eps}
\caption{\small \textrm{\ch{Si}的$G_0W_0$计算的控制文件.}}%(与文献\cite{EPJB33-47_2003}图1对比)
\label{Si-G0W0-INCAR}
\end{figure}
在\textit{GW}计算的输出文件\textrm{OUTCAR}中,准粒子能量如图\ref{Si-G0W0-OUTCAR}所示。
\begin{figure}[h!]
\centering
\includegraphics[height=5.5in,viewport=0 40 1190 1240,clip]{Si-G0W0-OUTCAR.png}
%\includegraphics[height=1.8in,width=4.in,viewport=30 210 570 440,clip]{PAW_projector.eps}
\caption{\small \textrm{\ch{Si}的$G_0W_0$计算输出文件中的准粒子能量本征值.}}%(与文献\cite{EPJB33-47_2003}图1对比)
\label{Si-G0W0-OUTCAR}
\end{figure}

如果\textit{GW}计算中,准粒子能量是通过迭代计算得到的,则是$GW_0$计算,只要在$G_0W_0$基础上引入迭代控制参数\textit{NELM}即可。如本算例中,迭代次数为4次(\textit{NELM}=4),如图\ref{Si-GW0-INCAR}所示。
\begin{figure}[h!]
\centering
\includegraphics[height=2.5in,viewport=0 10 595 465,clip]{Si-GW0-INCAR.png}
%\includegraphics[height=1.8in,width=4.in,viewport=30 210 570 440,clip]{PAW_projector.eps}
\caption{\small \textrm{\ch{Si}的$GW_0$计算的控制文件.}}%(与文献\cite{EPJB33-47_2003}图1对比)
\label{Si-GW0-INCAR}
\end{figure}

图\ref{Si-GW-Band}给出\textit{GW}计算和\textrm{DFT}计算的\ch{Si}的能带变化,可以看出,考虑\textit{GW}影响之后,带隙增大。这部分作为课后练习,留待用户完成。
\begin{figure}[h!]
\centering
\includegraphics[height=2.5in, width=2.3in, viewport=0 0 375 490,clip]{Si-BSE-band.png}
%\includegraphics[height=1.8in,width=4.in,viewport=30 210 570 440,clip]{PAW_projector.eps}
\caption{\small \textrm{\ch{Si}的\textrm{DFT}和$GW$计算的能带(局部)和$\Gamma$点的带隙变化.}}%(与文献\cite{EPJB33-47_2003}图1对比)
\label{Si-GW-Band}
\end{figure}

有关旋-轨耦合(\textrm{Spin-Orbital coupling, SOC})、非共线磁性(\textrm{nonlinear magnetization})等相关的计算,与\textrm{LDA}+$U$、光学介电函数和\textit{GW}计算类似,主要是控制参数文件\textrm{INCAR}的修改,有兴趣的用户请参阅有关算例。

\clearpage\newpage
\textbf{课后作业}
\begin{enumerate}
	\item 材料模拟的重要任务之一就是在开展实验之前筛选更合适的候选组成,在实验室在开发新的催化剂时,通常需要一系列的研究工作,因此“计算先行”策略显得尤为重要。如果我们需要寻找\textrm{Pt}-基体系的应力效应,请描述如何在应用层面上如何模拟该体系,并列出需要考虑的重要问题。
	\item 长期以来,\textrm{Pt}是非常重要的催化剂。研究表明,从催化活性、结构稳定性以及价格经济等角度考虑,\textrm{Pt3Y}($\mathrm{L1}_2$结构)都是很好的催化剂性能。假设我们已知\textrm{Pt}和\textrm{Y}的体相基态能分别是~--6.056\textrm{~eV/atom}和\textrm{--6.466}\textrm{~eV/atom},根据\ref{Sec:FCC-Pt}介绍的计算流程,构建\textrm{FCC}晶格(四个原子)的\ch{Pt3Y},并确定平衡态晶格参数。
	\item 根据上题结果,用$4\times4\times2.5$的\ch{Pt3Y}(111)薄层(5层,80个原子)和$15\mathrm{\AA}$真空层模拟\ch{Pt3Y}表面,计算一个氧原子在\textrm{FCC}间隙位(两个\textrm{Pt}和一个\textrm{Y}原子)的吸附能和\textrm{Pt-O}键长。\footnote{如果需要,为节省时间,可以给学生提供已收敛的\textrm{CONTCAR}文件}因为该吸附位是已知的氧原子在\ch{Pt3Y}(111)表面的最优先吸附位,假设我们已知如下能量:~\\
		\begin{itemize}
			\item 氧的原子能量:~--1.5514\textrm{~eV}
			\item \ch{Pt3Y}表面的基态能量:~548.6599\textrm{~eV}
		\end{itemize}
	\item 根据上题,说明为什么计算\ch{Pt3Y}(111)表面的吸附能时,我们需要构建这样大的超晶胞。
	\item 锗(\textrm{Ge})是一种和硅(\textrm{Si})类似的半导体材料(曾被称为“亚硅”)。根据\ref{Sec:Si-band}介绍的步骤算并绘制\textrm{Ge}的能带图(选择平衡晶胞参数$5.784\mathrm{\AA}$),你能从能带图中看到带隙(\textrm{band gap})吗?如果看不到带隙,找到可能的主要原因,并说明怎么才能克服这些问题。
	\item 完成\ch{SiC}的动态光学介电函数计算练习,用有限差分方法计算$\nabla_{\vec k}\tilde u_{n\vec k}$并获得动态介电函数,并与\textrm{RPA}近似动态介电函数对比。
	\item 完成\ch{Si}的\textit{GW}计算和\textrm{DFT}计算的能带图,并在能带图上标出$\Gamma$点的带隙变化情况。
\end{enumerate}
