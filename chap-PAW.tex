\section{投影子缀加波(Projector Augmented Wave, PAW)方法}
PAW方法是\textrm{Bl\"ochl}于1994年独立提出来的一种计算方法\cite{PRB50-17953_1994},该方法的基本思想与OPW很相似,但同时结合了赝势方法和APW方法的优点,达到平衡计算效率和精度的目的。PAW方法刚提出来的时候并未引起注意,直到1999年\textrm{Kresse}指明了PAW方法和USPP方法的密切关联,USPP方法的计算程序只要经过简单改造就能扩展为PAW方法,有力地推动了PAW方法的广泛应用\cite{PRB59-1758_1999}。现在PAW方法已经成为最主要的可支持第一原理分子动力学(Ab initio Molecular Dynamics, AIMD)。

\subsection{PAW方法的基本思想}
与一般赝势方法不同,PAW方法的目标是全电子(all-electron)波函数\footnote{注意,“全电子”在\textrm{Bl\"ochl}原始文献\inlinecite{PRB50-17953_1994}中与“真实电子波函数”意义相近,强调电子在原子核附近的振荡行为,但并未严格区分价电子与芯电子;~而在\textrm{Kresse}的文献\inlinecite{PRB59-1758_1999}中则是明确指价电子波函数。%强调价电子波函数因与芯层电子正交而在原子核附近振荡;
此外,与“全电子”概念密切关联的是“全势(full-potential)”,两者在具体语境中有一定的区别,“全势”强调的是重现价电子感受到的势函数的效果。},体系中全部电子构成\textrm{Hilbert}空间,价电子与芯层态彼此正交,使得波函数在\textrm{Muffin-tin}球内振荡。
\textrm{Bl\"ochl}假设全电子波函数$|\Psi\rangle$与赝波函数$|\tilde\Psi\rangle$满足线性变换,即满足:
\begin{equation}
	|\Psi\rangle=\mathbf{\tau|}\tilde\Psi\rangle
	\label{eq:PAW-Blochl-01}
\end{equation}
%	$$\tau=\mathbf{1}+\sum_{\mathrm R}\hat\tau_{\mathrm R}$$
在原子核附近的$r_c$范围内\footnote{习惯上这个区域称为缀加区(\textrm{Augmentation region}).},除了平面波,还引入原子分波函数展开来表示波函数:
\begin{equation}
	|\Psi\rangle=|\tilde\Psi\rangle+\sum_i(|\phi_i\rangle-|\tilde\phi_i\rangle)\langle\tilde p_i|\tilde\Psi\rangle
	\label{eq:PAW-Blochl-02}
\end{equation}
在$r_c$外$|\tilde\Psi\rangle$与$|\Psi\rangle$变换前后保持不变,因此线性变换$\mathbf{\tau}$可表示为:
\begin{equation}
	\mathbf{\tau}=\mathbf{1}+\sum_i(|\phi_i\rangle-|\tilde\phi_i\rangle)\langle\tilde p_i|
	\label{eq:PAW-Blochl-03}
\end{equation}
其中$|\tilde p_i\rangle$是\textrm{MT}球内的投影函数,$i$表示原子位置$\vec R$、原子轨道($l,m$)和能级$\epsilon_k$的指标。\textrm{PAW}的波函数与赝波函数的关系,可以用图\ref{PAW_basic}表示。
\begin{figure}[h!]
\centering
\includegraphics[height=2.35in,width=4.1in,viewport=0 0 1280 745,clip]{PAW-baseset.png}
%\includegraphics[height=1.8in,width=4.in,viewport=30 210 570 440,clip]{PAW_projector.eps}
\caption{\small \textrm{The analysis of PAW basic function.}}%(与文献\cite{EPJB33-47_2003}图1对比)
\label{PAW_basic}
\end{figure}

\subsection{PAW表示下的能量表示与补偿电荷}
\textrm{PAW}方法通过投影算符和赝波函数实现全电子波函数计算,因此完整的\textrm{PAW}方法下,一般算符的期望值计算为
\begin{equation}
	\langle A \rangle=\langle\Psi|\mathbf{A}|\Psi\rangle=\langle\tilde\Psi|\mathbf{\tau}^{\dag}\mathbf{A}\mathbf{\tau}|\tilde\Psi\rangle=\langle\tilde\Psi|\tilde{\mathrm{A}}|\tilde\Psi\rangle
	\label{eq:PAW-Blochl-04}
\end{equation}
式中将$\mathrm{\tau}$用式\eqref{eq:PAW-Blochl-03}展开,因此赝算符$\tilde A$可表示为
\begin{equation}
	\tilde A=\mathbf{A}+\sum_i|\tilde p_i\rangle(\langle\phi_i|\mathbf{A}|\phi_i\rangle-\langle\tilde\phi_i|\mathbf{A}|\tilde\phi_i\rangle)\langle\tilde p_i|
	\label{eq:PAW-Blochl-05}
\end{equation}
%不难看出,赝重叠算符$\tilde O$可展开为
%\begin{equation}
%	\tilde O=\mathbf{1}+\sum_i|\tilde p_i\rangle(\langle\phi_i|\phi_i\rangle-\langle\tilde\phi_i|\tilde\phi_i\rangle)\langle\tilde p_i|
%	\label{eq:PAW-Blochl-06}
%\end{equation}
实空间中,电荷密度的算符为$|r\rangle\langle r|$,根据式\eqref{eq:PAW-Blochl-03},则电荷密度的计算
\begin{equation}
	n(\vec r)=\tilde n(\vec r)+n^1(\vec r)-\tilde n^1(\vec r)
	\label{eq:PAW-Blochl-07}
\end{equation}
这里
\begin{displaymath}
	\begin{aligned}
		\tilde n(\vec r)=&\sum_nf_n\langle\tilde\Psi_n|\vec r\rangle\langle\vec r|\tilde\Psi_n\rangle \\
n^1(\vec r)=&\sum_{n,(i,j)}f_n\langle\tilde\Psi_n|\tilde p_i\rangle\langle\phi_i|\vec r\rangle\langle\vec r|\phi_j\rangle\langle\tilde p_j|\tilde\Psi_n\rangle \\
\tilde n^1(\vec r)=&\sum_{n,(i,j)}f_n\langle\tilde\Psi_n|\tilde p_i\rangle\langle\tilde\phi_i|\vec r\rangle\langle\vec r|\tilde\phi_j\rangle\langle\tilde p_j|\tilde\Psi_n\rangle
	\end{aligned}
\end{displaymath}
$f_n$是占据态的电子数。注意,这里的$n^1$和$\tilde n^1$中包含了芯层电荷的贡献,即$\sum_n\langle\phi_n^c|\vec r\rangle\langle\vec r|\phi_n^c\rangle$,$\sum_n\langle\tilde\phi_n^c|\vec r\rangle\langle\vec r|\tilde\phi_n^c\rangle$和$\sum_n\langle\tilde\Psi_n^c|\vec r\rangle\langle\vec r|\tilde\Psi_n^c\rangle$。但是实际应用中,一般都是直接构造芯层态电荷密度,并不考虑芯层态波函数。

类似地,根据\textrm{DFT}理论,体系总能量泛函可以表示为
\begin{equation}
	\begin{aligned}
		E&=\sum_nf_n\langle\Psi_n|-\dfrac12\nabla^2|\Psi_n\rangle\\
		 &+\dfrac12\int\mathrm{d}\vec r\int\mathrm{d}\vec r^{\prime}\dfrac{(n+n^Z)(n+n^Z)}{|\vec r-\vec r^{\prime}|}+\int\mathrm{d}\vec r n\epsilon_{\mathrm{XC}}(n)
	\end{aligned}
	\label{eq:PAW-Blochl-08}
\end{equation}
在\textrm{PAW}框架下,总能可分解为$E=\tilde E+E^1-\tilde E^1$,每一项分别表示为:
\begin{equation}
	\begin{aligned}
		\tilde E&=\sum_nf_n\langle\tilde\Psi_n|-\dfrac12\nabla^2|\tilde\Psi_n\rangle\\
		 &+\dfrac12\int\mathrm{d}\vec r\int\mathrm{d}\vec r^{\prime}\dfrac{(\tilde n+\hat n)(\tilde n+\hat n)}{|\vec r-\vec r^{\prime}|}+\int\mathrm{d}\vec r \tilde n\bar v+\int\mathrm{d}\vec r \tilde n\epsilon_{\mathrm{XC}}(\tilde n)
 	\end{aligned}
	\label{eq:PAW-Blochl-09}
\end{equation}
\begin{equation}
	\begin{aligned}
		E^1&=\sum_{n,(i,j)}f_n\langle\tilde\Psi_n|\tilde p_i\rangle\langle\phi_i|-\dfrac12\nabla^2|\phi_j\rangle\langle\tilde p_j|\tilde\Psi_n\rangle\\
		 &+\dfrac12\int\mathrm{d}\vec r\int\mathrm{d}\vec r^{\prime}\dfrac{(n^1+n^Z)(n^1+n^Z)}{|\vec r-\vec r^{\prime}|}+\int\mathrm{d}\vec r n^1\epsilon_{\mathrm{XC}}(n^1)
 	\end{aligned}
	\label{eq:PAW-Blochl-10}
\end{equation}
\begin{equation}
	\begin{aligned}
		\tilde E^1&=\sum_{n,(i,j)}f_n\langle\tilde\Psi_n|\tilde p_i\rangle\langle\tilde\phi_i|-\dfrac12\nabla^2|\tilde\phi_j\rangle\langle\tilde p_j|\tilde\Psi_n\rangle\\
		 &+\dfrac12\int\mathrm{d}\vec r\int\mathrm{d}\vec r^{\prime}\dfrac{(\tilde n^1+\hat n)(\tilde n^1+\hat n)}{|\vec r-\vec r^{\prime}|}+\int\mathrm{d}\vec r \tilde n^1\bar v+\int\mathrm{d}\vec r \tilde n^1\epsilon_{\mathrm{XC}}(\tilde n^1)
 	\end{aligned}
	\label{eq:PAW-Blochl-11}
\end{equation}
式\eqref{eq:PAW-Blochl-09}和\eqref{eq:PAW-Blochl-11}中$\bar v$是缀加区内的任意局域函数,只要该区域内满足条件$\tilde n=\tilde n^1$,则$\bar v$对总能量的贡献为0。与赝势方法类似,$\bar v$取去屏蔽局域赝势。

类似地,式\eqref{eq:PAW-Blochl-09}和\eqref{eq:PAW-Blochl-11}中$\hat n$为补偿电荷,取值范围限于缀加区。与超软赝势中的补偿电荷的要求类似,\textrm{PAW}方法中的$\hat n$存在使得电荷密度$(n^1+n^Z)$和$(\tilde n^1+\hat n)$在缀加区外的多极矩贡献相等,因此不必再考虑电荷在缀加区外的相互作用,只需考虑$\hat n$在$\tilde E$中的贡献。$\hat n$可表示为单个原子截断区间的补偿电荷之和,即$\hat n=\sum_R\hat n_R$,并且$\hat n_R(r)$可用\textrm{Gaussian}函数展开,即
\begin{equation}
	\hat n_R(r)=\sum_{L=(l,m)}g_{RL}(r)Q_{RL}
	\label{eq:PAW-Blochl-12}
\end{equation}
式中$g_{RL}(r)$是广义的\textrm{Gaussian}函数,表示如下
	$$g_{RL}(r)=C_l|r-R|^lY_L(r-R)\mathrm{e}^{-(|r-R|/r_c)^2}$$
其系数$C_l$是归一化系数,由条件
	$\int\mathrm{d}rr^lY_L(r)g_L(r)=1$
确定。
式\eqref{eq:PAW-Blochl-12}的$Q_{RL}$由构造补偿电荷所须满足的多极矩条件确定
	$$Q_{RL}=\int\mathrm{d}r|r-R|^l\big[n_R^1(r)+n_R^Z(r)-\tilde n_R^1(r)\big]Y_L^{\ast}(r-R)$$

一般来说,原子的补偿电荷的\textrm{Gaussian}展开在缀加区会快速衰减,这意味着最终需要很高的平面波来展开。在\textrm{Bl\"ochl}的方案中,建议引入新的补偿电荷$\hat n^{\prime}$,满足条件:~\footnote{这是\textrm{Bl\"ochl}方案与后来\textrm{Kresse}方案处理补偿电荷的主要区别。}
	\begin{itemize}
		\item $\hat n^{\prime}$与$\hat n$具有相同的多极矩(保留原来的补偿电荷的基本要求)
		\item $\hat n^{\prime}$的\textrm{Gaussian}展开的衰减半径$r_c^{\prime}$比$r_c$大得多,可以用很少的平面波展开
	\end{itemize}
因此,能量$\tilde E$中的静电相互作用可以表示为
	\begin{equation}
		\begin{aligned}
			&\dfrac12\int\mathrm{d}r\int\mathrm{d}r^{\prime}\dfrac{(\tilde n+\hat n)(\tilde n+\hat n)}{|r-r^{\prime}|}\\
			=&\underline{\dfrac12\int\mathrm{d}r\int\mathrm{d}r^{\prime}\dfrac{(\tilde n+\hat n^{\prime})(\tilde n+\hat n^{\prime})}{|r-r^{\prime}|}}
			+\underline{\int\mathrm{d}r\tilde n(r)\hat v(r)}+\underline{\sum_{R,R^{\prime}}U_{R,R^{\prime}}}
		\end{aligned}
		\label{eq:PAW-Blochl-13}
	\end{equation}
其中第一项是平滑函数,可以在\textrm{Fourier}空间计算
	$$2\pi V\sum_G\dfrac{|\tilde n(G)+\hat n^{\prime}(G)|^2}{G^2}$$
第二项的$\hat v(r)$表示为
	$$\hat v(r)=\int\mathrm{d}r^{\prime}\dfrac{\hat n(r^{\prime})-\hat n^{\prime}(r^{\prime})}{|r-r^{\prime}|}$$
虽然$\hat v(r)$和$n(r)$一样有高\textrm{Fourier}截断,但因为$\tilde n(G)$只需要少量平面波展开,所以$\hat v(r)$的高阶部分不会对$\int\mathrm{d}r\tilde n(r)\hat v(r)$有贡献。最后一项中$U_{R,R^{\prime}}$是原子间的短程成对势
	$$U_{R,R^{\prime}}=\dfrac12\int\mathrm{d}r\int\mathrm{d}r^{\prime}\dfrac{\hat n_R(r)\hat n_{R^{\prime}}(r^{\prime})-\hat n_R^{\prime}(r)\hat n_{R^{\prime}}^{\prime}(r^{\prime})}{|r-r^{\prime}|}$$
这一项可以通过\textrm{Ewald}求和方法计算。

\subsection{PAW方法的Kohn-Sham方程与原子分波、投影函数}
在\textrm{DFT}框架下,应用\textrm{PAW}方法,除了总能量的表示外,还要考虑\textrm{Kohn-Sham}方程。根据式\eqref{eq:PAW-Blochl-05},平面波基表示的重叠算符可以写成:
\begin{equation}
	\tilde O=\mathbf{1}+\sum_{i,j}|\tilde p_i\rangle\bigg[\langle\phi_i|\phi_j\rangle-\langle\tilde\phi_i|\tilde\phi_j\rangle\bigg]\langle\tilde p_j|
	\label{eq:PAW-Blochl-14}
\end{equation}
类似地,可以得到\textrm{Hamilitonian}算符\cite{PRB50-17953_1994}:
%\begin{itemize}
%	\item 动能算符:
%		$$\tilde T=-\dfrac12\nabla^2+\sum_{i,j}|\tilde p_i\rangle[\langle\phi_i|-\dfrac12\nabla^2|\phi_j\rangle-\langle\tilde\phi_i|-\dfrac12\nabla^2|\tilde\phi_j\rangle]\langle\tilde p_j|$$
%	\item 完全势\textrm{(full-potential)}算符:
%$$v(\vec r)=\tilde v(\vec r)+v^1(\vec r)-\tilde v^1(\vec r)$$
%	\item \textrm{Hamilton}算符:
\begin{equation}
	\begin{aligned}
		\tilde H=&-\dfrac12\nabla^2+\tilde v+\sum_{i,j}|\tilde p_i\rangle\bigg[\langle\phi_i|-\dfrac12\nabla^2+v^1|\phi_j\rangle\\
			&-\langle\tilde\phi_i|-\dfrac12\nabla^2+\tilde v^1|\tilde\phi_j\rangle\bigg]\langle\tilde p_j| 
	\end{aligned}
	\label{eq:PAW-Blochl-15}
\end{equation}
%\end{itemize}
在\textrm{PAW}方法中,完整的势函数\textrm{(full potential)}算符表示为:
\begin{equation}
	v(\vec r)=\tilde v(\vec r)+v^1(\vec r)-\tilde v^1(\vec r)
	\label{eq:PAW-Blochl-16}
\end{equation}
与电荷密度的表示类似,势函数是由平滑的平面波表示赝势函数$\tilde v$和位于每个原子缀加区的单中心局域的原子势$v^1$和$\tilde v^1$叠加而成。平滑势可用平面波展开,而原子势则表示成径向函数和角度部分乘积。

总能量泛函对赝波函数求导,可有
\begin{equation}
	\left.\dfrac{\partial E[\tilde\Psi, R]}{\partial\langle\tilde\Psi_n|}\right|_R=\tilde H|\tilde\Psi_n\rangle f_n
	\label{eq:PAW-Blochl-17}
\end{equation}

\textrm{PAW}方法除了采用平面波作为基函数,在每个原子核附近$r_c$范围内的缀加区,还保留了原子波函数和以平面波表示的赝波函数,它们构成\textrm{PAW}方法的基函数的一部分,%在\textrm{PAW}方法中,将与原子分波展开所有相关的数据统称原子数据集(data set),
也是\textrm{PAW}计算的基础。

\textrm{PAW}与原子分波展开所有相关的数据主要包括:~
	\begin{itemize}
		\item 原子分波信息:~原子分波函数$\phi_i$、赝分波函数$\tilde\phi_i$和投影波函数$p_i$
		\item 电荷密度信息:~$r_c$内的电荷密度$n^1$、赝电荷密度$\tilde n^1$和补充电荷$\hat n$
		\item 赝势信息:~原子局域赝势$\tilde v_{\mathrm{loc}}(\vec r)$
	\end{itemize}
\textrm{PAW}方法的原子分波与赝势方法相似,一套原子数据集将用于各种化学环境下的\textrm{PAW}计算,即要求原子数据集有良好的可移植性;~与赝势方法不同之处在于\textrm{PAW}原子集中除了赝原子的信息,还包含了真实原子的信息。

原子分波函数由原子\textrm{Schr\"odinger}方程确定
\begin{equation}
	\bigg(-\dfrac12\nabla^2+v_{at}-\varepsilon_i^1\bigg)|\phi_i\rangle=0
	\label{eq:PAW-Blochl-18}
\end{equation}
根据\textrm{Bl\"ochl}的建议,为确定分波函数$\phi_i$,一般通过适当选择$\varepsilon_i^1$(原子价电子能量附近),并要求在缀加区与芯层分波函数正交。实际应用中,还可以类似\textrm{LAPW}方法,对每个分波引入多个(一般是两个)分波函数。

对于赝原子分波,\textrm{Bl\"ochl}建议的赝化方案与传统赝势构造类似\cite{PRL43-1494_1979,PRB26-4119_1982,PRB40-2980_1989}:
通过引入局域赝势:%$$w_i(r)=\tilde v_{at}(r)+c_ik(r)=\tilde v_{at}(0)\mathrm{e}^{-(r/r_k)^{\lambda}}+[1-k(r)]v_{at}(r)+c_i\mathrm{e}^{-(r/r_k)^{\lambda}}$$
\begin{equation}
	w_i(r)=\tilde v_{\mathrm{at}}(0)\mathrm{e}^{-(r/r_k)^{\lambda}}+[1-\mathrm{e}^{-(r/r_k)^{\lambda}}]v_{\mathrm{at}}(r)+c_i\mathrm{e}^{-(r/r_k)^{\lambda}}
	\label{eq:PAW-Blochl-19}
\end{equation}
其中$\tilde v_{\mathrm{at}}$是用多项式近似的原子赝势,参数$r_k$和$\lambda$由条件在$r_c$外赝势与原子势相等确定。赝分波函数由方程
\begin{equation}
	\bigg(-\dfrac12\nabla^2+w_i(r)-\varepsilon_i^1\bigg)|\tilde\phi_i\rangle=0
	\label{eq:PAW-Blochl-20}
\end{equation}
确定。式\eqref{eq:PAW-Blochl-19}的系数$c_i$根据赝分波函数$\tilde\phi_i$与分波函数$\phi_i$在$r_c$外相等确定。
\begin{figure}[h!]
\centering
\includegraphics[height=2.6in,width=3.2in,viewport=0 0 570 545,clip]{PAW-partical.png}
\caption{\small \textrm{The partical-wave of Fe atom.}}%(与文献\cite{EPJB33-47_2003}图1对比)
\label{PAW_partical_Fe}
\end{figure}

按照\textrm{Bl\"ochl}建议的投影函数的构造方案,初始形式与超软赝势的辅助函数计算类似:
\begin{equation}
	|\tilde p_i\rangle=\bigg(-\dfrac12\nabla^2+\tilde v_{at}-\varepsilon_i^1\bigg)|\tilde\phi_i\rangle
	\label{eq:PAW-Blochl-21}
\end{equation}
考虑到投影函数与赝分波函数的正交性$\langle\tilde p_i|\tilde\phi_j\rangle=\delta_{ij}$,对于给定下标的投影函数$\tilde p_i$,要求与所有$j<i$的赝分波函数正交,因此\textrm{Bl\"ochl}采用\textrm{Gram-Schmidt}正交化方法,得到正交的投影函数
$$|\tilde p_i\rangle=|\tilde p_i\rangle-\sum_{j=1}^{i-1}|\tilde p_j\rangle\langle\tilde\phi_j|\tilde p_i\rangle$$
%	$$|\phi_i\rangle=|\phi_i\rangle-\sum_{j=1}^{i-1}|\phi_j\rangle\langle\tilde p_j|\tilde\phi_i\rangle$$
%	$$|\tilde\phi_i\rangle=|\tilde\phi_i\rangle-\sum_{j=1}^{i-1}|\tilde\phi_j\rangle\langle\tilde p_j|\tilde\phi_i\rangle$$
在此基础上,分波函数与赝分波函数也分别与投影函数正交
$$|\phi_i\rangle=|\phi_i\rangle-\sum\limits_{j=1}^{i-1}|\phi_j\rangle\langle\tilde p_j|\tilde\phi_i\rangle$$
$$|\tilde\phi_i\rangle=|\tilde\phi_i\rangle-\sum\limits_{j=1}^{i-1}|\tilde\phi_j\rangle\langle\tilde p_j|\tilde\phi_i\rangle$$
最终作为基函数的$\phi_i$、$\tilde\phi_i$、$\tilde p_i$是通过迭代得到的。
%\begin{figure}[h!]
%\centering
%\vspace*{-0.4in}
%\includegraphics[height=1.5in,width=2.3in,viewport=0 0 1100 745,clip]{PAW_projector-2.png}
%\caption{\small \textrm{The projector of PAW.}}%(与文献\cite{EPJB33-47_2003}图1对比)
%\label{PAW_projector}
%\end{figure}
%\begin{itemize}
%	\item 与分波具有相同的角动量$l$
%	\item 局域在缀加区域(Augmentation region)
%	\item 节点依次增加
%\end{itemize}

与赝势的去屏蔽过程类似,%在得到赝势$\tilde v_{\mathrm{at}}$的基础上,可以计算
缀加区的局域势表示形式为:
\begin{equation}
	\bar v(r)=\tilde v_{at}(r)-\int\mathrm{d}r^{\prime}\dfrac{\tilde n(r^{\prime})+\hat n(r^{\prime})}{|r-r^{\prime}|}-\mu_{xc}[\tilde n(r)]
	\label{eq:PAW-Blochl-22}
\end{equation}
注意到在缀加区,式\eqref{eq:PAW-Blochl-22}由束缚态计算得到,计算交换-相关势的贡献时,赝电荷密度$\tilde n(r)$包括赝芯电荷密度$\tilde n^c$和赝价电子密度。其中赝芯电荷的构造与赝势方法相同,价电子密度来自束缚态赝分波函数$|\tilde\Psi_j\rangle$,它由方程\cite{PRB50-17953_1994}
%中的赝电荷密度$\tilde n(r)$由赝分波波函数$|\tilde\Psi_j\rangle$确定。对于束缚态,
\begin{equation}
	\bigg[-\dfrac12\nabla^2+\tilde v_{at}-\varepsilon+\sum_{(i,j)}|\tilde p_i\rangle\big(\mathrm{d}H_{ij}-\varepsilon\mathrm{d}O_{ij}\big)\langle\tilde p_j|\bigg]|\tilde\Psi_j\rangle=0
	\label{eq:PAW-Blochl-23}
\end{equation}
确定,式\eqref{eq:PAW-Blochl-23}中$\mathrm{d}H_{ij}$和$\mathrm{d}O_{ij}$分别是
$$\mathrm{d}H_{ij}=\langle\phi_i|-\dfrac12\nabla^2+v_{at}|\phi_j\rangle-\langle\tilde\phi_i|-\dfrac12\nabla^2+\tilde v_{at}|\tilde\phi_j\rangle$$
$$\mathrm{d}O_{ij}=\langle\phi_i|\phi_j\rangle-\langle\tilde\phi_i|\tilde\phi_j\rangle$$
式\eqref{eq:PAW-Blochl-23}的求解,参见文献\inlinecite{PRB50-17953_1994}。

上述讨论不难看出,\textrm{PAW}方法的原子信息计算与赝势方法很相似,其中包括
\begin{itemize}
	\item 缀加局域区截断半径$r_c$
	\item 芯电荷密度$n^c$和赝芯电荷密度$\tilde n^c$
	\item (价)电子分波函数$|\phi_i\rangle$和赝分波函数$|\tilde\phi_i\rangle$
	\item 矩阵元$\langle\phi_i|-\dfrac12\nabla^2|\phi_j\rangle-\langle\tilde\phi_i|-\dfrac12\nabla^2|\tilde\phi_j\rangle$和$\langle\phi_i|\phi_j\rangle-\langle\tilde\phi_i|\tilde\phi_j\rangle$
\end{itemize}
%如果解$$|\tilde\Psi\rangle=|u\rangle+\sum_i|w_i\rangle c_i$$
%其中$|u\rangle$和$|w_i\rangle$的定义分别为
%$$\big(-\frac12\nabla^2+\tilde v-\varepsilon\big)|u\rangle=0$$
%$$\big(-\frac12\nabla^2+\tilde v-\varepsilon\big)|w_i\rangle=|\tilde p_i\rangle$$
%因此可以得到
%$$c_i=-\sum_{j,l}\bigg[\delta_{ij}+\sum_k\mathrm{d}H_{ik}-\varepsilon\mathrm{d}O_{ik}\langle\tilde p_k|w_j\rangle\bigg]^{-1}\big(\mathrm{d}H_{jl}-\varepsilon\mathrm{d}O_{jl}\big)\langle p_l|u\rangle$$

%\frame
%{
%%	\frametitle{\textrm{PAW}原子数据集}
%	\frametitle{\textrm{PAW Augmentation}}
%\begin{figure}[h!]
%\centering
%\includegraphics[height=2.3in,width=4.0in,viewport=0 0 1280 745,clip]{Figures/PAW-baseset.png}
%\caption{\small \textrm{The Augmentation of PAW.}}%(与文献\cite{EPJB33-47_2003}图1对比)
%\label{PAW_baiseset}
%\end{figure}
%}

%\frame
%{
%	\frametitle{\textrm{PAW Augmentation}}
%\begin{figure}[h!]
%\centering
%\includegraphics[height=2.3in,width=4.0in,viewport=0 0 1100 745,clip]{Figures/PAW-projector.png}
%\caption{\small \textrm{The projector of PAW.}}%(与文献\cite{EPJB33-47_2003}图1对比)
%\label{PAW_projector}
%\end{figure}
%}

\subsection{\rm{PAW}方法与\rm{USPP}的内在联系}
%\begin{figure}[h!]
%\centering
%\includegraphics[height=2.3in,width=4.0in,viewport=0 0 1280 745,clip]{PAW-baseset.png}
%\caption{\small \textrm{The Augmentation of PAW.}}%(与文献\cite{EPJB33-47_2003}图1对比)
%\label{PAW_baiseset}
%\end{figure}
%
%\frame
%{
%	\frametitle{\textrm{PAW Augmentation}}
%\begin{figure}[h!]
%\centering
%\includegraphics[height=2.3in,width=4.0in,viewport=0 0 1100 745,clip]{Figures/PAW-projector.png}
%\caption{\small \textrm{The projector of PAW.}}%(与文献\cite{EPJB33-47_2003}图1对比)
%\label{PAW_projector}
%\end{figure}
%}
\textrm{PAW}方法在提出后的很长一段时间内都没有得到足够重视,直到\textrm{G. Kresse}等\cite{PRB59-1758_1999}将\textrm{Bl\"ochl}的原始方案中电荷密度计算方法作出调整,明确了\textrm{PAW}方法与\textrm{USPP}方法的内在联系后,特别是伴随着\textrm{Kresse}等将\textrm{PAW}方法引入第一原理分子动力学模拟软件包\textrm{VASP~(Vienna Ab-initio Simulation Package)}\cite{VASP_manual}中,有力地推动了\textrm{PAW}方法的广泛应用。

\textrm{Kresse}方案中将
\begin{itemize}
	\item 芯层电荷与核电荷构成离子实电荷:$n_{Zc}=n_Z+n_c$
	\item 赝离子实电荷的构造$$\int_{\Omega_c}n_{Zc}(\vec r)\mathrm{d}^3\vec r=\int_{\Omega_c}\tilde n_{Zc}(\vec r)\mathrm{d}^3\vec r$$
\end{itemize}
在此基础上,\textrm{Bl\"ochl}方案中的电荷可以分解为:
\begin{equation}
	\begin{aligned}
		n_T=n+n_{Zc}\equiv&\underbrace{(\tilde n+\hat n+\tilde n_{Zc})}\\
				 		&\quad\qquad\tilde n_T\\
				  &+\underbrace{(n^1+\hat n+n_{Zc})}-\underbrace{(\tilde n^1+\hat n+\tilde n_{Zc})}\\
				                  &\quad\qquad n_T^1\qquad\qquad\qquad\tilde n_T^1
	\end{aligned}
\end{equation}
\textcolor{red}{注意}:\textrm{G. Kresse}方案中补偿电荷$\hat n$局域在每个缀加球内。

\begin{equation}
	\begin{aligned}
		\dfrac12(n_T)(n_T)=&\dfrac12(\tilde n_T)(\tilde n_T)+(n_T^1-\tilde n_T^1)(\tilde n_T)\\
				&+\dfrac12(n_T^1-\tilde n_T^1)(n_T^1-\tilde n_T)
	\end{aligned}
\end{equation}
这里$$(a)(b)=\int\mathrm{d}\vec r\mathrm{d}\vec r^{\prime}\dfrac{a(\vec r)b(\vec r\,^{\prime})}{|\vec r-\vec r\,^{\prime}|}$$
\textcolor{red}{近似}:$\tilde n_T$用$\tilde n_T^1$替换:
\begin{equation}
	\dfrac12(n_T)(n_T)=\dfrac12(\tilde n_T)(\tilde n_T)-\dfrac12\overline{(n_T^1(\tilde n_T^1)}+\dfrac12\overline{(n_T^1)(n_T^1)}
\end{equation}

由于交换-相关能泛函是非线性的,\textrm{G. Kresse}方案中电荷密度分解为
\begin{equation}
	n_c+n=(\tilde n+\hat n+\tilde n_c)+(n^1+n_c)-(\tilde n^1+\hat n+\tilde n_c)
\end{equation}
原始的\textrm{Bl\"ochl}方案中电荷分解为
\begin{equation}
	n_c+n=(\tilde n)+(n^1+n_c)-(\tilde n^1)
\end{equation}
\textcolor{blue}{两种不同的电荷密度分解方案根源}:\\\textrm{G. Kresse}方案中赝离子实电荷$\tilde n_{Zc}$与\textrm{Bl\"ochl}方案中$\tilde n_c$的约束条件不同!
\begin{equation}
	E_{\mathrm{XC}}[\tilde n+\hat n+\tilde n_c]+\overline{E_{\mathrm{XC}}[n^1+n_c]}-\overline{E_{\mathrm{XC}}[\tilde n^1+\hat n+\tilde n_c]}
\end{equation}

	根据总能量表达式$$E=\tilde E+E^1-\tilde E^1$$其中
	\begin{equation}
		\begin{aligned}
			\tilde E=&\sum_nf_n\langle\tilde\Psi_n|-\frac12\nabla^2|\tilde\Psi_n\rangle+E_{\mathrm{XC}}[\tilde n+\hat n+\tilde n_c]+E_H[\tilde n+\hat n]\\
			&+\int v_H[\tilde n_{Zc}][\tilde n(\vec r)+\hat n(\vec r)]\mathrm{d}\vec r+U(\vec R,Z_{\mathrm{ion}})\\
			\tilde E^1=&\sum_{(i,j)}\rho_{ij}\langle\tilde\phi_i|-\frac12\nabla^2|\tilde\phi_j\rangle+\overline{E_{\mathrm{XC}}[\tilde n^1+\hat n+\tilde n_c]}+\overline{E_H[\tilde n^1+\hat n]}\\
			&+\int_{\Omega_r}v_H[\tilde n_{Zc}][\tilde n^1(\vec r)+\hat n(\vec r)]\mathrm{d}\vec r
		\end{aligned}
	\end{equation}

	\begin{equation}
		\begin{aligned}
			E^1=&\sum_{(i,j)}\rho_{ij}\langle\phi_i|-\frac12\nabla^2|\phi_j\rangle+\overline{E_{\mathrm{XC}}[n^1+n_c]}+\overline{E_H[n^1]}\\
			&+\int_{\Omega_r}v_H[n_{Zc}]n^1(\vec r)\mathrm{d}\vec r
		\end{aligned}
	\end{equation}
	$v_H$是电荷密度$n$产生的静电势
	$$v_H[n](\vec r)=\int\dfrac{n(\vec r\,^{\prime})}{|\vec r-\vec r\,^{\prime}|}\mathrm{d}\vec r\,^{\prime}$$
	$E_H[n]$是对应的静电能
	$$E_H[n]=\dfrac12(n)(n)=\dfrac12\int\mathrm{d}\vec r\mathrm{d}\vec r\,^{\prime}\dfrac{n(\vec r)n(\vec r\,^{\prime})}{|\vec r-\vec r\,^{\prime}|}$$ 
	$U(\vec R,Z_{\mathrm{ion}})$\textcolor{blue}{由\textrm{Ewald}求和计算}

	根据约束条件
	\begin{equation}
		\int_{\Omega_c}(n^1-\tilde n^1-\hat n)|\vec r-\vec R|^lY_{lm}^{\ast}(\widehat{\vec r-\vec R})\mathrm{d}\vec r=0
	\end{equation}
	定义电荷密度差
	\begin{equation}
		Q_{ij}(\vec r)=\phi_i^{\ast}(\vec r)\phi_j(\vec r)-\tilde\phi_i^{\ast}(\vec r)\tilde\phi_j(\vec r)
	\end{equation}
	电荷密度差的多极矩为
	\begin{equation}
		q_{ij}^L(\vec r)=\int_{\Omega_r}Q_{ij}(\vec r)|\vec r-\vec R|^lY_{lm}^{\ast}(\widehat{\vec r-\vec R})\mathrm{d}\vec r
	\end{equation}
	因此,补充电荷的计算为:
	\begin{equation}
		\begin{aligned}
			\hat n=\sum_{(i,j),L}\sum_n f_n\langle\tilde\Psi_n|\tilde p_i\rangle\langle\tilde p_j|\Psi_n\rangle\hat Q_{ij}^L(\vec r)\\
			\hat Q_{ij}^L(\vec r)=q_{ij}^Lg_l(|\vec r-\vec R|)Y_{lm}(\widehat{\vec r-\vec R})
		\end{aligned}
	\end{equation}

重叠矩阵
	\begin{equation}
		\langle\tilde\Psi_n|\mathbf{S}|\tilde\Psi_m\rangle=\delta_{nm}
	\end{equation}
	其中重叠矩阵$$S[\{\mathbf{R}\}]=1+\sum_i|\tilde p_i\rangle q_{ij}\langle\tilde p_j|$$
	而$$q_{ij}=\langle\phi_i|\phi_j\rangle-\langle\tilde\phi_i|\tilde\phi_j\rangle$$
	\textrm{Hamiltonian}的计算
	\begin{equation}
		H[\rho,\{\mathbf{R}\}]=-\dfrac12\nabla^2+\tilde v_{eff}+\sum_{(i,j)}|\tilde p_i\rangle(\hat D_{ij}+D_{ij}^1-\tilde D_{ij}^1)\langle\tilde p_j|	
	\end{equation}
	$$\tilde v_{eff}=v_H[\tilde n+\hat n+\tilde n_{Zc}]+v_{\mathrm{XC}}[\tilde n+\hat n+\tilde n_{Zc}]$$

	$$\hat D_{ij}=\dfrac{\partial\tilde E}{\partial\rho_{ij}}=\int\dfrac{\delta\tilde E}{\delta\hat n(\vec  r)}\dfrac{\partial\hat n(\vec r)}{\partial\rho_{ij}}\mathrm{d}\vec r=\sum_{L}\int\tilde v_{eff}\hat Q_{ij}^L(\vec r)\mathrm{d}\vec r$$
	$$D_{ij}^1=\dfrac{\partial E^1}{\partial\rho_{ij}}=\langle\phi_i|-\dfrac12\nabla^2+v_{eff}^1|\phi_j\rangle$$
	其中$$v_{eff}^1[n^1]=v_H[n^1+n_{Zc}]+v_{\mathrm{XC}}[n^1+n_c]$$
	$$\tilde D_{ij}^1=\dfrac{\partial\tilde E^1}{\partial\rho_{ij}}=\langle\tilde\phi_i|-\dfrac12\nabla^2+\tilde v_{eff}^1|\tilde\phi_j\rangle+\sum_L\int_{\Omega_r}\mathrm{d}\vec r\tilde v_{eff}^1(\vec r)\hat Q_{ij}^L$$
	其中$$\tilde v_{eff}^1[\tilde n^1]=v_H[\tilde n^1+\hat n+\tilde n_{Zc}]+v_{\mathrm{XC}}[\tilde n^1+\hat n+\tilde n_c]$$

	能带计算中,总能量可通过\textrm{Kohn-Sham}本征值求和扣除\textrm{Double counting}计算更方便,其中修正项
	\begin{equation}
		\begin{aligned}
			\tilde E_{dc}=&-E_H[\tilde n+\hat n]+E_{\mathrm{XC}}[\tilde n+\hat n+\tilde n_c]\\
			&-\int v_{\mathrm{XC}}[\tilde n+\hat n+\tilde n_c](\tilde n+\hat n)\mathrm{d}\vec r\\
			E_{dc}^1=-\overline{E_H[n^1]}&+\overline{E_{\mathrm{XC}}[n^1+n_c]}-\int_{\Omega_r}v_{\mathrm{XC}}[n^1+n_c]n^1\mathrm{d}\vec r\\
			\tilde E_{dc}^1=&-\overline{E_H[\tilde n^1+\hat n]}+\overline{E_{\mathrm{XC}}[\tilde n^1+\hat n+\tilde n_c]}\\
			&-\int v_{\mathrm{XC}}[\tilde n^1+\hat n+\tilde n_c](\tilde n^1+\hat n)\mathrm{d}\vec r
		\end{aligned}
	\end{equation}
	因此总能量的计算表达式是
	$$E=\sum_nf_n\langle\tilde\Psi_n|H|\tilde\Psi_n\rangle+\tilde E_{dc}+E_{dc}^1-\tilde E_{dc}^1+U(\vec R,Z_{\mathrm{ion}})$$

\subsection{\rm{PAW~}原子数据集}
	平滑赝原子分波函数
	\begin{equation}
		\tilde\phi_{i=Lk}(\vec r)=Y_L(\widehat{\vec r-\vec R})\tilde\phi_{lk}(|\vec r-\vec R|)
	\end{equation}
	根据\textrm{RRKJ}赝势构造,赝分波函数由球\textrm{Bessel}函数线性组合
	\begin{equation}
		\tilde\phi_{lk}(r)=\left\{
		\begin{aligned}
			&\sum_{i=1}^2\alpha_ij_l(q_ir)\quad &r<r_c^l\\
			&\phi_{lk}(r)\quad&r>r_c^l
		\end{aligned}
		\right.
	\end{equation}
	调节系数$\alpha_i$和$q_i$赝分波函数$\phi_{lk}(r)$在截断半径$r_c^l$处两阶连续可微
	投影子波函数$\tilde p_i$由\textrm{Gram-Schmidt}正交条件$\langle\tilde p_i|\tilde\phi_j\rangle=\delta_{ij}$确定

	\textcolor{blue}{构造原子局域赝势$\tilde v_{eff}^a$}(\textcolor{red}{为防止\textrm{ghost band}}):\\在截断半径$r_{loc}$内的定义为
	$$\tilde v_{eff}^a=A\dfrac{\sin(q_{loc}r)}r\quad r<r_{loc}$$
	其中$q_{loc}$和$A$要求局域赝势在截断半径$r_{loc}$处连续到一阶导数

	\textcolor{blue}{构造赝芯电荷密度$\tilde n_c$}:~在截断半径$r_{pc}$内的定义为
	$$\sum_{i=1,2}B_i\dfrac{\sin(q_ir)}r\quad r<r_{pc}$$
	调节系数$q_i$和$B_i$使得赝芯电荷密度$\tilde n_c(r)$在截断半径$r_{pc}$处的两阶导数连续

	局域离子赝势$v_H[\tilde n_{Zc}]$可由原子局域赝势去屏蔽得到
	$$v_H[\tilde n_{Zc}]=\tilde v_{eff}^a-v_H[\tilde n_a^1+\hat n_a]-v_{\mathrm{XC}}[\tilde n_a^1+\hat n_a+\tilde n_c]$$
	在\textrm{VASP}的\textrm{POTCAR}生成过程中,各截断半径的确定条件
	$r_{rad}=\max({r_c^l})$,$r_{pc}\approx r_{rad}/1.2$,$r_{loc}<r_{rad}/1.2$

	在每个原子球内用球\textrm{Bessel}函数构造补偿电荷$g_l(r)$
	$$g_l(r)=\sum_{i=1}^2\alpha_i^lj_l(q_i^lr)$$
	调节系数$q_i^l$和$\alpha_i^l$使得补偿电荷$g_l(r)$在截断半径$r_{comp}$处的数值和前两阶导数值都是0,因此可以选择$q_i^l$使得多极矩
	$$\int_0^{r_{comp}}g_l(r)r^{l+2}\mathrm{d}r=1$$
	并且有
	$$\dfrac{\mathrm{d}}{\mathrm{d}r}j_l(q_i^lr)\bigg|_{r_{comp}}=0$$
	设置$\alpha_i^l$,因此$g_l(r_{comp})=0$,$r_{comp}=r_{rad}/1.3\sim r_{rad}/1.2$

\begin{figure}[h!]
	\vspace{-0.2in}
\centering
%\includegraphics[height=2.7in,width=4.0in,viewport=0 0 1180 875,clip]{Figures/dual_grid.png}
\includegraphics[height=2.7in,width=4.0in,viewport=0 0 800 600,clip]{dual_grid-2.png}
\caption{\small \textrm{The Schematic description of the dual grid technique.}}%(与文献\cite{EPJB33-47_2003}图1对比)
\label{PAW_baiseset}
\end{figure} 

\begin{figure}[h!]
	\vspace{-0.2in}
\centering
\includegraphics[height=2.3in,width=2.3in,viewport=0 0 420 420,clip]{VASP_G_max.png}
\caption{{\textrm{The small sphere contains all plane waves included in the basis set $\vec G<\vec G_\mathrm{cut}$. The charge density contains components up to $2\vec G$ cut (second sphere), and the acceleration a components up to $3\vec G$ cut , which are reflected in (third sphere) because of the finite size of the FFT-mesh. Nevertheless the components a $\vec G$ with $|\vec G|<\vec G$ cut are correct i.e. the small sphere does not intersect with the third large sphere.}}}%(与文献\cite{EPJB33-47_2003}图1对比)
\label{VASP_G_max}
\end{figure} 

%\frame
%{
%\frametitle{发展统一理论框架下的材料计算程序}
%\begin{itemize}
%	\item
%\end{itemize}
%}

\subsection{\rm{PAW}方法与\rm{US-PP}的关系}
	根据\textrm{D. Vanderbilt}的超软赝势(\textrm{US-PP})方案
	原子波函数满足$$(T+V_{\mathrm{AE}}-\varepsilon_n)|\psi_n\rangle=0$$
	据此构造原子赝波函数$\phi_n$,在截断半径$r_c^l$处,$\phi_n$与$\psi_n$平滑衔接(不需要模守恒条件)

	类似地,构造局域平滑势$V_{loc}(r)$,在截断半径$r_c^{loc}$处,$V_{loc}(r)$与$V_{\mathrm{AE}}(r)$平滑衔接
	
	构造辅助轨道
	$$|\chi_n\rangle=(\varepsilon_n-T-V_{loc})|\phi_n\rangle$$
	由此构造内积矩阵,矩阵元$$B_{nm}=\langle\phi_n|\chi_m\rangle$$
	并有$$|\beta_n\rangle=\sum_m(B^{-1})_{mn}|\chi_m\rangle$$
	这里$\beta_n$是局域函数,并与$\phi_n$垂直

	定义\textcolor{orange}{缀加函数}$$Q_{nm}(\vec r)=\psi_n^{\ast}(\vec r)\psi_m(\vec r)-\phi_n^{\ast}(\vec r)\phi_m(\vec r)$$
	\textcolor{blue}{可赝化的补偿电荷}$$q_{nm}=\langle\psi_n|\psi_m\rangle_R-\langle\phi_n|\phi_m\rangle_R$$
	由此可以导出$\phi_n$满足久期方程
	\begin{equation}
		\begin{aligned}
			&\left(T+V_{loc}+\sum_{nm}D_{nm}|\beta_n\rangle\langle\beta_m|\right)|\phi_n\rangle\\
			=&\varepsilon_n\left(1+\sum_{nm}q_{nm}|\beta_n\rangle\langle\beta_m|\right)|\phi_n\rangle
		\end{aligned}
	\end{equation}
	其中$$D_{nm}=B_{nm}+\varepsilon q_{nm}$$

	在超软赝势方法中,包含$N_v$个价电子体系的总能量\upcite{PRB47-10142_1993}
	\begin{equation}
		\begin{aligned}
			E_{\mathrm{tot}}[\{\phi_i\},\{\vec R_I\}]=&\sum_i\langle\phi_i|-\frac12\nabla^2+V_{\mathrm{NL}}|\phi_i\rangle\\
			&+\frac12\iint\mathrm{d}\vec r\mathrm{d}\vec r\,^{\prime}\dfrac{n(\vec r)n(\vec r\,^{\prime})}{|\vec r-\vec r\,^{\prime}|}\\
			&+\int\mathrm{d}\vec r V_{loc}^{\mathrm{ion}}(\vec r)n(\vec r)+E_{\mathrm{XC}}[n]\\
			&+U(\{\vec R_I\})
		\end{aligned}
	\end{equation}
这里$\phi$是体系波函数,$n(\vec r)$是电子密度,$E_{\mathrm{XC}}$是交换-相关能,$U(\{\vec R_I\})$是离子-离子相互作用能

	电荷密度$$n(\vec r)=\sum_i\big[|\phi_i(\vec r)|^2+\sum_{nm,I}Q_{nm}^I(\vec r)\langle\phi_i|\beta_n^I\rangle\langle\beta_m^I|\phi_i\rangle\big]$$
	局域势$$V_{loc}^{\mathrm{ion}}(\vec r)=\sum_IV_{loc}^{\mathrm{ion}}(\vec r-\vec R_I)$$
	$V_{loc}^{\mathrm{ion}}$由$V_{loc}$去屏蔽后得到$$V_{loc}^{\mathrm{ion}}(r)=V_{loc}(r)-\int\mathrm{d}\vec r\,^{\prime}\dfrac{n(\vec r\,^{\prime})}{|\vec r-\vec r\,^{\prime}|}-\mu_{\mathrm{XC}}(r)$$
	非局域部分$$V_{\mathrm{NL}}=\sum_{nm,I}D_{nm}^{(0)}|\beta_n^I\rangle\langle\beta_m^I|$$
	这里$$D_{nm}^{(0)}=D_{nm}-\int\mathrm{d}\vec r\,^{\prime}V_{loc}(\vec r\,^{\prime})n(\vec r\,^{\prime})$$

	\textrm{G. Kresse}指出只要总能量表达式中$E^1$和$\tilde E^1$在原子构象附近作线性化即可得到\textrm{US-PP}的表达式。
	
	如果原子构象由占据数$\rho_{ij}^a$、电荷密度分布$n_a^1$、$\tilde n_a^1$和$\hat n_a$确定,$E^1$中的\textcolor{blue}{\underline{\textrm{Hartree}能和交换-相关能项}}在$n_a^1$附近线性化得
	\begin{equation}
		E_{\mathrm{XC}}(\textcolor{red}{n_a^1+n_c})+E_H(\textcolor{red}{n_a^1})+\int(v_{\mathrm{XC}}[n_a^1+n_c]+v_H[n_a^1])\underline{\textcolor{red}{[n^1(\vec r)-n_a^1(\vec r)]}\mathrm{d}\vec r}
	\end{equation}
	在\textrm{PAW}方法中,电子密度$n^1(\vec r)$%和$\tilde n^1(\vec r)$
	的表达式
	\begin{equation}
		\begin{aligned}
			n^1(\vec2 r)=&\sum_{(i,j)}\rho_{ij}\langle\phi_i|\vec r\rangle\langle\vec r|\phi_j\rangle\\
	%		\tilde n^1(\vec r)=&\sum_{(i,j)}\rho_{ij}\langle\tilde\phi_i|\vec r\rangle\langle\vec r|\tilde\phi_j\rangle 
		\end{aligned}
	\end{equation}
	因此,\textrm{Hartree}能和交换-相关能项为
	$$\textcolor{blue}{C}+\sum_{(i,j)}\rho_{ij}\langle\phi_i|\textcolor{red}{v_{\mathrm{XC}}[n_a^1+n_c]+v_H[n_a^1]}|\phi_j\rangle$$
	这里\textcolor{blue}{\textrm{C}}是常数

	因此$E^1$用占据数$\rho_{ij}$近似到一阶可有
	\begin{equation}
		E^1\approx\textcolor{blue}{C}+\sum_{(i,j)}\langle\phi_i|-\frac12\nabla^2+v_{eff}^a|\phi_j\rangle
	\end{equation}
	其中
	\begin{equation}
		v_{eff}^a=\textcolor{red}{v_H[n_a^1+}\textcolor{blue}{n_{Zc}}\textcolor{red}{]+v_{\mathrm{XC}}[n_a^1+n_c]}
	\end{equation}
	这里\textcolor{red}{$v_{eff}^a$}是\textcolor{blue}{原子构象下的全电子有效势}

	类似地可有
	\begin{equation}
		\tilde E^1\approx\textcolor{blue}{\tilde C}+\sum_{(i,j)}\left[\langle\tilde\phi_i|-\frac12\nabla^2+\tilde v_{eff}^a|\tilde\phi_j\rangle+\underline{\int\tilde Q_{ij}^L(\vec r)\tilde v_{eff}^a(\vec r)\mathrm{d}\vec r}\right]
	\end{equation}
	其中$$\tilde v_{eff}^a=v_H[\tilde n_a^1+\textcolor{blue}{\hat n_a}+\tilde n_{Zc}]+v_{\mathrm{XC}}[\tilde n_a^1+\textcolor{blue}{\hat n_a}+\tilde n_c]$$
	这里\textcolor{red}{$\tilde v_{eff}^a$}是\textcolor{blue}{原子构象下的局域原子赝势}

	在此近似下,包含$\tilde E$可得体系总能量$E$:
	\begin{equation}
		\begin{aligned}
			E=&\sum_nf_n\langle\tilde\Psi_n|-\frac12\nabla^2+\sum_{(ij)}|\tilde p_i\rangle\langle\tilde p_j|G_{ij}^{\mathrm{US}}|\tilde\Psi_n\rangle\\
			&+E_{\mathrm{XC}}[\tilde n+\hat n+\tilde n_c]+E_H[\tilde n+\hat n]\\
			&+\int v_H[\tilde n_{Zc}][\tilde n(\vec r)+\hat n(\vec r)]\mathrm{d}\vec r+U(\vec R,Z_{\mathrm{ion}})
		\end{aligned}
	\end{equation}
	其中
	\begin{equation}
		\begin{aligned}
			G_{ij}^{\mathrm{US}}=&\langle\phi_i|-\frac12\nabla^2+v_{eff}^a|\phi_j\rangle-\langle\tilde\phi_i|-\frac12\nabla^2+\tilde v_{eff}^a|\tilde\phi_j\rangle\\
			&-\int\hat Q_{ij}^L(\vec r)\tilde v_{eff}^a(\vec r)\mathrm{d}\vec r
		\end{aligned}
	\end{equation}
	\textcolor{red}{当补偿电荷$\hat n$用\textrm{US-PP}方案的赝化补偿电荷表示时,$E\rightarrow E_{\mathrm{tot}}$}

	在$G_{ij}^{\mathrm{US}}$中,
	\begin{equation}
		\begin{aligned}
			&\textcolor{blue}{\langle\phi_i|-\frac12\nabla^2+v_{eff}^a|\phi_j\rangle-\langle\tilde\phi_i|-\frac12\nabla^2+\tilde v_{eff}^a|\tilde\phi_j\rangle}\\
			\rightarrow&\textcolor{red}{D_{nm}=B_{nm}+\varepsilon_mq_{nm}}
		\end{aligned}
	\end{equation}
	\begin{equation}
		\textcolor{blue}{\int\hat Q_{ij}^L(\vec r)\tilde v_{eff}^a(\vec r)\mathrm{d}\vec r}\rightarrow\textcolor{red}{D_{nm}^{(0)}=D_{nm}-\int\mathrm{d}\vec r\,^{\prime}V_{loc}(\vec r\,^{\prime})n(\vec r\,^{\prime})}
	\end{equation}

	在\textrm{PAW}方法能量泛函中,如果缀加区补偿电荷$\hat n$满足$$\hat n=n^1-\tilde n^1$$
	并且如果满足$\tilde n_{Zc}=n_{Zc}$和$\tilde n_c=n_c$,则“在位”(\textrm{on-site})动能对总能的贡献
	\begin{equation}
		E^1-\tilde E^1=\sum_{(i,j)}\rho_{ij}\big(\langle\phi_i|-\frac12\nabla^2|\phi_j\rangle-\langle\tilde\phi_i|-\frac12\nabla^2|\tilde\phi_j\big)
	\end{equation}

	\textcolor{red}{在这种极限情况下\textrm{PAW}与\textrm{US-PP}完全等价}

	在\textrm{US-PP}中,\textcolor{orange}{缀加函数}满足条件
	$$\hat Q_{ij}^L(\vec r)=Q_{ij}(\vec r)=\phi_i^{\ast}(\vec r)\phi_j(\vec r)-\tilde\phi_i^{\ast}(\vec r)\tilde\phi_j(\vec r)$$
	上述推导表明:~\textcolor{blue}{在\textrm{US-PP}方案中,可以通过提高\underline{赝化缀加函数}的精度,系统提升总能量的计算精度}

\begin{figure}[h!]
	\vspace{-0.2in}
\centering
%\includegraphics[height=2.7in,width=4.0in,viewport=0 0 1300 960,clip]{Figures/VASP_procedure-full.png}
\includegraphics[height=2.7in,width=2.2in,viewport=0 0 480 630,clip]{VASP_procedure.png}
\caption{\small \textrm{The Flow of calculation for the KS-ground states.}}%(与文献\cite{EPJB33-47_2003}图1对比)
\label{PAW_baiseset}
\end{figure} 

