王守竞(1904-1984)
王守竞,我国早期著名理论物理学家。他在量子力学刚刚兴起的 1928—1929年,相继发表三篇至今仍被广泛引用的重要论文,享營国际物理学界。1929年归国后,先后担任浙江大学物理系、北京大学物理学系主任,为近代物理在我国的传播做出重要贡献。他是中国物理学会创始人之一,长期担任中国物理学会评议员及理事,为中国物理学会的建设做出重要贡献。他从20世纪30年代中期开始转入工业界,筹办并领导了中央机器厂,为抗日战争胜利做出重大贡献。他传奇的一生,体现了我国爱国知识分子为追求国家富强而不奋斗的高贵精神,值得后辈永远铭记。

王守竞,物理学家、教育家和企业家,中国近代物理学先驱,中国现代机械工业开拓者之一,中国物理学会创始人之一。1924年赴美留学,1928年获哥伦比亚大学博士学位。在美期间,王守竞先后发表了《两个氢原子的相互作用》《新量子力学中的正常氢分子问题》和《论量子力学中的非对称陀螺》等开创性论文,是早期对量子力学应用做出重要成果的唯一中国物理学家。1929年回国后任浙江大学物理系教授、系主任。1931年任北京大学物理系教授、研究教授、讲座教授和系主任、物理研究所所长。1931年当选为中国物理学会临时执行委员会委员。1933年任中国物理学会副会长和《物理学报》编委、译名委员会委员。全面抗战前夕,为了国家的需要,王守竟从科教界转入了工业界,走上了“工业救国”之路。1933年任军政部兵工署技术司光学组主任,1935年任资源委员会专员,1936年任机器制造厂筹委会主任,1939年任中央机器厂总经理。他在组建中国第一个光学工厂和机械工厂、支援抗日战争方面做出了突出贡献。1943年奉派赴美,先后担任资源委员会驻美办事处主任、中国驻美大使馆科技参赞、美国租借法案中方负责人、国民政府驻美物资供应委员会主任、中国石油有限公司董事等职务。1951年定居美国,后任职于美国林肯实验室。

赴美留学,学识俱进
王守竞,字井然,江苏苏州人,生于1904年1月15日(据王守竞传,生于1904年12月24日,即清光绪三十年十一月十八日)。其父王季同清末赴英,于牛津大学研习数学和物理,为民国初期科学界先驱。光绪年间,祖母谢长达在苏州创办振华女校,为国内妇女教育之前辈。姑母王季茝(芝加哥大学化学博士)于民国前去美国,入麻省韦尔斯利女校(Wellesley College),后在美工作50年,是美国公共卫生权威。
王守竞出生于笃实好学世家,天赋优异,聪颖过人。幼年受启蒙教育于苏州私立彭氏小学。1921年毕业于苏州工业专科学校,同年考入北京清华学校(留美预备学校、清华大学前身)。他一入学,才艺出众、头角峥嵘,被甲子级(1924年毕业班级)同学及师长一致公认为全校最优秀的学生。1924年秋毕业后即与同学周培源等赴美留学。到美国后,他选择的是东部著名大学康奈尔大学。
入学考试,王守竞成绩出众,被特许直升康奈尔大学研究院物理研究所攻读硕士。这在康奈尔大学历史上是少有的,在清华学校历史上更是绝无仅有。因为按照惯例,清华学校高等科毕业的学生到美后可插人美国大学二、三年级学习,大学毕业得了学士学位后再读两年研究生,才可得硕士学位。而像王守竞这样未重入大学部,直升研究院者,是清华学校自1909年到1929年 20年间保送赴美的留学生中唯一的一个。第二年夏天,不到一年的时间,王守竞就得到了(物理学)硕士学位,更令全校师生对这位中国学生刮目相看,钦佩不已。
王守竞是一个文理兼优、全面发展的学生。由于他刚好诞生在一个新旧交替的时代,所受的教育也刚好赶上中西交融的开始,他不但接受了传统的中国文化教育,也接受了一些西方文化的熏陶,再加上他天生聪颖又勤奋好学,所以他不但有深厚的中国文化基础,西方文化也有相当功底。在康奈尔大学获得物理学硕士学位后,他并不急于进而攻读博士学位,而是于1925年的秋天,转入哈佛大学研习欧洲文学。之所以弃理从文,一则是兴趣所在,二则是他的远见——为今后专攻和深攻物理做好语文表达方面的充分准备。他之所以选择哈佛大学,自然是考虑到了哈佛大学在美国的崇高地位和优越的学习条件。
仅一年时间,他又拿到了哈佛大学欧洲文学的硕士学位。文学修养大为提高,外语水平也大大提高了一步,无论是英文、德文、法文,均达到了写说俱佳、运用自如的程度,不仅在留学生中出类拔萃,就是与说这些语言的本国人士相比也毫不逊色。
20世纪20年代,世界科学中心仍在欧洲,而美国东部诸州得风气之先,吸引了诸多外来名教授。正是考虑到将来与这些欧洲名教授的交流,王守竞才专门去研习了一年欧洲文学,打下了扎实的基础。而他心中的主攻方向仍然是物理学。1926年夏,他转入外来名教授较多的哥伦比亚大学物理研究所,潜心钻研物理学。
在哥伦比亚大学期间,得名师的指导,学友的互助,年轻的王守竞锐意进取,时发妙想。他更加抱定了探索和理解自然奥秘的理想,并逐步形成了那不失赤子之心的天真和热爱自然而又好学深思、追求真理的纯真性格。由于他无论是文学修养还是外语水平都很高,并具备了丰富的物理和化学知识,掌握了熟练的数学技巧,因此,他很快就成为同学中的佼佼者。
攻读博士期间,王守竞于1927年在短期来访的著名荷兰物理学家、诺贝尔奖获得者德拜(P. Debye)建议下,采用了一个简单的平面模型,计算了两个氢原子的诱导偶极相互作用,首先得出两原子以距离负6 次方势相互吸引的结果,并定出此作用能量的表达式为:243/28(8u/R)。其中。为电子电荷,Q。为波尔半径,R。为两原子核距离。这项工作发表在当年的德国《物理学杂志》〔Phys. Z.28,663(1927)〕。他的这个结果,直到1930年才被艾森席兹(R. Eisenschitz)和伦敦(F. London)改进,并被命名为色散力或伦敦力。之后,他独立于海特勒(w. Heiller)等用变分法计算了氢分子的基态波函数和能级,由于采用了不同的试探波函数并引进新的变分参量,使能量计算值和实验值间的差异从1.58电子伏降到0.96电子伏。这项工作发表在次年的《物理评论》〔Phys. Rev.31, 579 (1928)〕 上,该工作是用量子力学解释化学中的共价键、开创量子化学的首批先驱性论文之一。在从事理论工作的同时,王守竞还跟随该系韦伯(H. W. Webb)教授从事了“钠蒸汽和汞原子碰撞中的激发态”的实验研究,研究结果发表在1929年的《物理评论》〔Phys. Rev. 33, 329 (1929)〕 上。
1928年夏,入哥伦比亚大学仅两年,王守竟即被授子博士学位。获得博士学位后,他得到美国国家研究委员会的资助,在威斯康星大学从事博士后研究一年。这一年他集中精力,解决了量子力学中的非对称陀螺问题,给出了非对称陀螺的能级公式和能级跃迁规则的严格结果。这项工作前期结果的摘要曾递交1928年 11 月30日至12月1 日召开的美国物理学会153 次例会,最后结果正式发表在《物理评论》〔Phys. Rex. 34, 243 (1929)]上。他的这项工作为非对称多原子分子转动光谱研究打下了理论基础,他所求出的能级公式被称为“王守竞方程”,至今仍被应用。

三篇论文,历史留痕
王守竞赴美留学,以不到三年的时光发表了三篇开创性的量子力学论文,成为参与量子力学初期发展并获得重要结果的唯一中国物理学家。吴大猷先生在《对中国早期物理学发展的回忆》中曾高度评价王守竞的这三篇论文,他说:“量子论在中国的发展,初期有一位王守竞先生(1904—1984),是一位很难得的聪明的年轻人,1924年从清华旧制的留美预备班的那种情形下到美国去的。在1927年、1928年这两年(1926年量子力学刚刚开始),王守竞就在这两年之内做了三篇可以说是很好很好的文章。”(根据王守竞第三篇量子力学论文收稿日期为1929年5月1日看,吴先生此处时间计算似不够准确,王守竞完成这三篇文章实际用了近两年半时间。)
王守竞这三篇论文好在何处?必须从它们在量子力学的发展中的作用和地位说起。
王守竞的第一篇论文《两个氢原子的相互作用力》解决了困惑了物理学界相当长时间的非极性原子间相互吸引的问题。早在1873年范德瓦耳斯导出有名的气体方程时,人们就一直在思考气体原子间吸引力的微观起源。1912年葛生(W. HL. Keesom)提出分子间吸引力的偶极 -偶极作用模型,算出原子间作用势与原子间距离R成R-°的结果,称为葛生力。但这个结果只适用于具有固有偶极矩的极性原子。1920年德拜提出一个原子的固有偶极矩可以诱导另一个原子偶极矩的模型,也得出原子间距离负6次方吸引势的结果,称为德拜诱导力。这个结果对没有固有偶极矩的原子同样不适用。对于没有固有偶极矩的原子,德拜曾提出诱导偶极-诱导偶极相互作用模型,但经典静电学计算的结果为零。1926年薛定谔建立波动力学不久,德拜在1927年访问哥伦比亚大学时遇到王守竞并建议他研究这一问题,王守竞经过不长时间的计算最后解决了这个问题。尽管这个力在后来的文献中被称为伦敦色散力,但近年来两位著名物理学史专家布拉希(S.G. Brush)和劳林森(JS. Rowinson)先后以确凿的史实(见Stephen G Brush. Statistical physics and the atomic theory of matterfrom Boyle and Newton to Landau and Onsager. Princeton: PrincetonUniversity Press, 1983:210; J.S. Rowlinson. Cohesion-A scientifichistory of intermoleeular forces, Cambridge: Cambridge UniversityPress,2002:236-237) 证明:是王守竞首先算出这个力,并公正地提出应当将所谓伦敦色散力改称为王守竞力。
王守竞的第二篇论文《量子力学中的正常氢分子》独立于略早于他的海特勒和伦敦,以精确的数学方法计算了氢分子的基态能级,给出了比前者更接近实验值的基态能量。尤其值得指出的是,王守竞在计算中首次引进了有效电荷Z作为变分参量,为以后研究和改进氢分子结构计算的研究者们所一致采用,而且他的结果满足量子位力定理,而海特勒一伦敦的结果却不满足。美国著名理论物理学家坎布尔(E. C.Kemble)、斯菜特(J.C. Slater)以及著名化学家、两次诺贝尔奖(化学奖、和平奖)获得者鲍林(L.Pauling)等人的专著中都对这项工作做了介绍,尤其是斯莱特用大篇幅叙述了王守竞对海特勒-伦敦工作的改进(见 John C. Slater. Quantum theory of molecules and solids.Vol 1. New York: MeGrall-Hhill, 1963:54 -59)。氢分子结构的量子力学计算在量子力学的应用中是具有里程碑意义的工作,正是从这项工作开始,量子化学得以创立。王守竞的这篇工作作为创立量子化学先驱性论文之一,已永久记载在科学史上。
王守竞的第三篇论文《论量子力学中的非对称陀螺》所涉及的是一项难度极高的理论工作,同时也是一项为非对称分子转动光谱研究奠定基础的工作。量子力学一创立,人们相继解出了解释双原子分子转动光谱的转子解以及解释对称多原子分子转动光谱的对称陀螺解,然而对于与非对称多原子分子转动光谱对应的非对称陀螺,由于其三个转动惯量各异,在求解薛定谔方程上遇到了极大困难。王守竞在做博士后期间,以惊人的毅力和纯熟的数学技巧严格地求出了非对称陀螺的能级公式:F(J)=1/2(B+C) J(J+1) +(A-1/2)(B+C)W,其中」为角量子数,A,B,C分别为与三个转动惯量有关的常数,W则为2J+L阶矩阵的根。同时他还给出了非对称陀螺能级间跃迁的完整跃迁规则。诺贝尔奖获得者赫兹堡(G. Herzberg)在他著名的五卷巨著《分子光谱和分子结构》第二卷中,将王守竞的以上能级公式称为“王守竞方程”(见 Gerhard Herzberg. Molecular spectraand molecular structure, IL. Infrared and Raman spectra of polyatomicmolecules. New York: D. van Nostrand company, 1945: 46 -49),作为讨论多原子分子转动光谱的基本公式,并特别指明“王守竞方程”是诸多非对称陀螺能级公式中可进行数值计算的两个公式之一。王守竞的这个能级公式至今一直在多原子分子光谱的教科书和专著中使用。
一位年方24 岁的青年,竟然在量子力学初创时期做出三项在物理学历史上存留下来的如此重要的工作,实可谓成绩卓著。
王守竞之所以能够取得如此杰出的成就,一方面固然是他的天赋优异、才华横溢,另一方面与他“天时、地利、人和”三大条件具备也大有关系。天时者,即量子力学创立之初,他能够抓住机遇,勇闯前沿;地利者,即美国大学中开放的观念和雄厚的物质条件;人和者,即他能够选择和结交到优秀的老师和同学。
1926年夏天,王守竞一来到哥伦比亚大学马上就结交了几位志同道合的朋友,他们中有刚刚结束在欧洲的博士后之旅回到哥伦比亚大学物理系任讲师的克罗尼格(Ralph Kronig),还有同在哥伦比亚大学物理系做博士论文的拉比(1L. Rabi)等人。拉比因核磁共振获得1944年诺贝尔物理学奖,克罗尼格是在固体理论和x射线吸收谱研究中做出重要贡献的著名理论物理学家。
1926年前后,量子力学刚刚创立,当时物理学研究的中心在欧洲,美国的大学尚无人讲授量子力学。然而这几位年轻人敏锐地感觉到量子力学的创立对传统物理学的革命性冲击,渴望掌握量子力学的最新进展,于是便自发地成立了一个理论物理的自学小组。成员有克罗尼格、拉比、王守竞、比特尔(Biter)和蔡曼斯基(Zemansky)五人,除他们之外,纽约大学的一些教授也常来参加讨论。小组成员的聚会大约每周一次,一般安排在星期六或星期天,每次由一人报告近期读到的重要理论物理文献,大家就此展开讨论。这一时期德国《物理学杂志》上的新东西层出不穷,他们也就没完没了地进行着讨论。聚会从上午开始进行到下午,讨论结束后,大家通常去中餐馆用餐,这时王守竞就成了唯一的行家,他会让大家品尝到正宗的中国菜肴。
薛定谔的波动力学论文发表以后,克罗尼格、拉比和王守竞花费了相当多的时间来研究它,在欧洲访问期间已经和海森堡、克拉默斯(H. A. Kramers)共同发表过论文的克罗尼格建议,为了更好地理解和掌握波动力学,大家应该尝试用它来处理一些实际问题。于是,他们对近期出版的书刊进行了检素,以寻找合适的研究题目。薛定谔已经处理了单原子系统的能谱,他们便尝试把薛定谔的理论推广到分子体系。他们很快建立起对称陀螺的薛定谔方程,1926年底,距离薛定谔理论发表仅几个月的时间,克罗尼格和拉比就用它解出了对称陀螺能谱,在1927年2月的美国《物理评论》发表。而王守竞则在不久之后,接受德拜的建议,完成了他那篇有名的《两个氢原子的相互作用》,在1927年的德国的《物理学杂志》发表。这表明哥伦比亚大学理论物理自学小组是能够紧跟量子力学发展的潮流。而王守竞与这样的精英在一起比翼齐飞,自是受益良多、进步神速。
王守竞的第二篇量子力学论文在正式发表之前,曾在 1927年11月25日至26日于芝加哥大学举行的美国物理学会 147 次年会上,被选为大会仅有的7篇口头报告之一全文宣读。这篇论文也成了王守竞的博士论文。1928年前美国大学里还没有人完成过以量子力学为研究课题的博士论文。至1928年始有王守竞等7人以量子力学的研究为博士论文选题,成为美国大学最早的一批因研究量子力学而被授子博士学位的学者。他的论文也是哥伦比亚大学物理系的第一篇纯理论的博士论文,他在进行这项工作时,完全没有得到博士导师方面的指导,而是得益于与自学小组成员的讨论和自己的认真钻研。这既体现了他的能耐,也体现了当年哥大的开明。
王守竞不仅善手自学专研,而且也善于向他人学习。他在博主后期间转到威斯康星大学师从美国著名理论物理学家范弗莱克(J.H.van Vleck)进行研究,充分说明了这一点。范弗莱克手1922年获哈佛大学物理学博士学位,他与同为哈佛出身的坎布尔、斯莱特等人一起为量子力学在美国的建立和发展做出了非常重要的贡献,他后来因磁学方面的理论研究成果获1977年诺贝尔物理学奖。他在当时是美国本土最好的理论物理学家之一。正是在范弗莱克的指导下,王守竞完成了极为困难的非对称陀螺的能谱计算。考虑到此前曾与克罗尼格和拉比研究过对称陀螺,他做出这一选择可以认为是很自然的事。开始时,他曾尝试按克拉默斯的建议,求解非对称陀螺的薛定谔方程。他证明,在空间量子数为零时,经过合理选择椭圆坐标系,该薛定谔方程可以完全分离成两个类似的常微分方程,每个方程只含有一个椭圆坐标。但是,由于考虑边界条件的一点疏忽,求解过程没有进行下去。后来,他改用范弗莱克擅长的矩阵力学的方法,在较短期间内取得成功,为此他特别在论文的最后一节,诚挚地感谢了范弗莱克的鼓励、建议和帮助。王守竞这篇工作论文发表先于同时开始研究同一问题的荷兰学者克拉默斯和伊特曼(G.P.Iumann),后者就此问题发表的三篇文章中都引用了王守竞的工作。美国学者曾以此作为美国本土物理研究已引起欧洲关注的范例加以褒扬。

传授物理,筹建学会
王守竞学业期满,获得博士学位并取得杰出成就以后,抱定“学成归国,报效祖国”的信念和“宁怀故国土,不恋他乡金”的情操,毅然回国。
1929年夏天,王守竞携爱侣费令宜女士回到了祖国。原来,在哈佛大学时王守竞与同在该校攻读欧洲文学硕士的苏州同乡、同样出身名门的大家闺秀费小姐彼此相爱,异邦结情缘。
回国后,王守竞先任浙江大学物理系主任,1931年被北京大学聘为物理系主任。任北京大学物理系主任期间,曾为北京大学物理系的实验室建设做了重要贡献,建立了真空系统、阴极溅射设备,磨制精密光学元件等,为物理系奠定了科研基础。从此,北京大学有了王守竞、吴有训、周培源、周同庆等七八位名师来校授课。由于学生兴趣高和知识界评论极佳,北京大学物理系的学术地位得以复列前茅,与清华物理系并驾齐驱。1932年春,北京大学设立“研究教授”一职,首批15人,王守竟是其中之一。1933年春季,王守竞辞去系主任职务,创立北京大学物理研究所并任所长,与饶航泰、萨本栋等专任研究教授。系主任一职由萨本栋接任。其后又有朱物华、张宗蠡到物理系任教授,北京大学物理系教师阵容盛极一时。1933年至1934年间,北京大学文学院开“科学概论”课,其中物理学方法论部分由萨本栋、王守竞担任讲座教授。1933年起,清华大学开始公开招考公费留美生,王守竞被聘为招生考试委员会委员,他与叶企孙、吴有训、严济慈、丁西林等委员一起,主要担负物理学方面的选拔工作。委员们在通盘考虑中国科学事业的发展规划并兼顾国家当前急需的情况下,设置了一些应用物理招考学科,选拔了各科人才,这批学生学成回国后,大都成为学科创始人和学术带头人。
王守竞是中国现代物理学的先驱者之一,他积极参与了中国物理学会的筹建。
1931年12月,王守竞经通信选举当选为中国物理学会临时执行委员会委员。除王守竞外,其他6 名委员是:夏元瑮、胡刚复、叶企孙、文元模、严济慈、吴有训。临时执行委员会于1932年3月29日和7月9日开会两次,决定当年8月22日召开中国物理学会成立大会。在成立大会上,王守竞当选为评议员。评议员共9名,包括会长李书华、副会长叶企孙、秘书吴有训、会计萨本栋,以及王守竞、严济慈、胡刚复、张贻惠、丁西林。1933年,《物理学报》创刊,他是编委之一。中国物理学会成立后即设立译名委员会,他是委员之一。第二次会上他继续当选为评议员和名词委员会委员。1933年8月21日至9月2日期间,名词委员会开会9天,审定物理学名词5000余则。在1934年召开的中国物理学会第3次年会上,王守竞被选派出席当年10月在英国伦敦召开的国际纯粹与应用物理学联合会会议。此后,在他投身工业、主持中央机器厂期间和奉派赴美以后,仍继续当选为中国物理学会理事,说明他在中国物理学界的地位和影响并不因他转往工业界而削弱。同时,他也仍然关心、支持和参加物理学界的活动,继续担任公费留美生招生考试委员会委员,因为他毕竟是一位物理学家,物理学始终是他的专业和特长。

应用研究,工业救国
1931年“九一八〞事变后,日本侵略中国的野心暴露无遗,有识之士深感国难当头。许多科学家都纷纷改变了自己的研究方向,以图救国。不少物理学教授开展了应用研究,如吴有训在金属学方面、周培源在流体力学方面开展工作。在这个时期,王守竞也不再从事量子力学研究,他指导助教赵元磨制光学平面玻璃,后来赵元掌握的技术在中央研究院发挥了作用。当时北京大学一位生理学教授请王守竞修理进口仪器,经检查后发现其中的铂丝断了,他指导赵广增将较粗铂丝用熔化的银铸进铜套管的中心,为了避免气泡,在铜套管上打了许多细孔,随后送到北京前门拉丝作坊拉丝,最后制成了直径仅几微米的铂丝,修好了仪器。以后赵广增和谭承泽利用不同直径的铂丝测定了它的滞弹性行为,发表了论文。1932年,王守竞在中国物理学会成立大会上,宣读了论文《试验玻璃平面之绝对方法》。1933年,在第二次中国物理学会年会上,他和张仲桂宣读了论文《测定短小物体杨氏弹性系数的方法》,他和赵广增宣读了论文《晶体切面上形状分布之或然率》。
同时,王守竞对国防建设、资源开发等救国方略提出了许多独到的见解,受到了政府当局的重视。特别是他在应用研究中一开始就选择了光学玻璃的磨制这一课题,并且富有成果,与当时国家的急需不谋而合,亦表现出了他的过人之处。1933年秋,军政部兵工署署长俞大维识其异才,亲自登门,请其担任兵工署技术司光学组主任,主持筹划建立中国的光学工业。想到之所以研究光学玻璃,正是为了要建立中国自己的光学工业,王守竞自然一口答允。
从此,王守竞也就从科教界转人了工业界,走上了工业救国之路。在王守竞主持下,在一年多的时间里,进行了大量的前期准备工作,为后来正式建厂奠定了基础。其后王守竞虽然不再主持其事,但这段时间的工作后人是不会忘记的。
1935年4月,在资源委员会兼委员长的蒋介石和秘书长翁文灏的指令下,王守竞被从兵工署调往资源委员会,任少将级专门委员,参与国家重工业厂矿建设。
1936年9月,国民政府军事委员会训令成立机器制造厂筹备委员会,并指定王守竞为主任委员。从此王守竞肩负重任,走上了一条开创中国机械工业的光荣而又艰难的道路。建设机器制造厂是国家重工业三年计划中的十大工程之一,该厂又是资源委员会与同属军事委员会(委员长蒋介石)的航空委员会(秘书长宋美龄)的合作项目,以制造航空发动机为首要目标,兼及其他。如此重任交给王守竟来承担,可见其人才难得,不同凡响。机器厂初设湖南湘潭,后迁往云南,重建于昆明北郊茨坝。在王博士(时人都如此称呼他)的领导和精神感召下,全体建厂人员上下一心,共同努力,拓荒创业,披荆斩棘,1939年9月9日,工厂终于宣告正式建成,命名为中央机器厂(即今昆明机床股份有限公司之前身),王守竞任总经理。“中央机器厂为国营机器工业中最早之工厂,其规模设备,在全国首屈一指”(时任资源委员会主任、副主任的翁文灏、钱昌照语)。
在抗战期间,中央机械厂以其雄厚的设备和技术实力,生产了大量军民用机械产品,为抗战胜利贡献甚巨。同时,所生产的产品中,很多都属“中国第一”,在中国机械工业发展史上也有极其重要的地位。而且王守竞以学者办厂,使厂如学校,既出产品又出人才,意义十分重大,至今为人赞誉。
原中央机械厂工程师、后为北京机床研究所总工程师的张克品孟盖當昌光生曾经精解地機指和总结道:牛中机器),建立中中华民危难之秋,在上分艰苦的条件下,培育」我国现代机械工业后的人才,抗战初期在大后方的春城建立了中央机器广与西南甲大学,不能不说是对我民族复兴的伟大功绩。中央机器厂与西联大二者相辅相成,形成了不仅是象征意义的而且是实实在在的复兴工业与文化的摇篮。西南联大工学院机械系,新建立的航空系,乃至机电系的高班学生都以中央机器厂为实习基地。中华人民共和国成立后新建起来的机器厂几乎都有原中央机器厂出来的员工,而且都是创业的骨干力量。高等院校和工业部门的研究院所的创业者则多有来自西南联大和中央机器厂的人员。昆明机床厂在中华人民共和国成立初期能定点制造铣床、镗床,自然与原有的基础有关,以后不断发展壮大,至今成为国家精密机床的主要基地。”
原中央机械厂工程师、后为国家第一机械工业部二局(机床和工具局)总工程师的韩云岑先生曾经说过:“当时中机厂一是提倡读书,二是提倡钻研技术,三是提倡‘什么都能做’。人和物的条件具备,所以做了不少事,也培养了不少人,后来在技术界起了很大的作用。”
原中央机械厂工程师、后为国家机械工业部机械研究院总工程师、中国科学院学部委员(院士)的雷天觉先生曾经说过:“当时那个厂还起了一个很大的作用,就是培养人才,这个作用恐怕比造机器的作用还要大。”
1989年江泽民总书记在视察了昆明机床厂后,对昆明机床厂的现状给子高度赞扬,也对其前身—中央机器厂给予了历史性的肯定。王守竞作为中央机器厂的创始人和负责人,自是功不可没。昆明机床股份有限公司在建厂70 周年之际,为工厂奠基人王守竞敬塑了一尊半身铜像,安放于厂区花园中,以供人们永远怀念和瞻仰。
1943年6月,王守竞奉派赴美,任资源委员会驻美办事处主任、中国驻美大使馆科技参赞。1944年应马歇尔将军之请,出任租借法案中方实际主持人(名义主持人宋子文)。1945年国民政府成立驻美物资供应委员会,他出任主任委员。1946年任以翁文灏为董事长的中国石油有限公司七名董事之一。在此期间王守竞曾代表中国政府在华盛顿就日本战败赔偿问题与美国代表举行会谈,与企图扶植日本军国主义的美国政府进行了抗争,争得拆卸部分日本工厂机械设备运回国内。
1951年,王守竞脱离国民党政权,移居美国麻省波士顿近郊水城(Water Town)。后任职于美国国防部与麻省理工学院合作的林肯实验室(Lincoin Laboratory,MIT),从事太空和军事系统之研究,回归他所热爱和擅长的物理和机械专业。
1969年,王守竞从林肯实验室退休。退休后他以摹苏体书法自娱。
由于王守竞早年及后来所享有的极高声誉,特别是他在美国高科技领域工作多年所取得的成就和经验,1980年之后国内有关方面曾数次通过多种渠道邀请他回国访问。例如周培源先生就曾多次以老同学和北京大学校长、全国政协副主席名义致函邀请,切盼回国相聚,并曾说过:如果王守竞先生回来,中国物理学界要召开一个盛大的欢迎会。中央有关部门已做好接待准备,如果王守竞回来,将按接待海外华人科学家的最高规格接待。王守竞本人也极想在有生之年重返故土一行,只是因病情日剧,终未能如愿成行。
1984年6月19日,王守竞以脑痪辞世,享年八十有一。综观王守竞先生的一生,他不仅具有老一辈知识分子所共有的忧国忧民、一切以国家民族为重的爱国情怀和传统美德,而且由于历史机遇和本人的才干素养,较很多同代人有更为丰富多彩的经历。他先后走过了“科学救国”“教育救国”“工业救国”几条道路,并在所从事的每个领域都取得了卓越的成绩。他为国家民族所作出过的贡献,为现代物理学发展所作的贡献,值得我们永远铭记。
(刘寄星 余少川)

简历
1904年1月15日生于江苏苏州
1921—1924年 北京清华学堂学生
1924—1925年 美国康奈尔大学研究生院学生,获理学硕士学位
1925—1926年 美国哈佛大学研究生院学生,获欧洲文学硕士学位
1926-1928年 美国哥伦比亚大学研究生院学生,获理学博士学位
1928年 美国威斯康星大学物理系博士后研究员
1929—1930年 浙江大学物理系教授、系主任
1931—1933年 北京大学物理系教授、系主任、研究教授
1933—1934年北京大学物理研究所所长
1931—1932年 参与筹建中国物理学会,被选为中国物理学会评议员
1933年被选为《物理学报》编委及物理译名委员会委员
1834年 代表中国物理学会出席国际纯粹与应用物理联合会会议
1933年 出任国民政府兵工署技术司光学组主任
1935-—1943年 任国民政府资源委员会委员,主持筹建机器制造厂,1939年起任中央机器厂总经理
1943—1951年 先后担任资源委员会驻美办事处主任、中国驻美大使馆科技参赞、美国租借法案中方实际主持人、国民政府驻美物资供应委员会主任委员、中国石油有限公司董事
1951—1969年 退出国民党政府,任美国麻省理工学院林肯实验室研究员
1984年6月19日逝世

主要论著
1. Wang S C. Die gegenseitige einwirkung zweier wasserstoffatome. Phys. Zeit.,1927, 28: 663.
2. Wang S C. The problem of the normal hydrogen molecule in the new quantum mechanics. Phys. Rev. , 1928, 31: 579.
3. Webb H W, Wang S C. The exeitation of sodium by ionized mercury vapor. Phys. Rev. 1929, 33: 329
4. Wang S C. On the separability fo Schroedinger's equation for the asymmetrical top. Phys. Rev., 1929, 33: 123.
5. Wang S C. On the asymmetrical top in quantum mechanics. Phys. Rev. 1929,34:243.
6. 王守竞.试验玻璃平面之方法//梅贻琦等.中国物理学会第一次年会论文集.北平:清华大学,1932.
7. 王守竞,张仲桂.测量短小物体杨氏弹性系数的方法//李书华,叶企孙等.中国物理学会第二次年会论文集.上海:交通大学,1933.
8. 王守竞.在氢分子结构计算中引入屏蔽效应//李书华,叶企孙等.中国物理学会第三次年会论文集.南京:中央大学,1934.

参考文献
〔1〕 赵广增,王守武,王明贞.纪念王守竞先生。物理,1985,14:382.
〔2〕 钱维翔.王守竞//刘绍唐.民国人物小传、台北:传记文学出版社,1987.
〔3〕 王守武.王守竞//《科学家传记大辞典》编辑组.中国现代科学家传记.第三集.北京:科学出版社,1992:89.
〔4〕 余少川.中国机械工业的拓荒者—王守竞.昆明:云南大学出版社,1999.
〔5〕 胡升华,王守竞的量子力学研究成果及其学术背景.中国科技史料,2000,21 (3):235.

《中国科学技术专家传略 理学编 物理学卷 5》 
