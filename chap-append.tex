\section{量子化学计算中的密度泛函计算{程序}}
密度泛函理论应用于实际体系的计算通常是通过自洽迭代求解Kohn-Sham方程实现的。Kohn-Sham方程是二阶偏微分方程{。}对于原子体系,%由于其本质上是})
{利用其球形对称性可以转化为}一维问题,%可以})
通过数值%微分的})
方法{可以}很精确地求解Kohn-Sham方程,对于分子和固体等体系,通常需要采用基组展开的%办})
{方}法把它转化为%求解})
广义本征值%加以})
{问题}求解。%具体过程如下:})

设Kohn-Sham方程的本征态$\varphi_i$可用%一套})
基组$\{\chi_{\mu}\}$的线性组合近似表示;$\varphi_i$\,=\,$\sum\limits_{\mu}C_{\mu i}\chi_{\mu}$,把$\varphi_i$的表达式代入Kohn-Sham方%法的总能量表达式\eqref{eq:dft-1},再对总能量作变分,可以})
{程,在\{$\varphi_i$\}满足正交归一条件的约束下变分使得基组展开的误差取极值,就}得到求解组合系数$\{C_{\mu i}\}$的广义本征方程:
\begin{equation} \label{eq:dft-12}
{\mathbf HC}={\mathbf SC\varepsilon}
\end{equation}
其中$\mathbf H$是Kohn-Sham方程%对应})
的Hamiltonian%。})
在这%套})
{一}基组下的表示矩阵,$\mathbf S$是重叠矩阵,$\mathbf C$是组合系数矩阵,${\mathbf\varepsilon}$是对角元为近似本征值的对角化矩阵%元,})
{:}%各矩阵的矩阵元具体表示形式为:})
$H_{\mu\nu}$\,=\,$\langle\chi_{\mu}|\hat{H}|\chi_{\nu}\rangle$,\linebreak$S_{\mu\nu}$\,=\,$\langle\chi_{\mu}|\chi_{\nu}\rangle$,$\varepsilon_{ij}=\varepsilon_i\delta_{ij}$。对于原子、分子和固体等多电子体系,$\hat{H}$的具体表达式为:
\begin{equation} \label{eq:dft-13}
  \hat{H}=\hat{T}+\hat{V}_N+\hat{V}_{\mathrm{Coul}}+\hat{V}_{XC}
\end{equation}
其中$\hat{T}$\,=\,$-\dfrac12\nabla^2$是动能算符,$V_N(\vec{r})$\,=\,$-\sum\limits_A\dfrac{Z_A}{|\vec{r}-\vec{R}_A|}$为核吸引势,\\$V_{\mathrm{Coul}}(\vec{r})$\,=\,$\displaystyle\int{\dfrac{\rho(\vec r_l)}{|\vec{r}-\vec r_l|}\textrm{d}^3\vec{r_l}}$为电子间的Coulomb势,$V_{XC}(\vec{r})$\,=\,$\dfrac{\delta E_{XC}}{\delta\rho(\vec{r})}$为电子间的交换-相关势{,$\rho(\vec r)$\,=\,$\sum\limits_in_i|\varphi_i|^2$,$n_i$为$\varphi_i$轨道上的电子占据数}。从\eqref{eq:dft-13}式可以知道$\hat{H}$与{\{}$\varphi_i${\}}有关,式\eqref{eq:dft-12}需要通过自洽迭代的方法求解。由于交换-相关势中包含$\rho^{1/3}$项,%很})
难{于}利用解析方法%求解})
{计算}交换-相关势的矩阵元和交换-相关能,通常{采}用数值积分%的})
方法%加以处理})
{计算积分}。目前最%流行})
{常用}的数值积分方法是把被积函数分割为%单})
{以原子核为}中心{的各}部分,对每%个单中心})
部分的积分用高斯求积公式{计算,各部分的积分值之和给出整体的积分值。}分割方案主要%由})
{有}Boerrigter等~\cite{IJQC33-87_1988}~提出的把空间按原子%核})
分割成多面体的方法和Becke\cite{JCP88-2547_1988}提出的{有}重叠区间{的}分割方法。对于非局域密度近似的交换-相关势的矩阵元计算,%如果})
直接%采用对非局域密度近似的交换-相关能泛函变分的方法得到})
{按}交换-相关势%再})
{表达式}计算,需要计算电子密度的二阶微分,%也})
{亦}即需要计算基函数的二阶微分,计算量%很})
大,但可以通过分部积分的方法%加以})
避免{。}通常采用的计算非局域近似下交换-相关势矩阵{元}的表达式%可以写成})
{为}~\cite{PRA43-5810_1991}~:
\begin{equation} 
\label{eq:dft-14}
\langle F_{XC}^{GGA}\rangle_{\mu\nu}=\int{\left[\dfrac{\delta E_{XC}^{GGA}}{\delta \rho}\chi_{\mu}^{\ast}\chi_{\nu}+\dfrac1{|\nabla\rho|}\dfrac {\delta E_{XC}^{GGA}}{\delta|\nabla\rho|}\nabla\rho\cdot\nabla(\chi_{\mu}^{\ast}\chi_{\nu})\right]\textrm{d}^3\vec r}
\end{equation}

各种不同的密度泛函计算程序的主要区别在于所用的基组不同,计算Coulomb势的方法不同,以及由于采用不同基组引起的计算矩阵元的方法不同。通常采用的基组有%所谓})
Gaussian基%函数})
{组},Slater基%函数}{组}各和数值基%函数})
{组}和数值基组。采用%收})
{简}缩%的})
Gaussian基%函数})
{组}的密度泛函计算程序主要有GAUSSIAN\cite{CPL199-557_1992}中的密度泛函计算部分,Dgauss\cite{JCP96-1280_1992}、DeFT\cite{Http-DFT} 和deMon\cite{CPL169-387_1990}等{。}由于采用Gaussian基组,除了交换-相关势的矩阵元外,其他矩阵元都可以用解析方法精确%求得})
{计算},可以得到比较精确的总能量%的})
值。但是如果%要})
精确计算Coulomb势的矩阵元则需要用解析方法计算四中心积分,计算量很大。

在上面列出的计算程序中,GAUSSIAN采用解析方法精确计算的Coulomb积分,其计算量与基组大小的四次方成比例,%但是在使})
{优点是可}用%交换-相关泛函上可以采用})
\eqref{eq:dft-11}式表示的%混合})
{杂化能量密度泛函进行计算}泛函%的形式})
。其它几种程序都采用对电子密度%采})
用附加基组拟合{的方法},把四中心积分变成三中心积分,计算量只与基组大小的三次方成比例,但是%不能计算四中心积分,使得这些程序在使用泛函上,})
不能%采用混合})
{进行}{使用杂化能量密度}泛函%形式})
{的计算}。采用Slater基函数的密度泛函计算程序主要是ADF程序\cite{ADF-UG_1995}{。}由于采用Slater函数计算多中心积分的计算量很大,ADF程序中除重叠矩阵和动能矩阵这两种最多涉及双中心积分的矩阵元采用解析方法计算外,其他积分都采用数值积分的方法计算{;}%同样})
在处理Coulomb势的矩阵元上,也是先用%一套})
附加基组拟合电子密度,%再用这套附加基})
{用以}计算各积分点上Coulomb势的值。%这种方法})
{ADF程序}也不能{进行}采用%混合})
{杂化能量密度}泛函的%形式})
{计算},而且由于主要的积分项都%适})
用数值积分方法计算%得到})
,%而体系总能量一般是很大的值,})
直接用数值积分方法计算体系总能量{误差}比较大%的误差})
{。}通常采用Ziegler\cite{TCA46-1_1977}提出的过渡态方法计算体系与{平}均态原子的能量差{,再加上容易精确计算的原子平均态的能量和它们之间的相互作用能,得出体系总能量,}以减少计算误差。采用Slater基函数的另一个优点是由于Slater函数与原子轨道形状类似,对每个内层轨道%采})
{可以}用一个Slater函数%去})
模拟,%就可以})
采用冻芯近似,%所增加的基组的数目不多而能在很大程度上})
{减少使用的基函数从而}减少计算量。采用数值基函数%方法})
的程序主要有Dmol\cite{JCP92-508_1990}{。}我们实验室自己编写的BDF程序\cite{TCA96-75_1997}和非相对论密度泛函程序\cite{PKUAS32-172_1996}采用的基组是数值基函数和Slater基函数的混合基{组}。这%种方法})
{些程序}因为采用了数值基,所有%的})
矩阵元通过数值积分的方法%得到})
{计算},Coulomb势在数值积分点上的值%是})
通过求解Possion方程%的办法})
得到:
\begin{equation} \label{eq:dft-15}
  \nabla^2V_{\mathrm{Coul}}=-4\pi\rho
\end{equation}
%因为})
{在分子和固体中}电子密度$\rho$是多中心项,直接求解\eqref{eq:dft-15}式%还})
比较困难,一般是%通过})
{先}把电子密度%先})
分割成%单中心})
{以原子核为中心的各}部分,然后用球谐函数对%单中心})
{各部分}的电子密度{以原子核为中心在一系列径向点上}作多极展开,再用数值方法求解Possion方程,{得到各数值积分点上的Coulomb势值。}这%个})
{种}方法%的本质还是对电子密度作拟合,但是拟合函数的径向部分形式是由数值函数表示的,在拟合到等同阶球谐函数的情况下这种拟合方式要比采用附加基组的方法精度更高。})
{的优点是可以根据需要通过提高多极展开的阶数和增加径向点的数目提高计算精度。}%同样,这种方法也不能采用混合泛函的形式。由于所有矩阵元都是用数值方法计算得到,})
{同样,使用数值(或混合)基组的程序也不能进行使用杂化能量密度泛函的计算,}也不能通过直接计算得到比较精确的总能量,需要使用过渡态方法才能得到精度比较高的%相对})
{总}能量值。%这种方法也可以采用冻芯近似减少计算量,与用Slater函数为基的方法相比,可以直接使价层基函数与芯层轨道正交而不用引入新的基函数,这种冻芯方法比用Slater基组的方法有一定的优越性。})

%\newpage
%  \phantomsection\addcontentsline{toc}{section}{Bibliography}
%  \phantomsection\addcontentsline{toc}{section}{(本章)参考文献}
%  {\bibliography{chapters/bib/Myref}}
%  \bibliographystyle{mythesis}
%  \nocite{*}
%\begin{thebibliography}{99}
%\bibitem{Parr-Yang}R. G. Parr and W. Yang, {\it Density Functional Theory of Atoms and Molecules}\ (Oxford University, Oxford, 1989)
%\bibitem{CR91-651_1991}T. Ziegler, {\it Chem. Rev.} {\bf 91}, 651 (1991)
%\bibitem{PCPS23-542_1927}L. H. Thomas, {\it Proc. Cambridge Philos. Soc.} {\bf 23}, 542 (1927)
%\bibitem{ZP48-73_1928}E. Fermi, {\it Z. Phys.} {\bf 48}, 73 (1928)
%\bibitem{PR81-385_1951}J. C. Slater, {\it Phys. Rev.} {\bf 81}, 385 (1951)
%\bibitem{PRB136-864_1964}P. Hohenberg and W. Kohn, {\it Phys. Rev. B} {\bf 136}, 864 (1964)
%\bibitem{PRA140-1133_1965}W. Kohn, L. J. Sham, {\it Phys. Rev. A} {\bf 140}, 1133 (1965)
%\bibitem{CJP58-1200_1980}S. H. Vosko, L. Wilk and M. Nusair, {\it Can. J. Phys.} {\bf 58}, 1200 (1980)
%\bibitem{PRB45-13244_1992}J. P. Perdew and Y. Wang, {\it Phys. Rev. B} {\bf 45}, 13244 (1992)
%\bibitem{PRA38-3098_1988}A. D. Becke, {\it Phys. Rev. A} {\bf 38}, 3098 (1988)
%\bibitem{PRB33-8822_1986}J. P. Perdew, {\it Phys. Rev. B} {\bf 33}, 8822 (1986)
%\bibitem{PRB37-785_1988}C. Lee, W. Yang and R. G. Parr, {\it Phys. Rev. B} {\bf 37}, 785 (1988)
%\bibitem{PCPS26-376_1930}P. A. M. Dirac, {\it Proc. Cambridge Philos. Soc.} {\bf 26}, 376 (1930)
%\bibitem{ZP96-431_1935}C. F. von Weizsacker, {\it Z. Phys.} {\bf 96}, 431 (1935)
%\bibitem{Bassani-Fumi-Tosi}{\it W. Kohn in Highlights of Condensed Matter Theory}, ed. By F. Bassani, F.  Fumi, M. P. Tosi, p.1
%\bibitem{PNAS76-6062_1979}M. Levy, {\it Proc. Natl. Acad. Sci. U.S.A}, {\bf 76}, 6062 (1979)
%\bibitem{PRB12-2111_1975}T. L. Gilbert, {\it Phys. Rev. B} {\bf 12}, 2111 (1975)
%\bibitem{IJQC24-243_1983}E. H. Lieb, {\it Int. J. Quant. Chem.} {\bf 24}, 243 (1983)
%\bibitem{AQC6-1_1972}J. C. Slater, {\it Adv. Quant. Chem.} {\bf 6}, 1 (1972)
%\bibitem{Slater-4_1974}J. C. Slater, {\it The Self-Consistent Field for Molecules and Solids: Quantum Theory of Molecules and Solids}, vol. 4 (1974)
%\bibitem{PRA20-397_1979}G. L. Oliver and J. P. Perdew, {\it Phys. Rev. A} {\bf 20}, 397 (1979)
%\bibitem{JCP104-1040_1996}A. D. Becke, {\it J. Chem. Phys.} {\bf 104}, 1040 (1996)
%\bibitem{JCP98-5648_1993}A. D. Becke, {\it J. Chem. Phys.} {\bf 98}, 5648 (1993)
%\bibitem{JCP98-5612_1993}B. G. Johnson, P. M. W. Gill and J. A. Pople, {\it J. Chem. Phys.} {\bf 98}, 5612 (1993)
%\bibitem{CPL316-160_2000}A. J. Cohn, N. C. Handy, {\it Chem. Phys. Lett.} {\bf 316}, 160 (2000)
%\bibitem{PRB43-6865_1991}F. W. Kutzler and G. S. Painter {\it Phys. Rev. B} {\bf 43}, 6865 (1991)
%\bibitem{CPL265-481_1997}E. J. Baerends, V. Branchadell and M. Sodupe, {\it Chem. Phys. Lett.} {\bf 265}, 481 (1997)
%\bibitem{IJQC33-87_1988}P. M. Boerrigter, G. T. Velde and E. J. Baerends, {\it Int. J. Quant. Chem.} {\bf 33}, 87 (1988)
%\bibitem{JCP88-2547_1988}A. D. Becke, {\it J. Chem. Phys.} {\bf 88}, 2547 (1988)
%\bibitem{PRA43-5810_1991}K. Kobayashi, N. Kurita, H. Kumnhora and K. Tago, {\it Phys. Rev. A} {\bf 43}, 5810 (1991)
%\bibitem{CPL199-557_1992}J. A. Pople, P. M. W. Gill and B. G. Johnson, {\it Chem. Phys. Lett.} {\bf 199}, 557 (1992)
%\bibitem{JCP96-1280_1992}J. Andzelm and E. Wimmer, {\it J. Chem. Phys.} {\bf 96}, 1280 (1992)
%\bibitem{Http-DFT}\url{http://www.chem.uottawa.ca/DeFT.html}
%\bibitem{CPL169-387_1990}A. St-Amant and D. R. Salahub, {\it Chem. Phys. Lett.} {\bf 169}, 387 (1990)
%\bibitem{ADF-UG_1995}Amsterdam Density Functional(ADF) (Theoretical Chemistry, Vrijie Universiteit, Amsterdam, 1995)
%\bibitem{TCA46-1_1977}T. Ziegler, A. Rauk, {\it Theor. Chem. Acta.} {\bf 46}, 1 (1977)
%\bibitem{JCP92-508_1990}B. Delly, {\it J. Chem. Phys.} {\bf 92}, 508 (1990)
%\bibitem{TCA96-75_1997}W. Liu, G. Hong, D. Dai, L. Li and M. Dolg, {\it Theor. Chem. Acc.} {\bf 96}, 75 (1997)
%\bibitem{PKUAS32-172_1996}洪功义,刘文剑,黎乐民, {\it 北京大学学报 (自然科学版)} {\bf 32}, 172 (1996)
%\end{thebibliography}

%%%%%%%%%%%%%%%%%%%%%%% 参考文献  %%%%%%%%%%%%%%%%%%%%%%%%%%%%%%%%%%
%%%%%%%%%%%% 采用reference.bib形式罗列参考文献 %%%%%%%%%%%%%%%     %
%\bibliographystyle{mythesis}% 参考文献罗列模式              %     %
%                              默认为plain/unsrt等模式       %     %
% \phantomsection\addcontentsline{toc}{section}{bibliography}%     %
%直接调用\addcontentsline命令可能导致超链指向不准确,         %     %
%一般需要在之前调用一次\phantomsection命令加以修正           %     %
%  {\small\bibliography{bib/myref}}                          %     %
%  \nocite{*}% 列出未被引用的myref.bib文件中的文献           %     %
%%%%%%%%%%%%%%%%%%%%%%%%%%%%%%%%%%%%%%%%%%%%%%%%%%%%%%%%%%%%%%     %
                                                                   %
%%%%%%%%%%%% 采用\bibitem方式引用参考文献  %%%%%%%%%%%%%%%%%%%     %
%\begin{thebibliography}{99}                                 %     %
%\bibitem{ref_lab}%文献                                      %     %
%\end{thebibliography}                                       %     %
%%%%%%%%%%%%%%%%%%%%%%%%%%%%%%%%%%%%%%%%%%%%%%%%%%%%%%%%%%%%%%     %
%%%%%%%%%%%%%%%%%%%%%%%%%%%%%%%%%%%%%%%%%%%%%%%%%%%%%%%%%%%%%%%%%%%%
