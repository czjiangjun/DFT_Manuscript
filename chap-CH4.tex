\chapter{甲烷催化燃烧机理研究进展}
天然气因其储量丰富和清洁低污染而被认为是2l世纪最具潜力的替代能源,被称为“第四代能源”。天然气的主要成分是甲烷,在常规的燃烧范围(5$\sim$15\%)内,着火温度为$530^{\circ}\mathrm{C}$左右,绝热燃烧温度达到$1900^{\circ}\mathrm{C}$左右。因热力型$\mathrm{NO}_x$在$1500^{\circ}\mathrm{C}$以上会大量快速生成,为了高效、低污染燃烧甲烷,就必须降低其着火温度,将其控制在$1400^{\circ}\mathrm{C}$以下。对此,催化燃烧是一种有效的方法,它可以在更低的燃料/空气比(1$\sim$5\%)范围内低温稳定燃烧。上世纪70年代,\textrm{Prefferle}等将催化氧化和气相自由基反应相结合,提出了多相催化气相燃烧过程,该过程提高了烃类燃烧效率,降低了燃烧污染物的排放\cite{ProcChem15-242_2003}。此外,矿井等工作区产生的瓦斯气体、天然气、汽车的尾气排放、水处理厂等都不可避免地排放一定量的甲烷,催化燃烧也是这些有害废气的最有效净化方法。\cite{ChemOL66-715_2003}

甲烷是最稳定的烃类,通常很难活化或氧化,而且甲烷催化燃烧工作温度较高,燃烧反应过程中会产生大量水蒸气,同时天然气中含少量硫,因此甲烷催化燃烧催化剂必须具备较高的活性和较高的水热稳定性,及一定的抗中毒能力.。通常催化剂的活性与稳定性是矛盾的,因此开发高效稳定的甲烷低温催化燃烧催化剂引起国内外研究者极大的兴趣,同时进行了大量相关研究,并取得了一定的成果。关于甲烷催化燃烧反应催化剂的制备及性能已多有报道,目前研究较多的是\textrm{Pd},\textrm{Pt},\textrm{Rh},\textrm{Au}等贵金属催化剂和金属氧化物催化剂。

\section{贵金属催化剂}
贵金属催化剂以其优异的起燃活性被广泛地应用于甲烷催化燃烧反应。贵金属催化剂与金属氧化物催化剂相比具有更高的催化活性和更好的抗中毒能力,但由于价格昂贵,其应用受到一定限制。其中负载型\textrm{Pd},\textrm{Pt}催化剂研究的最多,负载型\textrm{Rh},\textrm{Au}催化剂则报道较少。为了提高催化剂的催化性能,将\textrm{Pd}和一种或多种金属元素连用制成双、多贵金属催化剂,如负载型\textrm{Pd-Pt},\textrm{Pd-Rh}催化剂,应用于甲烷燃烧反应。

\subsection{负载型\rm{Pd},\rm{Pt}催化剂}
普遍使用的以$\mathrm{Al}_2\mathrm{O}_3$或$\mathrm{SiO}_2$为载体的担载型\textrm{Pd},\textrm{Pt}催化剂上的甲烷催化燃烧反应,通常被认为属于结构敏感反应\cite{CataT47-245_1999},这是由于催化活性随分散度的不同而不同,即粒子尺寸效应所致。通过对平均直径在$1.4\sim3.7$\textrm{nm}的\textrm{Pd}/$\mathrm{Al}_2\mathrm{O}_3$催化剂进行活性和分散度测试,一些研究者得出相反的结论,即认为甲烷氧化反应是结构不敏感\cite{ACB5-149_1994},反应初始或达到稳定状态时的催化活性与载体表面的\textrm{Pd}数量之间不存在任何联系,导致催化性能不同的原因是样品形态上的变化.

在氧化反应过程中,\textrm{Pd}和\textrm{Pt}分别向\textrm{PdO}和$\mathrm{PtO}_2$转化。$\mathrm{PdO}$在$300\sim400^{\circ}\mathrm{C}$形成,在$800^{\circ}\mathrm{C}$是稳定的,超过此温度,稳定相为金属\textrm{Pd}。与\textrm{PdO}相比,$\mathrm{PtO}_2$是高度不稳定的,它在非常低的温度(大约$400^{\circ}\mathrm{C}$)即分解。此外$\mathrm{PtO}_2$是高度易变的,这个特性可用来解释在纯氧条件下\textrm{Pt}表面的重整是靠$\mathrm{PtO}_2$在纳米尺度内输运\textrm{Pt}完成的。这就说明在富氧条件下,\textrm{PdO}极易形成,为活性相,而\textrm{Pt}主要保持金属状态。对负载型\textrm{Pd}催化剂来说,少量还原比完全氧化或完全还原处理的金属粒子催化活性高,且比完全还原的样品更易重新氧化\cite{ACA218-197_2001}。一旦氧化物薄膜形成,反应将明显减慢。金属分散度越低,\textrm{Pd}的氧化程度越低。根据\textrm{Burch}的报道\cite{ACB5-149_1994},\textrm{Pt}表面的氧化程度是其催化行为的一个关键因素,\textrm{Pt}表面氧化程度低的催化剂活性要优于表面氧化程度高的催化剂,因此在富$\mathrm{O}_2$条件下,应降低\textrm{Pt}表面的氧化程度。

\textrm{Cullis}和\textrm{Willatt}\cite{JCatal86-187_1984}首先报道了卤化物对负载型\textrm{Pd},\textrm{Pt}催化剂催化活性的阻抑效应,\textrm{Pd}催化剂比\textrm{Pt}催化剂更易中毒,而且活性的降低不能反转。尽管由$\mathrm{PdCl}_2$制备的催化剂有较高的分散度和较小的晶体尺寸,但由$\mathrm{Pd}(\mathrm{NO}_3)_2$制备的催化剂活性要更好。氯离子阻抑催化活性的机理\cite{ACA203-37_2000,JC197-394_2001}可能是$\mathrm{Cl}^-$以\textrm{HCl}形式进行脱附,而释放出\textrm{HCl}和反应物竞争金属表面的活性中心,堵塞这些活性中心从而抑制了催化剂活性。此外,水蒸气和硫化物($\mathrm{H}_2\mathrm{S}$)对活性的抑制也不可忽略。制备方法、焙烧温度、贵金属担载量、预处理条件等都将极大影响催化剂的各项性能。

负载型\textrm{Pd},\textrm{Pt}催化剂最早使用的载体为$\mathrm{Al}_2\mathrm{O}_3$。为改进其催化性能,可通过引入其他金属氧化物对其进行改性,或直接采用其他的金属氧化物$\mathrm{MO}_x$(\textrm{M=Ti,~Mn,~Co,~Zr,~Sn}等)以及它们的混合物作为载体。

\subsubsection{$\mathrm{Al}_2\mathrm{O}_3$为载体}
传统的担载在$\mathrm{Al}_2\mathrm{O}_3$上的\textrm{Pd}、\textrm{Pt}催化剂是靠浸渍法制备的。\textrm{Mizushima}\cite{ACA88-137_1992}比较了以下两种方法制备的担载在溶胶凝胶$\mathrm{Al}_2\mathrm{O}_3$上的\textrm{Pd}和\textrm{Pt}催化剂。方法\textrm{A}(共沉淀法):~混合金属前驱物($\mathrm{H}_2\mathrm{PtCl}_6$和$\mathrm{PdCl}_2$)到氧化铝溶胶中来制取;方法\mathrm{B}(沉积-沉淀法):~用金属盐溶液浸渍焙烧过的氧化铝溶胶凝胶来制取。然后将两种方法制备的产物与以$\gamma$-$\mathrm{Al}_2\mathrm{O}_3$作为载体的\mathrm{Pd},\mathrm{Pt}催化剂进行比较。结果显示,无论用溶胶凝胶氧化铝还是$\gamma$-$\mathrm{Al}_2\mathrm{O}_3$作为载体均对\textrm{Pt}催化剂的催化活性无影响,但对\mathrm{Pd}催化剂影响较大。前者作为载体的活性要优于后者,尤其方法\mathrm{B}制得的催化剂活性最好。遗憾的是,他们没有考察催化剂在$500^{\circ}\mathrm{C}$焙烧后剩余物中氯化物离子的影响,因此不能得出是否溶胶凝胶法对\textrm{Pd}/$\mathrm{Al}_2\mathrm{O}_3$的活性产生了有益的影响,或溶胶凝胶载体促进了$\mathrm{Cl}^-$从金属前驱物中离开。

\textrm{Escand\'on}等[8]由$\mathrm{Pd}(\mathrm{NO}_3)_2$的水溶液作为前驱物制得3种经过\mathrm{V}(0.1\%,~0.5\%和1\%)修饰的1\%\textrm{Pd}/$\gamma$-$\mathrm{Al}_2\mathrm{O}_3$催化剂,在$100^{\cirx}\mathrm{C}$干燥,空气中$550^{\circ}\mathrm{C}$焙烧2\textrm{h}.由催化剂转化率曲线看出,\textrm{V}的加入对催化活性产生了重要的影响。0.5\%\textrm{V}修饰的且没有经过还原的催化剂活性最好,达到50\%转化率时的温度$\mathrm{T}_{50}$为$335^{\circ}\mathrm{C}$,这可能是由于\textrm{Pd}-\textrm{V}之间的相互作用造成的。

\subsubsection{其他金属氧化物为载体}
采用大量不同的金属氧化物$\mathrm{MO}_x$(\textrm{M=Ti,~Mn,~Co,~Y,~Zr,~Nb,~In,~Sn}),金属氧化物混合物如$\mathrm{SnO}_2$-$\mathrm{ZrO}_2$,改性的$\mathrm{Al}_2\mathrm{O}_3$(向氧化铝中加入其他金属氧化物添加剂)等作为载体,也可被用来提高\textrm{Pd}和\textrm{Pt}的催化性能。相比之下,其他金属氧化物为载体的\textrm{Pt}催化剂的研究要比\textrm{Pd}催化剂少的多。

\textrm{Widjaja}等\cite{CataD59-69_2000}研究了担载在\textrm{In,~Sn,~Zr,~Ga,~Ti,~Si,~Y,~Nb}等金属氧化物上\textrm{Pd}催化剂的催化活性,并与\textrm{Pd}/$\mathrm{Al}_2\mathrm{O}_3$作了比较。所有这些催化剂(除\textrm{Pd}/$\mathrm{SiO}_2$外)比表面积(3$\sim$6$\mathrm{m}^2/\mathrm{g}$)均比\textrm{Pd}/$\mathrm{Al}_2\mathrm{O}_3$的比表面积(109$\mathrm{m}^2/\mathrm{g}$)低。它们对甲烷催化燃烧活性顺序为:~$\mathrm{Sn}>\mathrm{Zr}>\mathrm{Al}>\mathrm{Ga}>\mathrm{In}>\mathrm{Ti}>\mathrm{Si}>\mathrm{Y}>\mathrm{Nb}$。其中\textrm{Pd}/$\mathrm{SnO}_2$的活性最好($\mathrm{T}_{50}=360^{\circ}\mathrm{C}$,而\textrm{Pd}/$\mathrm{Al}_2\mathrm{O}_3$的$\mathrm{T}_{50}=430^{\circ}\mathrm{C}$)。\textrm{Pd}/$\mathrm{ZrO}_2$催化性能相当于或高于\textrm{Pd}/$\mathrm{Al}_2\mathrm{O}_3$.由高倍透射电子显微镜(\textrm{HTEM})图看出,\textrm{Pd}层整齐地覆盖在$\mathrm{SnO}_2$球形颗粒表面,二者强烈的相互作用使\mathrm{Pd}/$\mathrm{SnO}_2$的活性有显著提高.

\textrm{Li}等\cite{RKCL66-367_1999}经过大量实验后认为,用$\mathrm{Pd}(\mathrm{NO}_3)_2$溶液浸渍$\mathrm{Co}_3\mathrm{O}_4$,$280^{\circ}\mathrm{C}$焙烧得到的\textrm{Pd}/$\mathrm{Co}_3\mathrm{O}_4$活性较好,在$250^{\circ}\mathrm{C}$时$\mathrm{CH}_4$的转化率达到72\%,且连续测试90\textrm{min}后活性仍很稳定。最近,文献\cite{RKCL70-97_2000}报道由气体浓缩法制得了\textrm{n}型晶体结构的$\mathrm{TiO}_2$,$\mathrm{Mn}_3\mathrm{O}_4$,$\mathrm{CeO}_2$,$\mathrm{ZrO}_2$等为载体的\textrm{Pd}催化剂,即将金属(\textrm{Ti,~Mn})、次氧化物(\textrm{ZrO})和氧化物($\mathrm{CeO}_2$)在\textrm{He}气中蒸发,借助在过度饱和蒸汽相中的核形成\textrm{n}型粒子,然后在一个液氮冷却的不锈钢盘中收集。在合成室里用$\mathrm{O}_2$取代$\mathrm{H}_2$得到\textrm{n}型晶体结构的氧化物,接着在$300^{\circ}\mathrm{C}$焙烧,最后将这些载体用$\mathrm{Pd}(\mathrm{NO}_3)_2$溶液浸渍,在$280^{\circ}\mathrm{C}$或$500^{\circ}{C}$焙烧。结果显示,以$\mathrm{CeO}_2$为载体的催化剂活性要优于以$\mathrm{Mn}_3\mathrm{O}_4$为载体的催化剂,\textrm{Pd}/$\mathrm{ZrO}_2$的催化性能与\textrm{Pd}/$\mathrm{CeO}_2$相当。\textrm{n}型晶体结构的载体具有优良的催化活性,而\textrm{p}型晶体结构的载体几乎是非活性的。

金属氧化物混合物$\mathrm{SnO}_2$-$\mathrm{MO}_x$(\textrm{M=Al,~Ce,~Fe,~Mn,~Ni,~Zr})为载体的\textrm{Pd}催化剂\cite{CataD59-19_2000},对甲烷的催化氧化结果显示,\textrm{Pd}/$\mathrm{SnO}_2$-$\mathrm{MO}_x$的活性均低于\textrm{Pd}/$\mathrm{SnO}_2$,此现象同时出现在$\mathrm{Al}_2\mathrm{O}_3$,$\mathrm{CeO}_2$及$\mathrm{ZrO}_2$与其他金属氧化物混合物的催化剂上,其原因可能是因为载体从一相到另一相密度不同造成的。\textrm{Bozo}等\cite{CataT59-33_2000}由共沉淀法制备的\textrm{Pt}/$\mathrm{CeO}_2$-$\mathrm{ZrO}_2$催化剂($\mathrm{T}_{50}=335^{\circ}\mathrm{C}$),活性高于Pt/$\mathrm{Al}_2\mathrm{O}_3$催化剂($\mathrm{T}_{50}=470^{\circ}\mathrm{C}$),与Pd/$\mathrm{Al}_2\mathrm{O}_3$催化剂($\mathrm{T}_{50}=320^{\circ}\mathrm{C}$)相当,但在较高温度下,其催化活性随反应进行而显著降低。这是由于\textrm{Pt}的氧化物或\textrm{Pt}-\textrm{O}-\textrm{Ce}相阻挡了$\mathrm{O}_2$的离解和逸出。

为了提高活性中心的稳定性,一些研究者进行了如下的尝试,即向传统的载体$\mathrm{Al}_2\mathrm{O}_3$上引入其他金属氧化物作为活性相。文献\inlinecite{ACA222-359_2001}考察了各种不同添加剂(\textrm{Ni,~Sn,~Ag,~Rh,~Mn,~Pt,~Pb,~Co,~Fe,~Cr,~Ce,~Cu})对\textrm{Pd}/$\mathrm{Al}_2\mathrm{O}_3$催化剂的影响。1\%\mathrm{Pd}/$\mathrm{Al}_2\mathrm{O}_3$催化剂由前驱物$\mathrm{PdCl}_2$和金属硝酸盐以共沉淀法沉积在$\mathrm{Al}_2\mathrm{O}_3$上,然后在$500^{\circ}\mathrm{C}$经$\mathrm{H}_2$还原,在空气中$800^{\circ}\mathrm{C}$焙烧制备。作者认为这里\textrm{Pd}粒子分散度对催化活性的影响可忽略。改性后催化剂的活性比\textrm{Pd}/$\mathrm{Al}_2\mathrm{O}_3$催化剂有显著提高。尤其是经\textrm{NiO}改性的\textrm{Pd}/$\mathrm{Al}_2\mathrm{O}_3$活性最高($\mathrm{T}_{30}=350^{\circ}\mathrm{C}$,$\mathrm{T}_{50}=380^{\circ}\mathrm{C}$),这是由于\textrm{NiO}加强了\textrm{PdO}的稳定性,使\textrm{PdO}分解为\textrm{Pd}的温度提高。而\textrm{Thevenin}\cite{JCata215-78_2003}等研究认为,掺杂\textrm{Ce}的2.5\%\textrm{Pd}/$\gamma$-$\mathrm{Al}_2\mathrm{O}_3$,\textrm{Pd}/\textrm{Ba}-$\mathrm{Al}_2\mathrm{O}_3$,\textrm{Pd}/\textrm{La}-$\mathrm{Al}_2\mathrm{O}_3$催化剂均表现出较好的甲烷催化燃烧活性及热力学稳定性,原因是$\mathrm{CeO}_2$加强了\textrm{PdO}的稳定性,使\textrm{PdO}分解为\textrm{Pd}的温度提高。

其他低碳烷烃在载体上的催化燃烧反应,对深入研究甲烷催化燃烧反应具有借鉴意义。如丙烷在各种金属氧化物载体上的氧化反应,其载体活性顺序如下:~$\mathrm{Al}_2\mathrm{O}_3>\mathrm{SiO}_2>\mathrm{TiO}_2>\mathrm{CeO}_2$,$\mathrm{Al}_2\mathrm{O}_3>\mathrm{ZrO}_2>\mathrm{La}_2\mathrm{O}_3>\mathrm{MgO}$.催化剂活性不同是由于载体酸碱度不同,酸性载体活性高于碱性载体\cite{ACA211-159_2001}。尽管丙烷要比甲烷对$\mathrm{O}_2$的反应活性高,但对甲烷而言,载体的酸碱度也是影响其催化燃烧活性的一个因素。此外,水及硫化物对催化剂活性和稳定性的影响也是需要考虑的因素。

\subsection{负载型\rm{Rh},\rm{Au}催化剂}
相对\textrm{Pd},\textrm{Pt}催化剂来说,负载型\textrm{Rh}\cite{CataD83-71_2003},\textrm{Au}催化剂则报道较少。\textrm{Pecchi}等\cite{JCTB74-897_1999}用溶胶-凝胶法制取的\textrm{Rh}/$\mathrm{ZrO}_2$和\textrm{Rh}/$\mathrm{ZrO}_2$-$\mathrm{SiO}_2$催化剂,认为\textrm{Rh}较高的分散度和$\mathrm{Rh}^{\delta+}$的存在提高了催化反应活性,但反应物中氯离子降低了催化活性。

2000年以后,\textrm{Au}催化剂开始受到人们的关注。已经发现由共沉淀法、沉积-沉淀等方法制备的过渡金属氧化物负载型\textrm{Au}催化剂,对甲烷燃烧有较好的催化活性\cite{JCata191-430_2000,CataD64-69_2001}。由共沉淀法制备的负载型金催化剂,对甲烷燃烧反应的催化活性顺序如下:~$\mathrm{Au}/\mathrm{Co}_3\mathrm{O}_4>\mathrm{Au}/\mathrm{NiO}>\mathrm{Au}/\mathrm{MnO}_x>\mathrm{Au}/\mathrm{Fe}_2\mathrm{O}_3\gg\mathrm{Au}/\mathrm{CeO}_2$。对于甲烷燃烧反应,负载型金催化剂的活性高于\mathrm{Pt}/$\mathrm{Al}_2\mathrm{O}_3$的活性。活性最高的\mathrm{Au}/$\mathrm{Co}_3\mathrm{O}_4$催化剂具有已商品化的\textrm{Pd}/$\mathrm{Al}_2\mathrm{O}_3$催化剂相当的活性。但如何保持高分散的金催化剂在催化燃烧反应中的稳定性值得进一步深入研究。

\subsection{双、多贵金属催化剂体系}
为了保持催化活性在一个较高的水平,将\textrm{Pd}和一种或多种铂族元素连用制成双、多贵金属催化剂,应用于甲烷燃烧反应。负载型\textrm{Pd}-\textrm{Pt},\textrm{Pd}-\textrm{Rh}\cite{ACA226-281_2002,CataT83-265_2003}催化剂被提出用来提高催化性能。当用\textrm{Pd}-\textrm{Pt}代替\textrm{Pd}时,催化性能得到进一步提高。\textrm{Yamamoto}和\textrm{Uchida}\cite{CataD45-147_1998}研究了沉淀法制备的\textrm{Pt}-\textrm{Pd}/$\mathrm{Al}_2\mathrm{O}_3$,Pt/$\mathrm{Al}_2\mathrm{O}_3$和Pd/$\mathrm{Al}_2\mathrm{O}_3$催化剂,并作一比较。发现在催化活性和催化稳定性方面,5\%\textrm{Pt}-5\%\textrm{Pd}/$\mathrm{Al}_2\mathrm{O}_3$催化剂较5\%\textrm{Pd}/$\mathrm{Al}_2\mathrm{O}_3$及5\%\textrm{Pt}/$\mathrm{Al}_2\mathrm{O}_3$催化剂有显著提高。提高\textrm{Pt}-\textrm{Pd}催化剂中\textrm{Pd}的含量可以提高催化活性和催化稳定性。经研究认为催化稳定性的提高是由于复合催化剂中的\textrm{Pt}抑制了\textrm{Pd}/\textrm{PdO}的烧结,但这种提高与催化剂中\textrm{Pt}的含量无关。

当把以$\mathrm{Co}_3\mathrm{O}_4$为载体的\textrm{Au}、\textrm{Pt}和\textrm{Pd}催化剂应用于甲烷催化燃烧\cite{ACB31-L1_2001},发现在$\mathrm{Co}_3\mathrm{O}_4$担载的\textrm{Au}催化剂中引入\textrm{Pt},可以明显降低甲烷完全氧化反应的起燃温度和完全转化温度。对甲烷完全氧化反应的活性优于贵金属担载量相近的\textrm{Pd}/$\mathrm{Co}_3\mathrm{O}_4$催化剂,这是由于在$\mathrm{Co}_3\mathrm{O}_4$载体上\textrm{Pt}和\textrm{Au}之间存在的协同作用提高了氧化甲烷的活性。\textrm{Au}-\textrm{Pt}/$\mathrm{Co}_3\mathrm{O}_4$(起燃温度约为$218^{\circ}\mathrm{C}$,完全转化温度约为$360^{\circ}\mathrm{C}$)是一个很有应用潜力的甲烷低温燃烧催化剂。

\section{金属氧化物催化剂}
贵金属催化剂具有良好的低温起燃活性和催化性能,但价格昂贵,热稳定性相对较差,易烧结,其应用受到一定限制。而金属氧化物催化剂原料价廉易得且可抑制$\mathrm{NO}_x$的生成,燃烧活性接近贵金属催化剂,热稳定性更高,有望在将来部分甚至完全取代贵金属催化剂。其中钙钛矿型催化剂和六铝酸盐型催化剂是金属氧化物催化剂研究的焦点。以\textrm{Cu},~\textrm{Co},~\textrm{Mn},~\textrm{Cr},\textrm{Ni}等单一过渡金属氧化物为活性组分的催化剂,对甲烷催化燃烧也有较好的活性,对这些金属氧化物进行掺杂可以使其催化性能发生显著改变。

\subsection{钙钛矿型金属氧化物催化剂}
钙钛矿型金属氧化物催化剂,通式为$\mathrm{ABO}_3$,一般\textrm{A}为稀土金属,\textrm{B}为过渡金属。这类催化剂对甲烷的燃烧活性主要依赖\textrm{B}位组分的氧化物。\textrm{Mn}\cite{ACB24-193_2000}和\textrm{Co}因为高催化活性,已成为钙钛矿型金属氧化物催化剂B位组分研究的焦点,而\textrm{La}\cite{ACA262-167_2004}则是\textrm{A}位阳离子广泛研究的对象。当部分取代\textrm{A},~\textrm{B}位的阳离子时,催化剂的催化性能可发生显著改变,如\textrm{A}位取代($\mathrm{A}_x\mathrm{A}^{\prime}_{1-x}\mathrm{BO}_3$),\mathrm{B}位取代($\mathrm{AB}_x\mathrm{B}^{\prime}_{1-x}\mathrm{O}_3$),或者\mathrm{A},~\mathrm{B}位同时被取代($\mathrm{A}_x\mathrm{A}^{\prime}_{1-x}\mathrm{B}_x\mathrm{B}^{\prime}_{1-x}\mathrm{O}_3$)。

钙钛矿型金属氧化物催化剂的制备方法一般有柠檬酸盐法,硅酸盐法,共沉淀法,冷冻-干燥法,溅射干燥法等,而催化剂比表面积的大小主要依赖于其制备方法。由于硅酸盐法需要较高的焙烧温度,因此导致钙钛矿型催化剂比表面积较低。而柠檬酸盐法和冷冻-干燥法较低的焙烧温度,使得这两种方法制备的钙钛矿型催化剂比表面积较高(超过20$\mathrm{m}^2/\mathrm{g}$)。\textrm{Marchetti}等\cite{ACB15-179_1998}采用柠檬酸盐法制备的催化剂$\mathrm{La}_{1-x}\mathrm{A}_x\mathrm{MnO}_3$(\textrm{A=Sr,~Eu,~Ce}),完全转化温度在$500\sim$600$^{\circ}\mathrm{C}$之间,连续测试100\textrm{h}催化性能无显著改变。由此说明这些催化剂中的两种氧(低温高活性的吸附氧和高温高活性的晶格氧)是活性氧,它们促进了催化活性。\textrm{Song}等\cite{CataT47-155_1999}采用溅射分解法(冷冻-干燥法和柠檬酸盐法的结合)合成了$\mathrm{La}_{1-x}\mathrm{M}_x\mathrm{MnO}_3$催化剂。比起冷冻-干燥法和无定形柠檬酸盐法,这种方法既简单、价廉,且制得的催化剂有相当高的比表面积。\textrm{Leanza}等\cite{ACB28-55_2000}由火焰加水分解法合成了稳定的、具有\textrm{n}型晶体结构的钙钛矿型催化剂$\mathrm{La}_{1-x}\mathrm{M}_x\mathrm{CoO}_{3+\delta}$(\textrm{M=Ce,~Eu},$x$=0,~0.05,~0.1,~0.2),这两种方法较之传统方法得到的催化剂粒子尺寸小得多。

由于钙钛矿型金属氧化物催化剂在高温下反应易烧结,一些研究者采取在具有较高比表面积的载体上担载分散的钙钛矿物来解决这一问题。\textrm{Cimino}等\cite{CataD59-19_2000}制备的$\mathrm{LaMnO}_3$/\textrm{MgO}催化剂获得了预期效果。与$\mathrm{Al}_2\mathrm{O}_3$不同,高温时钙钛矿和\textrm{MgO}之间无相互作用。当在1373\textrm{K}进行预处理后,$\mathrm{LaMnO}_3$/\mathrm{MgO}显示了比$\mathrm{LaMnO}_3$/$\mathrm{Al}_2\mathrm{O}_3$更高的稳定性。

\subsection{六铝酸盐型金属氧化物催化剂}
六铝酸盐型催化剂可以用$\mathrm{AAl}_{12}\mathrm{O}_{19}$表示,\textrm{A}通常是碱金属、碱土金属或稀土金属。由于它们的薄层结构(由单分子氧化物分离的尖晶石块组成),六铝酸盐型催化剂具有高的热力学稳定性。\textrm{A}位阳离子的半径和价态决定了六铝酸盐催化剂的晶体结构类型($\beta$-$\mathrm{Al}_2\mathrm{O}_3$型或磁铅石型)。

六铝酸盐型催化剂的合成一般采用粉末固态反应法,醇盐水解法,改性的溶胶-凝胶法,共沉淀法以及微乳法\cite{ACA265-207_2004}等。醇盐水解法比固态反应法得到的样品的比表面积高。这是由于前者的前驱物中存在更多有利于形成较高比表面积的混合组分。然而,醇盐水解法的一个重要缺点是它在合成六铝酸盐时需要苛刻的反应条件(潮湿且不含氧)。相比较而言,共沉淀法既可以得到具有类似较高比表面积的材料,又对甲烷氧化反应无需特殊的环境条件。微乳法的优点在于它允许尽可能地重复利用微乳中剩余的反应组分,同时得到十分微小的催化粒子,且粒度均匀分布,有很好的催化活性。该技术可应用于实际生产。

未经掺杂的催化剂具有高的热力学稳定性,但催化活性非常低,掺杂后($\mathrm{A}_{1-x}\mathrm{A}^{\prime}\mathrm{B}_x\mathrm{Al}_{12-x}\mathrm{O}_{19}$),催化活性得到提高\cite{CataD59-163_2000,MSEA384-324_2004}。Artizzu等\cite{CataD59-163_2000}用溶胶-凝胶法制备了$\mathrm{BaM}_x\mathrm{Al}_{12-x}\mathrm{O}_{19}$(\textrm{M=Mn,~Fe})催化剂,发现其中\textrm{Fe,~Mn}同时取代的催化剂$\mathrm{BaFeMn}_x\mathrm{Al}_{11-x}\mathrm{O}_{19}$甲烷燃烧性能较好,尤其是$\mathrm{BaFeMnAl}_{10}\mathrm{O}_{19}$在733\textrm{K}即达到10\%的转化率,983\textrm{K}达到90\%转化率,而无\textrm{Mn}取代的化合物$\mathrm{BaAl}_{12}\mathrm{O}_{19}$达到同样的转化率时分别为973\textrm{K}(10\%)和1073\textrm{K}(90\%),这是由于在化合物中同时掺杂$\textrm{Fe}^{3+}$和\textrm{Mn}离子(\textrm{Mn}可以在$\mathrm{Mn}^{2+}$和$\mathrm{Mn}^{3+}$之间转变)可以改变化合物的内部结构,且$\mathrm{Fe}^{3+}$在$\mathrm{Mn}^{2+}$的作用下,其活性会得到提高,从而提高催化剂的催化性能。

六铝酸盐还可以作为载体用于甲烷燃烧。如\textrm{Pd}/$\mathrm{Sr}_{0.8}\mathrm{La}_{0.2}\mathrm{Al}_{12}\mathrm{O}_{19}$催化剂\cite{JMCA186-135_2002}在反应开始时催化活性随着温度的上升而提高,但在温度超过973\textrm{K}后,由于\textrm{Pd}粒子的凝聚导致活性下降。而在用\textrm{Mn}进行部分取代后,在高温时\textrm{Pd}的烧结作用有所缓和。\textrm{Sidwell}\cite{ACA255-279_2003}等研究发现$\mathrm{Sr}_{0.5}\mathrm{Pd}_{0.5}\mathrm{La}_{0.2}\mathrm{Al}_{11}\mathrm{O}_{18-\alpha}$适于低温甲烷催化燃烧,而$\mathrm{La}_x\mathrm{Sr}_y\mathrm{Mn}_{0.4}\mathrm{Al}_{11}\mathrm{O}_{18-\alpha}$($x/y=0.8,~0.9,~1.1$)则是很好的高温甲烷催化燃烧材料。同时在温度超过1173\textrm{K}时,\textrm{Pd}取代的六铝酸盐催化剂比3\%\textrm{Pd}/$\gamma$-$\mathrm{Al}_2\mathrm{O}_3$具有更好的催化稳定性。

\subsection{其他金属氧化物催化剂}
以\textrm{Cu},~\textrm{Co},~\textrm{Mn},~\textrm{Cr},~\textrm{Ni}等单一过渡金属氧化物为活性组分的催化剂,对甲烷的催化燃烧也有较好的活性,如\textrm{CuO}/$\mathrm{Al}_2\mathrm{O}_3$,$\mathrm{Fe}_2\mathrm{O}_3$,$\mathrm{Mn}_3\mathrm{O}_4$等。对\textrm{CuO}/$\mathrm{Al}_2\mathrm{O}_3$这种类型催化剂来说,其主要的缺点是在高温时活性组分\textrm{CuO}和载体$\mathrm{Al}_2\mathrm{O}_3$发生反应,从而导致催化活性降低,且金属\textrm{Cu}担载量的增加也会使甲烷催化燃烧活性下降。这是由于催化剂表面孤立的\textrm{Cu}物种较晶体\textrm{CuO}活性高,而且较高的\textrm{Cu}担载量会导致分散的\textrm{CuO}晶体形成困难,因此使甲烷燃烧活性较低。但采用溶胶2凝胶法制备的催化剂\textrm{CuO}/\textrm{Zn}-$\mathrm{Al}_2\mathrm{O}_3$,则显示了较高的比表面积和较好的甲烷燃烧活性。\textrm{Choudhary}等发现在氧化锆中掺杂过渡金属如\textrm{Mn},~\textrm{Co},~\textrm{Cr},\textrm{Fe}等,使甲烷及丙烷的燃烧活性有惊人的提高。其研究表明,过渡金属掺杂的氧化锆催化剂,活性要高于钙钛矿型催化剂,与负载型贵金属催化剂相当。这是由于掺杂后$\mathrm{ZrO}_2$催化剂中的晶格氧反应活性得到提高。这种提高可能是由于晶格\cite{ACA253-65_2003}缺陷的产生和晶格氧迁移率提高造成的。此外,由\textrm{Ca},~\textrm{Mn},\textrm{Nd}等掺杂的$\mathrm{CeO}_2$催化剂也显示了较没有掺杂的$\mathrm{CeO}_2$催化剂更好的活性,而加入\textrm{PdO}却降低了催化活性。在\textrm{NiO}中加入\textrm{La}和\textrm{Zr}能够控制催化剂的晶体尺寸和还原性能,这是由于掺杂后的样品还原性较好,因此改性的\textrm{NiO}催化剂使甲烷氧化活性提高。加入过渡金属,如\textrm{Ag}和\textrm{Cu}也可以提高样品的甲烷燃烧活性。

在甲烷催化燃烧过程中所用催化剂存在两个关键问题:~热稳定性和低温活性。\texrrm{Pd},~\textrm{Pt},~\textrm{Rh},~\textrm{Au}等负载型贵金属是传统的甲烷低温催化燃烧催化剂,具有优良的低温起燃活性和催化性能,但目前还没有研制出能抗硫中毒的有效的负载型Pd催化剂。鉴于贵金属催化剂价格昂贵、易烧结,金属氧化物催化剂近年来吸引了研究者广泛地注意力,尤其是钙钛矿型金属氧化物催化剂。该类催化剂已经广泛地应用于甲烷催化燃烧。对于甲烷高温燃烧反应,贵金属及金属氧化物催化剂的热稳定性均较差,需对其改性或使用复合载体。如何提高催化剂的比表面积以及活性相和载体之间的协同效应,催化剂的制备方法尤为重要,这还需要更多大量的研究工作。
