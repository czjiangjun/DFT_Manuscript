
%%% Local Variables:
%%% mode: latex
%%% TeX-master: t
%%% End:

%\chapter{FP-LAPW与WIEN2k程序}
\section{Car-Parrinello方法简介}
标准\textrm{DFT}方法中,与电子波函数有关的自由度(即基函数的组合系数)与核坐标自由度无关。电子波函数被视为是核坐标的函数。用于确定电子与核坐标的计算方法完全不同。电子波函数的组合系数常通过对一套固定核坐标,求解自洽的\textrm{Kohn-Sham}方程得到,而核坐标自由度的确定,常常是核坐标位置对总能量变分(通常是力)得到。与此相反,\textrm{Car-Parrinello (CP)}方法将电子与核自由度同等对待,采用的是类似于经典的分子动力学方法,\textbf{模拟假想的包含两套粒子自由度的经典动力学体系}。

\textrm{Car}和\textrm{Parrinello}假设,电子运动的\textrm{Lagrangian}算符:
\begin{equation}
	L=\sum_j\dfrac12\mu\langle\dot{\psi_j}|\dot{\psi_j}\rangle-E[\{\psi\},\{\vec R\}]
	\label{eq:Lagrangian_orig}
\end{equation}
这里求和遍及所有电子占据态,$E$是总的电子态能量,$\mu$是假想粒子的质量,$\vec R$是核坐标,$\psi$是电子波函数,$\dot{\psi}$是$\psi$对时间的导数。该\textrm{Lagrangian}算符考虑了假想电子的动力学,当外加的包含动能的一项与核自由度有关,该算符就成为同时描述核与假想电子经典运动的算符。

不巧的是,由于\textrm{Kohn-Sham}方程要求电子轨道彼此正交,这就意味着变量$\dot{\psi}\delta t$是非任意的。在\textrm{CP}框架内,通过\textrm{Lagrangian}乘子$\Lambda_{ij}$引入该约束条件,因此\textrm{Lagrangian}算符变为:
\begin{equation}
	L=\sum_j\dfrac12\mu\langle\dot{\psi_j}|\dot{\psi_j}\rangle-E[\{\psi\},\{\vec R\}]+\sum_{ij}\Lambda_{ij}[\int_{\Omega}\mathrm{d}^3r\phi_i^{\ast}\phi_j-\delta_{ij}]
	\label{eq:Lagrangian}
\end{equation}
这里积分遍及整个原胞。

将此$L$代入\textrm{Lagrangian}运动方程,得到动力学运动方程:
\begin{equation}
	\mu\ddot{\psi_j}=-H\psi_j+\sum_i\Lambda_{ji}\phi_i
	\label{eq:dynamical_equation}
\end{equation}
这里$H$是\textrm{Kohn-Sham}方程的单粒子\textrm{Hamiltonian}。\textrm{Car}和\textrm{Parrinello}强调由该\textrm{Lagrangian}描述的是\textbf{假想的动力学过程}。\textrm{DFT}只在\textrm{Born-Oppenheimer}表面成立,即能量变分最小值处。不过该方程为求解电子结构问题提供了一个解决方案。

考虑核坐标固定的情况下,动力学可以描述初始态电子在\textrm{Kohn-Sham}下的\textrm{Hamiltonian}作用下的运动径迹。通过模拟退火\textrm{(annealing)}求解电子结构问题。根据此方法,运动方程增加一项,因此在模拟中的每个时间步长内,动能被\textrm{removed},动力学方程中的\textrm{Kohn-Sham}的\textrm{Hamiltonian}在每个时间步长内都更新,因为电子本征态波函数改变,电子密度和势能都需要重新计算;采用这种方式,当所有许可的能量都被\textrm{removed},电子结构问题也就求解完毕。极限状态下,每个时间步长内,动能的极大值被\textrm{removed},该拟合退火过程退化成能量最小化的“最陡下降法”。模拟退火的每一步,每个本征矢系数沿着作用在它上面的力的方向运动,运动的距离与作用力的大小、时间步长$\delta t$和假想电子质量$\mu$都有关。考虑到这种相关性,必须合理地选择步长$\delta t$和$\mu$:如果$\delta t$太短(或$\mu$太大),\textrm{move}能量过小,为求解步数过多;如果$\delta t$过长(或$\mu$过小),动力学积分方程将变得不稳定。

更一般的情况,同时考虑核坐标的运动。在这种情况下,退火过程包括核运动的自由度,在\textrm{Kohn-Sham}的\textrm{Hamiltonian}作用下,电子本征态的组合系数发生变化,由于电子结构导致的\textrm{Hellmann-Feynman}力将改变核坐标。当所有的能量都被\textrm{removed},电子与核运动的自由度都不再改变,因此核坐标将构成最小势能面,电子本征态系数在是此条件下最优解。由模拟退火算法和初始态、时间步长$\delta t$、$\mu$与核质量$M_i$,可以得到粒子径迹。
